% \subsection{准备}
%
%    \begin{macrocode}
%<@@=fdu>
%<*class|class-en>
%    \end{macrocode}
%
% 检查 \LaTeX3 编程环境。
%    \begin{macrocode}
\RequirePackage { xparse, xtemplate, l3keys2e }
\clist_map_inline:nn { expl3, xparse, xtemplate, l3keys2e }
  {
    \@ifpackagelater {#1} { 2018/05/12 }
      { } { \msg_error:nnn { fduthesis } { l3-too-old } {#1} }
  }
\msg_new:nnn { fduthesis } { l3-too-old }
  {
    Package~ "#1"~ is~ too~ old. \\\\
    Please~ update~ an~ up-to-date~ version~ of~ the~ bundles \\
    "l3kernel"~ and~ "l3packages"~ using~ your~ TeX~ package \\
    manager~ or~ from~ CTAN.
  }
%    \end{macrocode}
%
% 目前 \cls{fduthesis} 仅支持 \XeTeX{} 和 \LuaTeX{}。
%    \begin{macrocode}
\sys_if_engine_xetex:F
  {
    \sys_if_engine_luatex:F
      {
        \msg_fatal:nnx { fduthesis } { unsupported-engine }
          { \c_sys_engine_str }
      }
  }
\msg_new:nnn { fduthesis } { unsupported-engine }
  {
    The~ fduthesis~ class~ requires~ either~ XeTeX~ or~ LuaTeX. \\\\
    "#1"~ is~ not~ supported~ at~ present.~ You~ must~ change \\
    your~ typesetting~ engine~ to~ "xelatex"~ or~ "lualatex".
  }
%    \end{macrocode}
%
% \subsubsection{内部变量声明}
%
% \begin{variable}{\l_@@_tmpa_box,
%   \l_@@_tmpa_clist,\l_@@_tmpb_clist,
%   \l_@@_tmpa_dim,\l_@@_tmpb_dim,
%   \l_@@_tmpa_skip,
%   \l_@@_tmpa_tl,\l_@@_tmpb_tl}
% 临时变量。
%    \begin{macrocode}
\box_new:N   \l_@@_tmpa_box
\clist_new:N \l_@@_tmpa_clist
\clist_new:N \l_@@_tmpb_clist
\dim_new:N   \l_@@_tmpa_dim
\dim_new:N   \l_@@_tmpb_dim
\skip_new:N  \l_@@_tmpa_skip
\tl_new:N    \l_@@_tmpa_tl
\tl_new:N    \l_@@_tmpb_tl
%    \end{macrocode}
% \end{variable}
%
% \begin{variable}{\g_@@_thesis_type_int}
% 论文类型。取值 1、2、3 分别对应博士、硕士、本科(学士),这与学号
% 第三位是一致的。
%    \begin{macrocode}
\int_new:N \g_@@_thesis_type_int
%    \end{macrocode}
% \end{variable}
%
% \begin{variable}{\g_@@_to_ctexbook_clist,\g_@@_to_hyperref_clist}
% 分别保存由 \cls{fduthesis} 传入 \cls{ctexbook} 文档类和
% \pkg{hyperref} 宏包的选项列表。
%    \begin{macrocode}
\clist_new:N \g_@@_to_ctexbook_clist
\clist_new:N \g_@@_to_hyperref_clist
%    \end{macrocode}
% \end{variable}
%
% \begin{variable}{\g_@@_twoside_bool}
% 是否开启双页模式(默认打开)。
%    \begin{macrocode}
\bool_new:N \g_@@_twoside_bool
\bool_set_true:N \g_@@_twoside_bool
%    \end{macrocode}
% \end{variable}
%
% \begin{variable}{\g_@@_draft_bool}
% 是否开启草稿模式。
%    \begin{macrocode}
\bool_new:N \g_@@_draft_bool
%    \end{macrocode}
% \end{variable}
%
% \begin{variable}{\g_@@_config_tl}
% 保存配置文件名称。默认为空。
%    \begin{macrocode}
\tl_new:N \g_@@_config_tl
%    \end{macrocode}
% \end{variable}
%
% \subsubsection{内部函数}
%
% \begin{macro}{\cs_generate_variant:cn,
%   \file_input:V,
%   \int_to_arabic:v,
%   \keys_define:nx,
%   \tl_map_inline:xn}
% \begin{macro}[TF]{\tl_if_eq:Vn}
% \LaTeX3{} 函数变体。
%    \begin{macrocode}
\cs_generate_variant:Nn \cs_generate_variant:Nn { cn }
\cs_generate_variant:Nn \file_input:n           { V  }
\cs_generate_variant:Nn \int_to_arabic:n        { v  }
\cs_generate_variant:Nn \keys_define:nn         { nx }
\cs_generate_variant:Nn \tl_map_inline:nn       { xn }
\prg_generate_conditional_variant:Nnn \tl_if_eq:nn { Vn } { T, TF }
%    \end{macrocode}
% \end{macro}
% \end{macro}
%
% \begin{macro}{\@@_quad:,\@@_qquad:}
% 等价于 \LaTeXe{} 中的 \tn{quad} 和 \tn{qquad}。
%    \begin{macrocode}
\cs_new:Npn \@@_quad:  { \skip_horizontal:n { 1 em } }
\cs_new:Npn \@@_qquad: { \skip_horizontal:n { 2 em } }
%    \end{macrocode}
% \end{macro}
%
% \begin{macro}{\@@_vspace:N,\@@_vspace:n,\@@_vspace:c}
% 类似 \LaTeXe{} 中的 \tn{vspace*}。
%    \begin{macrocode}
\cs_new_protected:Npn \@@_vspace:N #1
  {
    \dim_set_eq:NN \l_@@_tmpa_dim \prevdepth
    \hrule height \c_zero_dim
    \nobreak
    \skip_vertical:N #1
    \skip_vertical:N \c_zero_skip
    \dim_set_eq:NN \prevdepth \l_@@_tmpa_dim
  }
\cs_new_protected:Npn \@@_vspace:n #1
  {
    \skip_set:Nn \l_@@_tmpa_skip {#1}
    \@@_vspace:N \l_@@_tmpa_skip
  }
\cs_generate_variant:Nn \@@_vspace:N { c }
%    \end{macrocode}
% \end{macro}
%
% \begin{macro}{\@@_symbol:n}
% 等价于 \LaTeXe{} 中的 \tn{symbol}。
%    \begin{macrocode}
\cs_new:Npn \@@_symbol:n #1 { \tex_char:D #1 \scan_stop: }
%    \end{macrocode}
% \end{macro}
%
% \begin{macro}{\@@_arabic:n}
% 等价于 \LaTeXe{} 中的 \tn{arabic}。
%    \begin{macrocode}
\cs_new:Npn \@@_arabic:n #1
  { \int_to_arabic:v { c@ #1 } }
%    \end{macrocode}
% \end{macro}
%
% \begin{macro}{\@@_patch_cmd:Nnn,\@@_appto_cmd:Nn}
% 补丁工具,来自 \pkg{ctexpatch} 宏包。
%    \begin{macrocode}
\cs_new_protected:Npn \@@_patch_cmd:Nnn #1#2#3
  {
    \ctex_patch_cmd_once:NnnnTF #1 { } {#2} {#3}
      { } { \ctex_patch_failure:N #1 }
  }
\cs_new_protected:Npn \@@_appto_cmd:Nn #1#2
  {
    \ctex_appto_cmd:NnnTF #1 { } {#2}
      { } { \ctex_patch_failure:N #1 }
  }
%    \end{macrocode}
% \end{macro}
%
% \begin{macro}{\@@_define_fn_style:nn,
%   \@@_define_punct:nn,
%   \@@_define_sep:nn,
%   \@@_define_format:nn,
%   \@@_define_name:nn,
%   \@@_define_name:nnn}
% 用来定义脚注样式、标点、默认名称的辅助函数。
%    \begin{macrocode}
\cs_new_protected:Npn \@@_define_fn_style:nn #1#2
  { \tl_const:cn { c_@@_fn_style_ #1 _tl } {#2} }
\cs_new_protected:Npn \@@_define_punct:nn #1#2
  { \tl_const:cn { c_@@_ #1 _tl } { \@@_symbol:n {#2} } }
% \cs_new_protected:Npn \@@_define_sep:nn #1#2
%   { \tl_const:cn { c_@@_ #1 _sep_tl } {#2} }
% \cs_new_protected:Npn \@@_define_format:nn #1#2
%   { \tl_const:cn { c_@@_ #1 _format_tl } {#2} }
\cs_new_protected:Npn \@@_define_name:nn #1#2
  { \tl_const:cn { c_@@_name_ #1 _tl } {#2} }
\cs_new_protected:Npn \@@_define_name:nnn #1#2#3
  {
    \tl_const:cn { c_@@_name_ #1    _tl } {#2}
    \tl_const:cn { c_@@_name_ #1 _en_tl } {#3}
  }
%    \end{macrocode}
% \end{macro}
%
% \begin{macro}{\@@_msg_new:nn,
%   \@@_error:n,\@@_error:nn,\@@_error:nx,\@@_error:nnn,
%   \@@_warning:n,\@@_warning:nn,\@@_warning:nxx,
%   \@@_info:nx}
% 各种信息函数的缩略形式。
%    \begin{macrocode}
\cs_new:Npn \@@_msg_new:nn  { \msg_new:nnn      { fduthesis } }
\cs_new:Npn \@@_error:n     { \msg_error:nn     { fduthesis } }
\cs_new:Npn \@@_error:nn    { \msg_error:nnn    { fduthesis } }
\cs_new:Npn \@@_error:nx    { \msg_error:nnx    { fduthesis } }
\cs_new:Npn \@@_error:nnn   { \msg_error:nnnn   { fduthesis } }
\cs_new:Npn \@@_warning:n   { \msg_warning:nn   { fduthesis } }
\cs_new:Npn \@@_warning:nn  { \msg_warning:nnn  { fduthesis } }
\cs_new:Npn \@@_warning:nxx { \msg_warning:nnxx { fduthesis } }
\cs_new:Npn \@@_info:nx     { \msg_info:nnx     { fduthesis } }
%    \end{macrocode}
% \end{macro}
%
% \subsection{选项处理}
%
% 定义 |fdu/option| 键值类。
%    \begin{macrocode}
\keys_define:nn { fdu / option }
  {
%    \end{macrocode}
%
% \changes{v0.7}{2018/02/01}{新增 \opt{type} 选项。}
%
% \begin{macro}{type}
% 设置论文类型。设为模板选项主要是为了以后的兼容性。论文类型可能会
% 影响很多设置,只是暂时还不考虑。默认为本科毕业论文。
%    \begin{macrocode}
    type .choice:,
    type .value_required:n = true,
    type .choices:nn = 
      { doctor, master, bachelor }
      { \int_set_eq:NN \g_@@_thesis_type_int \l_keys_choice_int },
    type .initial:n = bachelor,
%    \end{macrocode}
% \end{macro}
%
% \begin{macro}{oneside,twoside}
% 设置页面类型为单面或双面。
%    \begin{macrocode}
    oneside .value_forbidden:n = true,
    twoside .value_forbidden:n = true,
    oneside .code:n =
      {
        \clist_gput_right:Nn \g_@@_to_ctexbook_clist { oneside }
        \bool_set_false:N    \g_@@_twoside_bool
      },
    twoside .code:n =
      {
        \clist_gput_right:Nn \g_@@_to_ctexbook_clist { twoside }
        \bool_set_true:N     \g_@@_twoside_bool
      },
%    \end{macrocode}
% \end{macro}
%
% \changes{v0.5}{2017/09/23}{移除 \opt{nofonts} 选项。}
%
% \begin{macro}{draft}
% 是否开启草稿模式(默认关闭)。
%    \begin{macrocode}
    draft .choice:,
    draft / true  .code:n =
      {
        \bool_set_true:N     \g_@@_draft_bool
        \clist_gput_right:Nn \g_@@_to_ctexbook_clist { draft }
      },
    draft / false .code:n =
      { \bool_set_false:N    \g_@@_draft_bool },
    draft .default:n = true,
    draft .initial:n = false,
%    \end{macrocode}
% \end{macro}
%
% \changes{v0.7}{2018/01/31}{新增 \opt{config} 选项。}
%
% \begin{macro}{config}
% 配置文件名。
%    \begin{macrocode}
    config .tl_set:N = \g_@@_config_tl,
%    \end{macrocode}
% \end{macro}
%
% 处理未知选项。
%    \begin{macrocode}
    unknown .code:n = { \@@_error:n { unknown-option } }
  }
\@@_msg_new:nn { unknown-option }
  { Class~ option~ "\l_keys_key_tl"~ is~ unknown. }
%    \end{macrocode}
%
% 将文档类选项传给 |fdu/option|。
%    \begin{macrocode}
\ProcessKeysOptions { fdu / option }
%    \end{macrocode}
%
% 载入参数配置文件。
%    \begin{macrocode}
\file_input:n { fduthesis.def }
\tl_if_empty:NF \g_@@_config_tl
  {
    \file_input:V \g_@@_config_tl
    \@@_info:nx { load-config-file } { \g_@@_config_tl }
  }
\@@_msg_new:nn { load-config-file }
  { You~ are~ loading~ config~ file~ '#1'. }
%    \end{macrocode}
%
% \subsection{载入宏包、文档类}
%
% \changes{v0.7}{2018/01/19}{使用 \cls{ctexbook} 文档类,而非直接使用
%   标准文档类 \cls{book}。}
%
% 将选项传入 \cls{ctexbook} 文档类。
%    \begin{macrocode}
% TODO: (2017-07-22) Chinese fonts in fduthesis-en
\PassOptionsToClass
  {
    UTF8,
%<class-en>    scheme     = plain,
    heading    = true,
%<class>    fontset    = none,
%<class-en>    fontset    = fandol,
    zihao      = -4,
%<class>    linespread = \c_@@_line_spread_fp,
    \g_@@_to_ctexbook_clist
  }
  { ctexbook }
%    \end{macrocode}
%
% 传入各宏包选项。
%    \begin{macrocode}
\clist_map_inline:nn
  {
    { no-math           } { fontspec },
    { perpage           } { footmisc },
    { amsmath, thmmarks } { ntheorem }
  }
  { \PassOptionsToPackage #1 }
%    \end{macrocode}
%
% 本模板会在 \pkg{ctexhook} 提供的钩子 \cs{ctex_at_end_preamble:n}
% 中调用 \pkg{biblatex},而 \pkg{biblatex} 自身又会使用 \pkg{etoolbox}
% 的钩子 \tn{AtEndPreamble},因此需要在载入 \cls{ctexbook} 之前调用
% \pkg{etoolbox}。钩子的顺序为:
% \[ \text{\tn{CTEX@document@left@hook}}
%    \prec \text{\tn{@endpreamblehook}}
%    \prec \text{\texttt{\textbackslash begin\{document\}}}. \]
%    \begin{macrocode}
\RequirePackage { etoolbox }
%    \end{macrocode}
%
% 载入 \cls{ctexbook} 文档类。
% 在使用 \XeLaTeX{} 编译时,\cls{ctexbook} 的底层将调用 \pkg{xeCJK}
% 宏包;而在使用 \LuaLaTeX{} 编译时,则将调用 \pkg{LuaTeX-ja} 宏包。
% 两种情况下 \cls{ctexbook} 均会调用 \pkg{fontspec} 宏包。
%    \begin{macrocode}
\LoadClass { ctexbook }
%    \end{macrocode}
%
% \changes{v0.4}{2017/08/13}{提供彩色支持。}
%
% 载入各宏包。其中,\pkg{amsmath} 必须在 \pkg{unicode-math} 之前引入。
%    \begin{macrocode}
\RequirePackage
  {
    amsmath,
    unicode-math,
    geometry,
    fancyhdr,
    footmisc,
    ntheorem,
    graphicx,
    longtable,
    caption,
    xcolor
  }
%    \end{macrocode}
%
% \begin{macro}{\@@_check_package:nnn}
% 检查过时宏包。
%    \begin{macrocode}
\cs_new_protected:Npn \@@_check_package:nnn #1#2#3
  {
    \@ifpackagelater {#1} {#2}
      { } { \@@_error:nnn { package-too-old } {#1} {#3} }
  }
\@@_msg_new:nn { package-too-old }
  {
    Package~ "#1"~ is~ too~ old. \\
    The~ fduthesis~ class~ only~ supports~ "#1"~ with~ a~ version \\
    higher~ than~ v#2. \\\\
    Please~ update~ an~ up-to-date~ version~ of~ it~ using~ your \\
    TeX~ package~ manager~ or~ from~ CTAN.
  }
%    \end{macrocode}
% \end{macro}
%
%    \begin{macrocode}
\@@_check_package:nnn { ctex         } { 2017/08/07 } { 2.4.10 }
\@@_check_package:nnn { fontspec     } { 2017/09/22 } { 2.6e   }
\@@_check_package:nnn { unicode-math } { 2017/11/18 } { 0.8i   }
\sys_if_engine_xetex:T
  { \@@_check_package:nnn { xeCJK          } { 2017/08/07 } { 3.5.0 } }
\sys_if_engine_luatex:T
  { \@@_check_package:nnn { xunicode-addon } { 2018/04/30 } { 3.7.1 } }
%    \end{macrocode}
%
