% \iffalse meta-comment
% !TeX program  = XeLaTeX
% !TeX encoding = UTF-8
%
% Copyright (C) 2017 by Xiangdong Zeng <pssysrq@163.com>
%
% This work may be distributed and/or modified under the
% conditions of the LaTeX Project Public License, either
% version 1.3c of this license or (at your option) any later
% version. The latest version of this license is in:
%
%   http://www.latex-project.org/lppl.txt
%
% and version 1.3 or later is part of all distributions of
% LaTeX version 2005/12/01 or later.
%
% This work has the LPPL maintenance status `maintained'.
%
% The Current Maintainer of this work is Xiangdong Zeng.
%
% This work consists of the file fduthesis.dtx
%          and the derived files fduthesis.ins,
%                                fduthesis.cls,
%                                fduthesis-en.cls,
%                                fduthesis.def,
%                                fduthesis-user.def,
%                                fduthesis.pdf
%                            and README.md.
%
%<*internal>
\iffalse
%</internal>
%
%<*readme>
# fduthesis

[![Build Status](https://travis-ci.org/Stone-Zeng/fduthesis.svg?branch=master)](https://travis-ci.org/Stone-Zeng/fduthesis)

## 欢迎使用 fduthesis - 复旦大学论文模板!

在您使用 `fduthesis` 之前,请务必仔细阅读模板文档 `fduthesis.pdf`。
该文档可通过如下命令生成:

    xelatex fduthesis.dtx
    makeindex -s gind.ist fduthesis
    xelatex fduthesis.dtx
    xelatex fduthesis.dtx

若需要生成模板各部件,请执行如下命令:

    xetex fduthesis.dtx

### 模板组成

- `source/`             源代码
  - `fduthesis.dtx`       模板代码、注释以及文档
  - `ctxdoc-m.cls`        模板文档样式(修改自 `ctxdoc.cls`)
  - `latexmkrc`           Latexmk 配置文件

- `test/`               测试文件
  - `fduthesis.cls`       fduthesis 模板类
  - `fduthesis-en.cls`    fduthesis 模板类(英文版)
  - `fduthesis.def`       模板定义文件
  - `fduthesis-user.def`  用户定义文件
  - `test.tex`            测试论文
  - `test-en.tex`         测试论文(英文版)

- `support/`            Travis CI 支持文件
  - `texlive.sh`          安装最新版本 TeX Live
  - `texlive.profile`     TeX Live 环境变量配置
  - `local.sh`            安装本地宏包及相关字体
  - `run.sh`              执行测试命令

- `logo/`               复旦大学标识
  - `fudan-name.tex`      校名
  - `fudan-emblem.tex`    校徽
  - `fudan-motto.tex`     校训

- `.gitignore`          Git 忽略文件

- `.travis.yml`         Travis CI 配置文件

- `README.md`           自述文件(本文档)

### 许可证

本模板的发布遵守 [LaTeX Project Public License](http://www.latex-project.org/lppl.txt)
(版本 1.3c 或更高)。

<br></br>

## Welcome to fduthesis - LaTeX thesis template for Fudan University!

Before you using `fduthesis`, please read the document `fduthesis.pdf`
carefully. This file can be generated with the following commands:

    xelatex fduthesis.dtx
    makeindex -s gind.ist fduthesis
    xelatex fduthesis.dtx
    xelatex fduthesis.dtx

If you want to get all components of `fduthesis`, please excute the
following commands:

    xetex fduthesis.dtx

### License

This work may be distributed and/or modified under the conditions of
the [LaTeX Project Public License](http://www.latex-project.org/lppl.txt),
either version 1.3c of this license or (at your option) any later
version.
%</readme>
%
%<*internal>
\fi
\begingroup
  \def\NameOfLaTeXe{LaTeX2e}
\expandafter\endgroup\ifx\NameOfLaTeXe\fmtname\else
\csname fi\endcsname
%</internal>
%
%<*install>
\input l3docstrip.tex
\keepsilent
\askforoverwritefalse

\preamble

    Copyright (C) 2017 by Xiangdong Zeng <pssysrq@163.com>

    This work may be distributed and/or modified under the
    conditions of the LaTeX Project Public License, either
    version 1.3c of this license or (at your option) any later
    version. The latest version of this license is in:

      http://www.latex-project.org/lppl.txt

    and version 1.3 or later is part of all distributions of
    LaTeX version 2005/12/01 or later.

    This work has the LPPL maintenance status `maintained'.

    The Current Maintainer of this work is Xiangdong Zeng.

    This work consists of the file fduthesis.dtx
             and the derived files fduthesis.ins,
                                   fduthesis.cls,
                                   fduthesis-en.cls,
                                   fduthesis.def,
                                   fduthesis-user.def,
                                   fduthesis.pdf
                               and README.md.

\endpreamble

\generate{
  \usedir{tex/latex/fduthesis}
  \file{\jobname.cls}       {\from{\jobname.dtx}{class}}
  \file{\jobname-en.cls}    {\from{\jobname.dtx}{class-en}}
  \file{\jobname.def}       {\from{\jobname.dtx}{definition}}
  \file{\jobname-user.def}  {\from{\jobname.dtx}{user}}
%</install>
%<*internal>
  \usedir{source/latex/fduthesis}
  \file{\jobname.ins}       {\from{\jobname.dtx}{install}}
%</internal>
%<*install>
  \nopreamble\nopostamble
  \usedir{doc/latex/fduthesis}
  \file{README.md}          {\from{\jobname.dtx}{readme}}
}

\obeyspaces
\Msg{*************************************************************}
\Msg{*                                                           *}
\Msg{* To finish the installation you have to move the following *}
\Msg{* files into a directory searched by TeX:                   *}
\Msg{*                                                           *}
\Msg{* The recommended directory is TDS:tex/latex/fduthesis      *}
\Msg{*                                                           *}
\Msg{*     fduthesis.cls                                         *}
\Msg{*     fduthesis-en.cls                                      *}
\Msg{*     fduthesis.def                                         *}
\Msg{*     fduthesis-user.def                                    *}
\Msg{*                                                           *}
\Msg{* To produce the documentation run the file fduthesis.dtx   *}
\Msg{* through XeLaTeX.                                          *}
\Msg{*                                                           *}
\Msg{* Happy TeXing!                                             *}
\Msg{*                                                           *}
\Msg{*************************************************************}

\endbatchfile
%</install>
%
%<*internal>
\fi
%</internal>
%
%<class|class-en>\NeedsTeXFormat{LaTeX2e}
%<class|class-en>\RequirePackage{expl3}
%<*!(driver|install)>
%<!readme>\GetIdInfo $Id: fduthesis.dtx 0.2 2017-02-19 12:00:00Z Xiangdong Zeng <pssysrq@163.com> $
%<class>  {Thesis template for Fudan University}
%<class-en>  {Thesis template for Fudan University (English version)}
%<definition>  {Definition file for fduthesis}
%<user>  {User definition file for fduthesis}
%<class|class-en>\ProvidesExplClass{\ExplFileName}
%<definition|user>\ProvidesExplFile{\ExplFileName}
%<!readme>  {\ExplFileDate}{\ExplFileVersion}{\ExplFileDescription}
%</!(driver|install)>
%<*driver>
%
%%%%%%%%%% 版本历史 %%%%%%%%%%
% v0.1 2017/02/15
%   开始
% v0.2 2017/02/19
%   git 管理,使用 doc 和 DocStrip
%

\documentclass{ctxdoc-m}
\begin{document}
  \DocInput{\jobname.dtx}
\end{document}
%</driver>
% \fi
%
% \CheckSum{0}
%
% \CharacterTable
%  {Upper-case    \A\B\C\D\E\F\G\H\I\J\K\L\M\N\O\P\Q\R\S\T\U\V\W\X\Y\Z
%   Lower-case    \a\b\c\d\e\f\g\h\i\j\k\l\m\n\o\p\q\r\s\t\u\v\w\x\y\z
%   Digits        \0\1\2\3\4\5\6\7\8\9
%   Exclamation   \!     Double quote  \"     Hash (number) \#
%   Dollar        \$     Percent       \%     Ampersand     \&
%   Acute accent  \'     Left paren    \(     Right paren   \)
%   Asterisk      \*     Plus          \+     Comma         \,
%   Minus         \-     Point         \.     Solidus       \/
%   Colon         \:     Semicolon     \;     Less than     \<
%   Equals        \=     Greater than  \>     Question mark \?
%   Commercial at \@     Left bracket  \[     Backslash     \\
%   Right bracket \]     Circumflex    \^     Underscore    \_
%   Grave accent  \`     Left brace    \{     Vertical bar  \|
%   Right brace   \}     Tilde         \~}
%
%^^A 禁止使用 " 符号作为抄录文本缩略符
% \DeleteShortVerb\"
%
%
% \title{\textcolor{MaterialIndigo800}{%
%   \textbf{fduthesis:复旦大学论文模板}}}
% \author{曾祥东}
% \date{\today \quad v0.2^^A
%   \thanks{\url{https://github.com/Stone-Zeng/fduthesis}.}^^A
% }
%
% \newgeometry{
%   left   = 3.18 cm,
%   right  = 3.18 cm,
%   top    = 3.18 cm,
%   bottom = 3.18 cm
% }
%
% \maketitle
% \vfill
% \begin{center}
%   \includegraphics[width=8cm]{../logo/fduthesis-cover.pdf}
% \end{center}
% \vfill
% \thispagestyle{plain}
% \clearpage
%
% \tableofcontents
%
% \restoregeometry
%
% \section{介绍}
% 目前,在网上可以找到的复旦大学 \LaTeX{} 论文模板有以下这些:
% \begin{itemize}
%   \item 数学科学学院 2001 级的何力同学和李湛同学在 2005 年
% 根据学校要求所设计的\cls{毕业论文格式 tex04 版},以及 2008 年
% 张越同学修改之后的\cls{毕业论文格式 tex08 版},这是专为数院本科生
% 撰写毕业论文而设计的;
%   \item Pandoxie 编写的 \cls{FDU-Thesis-Latex},基本满足了博士
% (硕士)毕业论文格式要求,使用人数较多;
%   \item richarddzh 编写的 \cls{fudan-thesis};
%   \item hmshan 编写的 \cls{FDU_PhD_Thesis_Template}。
% \end{itemize}
% 以上这些模板大都没有经过系统的设计,也鲜有后续维护。相比之下,
% 清华大学、中国科学技术大学以及友校上海交通大学等,都有成熟、稳定的
% 解决方案,值得参考。
%
% 本模板将借鉴前辈经验,重新设计,并使用 \LaTeX3 编写,以适应
% \TeX{} 技术发展潮流;同时还将构建一套简洁的接口,方便大家使用。
%
% \section{使用说明}
% \subsection{编译方式}
% 本模板不支持 \pdfTeX{} 引擎,请使用 \XeLaTeX{} 或 \LuaLaTeX{} 编译。
% 推荐使用 \XeLaTeX{}。
%
% \subsection{模板选项}
% 所谓“模板选项”,指需要在引入文档类的时候指定的选项。
%
% \begin{function}{oneside,twoside}
%   指明论文的单双面模式,默认为 |twoside|。
%   该选项会影响章(chapter)出现的位置,还会影响页眉样式。
% \end{function}
%
% 在双面模式(|twoside|)下,按照通常的排版惯例,\tn{chapter}
% 只出现在奇数页(在右侧);而在单页模式(|oneside|)下,
% 则可以出现在任意页面。本模板中,目录、摘要、符号表等均视作章,
% 也按相同方式排版。
%
% 双面模式下,正文部分偶数页(在左)的左页眉显示章标题,奇数页
% (在右)的右页眉显示节标题;前置部分的页眉按同样格式显示,
% 但文字均为对应标题(如“\textit{目 \quad 录}”、
% “\textit{摘 \quad 要}”等)。
% 而在单面模式下,正文部分页面不分奇偶,均会同时显示左、右页眉,
% 文字分别为章标题和节标题;前置部分只有中页眉,显示对应标题。
%
% \begin{function}{draft}
%   \begin{syntax}
%     draft = <\TFF>
%   \end{syntax}
%   选择是否开启草稿模式,默认关闭。注意,单独一个 |draft| 的写法,
%   与 |draft = true| 是一致的。
% \end{function}
%
% 草稿模式为全局选项,会影响到很多宏包的工作方式。
% 开启之后,主要的变化有:
% \begin{itemize}
%   \item 把行溢出的盒子显示为黑色方块;
%   \item 不实际插入图片,只输出一个占位方框;
%   \item 关闭超链接渲染,也不添加 PDF 书签(还未调用
%     \pkg{hyperref},目前暂时不受影响);
%   \item 显示页面边框。
% \end{itemize}
%
% \subsection{参数设置}
% 本模板提供了一系列选项,可由用户自行配置。以下所有选项均可通过
% 统一的命令 \cs{fdusetup} 来设置。
%
% \begin{function}{\fdusetup}
%   \begin{syntax}
%     \cs{fdusetup}\Arg{键值列表}
%   \end{syntax}
%   这是本模板的通用控制命令,用来在载入文档类之后实现各种功能,
%   如修改论文格式、录入论文信息等。
% \end{function}
%
% \cs{fdusetup} 的参数是一组由(英文)逗号隔开的选项列表,列表中的
% 选项通常是 \meta{key} |=| \meta{value} 的形式。部分选项的
% \meta{value} 可以省略。对于同一项,后面的设置将会覆盖前面的设置。
% 在下文的说明中,我们将用\textbf{粗体}表示默认值。
%
% \cs{fdusetup} 采用 \LaTeX3 风格的键值设置,支持不同类型以及多种
% 层次的选项设定。键值列表中,“|=|”左右的空格不影响设置;但需注意,
% 参数列表中不可以出现空行。
%
% 有些选项为布尔型,即只能取 |true| 和 |false| 两个值。对于这些选项,
% \meta{key} |= true| 中的“|= true|”可以省略。
%
% 另有一些选项包含子选项,如 |style| 和 |info| 等。它们可以按如下
% 两种等价方式来设定:
% \begin{latexexample}
%   \fdusetup{
%     style = {font = adobe, fontsize = -4},
%     info  = {
%       author     = {张三},
%       department = {物理系},
%       title      = {论如何使用 fduthesis 写好论文}
%     }
%   }
% \end{latexexample}
% 或者
% \begin{latexexample}
%   \fdusetup{
%     style / font       = adobe,
%     style / fontsize   = -4,
%     info  / author     = {张三},
%     info  / department = {物理系},
%     info  / title      = {论如何使用 fduthesis 写好论文}
%   }
% \end{latexexample}
%
% 在第二种方式中,“|/|” 与 “|=|” 类似,前后的空白对设置没有影响。
%
% \subsubsection{论文格式}
% \begin{function}{style}
%   \begin{syntax}
%     style = \Arg{键值列表}
%     style / \meta{key} = \meta{value}
%   \end{syntax}
%   设置论文格式的通用选项,具体内容见下。
% \end{function}
%
% \begin{function}{style/font}
%   \begin{syntax}
%     font = <libertinus|lm|palatino|(times)>
%   \end{syntax}
%   设置西文字体以及相应的数学字体。
% \end{function}
%
% \begin{function}{style/cjkfont}
%   \begin{syntax}
%     cjkfont = <adobe|(fandol)|founder|linux|mac|windows>
%   \end{syntax}
%   设置中文字体。
% \end{function}
%
% \begin{function}{style/fontsize}
%   \begin{syntax}
%     fontsize = <(-4)|5>
%   \end{syntax}
%   设置论文的基础字号。
% \end{function}
%
% \begin{function}{style/fullwidthstop}
%   \begin{syntax}
%     fullwidthstop = <\TFF>
%   \end{syntax}
%   选择是否把全角实心句点“\symbol{"FF0E}”作为默认的句号形状。
%   这种句号一般用于科技类文章,以便与下标 o 或 0 区分。
% \end{function}
%
% \begin{function}{style/footnotestyle}
%   \begin{syntax}
%     footnotestyle = <plain|libertinus|libertinus*|libertinus-sans|
% \hspace{2.79cm}pifont|pifont*|pifont-sans|pifont-sans*|
% \hspace{2.79cm}xits|xits-sans|xits-sans*>
%   \end{syntax}
%   设置脚注样式。西文字体设置会影响其默认取值。因此,要使得该选项
%   生效,需将其放置在 |font| 选项之后。
% \end{function}
%
% \begin{table}[h]
% \caption{西文字体与脚注样式默认值的对应关系}
% \centering
% \begin{tabular}{ccccc}
%   \toprule
%   西文字体设置   & |libertinus| & |lm|     & |palatino| & |times| \\
%   \midrule
%   脚注样式默认值 & |libertinus| & |pifont| & |pifont|   & |xits|  \\
%   \bottomrule
% \end{tabular}
% \end{table}
%
% \begin{function}{style/automakecover}
%   \begin{syntax}
%     automakecover = <\TTF>
%   \end{syntax}
%   是否自动生成论文封面(封一)、指导小组成员名单(封二)和声明页
%   (封三)。封面中的各项信息,可通过 \cs{fdusetup} 录入,具体请参阅
%   \ref{subsubsec:信息录入}~节。
% \end{function}
%
% \begin{function}{\makecoveri,\makecoverii,\makecoveriii}
%   用于\emph{手动}生成论文封面、指导小组成员名单和声明页。这几个
%   命令不能确保页码的正确编排,因此除非必要,您应当始终使用自动生成
%   的封面。
% \end{function}
%
% \subsubsection{信息录入} \label{subsubsec:信息录入}
% \begin{function}{info}
%   \begin{syntax}
%     info = \Arg{键值列表}
%     info / \meta{key} = \meta{value}
%   \end{syntax}
%   录入论文信息的通用选项,具体内容见下。以下各选项中,带“|*|”的为
%   对应的英文字段。
% \end{function}
%
% \begin{function}{info/title,info/title*}
%   \begin{syntax}
%     title  = \Arg{中文标题}
%     title* = \Arg{English title}
%   \end{syntax}
%   论文标题。默认会在约 20 个汉字字宽处强制断行,但为了语义的连贯
%   以及排版的美观,如果您的标题长于一行,建议使用“|\\|”手动断行。
% \end{function}
%
% \begin{function}{info/author,info/author*}
%   \begin{syntax}
%     author  = \Arg{姓名}
%     author* = \Arg{English name}
%   \end{syntax}
%   作者姓名。
% \end{function}
%
% \begin{function}{info/supervisor}
%   \begin{syntax}
%     supervisor = \Arg{姓名}
%   \end{syntax}
%   导师姓名。
% \end{function}
%
% \begin{function}{info/department}
%   \begin{syntax}
%     department = \Arg{名称}
%   \end{syntax}
%   院系名称。
% \end{function}
%
% \begin{function}{info/major}
%   \begin{syntax}
%     major = \Arg{名称}
%   \end{syntax}
%   专业名称。
% \end{function}
%
% \begin{function}{info/studentid}
%   \begin{syntax}
%     studentid = \Arg{数字}
%   \end{syntax}
%   作者学号。
% \end{function}
%
% 复旦大学学号共 11 位,前两位为入学年份,之后一位为学生类型代码
% (本科生为 1,硕士生为 2,博士生为 3),接下来的五位为专业代码,
% 最后三位为顺序号。
%
% \begin{function}{info/schoolid}
%   \begin{syntax}
%     schoolid = \Arg{数字}
%   \end{syntax}
%   学校代码。默认值为 10246(这是复旦大学的学校代码)。
% \end{function}
%
% \begin{function}{info/date}
%   \begin{syntax}
%     date = \Arg{日期}
%   \end{syntax}
%   论文完成日期。默认值为文档编译日期(\tn{today})。
% \end{function}
%
% \begin{function}{info/secretlevel}
%   \begin{syntax}
%     secretlevel = <(none)|i|ii|iii>
%   \end{syntax}
%   密级。|i|、|ii|、|iii| 分别表示秘密、机密、绝密;|none| 表示论文不
%   涉密,即不显示密级与保密年限。
% \end{function}
%
% \begin{function}{info/secretyear}
%   \begin{syntax}
%     secretlevel = \Arg{年限}
%   \end{syntax}
%   保密年限。建议您使用中文,如“五年”。此选项在 |secretlevel = none|
%   时无效。
% \end{function}
%
% \begin{function}{info/instructors}
%   \begin{syntax}
%     instructors = \Arg{成员 1, 成员 2, ...}
%   \end{syntax}
%   指导小组成员。各成员之间需使用英文逗号隔开。为防止歧义,可以用
%   分组括号“|{...}|”把各成员字段括起来。
% \end{function}
%
% \begin{function}{info/keywords,info/keywords*}
%   \begin{syntax}
%     keywords  = \Arg{中文关键字}
%     keywords* = \Arg{English keywords}
%   \end{syntax}
%   关键字。
% \end{function}
%
% \begin{function}{info/clc}
%   \begin{syntax}
%     clc = \Arg{分类号}
%   \end{syntax}
%   中国图书馆分类号(CLC)。
% \end{function}
%
% \subsection{正文编写}
% \subsubsection{凤头}
% \begin{function}{\frontmatter}
%   声明前置部分开始。
% \end{function}
%
% 在本模板中,前置部分包含目录、中英文摘要以及符号表等。前置部分的
% 页码采用小写罗马字母,并且与正文分开计数。
%
% \begin{function}{\tableofcontents}
%   生成目录。为了生成完整、正确的目录,您至少需要编译\emph{两次}。
% \end{function}
%
% \begin{function}{abstract,abstract*}
%   \begin{syntax}
%     \tn{begin}\{abstract\}
%       \quad{}中文摘要。
%     \tn{end}\{abstract\}
%
%     \tn{begin}\{abstract*\}
%       \quad{}English abstract.
%     \tn{end}\{abstract*\}
%   \end{syntax}
%   中英文摘要。在摘要的最后,会显示关键字列表以及
%   中国图书馆分类号(CLC)。这两项可通过 \cs{fdusetup} 录入,具体
%   请参阅 \ref{subsubsec:信息录入}~节。
% \end{function}
%
% \begin{function}{notation}
%   \begin{syntax}
%     \tn{begin}\{notation\}\oarg{列格式说明}
%       \quad{}符号 1  \&  说明  \backslash\backslash
%       \quad{}符号 2  \&  说明  \backslash\backslash
%       \quad{}符号 3  \&  说明  \backslash\backslash
%       \quad{}...
%     \tn{end}\{notation\}
%   \end{syntax}
%   符号表。可选参数“列格式说明”与 \LaTeX 中标准表格的列格式相同,
%   默认值为“|l p{7.5 cm}|”,即第一列宽度自动调整,第二列限宽
%   \SI{7.5}{cm},均采用左对齐。
% \end{function}
%
% \subsubsection{猪肚}
%
% \subsubsection{豹尾}
%
% \section{宏包依赖情况}
%
% \StopEventually{
%   \begin{thebibliography}{9}
%     \bibitem{knuthtex1986}
%     \textsc{Donald~Ervin Knuth}.
%     \newblock \textit{The {{\TeX{}book}}}, \textit{Computers \& Typesetting},
%       volume~A.
%     \newblock Addison-Wesley, 1986
%^^A 
%     \bibitem{mittelbach2004}
%     \textsc{Frank Mittelbach} and \textsc{Michel Goossens}.
%     \newblock \textit{The {{\LaTeX}} Companion}.
%     \newblock Tools and Techniques for Computer Typesetting. Boston:
%       Addison-Wesley, second edition, 2004
%   \end{thebibliography}
%   \newgeometry{
%     left   = 2.54 cm,
%     right  = 2.54 cm,
%     top    = 3.18 cm,
%     bottom = 3.18 cm
%   }
%   \PrintIndex
% }
%
% \section{实现细节}
% 本模板使用 \LaTeX3 语法编写,依赖 \pkg{expl3} 环境,
% 并需调用 \pkg{l3packages} 中的相关宏包。
%
% 按照 \LaTeX3 语法,代码中的空格、换行符与制表符完全忽略,
% 而下划线“|_|”和冒号“|:|”则可作为一般字母使用。
% 正常的空格可以使用“|~|”代替;至于 |~| 原来所表示的“带子”,
% 则要用 \LaTeXe{} 的原始命令 \tn{nobreakspace} 代替。
%    \begin{macrocode}
%<*class|class-en>
\RequirePackage { xparse, l3keys2e }
%    \end{macrocode}
%
% 载入参数配置文件。
%    \begin{macrocode}
\file_input:n { fduthesis.def }
\file_input:n { fduthesis-user.def }
%    \end{macrocode}
%
% \subsection{内部变量声明}
% \begin{macro}{\l__fdu_tmpa_box,
%   \l__fdu_tmpa_dim,\l__fdu_tmpb_dim,
%   \l__fdu_tmpa_tl,\l__fdu_tmpb_tl,
%   \l__fdu_tmpa_int,
%   \l__fdu_tmpa_clist,\l__fdu_tmpb_clist}
% 临时变量。
%    \begin{macrocode}
\box_new:N   \l__fdu_tmpa_box
\dim_new:N   \l__fdu_tmpa_dim
\dim_new:N   \l__fdu_tmpb_dim
\tl_new:N    \l__fdu_tmpa_tl
\tl_new:N    \l__fdu_tmpb_tl
\int_new:N   \l__fdu_tmpa_int
\clist_new:N \l__fdu_tmpa_clist
\clist_new:N \l__fdu_tmpb_clist
%    \end{macrocode}
% \end{macro}
%
% \begin{macro}{\g__fdu_to_book_clist}
% 保存由 \cls{fduthesis} 传入 \cls{book} 文档类的选项列表。
%    \begin{macrocode}
\clist_new:N \g__fdu_to_book_clist
%    \end{macrocode}
% \end{macro}
%
% \begin{macro}{\g__fdu_twoside_bool}
% 是否开启双页模式(默认打开)。
%    \begin{macrocode}
\bool_new:N \g__fdu_twoside_bool
\bool_set_true:N \g__fdu_twoside_bool
%    \end{macrocode}
% \end{macro}
%
% \begin{macro}{\c__fdu_fn_style_plain_tl,
%   \c__fdu_fn_style_libertinus_tl,
%   \c__fdu_fn_style_libertinus_n_tl,
%   \c__fdu_fn_style_libertinus_sans_tl,
%   \c__fdu_fn_style_pifont_tl,
%   \c__fdu_fn_style_pifont_n_tl,
%   \c__fdu_fn_style_pifont_sans_tl,
%   \c__fdu_fn_style_pifont_sans_n_tl,
%   \c__fdu_fn_style_xits_tl,
%   \c__fdu_fn_style_xits_sans_tl,
%   \c__fdu_fn_style_xits_sans_n_tl}
% 各种脚注编号样式的名称。
%    \begin{macrocode}
\tl_const:Nn \c__fdu_fn_style_plain_tl           { plain           }
\tl_const:Nn \c__fdu_fn_style_libertinus_tl      { libertinus      }
\tl_const:Nn \c__fdu_fn_style_libertinus_n_tl    { libertinus*     }
\tl_const:Nn \c__fdu_fn_style_libertinus_sans_tl { libertinus-sans }
\tl_const:Nn \c__fdu_fn_style_pifont_tl          { pifont          }
\tl_const:Nn \c__fdu_fn_style_pifont_n_tl        { pifont*         }
\tl_const:Nn \c__fdu_fn_style_pifont_sans_tl     { pifont-sans     }
\tl_const:Nn \c__fdu_fn_style_pifont_sans_n_tl   { pifont-sans*    }
\tl_const:Nn \c__fdu_fn_style_xits_tl            { xits            }
\tl_const:Nn \c__fdu_fn_style_xits_sans_tl       { xits-sans       }
\tl_const:Nn \c__fdu_fn_style_xits_sans_n_tl     { xits-sans*      }
%    \end{macrocode}
% \end{macro}
%
% \begin{macro}{\c__fdu_thm_style_plain_clist,
%   \c__fdu_thm_style_break_clist}
% 保存 |plain|、|break| 两种类型的定理样式名称。
%    \begin{macrocode}
\clist_const:Nn \c__fdu_thm_style_plain_clist
  { plain, margin, change }
\clist_const:Nn \c__fdu_thm_style_break_clist
  { break, marginbreak, changebreak }
%    \end{macrocode}
% \end{macro}
%
% \subsection{选项处理}
% 定义 |fdu/option| 键值类。
%    \begin{macrocode}
\keys_define:nn { fdu / option }
  {
%    \end{macrocode}
%
% \begin{macro}{oneside,twoside}
% 设置页面类型为单面或双面。
%    \begin{macrocode}
    oneside .value_forbidden:n = true,
    twoside .value_forbidden:n = true,
    oneside .code:n = {
      \clist_gput_right:Nn \g__fdu_to_book_clist { oneside }
      \bool_set_false:N    \g__fdu_twoside_bool
    },
    twoside .code:n = {
      \clist_gput_right:Nn \g__fdu_to_book_clist { twoside }
      \bool_set_true:N     \g__fdu_twoside_bool
    },
%    \end{macrocode}
% \end{macro}
%
% \begin{macro}{draft,\g__fdu_draft_bool}
% 是否开启草稿模式(默认关闭)。
%    \begin{macrocode}
    draft .choice:,
    draft / true  .code:n = {
      \bool_set_true:N     \g__fdu_draft_bool
      \clist_gput_right:Nn \g__fdu_to_book_clist { draft }
    },
    draft / false .code:n = {
      \bool_set_false:N    \g__fdu_draft_bool
    },
    draft .default:n = true,
    draft .initial:n = false
  }
%    \end{macrocode}
% \end{macro}
%
% 将文档类选项传给 |fdu/option|。
%    \begin{macrocode}
\ProcessKeysOptions { fdu / option }
%    \end{macrocode}
%
% \subsection{载入宏包、文档类}
% 载入 \cls{book} 标准文档类,并传入相应的选项。
%    \begin{macrocode}
\PassOptionsToClass { \g__fdu_to_book_clist } { book }
\LoadClass { book }
%    \end{macrocode}
%
% \XeLaTeX{} \LuaLaTeX{} 下的字体选取。|no-math| 选项保证该宏包不参与
% 数学字体的设置。
%    \begin{macrocode}
\RequirePackage [ no-math ] { fontspec }
%    \end{macrocode}
%
% 中文排版支持。使用 \XeLaTeX{} 编译时,底层将调用 \pkg{xeCJK} 宏包;
% 使用 \LuaLaTeX{} 编译时,底层则将调用 \pkg{luatexja} 宏包。
%    \begin{macrocode}
\RequirePackage [
    UTF8,
%<class-en>    scheme     = plain,
    heading    = true,
%<class>    fontset    = none,
%<class-en>    fontset    = fandol,
    zihao      = \c__fdu_def_font_size_tl,
%<class>    linespread = \c__fdu_def_line_spread_fp
  ] { ctex }
%    \end{macrocode}
%
% 本模板使用 Unicode 编码的 OpenType 数学字体,此功能由
% \pkg{unicode-math} 宏包实现。为防止冲突,\pkg{amsmath} 必须在它
% 之前引入。
%    \begin{macrocode}
\RequirePackage { amsmath }
\RequirePackage { unicode-math }
%    \end{macrocode}
%
% 设置页面尺寸与页眉页脚。
%    \begin{macrocode}
\RequirePackage { geometry, fancyhdr }
%    \end{macrocode}
%
% 处理脚注。|perpage| 选项将使脚注编号每页清零。
%    \begin{macrocode}
\RequirePackage [ perpage ] { footmisc }
%    \end{macrocode}
%
% 定理环境。
%    \begin{macrocode}
\RequirePackage [ amsmath, thmmarks ] { ntheorem }
%    \end{macrocode}
%
% 插图、表格与浮动体控制。
%    \begin{macrocode}
\RequirePackage { graphicx }
\RequirePackage { longtable }
\RequirePackage { caption }
%    \end{macrocode}
%
% 下划线。
%    \begin{macrocode}
\RequirePackage { ulem }
%    \end{macrocode}
%
% \subsection{页面布局}
% 利用 \pkg{geometry} 宏包设置纸张大小、页面边距以及页眉高度。
%    \begin{macrocode}
\geometry
  {
    paper      = \c__fdu_def_paper_size_tl,
    top        = \c__fdu_def_page_margin_top_dim,
    bottom     = \c__fdu_def_page_margin_bottom_dim,
    left       = \c__fdu_def_page_margin_left_dim,
    right      = \c__fdu_def_page_margin_right_dim,
    headheight = \c__fdu_def_header_height_dim
  }
%    \end{macrocode}
%
% 草稿模式下显示页面边框及页眉、页脚线 。
%    \begin{macrocode}
\bool_if:NT \g__fdu_draft_bool
  { \geometry { showframe } }
%    \end{macrocode}
%
% \subsection{字体}
% 根据相关规定,数学表达式中,表示变量的拉丁字母和希腊字母均应当
% 使用斜体。
%    \begin{macrocode}
\unimathsetup { math-style = ISO, bold-style = ISO }
%    \end{macrocode}
%
% \subsubsection{西文字体、数学字体配置}
% 定义 |fdu/style| 键值类。
%    \begin{macrocode}
\keys_define:nn { fdu / style }
  {
%    \end{macrocode}
%
% \begin{macro}{style/font}
% 预定义西文字体。等宽字体使用 |Ligatures = NoCommon| 选项以禁用连字。
%    \begin{macrocode}
    font .choice:,
    font .value_required:n = true,
%    \end{macrocode}
% Libertinus 系列。
%    \begin{macrocode}
    font / libertinus .code:n = {
      \setmainfont { Libertinus~ Serif }
      \setsansfont { Libertinus~ Sans  }
      \setmonofont { TeX~ Gyre~ Cursor }
        [ Ligatures = NoCommon ]
      \setmathfont { Libertinus~ Math  }
      \keys_set:nn { fdu / style } { footnotestyle = libertinus }
    },
%    \end{macrocode}
% Latin Modern 系列。
%    \begin{macrocode}
    font / lm .code:n = {
      \setmainfont { Latin~ Modern~ Roman }
      \setsansfont { Latin~ Modern~ Sans  }
      \setmonofont { Latin~ Modern~ Mono  }
      \setmathfont { Latin~ Modern~ Math  }
      \keys_set:nn { fdu / style } { footnotestyle = pifont }
    },
%    \end{macrocode}
% Palatino 系列。
%    \begin{macrocode}
    font / palatino .code:n = {
      \setmainfont { TeX~ Gyre~ Pagella       }
      \setsansfont { TeX~ Gyre~ Heros         }
      \setmonofont { TeX~ Gyre~ Cursor        }
        [ Ligatures = NoCommon ]
      \setmathfont { TeX~ Gyre~ Pagella~ Math }
      \keys_set:nn { fdu / style } { footnotestyle = pifont }
    },
%    \end{macrocode}
% Times Roman 系列。
%    \begin{macrocode}
    font / times .code:n = {
      \setmainfont { XITS              }
      \setsansfont { TeX~ Gyre~ Heros  }
      \setmonofont { TeX~ Gyre~ Cursor }
        [ Ligatures = NoCommon ]
      \setmathfont { XITS~ Math        }
      \keys_set:nn { fdu / style } { footnotestyle = xits }
    },
%    \end{macrocode}
% \end{macro}
%
% \subsubsection{中文字体配置}
% \begin{macro}{style/cjkfont}
% 预定义中文(CJK)字体。对于没有粗体的字体,利用
% |AutoFakeBold = true| 获得伪粗体。
%    \begin{macrocode}
%<*class>
    cjkfont .choice:,
    cjkfont .value_required:n = true,
%    \end{macrocode}
% Adobe 系列。
%    \begin{macrocode}
    cjkfont / adobe .code:n = {
      \setCJKmainfont { Adobe~ Song~ Std~ L     }
        [
          ItalicFont   = Adobe~ Kaiti~ Std~ R,
          AutoFakeBold = true
        ]
      \setCJKsansfont { Adobe~ Heiti~ Std~ R    }
        [
          ItalicFont   = Adobe~ Heiti~ Std~ R,
          AutoFakeBold = true
        ]
      \setCJKmonofont { Adobe~ Fangsong~ Std~ R }
        [
          ItalicFont   = Adobe~ Fangsong~ Std~ R,
          AutoFakeBold = true
        ]
      \setCJKfamilyfont { song } { Adobe~ Song~     Std~ L }
      \setCJKfamilyfont { hei  } { Adobe~ Heiti~    Std~ R }
      \setCJKfamilyfont { fang } { Adobe~ Fangsong~ Std~ R }
      \setCJKfamilyfont { kai  } { Adobe~ Kaiti~    Std~ R }
    },
%    \end{macrocode}
% Fandol 系列。
%    \begin{macrocode}
    cjkfont / fandol .code:n = {
      \setCJKmainfont { FandolSong }
        [
          ItalicFont   = FandolKai
        ]
      \setCJKsansfont { FandolHei  }
        [
          ItalicFont   = FandolHei
        ]
      \setCJKmonofont { FandolFang }
        [
          ItalicFont   = FandolFang,
          AutoFakeBold = true
        ]
      \setCJKfamilyfont { song } { FandolSong }
      \setCJKfamilyfont { hei  } { FandolHei  }
      \setCJKfamilyfont { fang } { FandolFang }
      \setCJKfamilyfont { kai  } { FandolKai  }
    },
%    \end{macrocode}
% 方正系列。
%    \begin{macrocode}
    cjkfont / founder .code:n = {
      \setCJKmainfont { FZShuSong-Z01  }
        [
          BoldFont     = FZXiaoBiaoSong-B05,
          ItalicFont   = FZKai-Z03
        ]
      \setCJKsansfont { FZHei-B01      }
        [
          ItalicFont   = FZHei-B01,
          AutoFakeBold = true
        ]
      \setCJKmonofont { FZFangSong-Z02 }
        [
          ItalicFont   = FZFangSong-Z02,
          AutoFakeBold = true
        ]
      \setCJKfamilyfont { song } { FZShuSong -Z01 }
      \setCJKfamilyfont { hei  } { FZHei     -B01 }
      \setCJKfamilyfont { fang } { FZFangSong-Z02 }
      \setCJKfamilyfont { kai  } { FZKai     -Z03 }
    },
%    \end{macrocode}
% Linux 系列。(没做)
%    \begin{macrocode}
%    cjkfont / linux .code:n = {
%      \setCJKmainfont{ SimSun }
%        [ BoldFont = SimHei, ItalicFont   = KaiTi ]
%    },
%    \end{macrocode}
% Mac (华文)系列。
%    \begin{macrocode}
    cjkfont / mac .code:n = {
      \setCJKmainfont { STSong     }
        [
          BoldFont     = STZhongsong,
          ItalicFont   = STKaiti
        ]
      \setCJKsansfont { STHeiti    }
        [
          ItalicFont   = STHeiti,
          AutoFakeBold = true
        ]
      \setCJKmonofont { STFangsong }
        [
          ItalicFont   = STFangsong,
          AutoFakeBold = true
        ]
      \setCJKfamilyfont { song } { STSong     }
      \setCJKfamilyfont { hei  } { STHeiti    }
      \setCJKfamilyfont { fang } { STFangsong }
      \setCJKfamilyfont { kai  } { STKaiti    }
    },
%    \end{macrocode}
% Windows (中易)系列。
%    \begin{macrocode}
    cjkfont / windows .code:n = {
      \setCJKmainfont { SimSun   }
        [
          ItalicFont   = KaiTi,
          AutoFakeBold = true
        ]
      \setCJKsansfont { SimHei   }
        [
          ItalicFont   = SimHei,
          AutoFakeBold = true
        ]
      \setCJKmonofont { FangSong }
        [
          ItalicFont   = FangSong,
          AutoFakeBold = true
        ]
      \setCJKfamilyfont { song } { SimSun   }
      \setCJKfamilyfont { hei  } { SimHei   }
      \setCJKfamilyfont { fang } { FangSong }
      \setCJKfamilyfont { kai  } { KaiTi    }
    },
%</class>
%    \end{macrocode}
% \end{macro}
%
% \subsubsection{字号}
% \begin{macro}{style/fontsize}
% |fontsize| 不是文档类选项,不能传给 \pkg{ctex} 宏包
% 或者 \cls{book} 文档类,因此只能手动重定义字号命令。
%    \begin{macrocode}
    fontsize .choice:,
    fontsize .value_required:n = true,
    fontsize / -4 .code:n = { },
%    \end{macrocode}
% \end{macro}
%
% \begin{macro}{\tiny,\scriptsize,\footnotesize,\small,
%   \normalsize,\large,\Large,\LARGE,\huge,\Huge}
% 默认使用小四号字,所以只有五号字需要重新设置。
%    \begin{macrocode}
    fontsize /  5 .code:n = {
      \RenewDocumentCommand \tiny         { } { \zihao {  7 } }
      \RenewDocumentCommand \scriptsize   { } { \zihao { -6 } }
      \RenewDocumentCommand \footnotesize { } { \zihao {  6 } }
      \RenewDocumentCommand \small        { } { \zihao { -5 } }
      \RenewDocumentCommand \normalsize   { } { \zihao {  5 } }
      \RenewDocumentCommand \large        { } { \zihao { -4 } }
      \RenewDocumentCommand \Large        { } { \zihao { -3 } }
      \RenewDocumentCommand \LARGE        { } { \zihao { -2 } }
      \RenewDocumentCommand \huge         { } { \zihao {  2 } }
      \RenewDocumentCommand \Huge         { } { \zihao {  1 } }
%<class>    },
%<class-en>    }
%    \end{macrocode}
% \end{macro}
%
% \subsubsection{句号}
% \begin{macro}{style/fullwidthstop}
% 设置句号形状(圆圈或是圆点)。
% 本模板采用的实现方法是将“\symbol{"3002}”设置为活动符,
% 并定义为句点“\symbol{"FF0E}”。
%
% \pkg{xeCJK} 宏包提供了 |Mapping = fullwidth-stop| 和 |full-stop|
% 选项,也能实现两种句号的切换。此种方法是基于字体映射实现的,
% 当需要同时使用两种句号的时候(比如本文档)将会带来不便。
% 另外通过 \LuaLaTeX{} 编译时,底层使用 \pkg{luatexja} 而非
% \pkg{xeCJK},也必须采取 \tn{catcode} 的手段来切换。
%    \begin{macrocode}
%<*class>
    fullwidthstop .choice:,
    fullwidthstop / true  .code:n = {
      \char_set_active_eq:nN { "3002 }
        \c__fdu_full_stop_fullwidth_tl
      \char_set_catcode_active:n { "3002 }
    },
    fullwidthstop / false .code:n = { },
    fullwidthstop .default:n  = true
%</class>
  }
%    \end{macrocode}
% \end{macro}
%
% \subsection{章节标题结构}
% |\keys_set:nn {ctex}| 实际相当于 \cs{ctexset}。
%    \begin{macrocode}
\keys_set:nn { ctex }
  {
%    \end{macrocode}
%
% 设置章(chapter)、节(section)与小节(sub-section)标题样式。
% 此处使用 |fixskip = true| 选项来抑制前后的多余间距。
%    \begin{macrocode}
    chapter = {
%<class>      format      = \c__fdu_def_chapter_format_tl,
%<*class-en>
      format      = \c__fdu_def_chapter_format_en_tl,
      nameformat  = \c__fdu_def_chapter_name_format_en_tl,
      titleformat = \c__fdu_def_chapter_title_format_en_tl,
      aftername   = \c__fdu_def_chapter_after_name_en_tl,
%</class-en>
      beforeskip  = \c__fdu_def_chapter_before_sep_tl,
      afterskip   = \c__fdu_def_chapter_after_sep_tl,
      number      = { \arabic { chapter } },
      fixskip     = true
    },
    section = {
%<class>      format      = \c__fdu_def_section_format_tl,
%<class-en>      format      = \c__fdu_def_section_format_en_tl,
      beforeskip  = \c__fdu_def_section_before_sep_tl,
      afterskip   = \c__fdu_def_section_after_sep_tl,
      fixskip     = true
    },
    subsection = {
%<class>      format      = \c__fdu_def_subsection_format_tl,
%<class-en>      format      = \c__fdu_def_subsection_format_en_tl,
      beforeskip  = \c__fdu_def_subsection_before_sep_tl,
      afterskip   = \c__fdu_def_subsection_after_sep_tl,
      fixskip     = true
    }
  }
%    \end{macrocode}
%
% \subsection{页眉页脚}
% 清除默认页眉页脚格式。
%    \begin{macrocode}
\fancyhf { }
%    \end{macrocode}
%
% \begin{macro}{\l__fdu_header_center_mark_tl}
% 保存中间页眉的文字。正文中设置为空,目录、摘要、符号表等设置为
% 相应标题。
%    \begin{macrocode}
\tl_new:N \l__fdu_header_center_mark_tl
%    \end{macrocode}
% \end{macro}
%
% 构建页眉,要在单面或双面下分别设置。
% \cs{fancyhead} 的选项中,|E| 和 |O| 分别表示偶数(even)和奇数
% (odd), 而 |L|、|R| 和 |C| 则分别表示左(left)、右(right)
% 和中间(center)。按照通常的排版规则,在双面模式下,偶数页中页眉
% 文字在左,奇数页中则在右。单面模式下,左右页眉都要显示。
%    \begin{macrocode}
\bool_if:NTF \g__fdu_twoside_bool
%<*class>
  {
    \fancyhead [ EL ]
      { \small \nouppercase { \CJKfamily { kai } \leftmark  } }
    \fancyhead [ OR ]
      { \small \nouppercase { \CJKfamily { kai } \rightmark } }
  }
  {
    \fancyhead [ L ]
      { \small \nouppercase { \CJKfamily { kai } \leftmark  } }
    \fancyhead [ R ]
      { \small \nouppercase { \CJKfamily { kai } \rightmark } }
    \fancyhead [ C ]
      {
        \small \nouppercase
          { \CJKfamily { kai } \l__fdu_header_center_mark_tl }
      }
  }
%</class>
%<*class-en>
  {
    \fancyhead [ EL ] { \small \nouppercase { \itshape \leftmark  } }
    \fancyhead [ OR ] { \small \nouppercase { \itshape \rightmark } }
  }
  {
    \fancyhead [ L ] { \small \nouppercase { \itshape \leftmark  } }
    \fancyhead [ R ] { \small \nouppercase { \itshape \rightmark } }
    \fancyhead [ C ]
      {
        \small \nouppercase
          { \itshape \l__fdu_header_center_mark_tl }
      }
  }
%</class-en>
%    \end{macrocode}
%
% 构建页脚,用来显示页码。选项 |C| 表示居中(center)。
%    \begin{macrocode}
\fancyfoot [ C ] { \small \thepage }
%    \end{macrocode}
%
% 关闭横线显示(未启用)。
%    \begin{macrocode}
% \RenewDocumentCommand \headrulewidth { } { 0 pt }
%    \end{macrocode}
%
% \begin{macro}{\fdu_front_matter_header:n}
% 在单页模式下,设置前导部分(包括目录、摘要、符号表等)的页眉中间
% 为相应标题,左右为空。
%    \begin{macrocode}
\cs_new:Npn \fdu_front_matter_header:n #1
  {
    \bool_if:NTF \g__fdu_twoside_bool
      { \markboth { #1 } { #1 } }
      { 
        \markboth { } { }
        \tl_gset:Nn \l__fdu_header_center_mark_tl { #1 }
      }
  }
%    \end{macrocode}
% \end{macro}
%
% \begin{macro}{\cleardoublepage}
% 重定义 \tn{cleardoublepage},使得偶数页面在没有内容时也不显示
% 页眉页脚。\\
% 见 http://tex.stackexchange.com/q/1681 \\
% 最后清空中间页眉,确保正文部分页眉显示正确。
%    \begin{macrocode}
\RenewDocumentCommand \cleardoublepage { }
  {
    \clearpage
    \bool_if:NT \g__fdu_twoside_bool
      {
        \int_if_odd:nF \c@page
          { \hbox:n { } \thispagestyle { empty } \newpage }
      }
    \tl_gset:Nn \l__fdu_header_center_mark_tl { }
  }
%    \end{macrocode}
% \end{macro}
%
% \pkg{ctex} 宏包使用 |heading| 选项后,会把页面格式设置为
% |headings|。因此必须在 \pkg{ctex} 调用之后重新设置 \cs{pagestyle}
% 为 |fancy|。
%    \begin{macrocode}
\pagestyle { fancy }
%    \end{macrocode}
%
% \begin{macro}{\sectionmark}
% 重定义右侧页眉格式(否则貌似少了一个空格)。
%    \begin{macrocode}
\RenewDocumentCommand \sectionmark { m }
  { \markright { \CTEXthesection \negthinspace \quad #1 } }
%    \end{macrocode}
% \end{macro}
%
% \subsection{脚注}
% \subsubsection{编号样式}
% \begin{macro}{\l__fdu_fn_style_tl}
% 保存当前使用的脚注编号样式。
%    \begin{macrocode}
\tl_new:N \l__fdu_fn_style_tl
%    \end{macrocode}
% \end{macro}
%
% 脚注样式也归入 |fdu/style| 键值类。
%    \begin{macrocode}
\keys_define:nn { fdu / style }
  {
%    \end{macrocode}
%
% \begin{macro}{style/footnotestyle}
% 脚注类型共分四大类:
% |plain|:使用当前字体;
% |libertinus|:取自 Libertinus Serif 和 Libertinus Sans 字体;
% |pifont|:使用 \pkg{pifont} 宏包;
% |xits|:取自 XITS-Math 字体。
%
% 不带任何修饰的为衬线阳文符号,带“|sans|”的为无衬线符号,带“|*|”的
% 为阴文版本。
%    \begin{macrocode}
    footnotestyle .choices:nn = {
      plain,
      libertinus, libertinus*, libertinus-sans,
      pifont,     pifont*,     pifont-sans,     pifont-sans*,
      xits,                    xits-sans,       xits-sans*
    }
%    \end{macrocode}
%
% 若使用 |pifont| 类型,则需引入 \pkg{pifont} 宏包;
% 若使用 |xits| 类型,则需调用 XITS Math 字体。
%    \begin{macrocode}
    {
      \tl_gset_eq:NN \l__fdu_fn_style_tl \l_keys_choice_tl
      \int_compare:nTF
        { 5 <= \l_keys_choice_int <= 8 }
        { \RequirePackage { pifont } }
        {
          \int_compare:nT
            { 9 <= \l_keys_choice_int <= 11 }
            { \setmathfont { XITS~ Math } [ version = fn-XITS ] }
        }
    },
    footnotestyle .value_required:n = true
  }
%    \end{macrocode}
% \end{macro}
%
% \begin{macro}{\fdu_fn_symbol_libertinus:V}
% |libertinus| 普通版。\numrange{1}{20} 为数字,\numrange{21}{46}
% 为小写英文字母,\numrange{47}{72} 为大写英文字母。
%    \begin{macrocode}
\cs_new:Npn \fdu_fn_symbol_libertinus:V #1
  {
    \int_compare:nTF { #1 >= 21 }
      {
        \int_compare:nTF { #1 >= 47 }
          { \symbol { \int_eval:n { "24B6 - 47 + #1 } } }
          { \symbol { \int_eval:n { "24D0 - 21 + #1 } } }
      }
      { \symbol { \int_eval:n { "2460 - 1 + #1 } } }
  }
%    \end{macrocode}
% \end{macro}
%
% \begin{macro}{\fdu_fn_symbol_libertinus_n:V}
% |libertinus| 阴文衬线版。只含 \numrange{1}{20}。
%    \begin{macrocode}
\cs_new:Npn \fdu_fn_symbol_libertinus_n:V #1
  {
    \int_compare:nTF { #1 >= 11 }
      { \symbol { \int_eval:n { "24EB - 11 + #1 } } }
      { \symbol { \int_eval:n { "2776 -  1 + #1 } } }
  }
%    \end{macrocode}
% \end{macro}
%
% \begin{macro}{\fdu_fn_symbol_libertinus_sans:V}
% |libertinus| 阳文无衬线版。符号排列与普通版相同。
%    \begin{macrocode}
\cs_new:Npn \fdu_fn_symbol_libertinus_sans:V #1
  { \fdu_fn_symbol_libertinus:V { #1 } }
%    \end{macrocode}
% \end{macro}
%
% \begin{macro}{\fdu_fn_symbol_pifont:V}
% |pifont| 普通版。以下四种都只包含 \numrange{1}{10}。
%    \begin{macrocode}
\cs_new:Npn \fdu_fn_symbol_pifont:V #1
  { \ding { \int_eval:n { 171 + #1 } } }
%    \end{macrocode}
% \end{macro}
%
% \begin{macro}{\fdu_fn_symbol_pifont_n:V}
% |pifont| 阴文衬线版。
%    \begin{macrocode}
\cs_new:Npn \fdu_fn_symbol_pifont_n:V #1
  { \ding { \int_eval:n { 181 + #1 } } }
%    \end{macrocode}
% \end{macro}
%
% \begin{macro}{\fdu_fn_symbol_pifont_sans:V}
% |pifont| 阳文无衬线版。
%    \begin{macrocode}
\cs_new:Npn \fdu_fn_symbol_pifont_sans:V #1
  { \ding { \int_eval:n { 191 + #1 } } }
%    \end{macrocode}
% \end{macro}
%
% \begin{macro}{\fdu_fn_symbol_pifont_sans_n:V}
% |pifont| 阴文无衬线版。
%    \begin{macrocode}
\cs_new:Npn \fdu_fn_symbol_pifont_sans_n:V #1
  { \ding { \int_eval:n { 201 + #1 } } }
%    \end{macrocode}
% \end{macro}
%
% \begin{macro}{\fdu_fn_symbol_xits:V}
% |xits| 普通版。\numrange{1}{9} 为数字,\numrange{10}{35} 为小写
% 英文字母,\numrange{36}{61} 为大写英文字母。
%    \begin{macrocode}
\cs_new:Npn \fdu_fn_symbol_xits:V #1
  {
    \int_compare:nTF { #1 >= 10 }
      {
        \int_compare:nTF { #1 >= 36 }
          { \symbol { \int_eval:n { "24B6 - 36 + #1 } } }
          { \symbol { \int_eval:n { "24D0 - 10 + #1 } } }
      }
      { \symbol { \int_eval:n { "2460 - 1 + #1 } } }
  }
%    \end{macrocode}
% \end{macro}
%
% \begin{macro}{\fdu_fn_symbol_xits_sans:V}
% |xits| 阳文无衬线版。只包含 \numrange{1}{10}。
%    \begin{macrocode}
\cs_new:Npn \fdu_fn_symbol_xits_sans:V #1
  { \symbol { \int_eval:n { "2780 - 1 + #1 } } }
%    \end{macrocode}
% \end{macro}
%
% \begin{macro}{\fdu_fn_symbol_xits_sans_n:V}
% |xits| 阴文无衬线版。也只包含 \numrange{1}{10}。
%    \begin{macrocode}
\cs_new:Npn \fdu_fn_symbol_xits_sans_n:V #1
  { \symbol { \int_eval:n { "278A - 1 + #1 } } }
%    \end{macrocode}
% \end{macro}
%
% \begin{macro}{\thefootnote}
% 重定义脚注编号。
%    \begin{macrocode}
\RenewDocumentCommand \thefootnote { }
  {
    \int_set:Nn \l__fdu_tmpa_int { \value { footnote } }
    \tl_case:NnF \l__fdu_fn_style_tl
      {
%    \end{macrocode}
%
% |plain| 类型直接使用计数器 |footnote| 的值。
%    \begin{macrocode}
        \c__fdu_fn_style_plain_tl
          { \int_use:N \l__fdu_tmpa_int }
%    \end{macrocode}
%
% |libertinus| 类型需要使用 Libertinus Serif 或
% Libertinus Sans 字体。
%    \begin{macrocode}
        \c__fdu_fn_style_libertinus_tl
          {
            \fontspec { Libertinus~ Serif }
            \fdu_fn_symbol_libertinus:V      \l__fdu_tmpa_int
          }
        \c__fdu_fn_style_libertinus_n_tl
          {
            \fontspec { Libertinus~ Serif }
            \fdu_fn_symbol_libertinus_n:V    \l__fdu_tmpa_int
          }
        \c__fdu_fn_style_libertinus_sans_tl
          {
            \fontspec { Libertinus~ Sans }
            \fdu_fn_symbol_libertinus_sans:V \l__fdu_tmpa_int
          }
%    \end{macrocode}
%
% |pifont| 类型无需进行额外的操作。
%    \begin{macrocode}
        \c__fdu_fn_style_pifont_tl
          { \fdu_fn_symbol_pifont:V        \l__fdu_tmpa_int }
        \c__fdu_fn_style_pifont_n_tl
          { \fdu_fn_symbol_pifont_n:V      \l__fdu_tmpa_int }
        \c__fdu_fn_style_pifont_sans_tl
          { \fdu_fn_symbol_pifont_sans:V   \l__fdu_tmpa_int }
        \c__fdu_fn_style_pifont_sans_n_tl
          { \fdu_fn_symbol_pifont_sans_n:V \l__fdu_tmpa_int }
%    \end{macrocode}
%
% |xits| 类型需要临时切换数学字体。
%    \begin{macrocode}
        \c__fdu_fn_style_xits_tl
          {
            \mathversion { fn-XITS }
            $ \fdu_fn_symbol_xits:V        \l__fdu_tmpa_int $
          }
        \c__fdu_fn_style_xits_sans_tl
          {
            \mathversion { fn-XITS }
            $ \fdu_fn_symbol_xits_sans:V   \l__fdu_tmpa_int $
          }
        \c__fdu_fn_style_xits_sans_n_tl
          {
            \mathversion { fn-XITS }
            $ \fdu_fn_symbol_xits_sans_n:V \l__fdu_tmpa_int $
          }
      }
%    \end{macrocode}
%
% 变量 \cs{l__fdu_fn_style_tl} 保存的类型未知时,默认使用 |plain|
% 类型。
%    \begin{macrocode}
      { \int_use:N \l__fdu_tmpa_int }
  }
%    \end{macrocode}
% \end{macro}
%
% \subsubsection{整体样式}
% \begin{macro}{\@makefntext}
% 重定义内部脚注文字命令。
%    \begin{macrocode}
\RenewDocumentCommand \@makefntext { +m }
  {
%    \end{macrocode}
%
% 脚注编号不使用上标,宽度为 \SI{1.5}{em}。
%
% 见 http://tex.stackexchange.com/q/19844
%    \begin{macrocode}
    \dim_set:Nn \l__fdu_tmpa_dim { \textwidth - 1.5 em }
    \makebox [ 1.5 em ] [ l ] { \@thefnmark }
%    \end{macrocode}
%
% 脚注文字用 |parbox| 封装。首段无缩进,第二段起缩进 \SI{2}{em}。
%    \begin{macrocode}
    \parbox [ t ] { \l__fdu_tmpa_dim }
      {
        \everypar { \hspace* { 2 em } }
        \hspace* { -2 em } #1
      }
  }
%    \end{macrocode}
% \end{macro}
%
% \subsection{定理环境}
% \begin{macro}{theorem/style,
%   theorem/headerfont,
%   theorem/bodyfont,
%   theorem/qed,
%   theorem/counter,
%   \l__fdu_thm_style_tl,
%   \l__fdu_thm_header_font_tl,
%   \l__fdu_thm_body_font_tl,
%   \l__fdu_thm_qed_tl,
%   \l__fdu_thm_counter_tl}
% 定义 |fdu/theorem| 键值类。
%    \begin{macrocode}
\keys_define:nn { fdu / theorem }
  {
    style      .tl_set:N  = \l__fdu_thm_style_tl,
    headerfont .tl_set:N  = \l__fdu_thm_header_font_tl,
    bodyfont   .tl_set:N  = \l__fdu_thm_body_font_tl,
    qed        .tl_set:N  = \l__fdu_thm_qed_tl,
    counter    .tl_set:N  = \l__fdu_thm_counter_tl
  }
%    \end{macrocode}
% \end{macro}
%
% \begin{macro}{\fdu_thm_new:nnnn,\fdu_thm_new:Vnnn}
% 带编号的定理环境。|#1| = 样式, |#2| = 计数器,|#3| = 定理环境名称,
% |#4| = 定理头文字。
%    \begin{macrocode}
\cs_new:Npn \fdu_thm_new:nnnn #1 #2 #3 #4
  {
    \theoremstyle { #1 }
    \newtheorem { #3 } { #4 } [ #2 ]
  }
\cs_generate_variant:Nn \fdu_thm_new:nnnn { Vnnn }
%    \end{macrocode}
% \end{macro}
%
% \begin{macro}{\fdu_thm_no_number_new:nnn,
%   \fdu_thm_no_number_new:Vnn}
% 不带编号的定理环境。|#1| = 样式, |#2| = 定理环境名称,
% |#3| = 定理头文字。
%    \begin{macrocode}
\cs_new:Npn \fdu_thm_no_number_new:nnn #1 #2 #3
  {
    \theoremstyle { #1 }
    \newtheorem { #2 } { #3 }
  }
\cs_generate_variant:Nn \fdu_thm_no_number_new:nnn { Vnn }
%    \end{macrocode}
% \end{macro}
%
% \begin{macro}{\fdu_thm_set_qed:n,
%   \fdu_thm_set_header_font:n,\fdu_thm_set_body_font:n}
% 封装 \pkg{ntheorem} 宏包提供的若干命令,分别用以设置证毕符号、
% 定理头字体和定理正文字体。
%    \begin{macrocode}
\cs_new:Npn \fdu_thm_set_qed:n         #1
  { \theoremsymbol     { #1 } }
\cs_new:Npn \fdu_thm_set_header_font:n #1
  { \theoremheaderfont { #1 } }
\cs_new:Npn \fdu_thm_set_body_font:n   #1
  { \theorembodyfont   { #1 } }
%    \end{macrocode}
% \end{macro}
%
% \begin{macro}{\fdu_thm_set_qed:V,
%   \fdu_thm_set_header_font:V,\fdu_thm_set_body_font:V}
% 生成以上三个函数的变体。
%    \begin{macrocode}
\cs_generate_variant:Nn \fdu_thm_set_qed:n         { V }
\cs_generate_variant:Nn \fdu_thm_set_header_font:n { V }
\cs_generate_variant:Nn \fdu_thm_set_body_font:n   { V }
%    \end{macrocode}
% \end{macro}
%
% \begin{macro}{\fdunewtheorem,\fdunewtheorem*}
% 创建新的定理环境。
%    \begin{macrocode}
\NewDocumentCommand \fdunewtheorem { s o m m }
  {
%    \end{macrocode}
%
% 默认情况下,由 \cs{fdunewtheorem*} 创建的定理其证毕符号为
% \cs{QED},而由 \cs{fdunewtheorem} 创建的则不带证毕符号。符号
% \cs{QED} 由 \pkg{unicode-math} 宏包提供。
%    \begin{macrocode}
    \IfBooleanTF #1
      { \tl_set:Nn \l__fdu_thm_qed_tl { \ensuremath { \QED } } }
      { \tl_set:Nn \l__fdu_thm_qed_tl { } }
%    \end{macrocode}
%
% 设置默认样式为 |plain|。
%    \begin{macrocode}
    \tl_set:Nn \l__fdu_thm_style_tl { plain }
%    \end{macrocode}
%
% 处理可选参数。利用 |fdu/theorem| 键值对设置,并按此修改证毕符号、
% 定理头字体和定理正文字体。
%    \begin{macrocode}
    \IfValueT { #2 }
      { \keys_set:nn { fdu / theorem } { #2 } }
    \fdu_thm_set_header_font:V \l__fdu_thm_header_font_tl
    \fdu_thm_set_body_font:V   \l__fdu_thm_body_font_tl
    \fdu_thm_set_qed:V         \l__fdu_thm_qed_tl
%    \end{macrocode}
%
% \cs{fdunewtheorem} 负责创建编号定理,而 \cs{fdunewtheorem*}
% 则负责创建无编号定理。以下分这两种情况处理。
%    \begin{macrocode}
    \IfBooleanTF #1
      {
%    \end{macrocode}
%
% 带 |*| 的版本原则上只接受 |plain| 和 |break| 两种样式,其余样式
% 将被转换成这两者其中之一。
%
% TODO(20170602):给出重定义样式的警告。
%    \begin{macrocode}
        \clist_if_in:NVTF
          \c__fdu_thm_style_plain_clist
          \l__fdu_thm_style_tl
          { \tl_set:Nn \l__fdu_thm_style_tl { plain } }
          {
            \clist_if_in:NVTF
              \c__fdu_thm_style_break_clist
              \l__fdu_thm_style_tl
              { \tl_set:Nn \l__fdu_thm_style_tl { break } }
% TODO(20170602): 给出样式未定义错误。
              { }
          }
%    \end{macrocode}
%
% \pkg{ntheorem} 宏包提供的无编号定理带有 |nonumber| 前缀,
% 这里我们将其加上。
%    \begin{macrocode}
        \tl_put_left:Nn \l__fdu_thm_style_tl { nonumber }
        \fdu_thm_no_number_new:Vnn \l__fdu_thm_style_tl
          { #3 } { #4 }
      }
      {
%    \end{macrocode}
%
% 不带 |*| 的版本支持不含“|nonumber|”的所有定理样式。
%    \begin{macrocode}
        \clist_clear:N \l__fdu_tmpa_clist
        \clist_concat:NNN \l__fdu_tmpa_clist
          \c__fdu_thm_style_plain_clist \c__fdu_thm_style_break_clist
        \clist_if_in:NVF
          \l__fdu_tmpa_clist \l__fdu_thm_style_tl
% TODO(20170602): 给出样式未定义错误。
          { }
        \fdu_thm_new:Vnnn \l__fdu_thm_style_tl
          { \l__fdu_thm_counter_tl } { #3 } { #4 }
      }
  }
%    \end{macrocode}
% \end{macro}
%
% \subsection{图表绘制;浮动体}
% 分别设置浮动体 |figure| 和 |table| 的标题样式。
%    \begin{macrocode}
\captionsetup [ figure ]
  {
    font     = small,
    labelsep = quad
  }
\captionsetup [ table  ]
  {
    font     = { small, sf },
    labelsep = quad
  }
%    \end{macrocode}
%
% \begin{macro}{\thefigure,\thetable}
% 重定义图表编号。
%    \begin{macrocode}
\RenewDocumentCommand \thefigure { }
  { \arabic { chapter } - \arabic { figure } }
\RenewDocumentCommand \thetable  { }
  { \arabic { chapter } - \arabic { table  } }
%    \end{macrocode}
% \end{macro}
%
% \subsection{封面}
% \subsubsection{信息录入}
% 定义 |fdu/info| 键值类。
%    \begin{macrocode}
\keys_define:nn { fdu / info }
  {
%    \end{macrocode}
%
% \begin{macro}{info/title,info/title*,
%   \l__fdu_info_title_tl,\l__fdu_info_title_en_tl}
% 论文题目。以下带星号的项目均表示相应的英文字段。
%    \begin{macrocode}
    title       .tl_set:N    = \l__fdu_info_title_tl,
    title*      .tl_set:N    = \l__fdu_info_title_en_tl,
%    \end{macrocode}
% \end{macro}
%
% \begin{macro}{info/date,\l__fdu_info_date_tl}
% 论文完成日期。
%    \begin{macrocode}
    date        .tl_set:N    = \l__fdu_info_date_tl,
%    \end{macrocode}
% \end{macro}
%
% \begin{macro}{info/author,info/author*,
%   \l__fdu_info_author_tl,\l__fdu_info_author_en_tl}
% 作者姓名。
%    \begin{macrocode}
    author      .tl_set:N    = \l__fdu_info_author_tl,
    author*     .tl_set:N    = \l__fdu_info_author_en_tl,
%    \end{macrocode}
% \end{macro}
%
% \begin{macro}{info/supervisor,info/supervisor*,
%   \l__fdu_info_supervisor_tl,\l__fdu_info_supervisor_en_tl}
% 导师姓名。
%    \begin{macrocode}
    supervisor  .tl_set:N    = \l__fdu_info_supervisor_tl,
    supervisor* .tl_set:N    = \l__fdu_info_supervisor_en_tl,
%    \end{macrocode}
% \end{macro}
%
% \begin{macro}{info/instructors,\l__fdu_info_instructors_clist}
% 指导小组成员。
%    \begin{macrocode}
    instructors .clist_set:N = \l__fdu_info_instructors_clist,
%    \end{macrocode}
% \end{macro}
%
% \begin{macro}{info/department,info/department*,
%   \l__fdu_info_department_tl,\l__fdu_info_department_en_tl}
% 院系。
%    \begin{macrocode}
    department  .tl_set:N    = \l__fdu_info_department_tl,
    department* .tl_set:N    = \l__fdu_info_department_en_tl,
%    \end{macrocode}
% \end{macro}
%
% \begin{macro}{info/major,info/major*,
%   \l__fdu_info_major_tl,\l__fdu_info_major_en_tl}
% 专业。
%    \begin{macrocode}
    major       .tl_set:N    = \l__fdu_info_major_tl,
    major*      .tl_set:N    = \l__fdu_info_major_en_tl,
%    \end{macrocode}
% \end{macro}
%
% \begin{macro}{info/studentid,\l__fdu_info_student_id_tl}
% 学号。
%    \begin{macrocode}
    studentid   .tl_set:N    = \l__fdu_info_student_id_tl,
%    \end{macrocode}
% \end{macro}
%
% \begin{macro}{info/schoolid,\l__fdu_info_school_id_tl}
% 学校代码。
%    \begin{macrocode}
    schoolid    .tl_set:N    = \l__fdu_info_school_id_tl,
%    \end{macrocode}
% \end{macro}
%
% \begin{macro}{info/keywords,info/keywords*,
%   \l__fdu_info_keywords_clist,\l__fdu_info_keywords_en_clist}
% 论文关键字。
%    \begin{macrocode}
    keywords    .clist_set:N = \l__fdu_info_keywords_clist,
    keywords*   .clist_set:N = \l__fdu_info_keywords_en_clist,
%    \end{macrocode}
% \end{macro}
%
% \begin{macro}{info/clc,\l__fdu_info_clc_tl}
% 中国图书馆分类号。
%    \begin{macrocode}
    clc         .tl_set:N    = \l__fdu_info_clc_tl
  }
%    \end{macrocode}
% \end{macro}
%
% \subsubsection{密级}
% \begin{macro}{\l__fdu_secret_bool}
% 是否显示密级。
%    \begin{macrocode}
\bool_new:N \l__fdu_secret_bool
%    \end{macrocode}
% \end{macro}
%
% \begin{macro}{\l__fdu_info_secret_level_tl}
% 保存当前的密级。
%    \begin{macrocode}
\tl_new:N \l__fdu_info_secret_level_tl
%    \end{macrocode}
% \end{macro}
%
% 密级也放在 |fdu/info| 键值类中。
%    \begin{macrocode}
\keys_define:nn { fdu / info }
  {
%    \end{macrocode}
%
% \begin{macro}{info/secretlevel}
% 密级。|none| 表示不涉密,|i|、|ii|、|iii| 分别为秘密、机密、绝密。
% 密级与保密年限中间的五角星符号需要利用 XITS-Math 字体。
%    \begin{macrocode}
    secretlevel .choices:nn  = {
      none, i, ii, iii
    }
    {
      \int_compare:nTF
        { \l_keys_choice_int >= 2 }
        {
          \bool_set_true:N \l__fdu_secret_bool
          \setmathfont { XITS~ Math } [ version = secret-XITS ]
          \tl_set:Nn \l__fdu_info_secret_level_tl
            {
              \clist_item:Nn \c__fdu_def_secret_clist
                { \l_keys_choice_int - 1 }
            }
        }
        { \bool_set_false:N \l__fdu_secret_bool }
    },
    secretlevel .value_required:n = true,
%    \end{macrocode}
% \end{macro}
%
% \begin{macro}{info/secretyear}
% 保密年限。
%    \begin{macrocode}
    secretyear .tl_set:N = \l__fdu_info_secret_year_tl
  }
%    \end{macrocode}
% \end{macro}
%
% \subsubsection{定义内部函数}
% \begin{macro}{\fdu_spread_box:Nn,\fdu_spread_box:NV,
%   \fdu_spread_box:Nx}
% 分散对齐的盒子。|#1| = 长度, |#2| = 内容。
%
% 利用 \cs{tl_map_inline:nn} 在字符间插入 \tn{hfil};
% 紧随其后的 \tn{unskip} 将会去掉最后一个 \tn{hfil}。\\
% 见 http://tex.stackexchange.com/q/169689
%
% |NV| 以及 |Nx| 版本的命令可以避免展开 token list 时出现坏盒子。
%    \begin{macrocode}
\cs_new:Npn \fdu_spread_box:Nn #1 #2
  {
    \makebox [ #1 ] [ s ]
      { \tl_map_inline:nn { #2 } { ##1 \hfil } \unskip }
  }
\cs_generate_variant:Nn \fdu_spread_box:Nn { NV }
\cs_generate_variant:Nn \fdu_spread_box:Nn { Nx }
%    \end{macrocode}
% \end{macro}
%
% \begin{macro}{\fdu_center_box:Nn,\fdu_center_box:NV}
% 居中对齐的盒子。|#1| = 长度, |#2| = 内容。
%    \begin{macrocode}
\cs_new:Npn \fdu_center_box:Nn #1 #2
  { \makebox [ #1 ] [ c ] { #2 } }
\cs_generate_variant:Nn \fdu_center_box:Nn { NV }
%    \end{macrocode}
% \end{macro}
%
% \begin{macro}{\fdu_fixed_width_box:Nn}
% 限宽盒子。|#1| = 长度, |#2| = 内容。
%    \begin{macrocode}
\cs_new:Npn \fdu_fixed_width_box:Nn #1 #2
  { \parbox { #1 } { #2 } }
%    \end{macrocode}
% \end{macro}
%
% \begin{macro}{\fdu_fixed_width_center_box:Nn,
%   \fdu_fixed_width_center_box:NV}
% 居中对齐的限宽盒子。|#1| = 长度, |#2| = 内容。
%    \begin{macrocode}
\cs_new:Npn \fdu_fixed_width_center_box:Nn #1 #2
  { \fdu_fixed_width_box:Nn { #1 } { \centering #2 } }
\cs_generate_variant:Nn \fdu_fixed_width_center_box:Nn { NV }
%    \end{macrocode}
% \end{macro}
%
% \begin{macro}{\fdu_get_text_width:Nn}
% 获取文本宽度,并存入 |dim| 型变量。|#1| = |dim| 型变量,
% |#2| = 内容。
%    \begin{macrocode}
\cs_new:Npn \fdu_get_text_width:Nn #1 #2
  {
    \hbox_set:Nn \l__fdu_tmpa_box { #2 }
    \dim_set:Nn #1
      { \box_wd:N \l__fdu_tmpa_box }
  }
%    \end{macrocode}
% \end{macro}
%
% \begin{macro}{\fdu_get_max_text_width:NN}
% 获取多个文本中的最大宽度,并存入 |dim| 型变量。
% |#1| = |dim| 型变量,|#2| = 文本 |clist|。
%
% 当 \cs{l__fdu_tmpa_clist} 非空时,弹出最后一个元素
% 赋给 \cs{l__fdu_tmpa_tl},获取其长度后与 |#1| 进行比较,
% 二者中较大的那一个将成为 |#1| 的新值。
% 不断循环,直至 \cs{l__fdu_tmpa_clist} 为空。
%    \begin{macrocode}
\cs_new:Npn \fdu_get_max_text_width:NN #1 #2
  {
    \group_begin:
    \clist_set_eq:NN \l__fdu_tmpa_clist #2
    \bool_until_do:nn { \clist_if_empty_p:N \l__fdu_tmpa_clist }
      {
        \clist_pop:NN \l__fdu_tmpa_clist \l__fdu_tmpa_tl
        \fdu_get_text_width:Nn \l__fdu_tmpa_dim
          { \large \l__fdu_tmpa_tl }
        \dim_gset:Nn #1
          { \dim_max:nn { #1 } { \l__fdu_tmpa_dim } }
      }
    \group_end:
  }
%    \end{macrocode}
% \end{macro}
%
% \begin{macro}{\fdu_blank_underline:N}
% 下划线占位符。|#1| = 长度。
%    \begin{macrocode}
\cs_new:Npn \fdu_blank_underline:N #1
  { \uline { \hbox_to_wd:nn { #1 } { } } }
%    \end{macrocode}
% \end{macro}
%
% \begin{macro}{\fdu_line_spread:N,\fdu_line_spread:n}
% 设置行距。|#1| = 行距倍数。
%    \begin{macrocode}
\cs_new:Npn \fdu_line_spread:N #1
  { \linespread { #1 } \selectfont }
\cs_generate_variant:Nn \fdu_line_spread:N { n }
%    \end{macrocode}
% \end{macro}
%
% \subsubsection{封面各部件}
% \begin{macro}{\__fdu_cover_id:}
% 右上角的学校代码和学号。
%    \begin{macrocode}
\cs_new:Npn \__fdu_cover_id:
  {
    \begin{flushright}
      \dim_set:Nn \rightskip { \c__fdu_def_cover_id_margin_sep_tl }
      \fdu_fixed_width_box:Nn \c__fdu_def_cover_id_width_tl
        {
          \__fdu_cover_font_size_small:
          \bool_if:NT \l__fdu_secret_bool
            {
              \group_begin:
                \sffamily \mathversion { secret-XITS }
                \c__fdu_def_name_secret_level_tl
                \c__fdu_colon_fullwidth_tl
                \l__fdu_info_secret_level_tl
                \c__fdu_def_name_secret_star_tl
                \l__fdu_info_secret_year_tl
              \group_end:
              \par
            }
          \c__fdu_def_name_school_id_tl
          \c__fdu_colon_fullwidth_tl
          \l__fdu_info_school_id_tl
          \par
          \c__fdu_def_name_student_id_tl
          \c__fdu_colon_fullwidth_tl
          \l__fdu_info_student_id_tl
        }
    \end{flushright}
  }
%    \end{macrocode}
% \end{macro}
%
% \begin{macro}{\__fdu_cover_logo:}
% 插入校名(毛体“復旦大學”)。
%    \begin{macrocode}
\cs_new:Npn \__fdu_cover_logo:
  {
    \begin{center}
      \includegraphics [
        width = \c__fdu_def_cover_logo_width_tl
      ] { \c__fdu_def_cover_logo_file_name_tl }
    \end{center}
  }
%    \end{macrocode}
% \end{macro}
%
% \begin{macro}{\__fdu_cover_title:}
% 标题,共有四行。第一行是论文类型,第二行是学位类型,三、四两行分别
% 是中英文题目。
%    \begin{macrocode}
\cs_new:Npn \__fdu_cover_title:
  {
    \begin{center}
      {
        \__fdu_cover_font_size_huge:
        \fdu_spread_box:NV
          \c__fdu_def_cover_type_width_tl
          \c__fdu_def_name_thesis_type_tl
      }
      \par \vspace { \c__fdu_def_cover_v_sep_iii_tl }
      {
        \__fdu_cover_font_size_normal:
        \c__fdu_def_name_degree_type_tl
      }
      \par \vspace { \c__fdu_def_cover_v_sep_iv_tl }
      {
        \__fdu_cover_font_size_large:  \sffamily
        \fdu_fixed_width_center_box:NV
          \c__fdu_def_cover_title_width_tl
          \l__fdu_info_title_tl
      }
      \par \vspace { \c__fdu_def_cover_v_sep_v_tl }
      {
        \__fdu_cover_font_size_normal: \bfseries
        \fdu_fixed_width_center_box:Nn
          \c__fdu_def_cover_title_en_width_tl
          {
            \fdu_line_spread:N
              \c__fdu_def_cover_title_en_line_spread_tl
            \l__fdu_info_title_en_tl
          }
      }
    \end{center}
  }
%    \end{macrocode}
% \end{macro}
%
% \begin{macro}{\__fdu_cover_info:}
% 信息栏。
%    \begin{macrocode}
\cs_new:Nn \__fdu_cover_info:
  {
    \begin{center}
%    \end{macrocode}
%
% 读取左侧名称字段。
%    \begin{macrocode}
      \clist_set:Nn \l__fdu_tmpa_clist
        {
          \c__fdu_def_name_department_tl,
          \c__fdu_def_name_major_tl,
          \c__fdu_def_name_author_tl,
          \c__fdu_def_name_supervisor_tl,
          \c__fdu_def_name_date_tl,
        }
%    \end{macrocode}
%
% 设置信息栏右侧宽度。读取各字段,并将最宽者的宽度赋给
% \cs{l__fdu_tmpb_dim}。
%    \begin{macrocode}
      \clist_set:Nn \l__fdu_tmpb_clist
        {
          \l__fdu_info_department_tl,
          \l__fdu_info_major_tl,
          \l__fdu_info_author_tl,
          \l__fdu_info_supervisor_tl,
          \l__fdu_info_date_tl
        }
      \fdu_get_max_text_width:NN
        \l__fdu_tmpb_dim \l__fdu_tmpb_clist
%    \end{macrocode}
%
% 在 |minipage| 环境中输出各字段。用循环实现。
%    \begin{macrocode}
      \begin{minipage} [ c ] { \textwidth }
        \centering \__fdu_cover_font_size_normal:
        \bool_until_do:nn
          { \clist_if_empty_p:N \l__fdu_tmpa_clist }
          {
            \clist_pop:NN \l__fdu_tmpa_clist \l__fdu_tmpa_tl
            \clist_pop:NN \l__fdu_tmpb_clist \l__fdu_tmpb_tl
            \fdu_spread_box:Nx
              \c__fdu_def_cover_info_left_width_tl \l__fdu_tmpa_tl
            \c__fdu_colon_fullwidth_tl
            \fdu_center_box:NV
              \l__fdu_tmpb_dim \l__fdu_tmpb_tl
            \par \vspace { \c__fdu_def_cover_v_sep_vii_tl }
          }
      \end{minipage}
    \end{center}
  }
%    \end{macrocode}
% \end{macro}
%
% \begin{macro}{\__fdu_decl_text:NNn}
% 构建声明文本。|#1| = 标题,|#2| = 内容,|#3| = 签名行。段前空格
% 需要用 \tn{qquad} 手动生成。
%    \begin{macrocode}
\cs_new:Npn \__fdu_decl_text:NNn #1 #2 #3
  {
    \begin{center}
%<class-en>      \fdu_line_spread:n { \fp_use:N \c__fdu_def_line_spread_fp }
      \__fdu_cover_font_size_large: \sffamily #1
    \end{center}
    \vspace { \c__fdu_def_decl_v_sep_iv_tl }
    \begin{center}
      \fdu_fixed_width_box:Nn \textwidth
        {
          \fdu_line_spread:N \c__fdu_def_decl_text_line_spread_tl
          \qquad #2
        }
    \end{center}
    \vspace { \c__fdu_def_decl_v_sep_iv_tl }
%    \end{macrocode}
%
% \tn{hfill} 用来确保签名行靠右对齐。
%    \begin{macrocode}
    { \hfill #3 }
  }
%    \end{macrocode}
% \end{macro}
%
% \subsubsection{绘制封面}
% \begin{macro}{\makecoveri}
% 生成封一,即真正的封面。各部件之间用橡皮长度隔开。
%    \begin{macrocode}
\NewDocumentCommand \makecoveri { }
  {
    \group_begin:
%<class-en>      \fdu_line_spread:n { \fp_use:N \c__fdu_def_line_spread_fp }
      \__fdu_cover_id:
      \vspace { \c__fdu_def_cover_v_sep_i_tl  }
      \__fdu_cover_logo:
      \vspace { \c__fdu_def_cover_v_sep_ii_tl }
      \__fdu_cover_title:
      \vspace { \c__fdu_def_cover_v_sep_vi_tl }
      \__fdu_cover_info:
      \vspace { \c__fdu_def_cover_v_sep_ix_tl }
    \group_end:
  }
%    \end{macrocode}
% \end{macro}
%
% \begin{macro}{\makecoverii}
% 生成封二,即指导小组成员名单。
%    \begin{macrocode}
\NewDocumentCommand \makecoverii { }
  {
    \group_begin:
%    \end{macrocode}
%
% 临时禁用 \tn{cleardoublepage} 带来的分页。
%    \begin{macrocode}
      \cs_set_eq:NN \cleardoublepage \relax
      \thispagestyle { empty }
%    \end{macrocode}
%
% 保持英文模板与中文模板的一致。
%    \begin{macrocode}
%<*class-en>
      \keys_set:nn { ctex }
        { chapter / titleformat = \c__fdu_def_chapter_format_tl }
      \fdu_line_spread:n { \fp_use:N \c__fdu_def_line_spread_fp }
%</class-en>
%    \end{macrocode}
%
% 为了关闭页眉页脚,此处使用了不编号章节的原始命令 \tn{@schapter}。
%    \begin{macrocode}
      \@schapter
        {
          \fdu_spread_box:NV
            \c__fdu_def_cover_instructors_width_tl
            \c__fdu_def_name_instructors_tl
        }
      \begin{center}
        \large
        \clist_use:Nn \l__fdu_info_instructors_clist { \par }
      \end{center}
    \group_end:
  }
%    \end{macrocode}
% \end{macro}
%
% \begin{macro}{\makecoveriii}
% 生成封三,即声明页。该页也需要关闭页眉、页脚显示。
%    \begin{macrocode}
\NewDocumentCommand \makecoveriii { }
  {
    \cleardoublepage
    \thispagestyle { empty }
    \vspace* { \c__fdu_def_decl_v_sep_i_tl }
%    \end{macrocode}
%
% 独创性声明。
%    \begin{macrocode}
    \__fdu_decl_text:NNn
      \c__fdu_def_name_originality_decl_tl
      \c__fdu_def_originality_decl_text_tl
      {
        \c__fdu_def_name_author_sign_tl
        \c__fdu_colon_fullwidth_tl
        \fdu_blank_underline:N \c__fdu_def_decl_sign_width_tl
        \quad
        \c__fdu_def_name_sign_date_tl
        \c__fdu_colon_fullwidth_tl
        \fdu_blank_underline:N \c__fdu_def_decl_date_width_tl
      }
    \vspace { \c__fdu_def_decl_v_sep_ii_tl }
%    \end{macrocode}
%
% 使用授权声明。
%    \begin{macrocode}
    \__fdu_decl_text:NNn
      \c__fdu_def_name_authorization_decl_tl
      \c__fdu_def_authorization_decl_text_tl
      {
        \c__fdu_def_name_author_sign_tl
        \c__fdu_colon_fullwidth_tl
        \fdu_blank_underline:N \c__fdu_def_decl_sign_width_tl
        \quad
        \c__fdu_def_name_supervisor_sign_tl
        \c__fdu_colon_fullwidth_tl
        \fdu_blank_underline:N \c__fdu_def_decl_sign_width_tl
        \quad
        \c__fdu_def_name_sign_date_tl
        \c__fdu_colon_fullwidth_tl
        \fdu_blank_underline:N \c__fdu_def_decl_date_width_tl
      }
    \vspace { \c__fdu_def_decl_v_sep_iii_tl }
  }
%    \end{macrocode}
% \end{macro}
%
% \begin{macro}{style/automakecover,\l__fdu_auto_make_cover_bool}
% 是否自动生成封面。
%    \begin{macrocode}
\keys_define:nn { fdu / style }
  {
    automakecover .bool_set:N = \l__fdu_auto_make_cover_bool,
    automakecover .default:n  = true
  }
%    \end{macrocode}
% \end{macro}
%
% 在 |document| 开始位置添加封面以及指导小组成员名单。
%    \begin{macrocode}
\AtBeginDocument
  {
    \bool_if:NT \l__fdu_auto_make_cover_bool
      {
        \begin{titlepage}
          \makecoveri \newpage \makecoverii
        \end{titlepage}
      }
  }
%    \end{macrocode}
%
% 在 |document| 结束位置添加声明页。
%    \begin{macrocode}
\AtEndDocument
  { \bool_if:NT \l__fdu_auto_make_cover_bool { \makecoveriii } }
%    \end{macrocode}
%
% \subsection{目录}
% 设置目录标题。
%    \begin{macrocode}
\keys_set:nn { ctex }
  {
%<class>    contentsname = { \c__fdu_def_name_toc_tl },
%<class-en>    contentsname = { \c__fdu_def_name_toc_en_tl },
%    \end{macrocode}
%
% 设置目录中章节标题的样式。
%    \begin{macrocode}
    chapter    / tocline = {
%<class>      \c__fdu_def_chapter_toc_format_tl    \CTEXnumberline { #1 } #2
%<class-en>      \c__fdu_def_chapter_toc_format_en_tl \CTEXnumberline { #1 } #2
    },
    section    / tocline = {
%<class>      \c__fdu_def_section_toc_format_tl    \CTEXnumberline { #1 } #2
%<class-en>      \c__fdu_def_section_toc_format_en_tl \CTEXnumberline { #1 } #2
    },
    subsection / tocline = {
%<class>      \c__fdu_def_subsection_toc_format_tl \CTEXnumberline { #1 } #2
%<class-en>      \c__fdu_def_subsection_toc_format_en_tl
%<class-en>        \CTEXnumberline { #1 } #2
    }
  }
%    \end{macrocode}
%
% \begin{macro}{\tableofcontents}
% 修改 \tn{tableofcontents} 的定义,使得页眉正确显示。第二个参数中的
% 代码来源于 \LaTeXe{} 标准文档类 \file{book.cls}。
%    \begin{macrocode}
\ctex_patch_cmd_once:NnnnTF \tableofcontents
  { }
  {
    \chapter*{\contentsname
      \@mkboth{%
        \MakeUppercase\contentsname}{\MakeUppercase\contentsname}}%
  }
  {
    \chapter* { \contentsname }
%<class>    \fdu_front_matter_header:n { \c__fdu_def_name_toc_tl }
%<class-en>    \fdu_front_matter_header:n { \c__fdu_def_name_toc_en_tl }
  }
  { } { \ctex_patch_failure:N \tableofcontents }
%    \end{macrocode}
% \end{macro}
%
% \begin{macro}{\@starttoc}
% 修改 \tn{@starttoc} 的定义以调整英文模板中的目录行距。
%    \begin{macrocode}
%<*class-en>
\ctex_patch_cmd_once:NnnnTF \@starttoc
  { }
  { \begingroup }
  {
    \begingroup
      \fdu_line_spread:n { \fp_use:N \c__fdu_def_line_spread_fp }
  }
  { } { \ctex_patch_failure:N \@starttoc }
%</class-en>
%    \end{macrocode}
% \end{macro}
%
% \subsection{摘要}
% \subsubsection{中文摘要}
% \begin{macro}{abstract}
% 中文摘要及关键字。
%    \begin{macrocode}
%<*class>
\NewDocumentEnvironment { abstract } { }
  {
%    \end{macrocode}
%
% 中文摘要标题为“摘 \quad 要”,需要修改页眉,并添加到目录。
%    \begin{macrocode}
    \chapter* { \c__fdu_def_name_abstract_tl }
    \fdu_front_matter_header:n { \c__fdu_def_name_abstract_tl }
    \addcontentsline { toc } { chapter }
      {
        \c__fdu_def_chapter_toc_format_tl
        \c__fdu_def_name_abstract_tl
      }
  }
  {
%    \end{macrocode}
%
% 摘要正文完成后,空行,输出关键字列表,之间用分号隔开。
%    \begin{macrocode}
    \par \mbox{} \par
    \noindent \hangindent = 4 em  \hangafter = 1
    {
      \normalfont \sffamily
      \c__fdu_def_name_keywords_tl \c__fdu_colon_fullwidth_tl
    }
    \clist_use:Nn \l__fdu_info_keywords_clist
      { \c__fdu_semicolon_fullwidth_tl }
    \par
%    \end{macrocode}
%
% 下一行输出中图分类号(CLC)。
%    \begin{macrocode}
    \noindent
    {
      \normalfont \sffamily
      \c__fdu_def_name_clc_tl \c__fdu_colon_fullwidth_tl
    }
    \l__fdu_info_clc_tl
  }
%</class>
%    \end{macrocode}
% \end{macro}
%
% \subsubsection{英文摘要}
% \begin{macro}{abstract*,abstract}
% 英文摘要及关键字。注意英文模板中的 |abstract| 环境与中文模板中的
% |abstract*| 环境是相同的,后者在英文模板中没有定义。
%    \begin{macrocode}
%<class>\NewDocumentEnvironment { abstract* } { }
%<class-en>\NewDocumentEnvironment { abstract } { }
  {
%    \end{macrocode}
%
% 英文摘要标题为“Abstract”,也要修改页眉并添加到目录。
%    \begin{macrocode}
    \chapter* { \c__fdu_def_name_abstract_en_tl }
    \fdu_front_matter_header:n { \c__fdu_def_name_abstract_en_tl }
    \addcontentsline { toc } { chapter }
      {
%<class>        \c__fdu_def_chapter_toc_format_tl
%<class-en>        \c__fdu_def_chapter_toc_format_en_tl
        \c__fdu_def_name_abstract_en_tl
      }
  }
  {
%    \end{macrocode}
%
% 空行,输出关键字,之间为全角空格。
%    \begin{macrocode}
    \par \mbox{} \par
    \noindent \hangindent = 4 em \hangafter = 1
    \textbf{\c__fdu_def_name_keywords_en_tl} \quad
    \clist_use:Nn \l__fdu_info_keywords_en_clist { \quad }
    \par
%    \end{macrocode}
%
% 下一行输出中图分类号(CLC)。
%    \begin{macrocode}
    \noindent
    \textbf{\c__fdu_def_name_clc_en_tl} \quad
    \l__fdu_info_clc_tl
  }
%    \end{macrocode}
% \end{macro}
%
% \subsection{符号表}
% \begin{macro}{notation}
% 符号表环境,利用 |longtable| 封装。可选参数为表格列格式说明符。
% 与摘要类似,符号表页需要修改页眉,并添加到目录。另外需要调整
% \cs{LTpre} 和 \cs{LTpost},以删去 |longtable| 前后的空白。
%    \begin{macrocode}
\NewDocumentEnvironment { notation }
  { O { \c__fdu_def_notation_arg_tl } }
  {
%<*class>
    \chapter* { \c__fdu_def_name_notation_tl }
    \fdu_front_matter_header:n { \c__fdu_def_name_notation_tl }
    \addcontentsline { toc } { chapter }
      {
        \c__fdu_def_chapter_toc_format_tl
        \c__fdu_def_name_notation_tl
      }
    \group_begin:
%</class>
%<*class-en>
    \chapter* { \c__fdu_def_name_notation_en_tl }
    \fdu_front_matter_header:n { \c__fdu_def_name_notation_en_tl }
    \addcontentsline { toc } { chapter }
      {
        \c__fdu_def_chapter_toc_format_en_tl
        \c__fdu_def_name_notation_en_tl
      }
    \group_begin:
      \cs_set_eq:NN \arraystretch
        \c__fdu_def_notation_line_stretch_en_tl
%</class-en>
      \dim_set_eq:NN \LTpre  \c_zero_dim
      \dim_set_eq:NN \LTpost \c_zero_dim
      \begin{longtable} { #1 }
  }
  {
      \end{longtable}
    \group_end:
  }
%    \end{macrocode}
% \end{macro}
%
% \subsection{文字绕排}
% WARNING:严重冲突,暂时不启用。
%    \begin{macrocode}
% \RequirePackage{xgalley}
%
%
% \box_new:N \l__fdu_tmpb_box
%
% \dim_new:N \l__fdu_wrap_width_dim
% \dim_new:N \l__fdu_wrap_height_dim
%
% \clist_new:N \l__fdu_wrap_indent_clist
%
% \int_new:N \l__fdu_tmpa_int
% \int_new:N \l__fdu_wrap_lines_int
%
% \fp_new:N \l__fdu_tmpa_fp
%
%
% \keys_define:nn { xwrapfig }
% {
%   cutout .code:n = {
%     \keys_set:nn { xwrapfig / cutout } { #1 }
%   }
% }
%
% \keys_define:nn { fdu / wrap / cutout }
% {
%   % 环境前不改变的行数
%   top~ lines    .int_set:N = \l__fdu_wrap_top_lines_int,
%   % 左右边距
%   left~  margin .dim_set:N = \l__fdu_wrap_L_margin_dim,
%   right~ margin .dim_set:N = \l__fdu_wrap_R_margin_dim,
%   % 上下行距
%   before~ lines .int_set:N = \l__fdu_wrap_before_lines_int,
%   after~  lines .int_set:N = \l__fdu_wrap_after_lines_int,
%   %
%   top~ lines    .initial:n = { 2 },
%   left~  margin .initial:n = { 0.5 em },
%   right~ margin .initial:n = { 0.5 em },
%   before~ lines .initial:n = { 1 },
%   after~  lines .initial:n = { 1 }
% }
%
%
% \cs_generate_variant:Nn \galley_cutout_right:nn { nV }
% \cs_generate_variant:Nn \galley_cutout_left:nn  { nV }
%
%
% % 预先准备
% % 参数:内容
% \cs_new_protected:Nn \fdu_wrap_prewrap:n
% {
%   % 清除列表,初始化
%   \clist_clear:N \l__fdu_wrap_indent_clist
%
%   % 装到 hbox
%   \hbox_set:Nn \l__fdu_tmpa_box { #1 }
%   % 总宽度 = 盒子宽 + 调整距离
%   \dim_set:Nn \l__fdu_wrap_width_dim
%     { \box_wd:N \l__fdu_tmpa_box }
%   \dim_add:Nn \l__fdu_wrap_width_dim
%     { \l__fdu_wrap_L_margin_dim + \l__fdu_wrap_R_margin_dim }
%
%   % 内容装到 vbox
%   \vbox_set:Nn \l__fdu_tmpb_box { #1 }
%   % 总高度 = 盒子高 + 盒子深
%   \dim_set:Nn \l__fdu_wrap_height_dim
%     { \box_ht:N \l__fdu_tmpb_box + \box_dp:N \l__fdu_tmpb_box }
%   % 总占据行数 = 总高度 / 行距 + 调整行数
%   \int_set:Nn \l__fdu_wrap_lines_int
%     {
%       ( \l__fdu_wrap_height_dim / \baselineskip )
%       + \l__fdu_wrap_before_lines_int
%       + \l__fdu_wrap_after_lines_int
%     }
%
%   % 循环:构建 clist,共 {行数} 个元素,每个元素均为 {总宽度}
%   \int_zero:N \l__fdu_tmpa_int
%   \int_do_while:nn
%     { \l__fdu_tmpa_int < \l__fdu_wrap_lines_int }
%     {
%       \int_incr:N \l__fdu_tmpa_int
%       \clist_put_right:Nn \l__fdu_wrap_indent_clist
%         { \l__fdu_wrap_width_dim }
%     }
% }
%
% % 右边插入内容
% % 参数1:不动的行数,参数2:内容
% \cs_new_protected:Nn \fdu_wrap_put_right:nn
% {
%   \fdu_wrap_prewrap:n { #2 }
%
%   % 开窗
%   \galley_cutout_right:nV { #1 } \l__fdu_wrap_indent_clist
%
%   % 内容存入盒子
%   \vbox_set:Nn \l__fdu_tmpa_box
%     {
%       % 垂直移动距离 = (不动的行数 + 0.5 * 调整行数) * 行距
%       \fp_set:Nn \l__fdu_tmpa_fp
%         {
%           ( #1 + \l__fdu_wrap_before_lines_int )
%           * \baselineskip
%         }
%       \skip_vertical:n  { \fp_to_dim:N \l__fdu_tmpa_fp }
%
%       % 插入盒子
%       % 宽度:行宽
%       % 内容:跳一个距离(行宽 - 内容总宽 + 左调整宽度)
%       %      内容
%       %      再跳一个距离(右调整宽度)
%       \hbox_to_wd:nn { \linewidth }
%         {
%           \skip_horizontal:n
%             {
%               \linewidth
%               - \l__fdu_wrap_width_dim
%               + \l__fdu_wrap_L_margin_dim
%             }
%           #2
%           % \skip_horizontal:n { \l__fdu_wrap_R_margin_dim }
%         }
%     }
%
%   \box_set_ht:Nn \l__fdu_tmpa_box { 0pt }
%   \box_set_dp:Nn \l__fdu_tmpa_box { 0pt }
%   \skip_vertical:n { -\baselineskip }
%   \box_use:N \l__fdu_tmpa_box
% }
%
% % 左边插入内容
% % 参数1:不动的行数,参数2:内容
% \cs_new_protected:Nn \fdu_wrap_put_left:nn
% {
%   \fdu_wrap_prewrap:n { #2 }
%
%   % 开窗
%   \galley_cutout_left:nV { #1 } \l__fdu_wrap_indent_clist
%
%   % 内容存入盒子
%   \vbox_set:Nn \l__fdu_tmpa_box
%     {
%       % 垂直移动距离 = (不动的行数 + 0.5 * 调整行数) * 行距
%       \fp_set:Nn \l__fdu_tmpa_fp
%         { ( #1 + \l__fdu_wrap_before_lines_int ) * \baselineskip }
%       \skip_vertical:n  { \fp_to_dim:N \l__fdu_tmpa_fp }
%
%       % 插入盒子
%       % 宽度:行宽
%       % 内容:跳一个距离(左调整宽度)
%       %      内容
%       \hbox_to_wd:nn { \linewidth }
%         {
%           \skip_horizontal:n {  \l__fdu_wrap_L_margin_dim }
%           #2
%         }
%     }
%
%   \box_set_ht:Nn \l__fdu_tmpa_box { 0pt }
%   \box_set_dp:Nn \l__fdu_tmpa_box { 0pt }
%   \skip_vertical:n { -\baselineskip }
%   \box_use:N \l__fdu_tmpa_box
% }
%
% \cs_generate_variant:Nn \fdu_wrap_put_right:nn { Vn }
% \cs_generate_variant:Nn \fdu_wrap_put_left:nn { Vn }
%
%
% % 参数1:选项,参数2:内容
% \NewDocumentCommand\putright { O { } +m }
% {
%   \keys_set:nn { fdu / wrap / cutout } { #1 }
%   \fdu_wrap_put_right:Vn \l__fdu_wrap_top_lines_int { #2 }
% }
% \NewDocumentCommand\putleft { O { } +m }
% {
%   \keys_set:nn { fdu / wrap / cutout } { #1 }
%   \fdu_wrap_put_left:Vn \l__fdu_wrap_top_lines_int { #2 }
% }
%
%
% \NewDocumentCommand\resetindents { }
% {
%   \galley_parshape_set_multi:nnnN
%     { 0 } { 0pt } { 0pt } \c_true_bool
% }
%    \end{macrocode}
%
% \subsection{用户接口}
% \begin{macro}{info,style}
% 定义元(meta)键值对。
%    \begin{macrocode}
\keys_define:nn { fdu }
  {
    info  .meta:nn = { fdu / info  } { #1 },
    style .meta:nn = { fdu / style } { #1 }
  }
%    \end{macrocode}
% \end{macro}
%
% 文档类初始设置。
%    \begin{macrocode}
\keys_set:nn { fdu }
  {
    style   / font          =  times,
%<class>    style   / cjkfont       =  fandol,
    style   / fontsize      =  -4,
%<class>    style   / fullwidthstop =  false,
    style   / automakecover =  true,
    info    / secretlevel   =  none,
    info    / date          =  \zhtoday,
    info    / schoolid      =  10246,
%<class>    theorem / headerfont    = { \sffamily },
%<class-en>    theorem / headerfont    = { \bfseries \upshape },
%<class>    theorem / bodyfont      = { \CJKfamily { kai } },
%<class-en>    theorem / bodyfont      = { \itshape },
    theorem / counter       = { chapter }
  }
%    \end{macrocode}
%
% \begin{macro}{\fdusetup}
% 用户设置接口。
%    \begin{macrocode}
\NewDocumentCommand \fdusetup { m }
  { \keys_set:nn { fdu } { #1 } }
%    \end{macrocode}
% \end{macro}
%
% \begin{macro}{proof,
%   assumption,axiom,conjecture,corollary,definition,example,
%   exercise,lemma,problem,proposition,remark,theorem}
% 模板预定义的常用数学环境。
% 其中的“证明”比较特殊,它不编号,但会添加证毕符号。
%    \begin{macrocode}
%<*class>
\fdunewtheorem* { proof       } { 证明 }
\fdunewtheorem  { assumption  } { 假设 }
\fdunewtheorem  { axiom       } { 公理 }
\fdunewtheorem  { conjecture  } { 猜想 }
\fdunewtheorem  { corollary   } { 推论 }
\fdunewtheorem  { definition  } { 定义 }
\fdunewtheorem  { example     } { 例   }
\fdunewtheorem  { exercise    } { 练习 }
\fdunewtheorem  { lemma       } { 引理 }
\fdunewtheorem  { problem     } { 问题 }
\fdunewtheorem  { proposition } { 命题 }
\fdunewtheorem  { remark      } { 评注 }
\fdunewtheorem  { theorem     } { 定理 }
%</class>
%<*class-en>
\fdunewtheorem* { proof       } { Proof       }
\fdunewtheorem  { assumption  } { Assumption  }
\fdunewtheorem  { axiom       } { Axiom       }
\fdunewtheorem  { conjecture  } { Conjecture  }
\fdunewtheorem  { corollary   } { Corollary   }
\fdunewtheorem  { definition  } { Definition  }
\fdunewtheorem  { example     } { Example     }
\fdunewtheorem  { exercise    } { Exercise    }
\fdunewtheorem  { lemma       } { Lemma       }
\fdunewtheorem  { problem     } { Problem     }
\fdunewtheorem  { proposition } { Proposition }
\fdunewtheorem  { remark      } { Remark      }
\fdunewtheorem  { theorem     } { Theorem     }
%</class-en>
%</class|class-en>
%    \end{macrocode}
% \end{macro}
%
% \subsection{模板参数配置文件}
%    \begin{macrocode}
%<*definition>
%    \end{macrocode}
%
% \subsubsection{通用配置}
% \begin{macro}{\c__fdu_full_stop_ideographic_tl,
%   \c__fdu_full_stop_fullwidth_tl,
%   \c__fdu_colon_fullwidth_tl,
%   \c__fdu_semicolon_fullwidth_tl}
% 一些标点符号:
% U+3002 是圆圈句号“\symbol{"3002}”(ideographic full stop),
% U+FF0E 是全角实心句点“\symbol{"FF0E}”(fullwidth full stop),
% U+FF1A 是全角冒号“\symbol{"FF1A}”(fullwidth colon),
% U+FF1B 是全角分号“\symbol{"FF1B}”(fullwidth semicolon)。
%    \begin{macrocode}
\tl_const:Nn \c__fdu_full_stop_ideographic_tl { \symbol { "3002 } }
\tl_const:Nn \c__fdu_full_stop_fullwidth_tl   { \symbol { "FF0E } }
\tl_const:Nn \c__fdu_colon_fullwidth_tl       { \symbol { "FF1A } }
\tl_const:Nn \c__fdu_semicolon_fullwidth_tl   { \symbol { "FF1B } }
%    \end{macrocode}
% \end{macro}
%
% \begin{macro}{\c__fdu_def_paper_size_tl}
% 纸张大小(A4)。
%    \begin{macrocode}
\tl_const:Nn \c__fdu_def_paper_size_tl { a4paper }
%    \end{macrocode}
% \end{macro}
%
% \begin{macro}{\c__fdu_def_page_margin_top_dim,
%   \c__fdu_def_page_margin_bottom_dim,
%   \c__fdu_def_page_margin_left_dim,
%   \c__fdu_def_page_margin_right_dim}
% 页面边距。这里,$\SI{2.54}{\centi\meter}=\SI{1}{in}$,
% $\SI{3.18}{\centi\meter}=\SI{1.25}{in}$。
%    \begin{macrocode}
\dim_const:Nn \c__fdu_def_page_margin_top_dim    { 2.54 cm }
\dim_const:Nn \c__fdu_def_page_margin_bottom_dim { 2.54 cm }
\dim_const:Nn \c__fdu_def_page_margin_left_dim   { 3.18 cm }
\dim_const:Nn \c__fdu_def_page_margin_right_dim  { 3.18 cm }
%    \end{macrocode}
% \end{macro}
%
% \begin{macro}{\c__fdu_def_header_height_dim}
% 页眉高度。此高度与五号字大致相配。
%    \begin{macrocode}
\dim_const:Nn \c__fdu_def_header_height_dim { 15 pt }
%    \end{macrocode}
% \end{macro}
%
% \begin{macro}{\c__fdu_def_font_size_tl}
% 字号(小四)。
%    \begin{macrocode}
\tl_const:Nn \c__fdu_def_font_size_tl { -4 }
%    \end{macrocode}
% \end{macro}
%
% \begin{macro}{\c__fdu_def_line_spread_fp}
% 行距倍数。行距倍数 $k$ 由下式确定:
% \begin{equation*}
%   \num{1.2} \times k \times \SI{12}{bp} = \SI{20}{pt}。
% \end{equation*}
% 式中,\num{1.2} 是基本行距与文字大小之比,\SI{12}{bp} 是小四号字
% 的大小,\SI{20}{pt} 是行距固定值。
%    \begin{macrocode}
\fp_const:Nn \c__fdu_def_line_spread_fp
  { ( 20 pt ) / ( 12 bp ) / 1.2 }
%    \end{macrocode}
% \end{macro}
%
% \subsubsection{章节标题}
% \begin{macro}{\c__fdu_def_chapter_format_tl,
%   \c__fdu_def_section_format_tl,
%   \c__fdu_def_subsection_format_tl}
% 中文模板章节标题样式。均使用黑体。章标题居中,节与小节标题左对齐
% (但需要使用 \tn{raggedright})。
%    \begin{macrocode}
\tl_const:Nn \c__fdu_def_chapter_format_tl
  { \huge  \normalfont \sffamily \centering   }
\tl_const:Nn \c__fdu_def_section_format_tl
  { \Large \normalfont \sffamily \raggedright }
\tl_const:Nn \c__fdu_def_subsection_format_tl
  { \large \normalfont \sffamily \raggedright }
%    \end{macrocode}
% \end{macro}
%
% \begin{macro}{\c__fdu_def_chapter_format_en_tl,
%   \c__fdu_def_chapter_name_format_en_tl,
%   \c__fdu_def_chapter_title_format_en_tl,
%   \c__fdu_def_chapter_after_name_en_tl,
%   \c__fdu_def_section_format_en_tl,
%   \c__fdu_def_subsection_format_en_tl}
% 英文模板章节标题样式。均使用粗体。
%    \begin{macrocode}
\tl_const:Nn \c__fdu_def_chapter_format_en_tl { \centering }
\tl_const:Nn \c__fdu_def_chapter_name_format_en_tl
  { \LARGE \bfseries }
\tl_const:Nn \c__fdu_def_chapter_title_format_en_tl
  { \huge  \bfseries }
\tl_const:Nn \c__fdu_def_chapter_after_name_en_tl
  { \par \nobreak \vskip 10 pt }
\tl_const:Nn \c__fdu_def_section_format_en_tl
  { \Large \bfseries \raggedright }
\tl_const:Nn \c__fdu_def_subsection_format_en_tl
  { \large \bfseries \raggedright }
%    \end{macrocode}
% \end{macro}
%
% \begin{macro}{\c__fdu_def_chapter_before_sep_tl,
%   \c__fdu_def_chapter_after_sep_tl,
%   \c__fdu_def_section_before_sep_tl,
%   \c__fdu_def_section_after_sep_tl,
%   \c__fdu_def_subsection_before_sep_tl,
%   \c__fdu_def_subsection_after_sep_tl}
% 章节标题前后间距。使用 |tl| 而非 |skip|,是为了防止在没有上下文的
% 时候 |ex| 被展开成 0。之后的不少间距也是这样定义的。
%    \begin{macrocode}
\tl_const:Nn \c__fdu_def_chapter_before_sep_tl { 50 pt }
\tl_const:Nn \c__fdu_def_chapter_after_sep_tl  { 40 pt }
\tl_const:Nn \c__fdu_def_section_before_sep_tl
  { 3.5  ex plus 1   ex minus 0.2 ex }
\tl_const:Nn \c__fdu_def_section_after_sep_tl
  { 2.7  ex plus 0.5 ex }
\tl_const:Nn \c__fdu_def_subsection_before_sep_tl
  { 3.25 ex plus 1   ex minus 0.2 ex }
\tl_const:Nn \c__fdu_def_subsection_after_sep_tl
  { 2.5  ex plus 0.3 ex }
%    \end{macrocode}
% \end{macro}
%
% \begin{macro}{\c__fdu_def_chapter_toc_format_tl,
%   \c__fdu_def_section_toc_format_tl,
%   \c__fdu_def_subsection_toc_format_tl,
%   \c__fdu_def_chapter_toc_format_en_tl,
%   \c__fdu_def_section_toc_format_en_tl,
%   \c__fdu_def_subsection_toc_format_en_tl}
% 章节目录在目录中的样式。
%    \begin{macrocode}
\tl_const:Nn \c__fdu_def_chapter_toc_format_tl
  { \normalfont \sffamily }
\tl_const:Nn \c__fdu_def_section_toc_format_tl
  { }
\tl_const:Nn \c__fdu_def_subsection_toc_format_tl
  { \CJKfamily { kai } }
\tl_const:Nn \c__fdu_def_chapter_toc_format_en_tl    { \bfseries }
\tl_const:Nn \c__fdu_def_section_toc_format_en_tl    { \bfseries }
\tl_const:Nn \c__fdu_def_subsection_toc_format_en_tl { }
%    \end{macrocode}
% \end{macro}
%
% \subsubsection{封面}
% \begin{macro}{\__fdu_cover_font_size_small:,
%   \__fdu_cover_font_size_normal:,
%   \__fdu_cover_font_size_large:,
%   \__fdu_cover_font_size_huge:}
% 字号,使用固定值。这里的定义与正文字号有所不同。
%    \begin{macrocode}
\cs_new:Nn \__fdu_cover_font_size_small:  { \zihao { -5 } }
\cs_new:Nn \__fdu_cover_font_size_normal: { \zihao {  4 } }
\cs_new:Nn \__fdu_cover_font_size_large:  { \zihao { -2 } }
\cs_new:Nn \__fdu_cover_font_size_huge:   { \zihao {  2 } }
%    \end{macrocode}
% \end{macro}
%
% \begin{macro}{\c__fdu_def_cover_id_width_tl,
%   \c__fdu_def_cover_id_margin_sep_tl,
%   \c__fdu_def_cover_logo_width_tl,
%   \c__fdu_def_cover_type_width_tl,
%   \c__fdu_def_cover_title_width_tl,
%   \c__fdu_def_cover_title_en_width_tl,
%   \c__fdu_def_cover_info_left_width_tl,
%   \c__fdu_def_cover_instructors_width_tl}
% 封面中的一些长度。
%    \begin{macrocode}
\tl_const:Nn \c__fdu_def_cover_id_width_tl          { 10 em }
\tl_const:Nn \c__fdu_def_cover_id_margin_sep_tl     { -2 em }
\tl_const:Nn \c__fdu_def_cover_logo_width_tl
  { 0.5  \textwidth }
\tl_const:Nn \c__fdu_def_cover_type_width_tl
  { 0.45 \textwidth }
\tl_const:Nn \c__fdu_def_cover_title_width_tl
  { 0.9  \textwidth }
\tl_const:Nn \c__fdu_def_cover_title_en_width_tl
  { 0.9  \textwidth }
\tl_const:Nn \c__fdu_def_cover_info_left_width_tl   { 6 em }
\tl_const:Nn \c__fdu_def_cover_instructors_width_tl { 7 em }
%    \end{macrocode}
% \end{macro}
%
% \begin{macro}{\c__fdu_def_cover_v_sep_i_tl,
%   \c__fdu_def_cover_v_sep_ii_tl,
%   \c__fdu_def_cover_v_sep_iii_tl,
%   \c__fdu_def_cover_v_sep_iv_tl,
%   \c__fdu_def_cover_v_sep_v_tl,
%   \c__fdu_def_cover_v_sep_vi_tl,
%   \c__fdu_def_cover_v_sep_vii_tl,
%   \c__fdu_def_cover_v_sep_ix_tl}
% 封面中的一些垂直间距,按自上而下的顺序排列。
%    \begin{macrocode}
\tl_const:Nn \c__fdu_def_cover_v_sep_i_tl   { \stretch { 1.5 } }
\tl_const:Nn \c__fdu_def_cover_v_sep_ii_tl  { \stretch { 0.8 } }
\tl_const:Nn \c__fdu_def_cover_v_sep_iii_tl { 0.4 cm }
\tl_const:Nn \c__fdu_def_cover_v_sep_iv_tl  { \stretch { 2   } }
\tl_const:Nn \c__fdu_def_cover_v_sep_v_tl   { 0.8 cm }
\tl_const:Nn \c__fdu_def_cover_v_sep_vi_tl  { \stretch { 2.5 } }
\tl_const:Nn \c__fdu_def_cover_v_sep_vii_tl { 1 ex }
\tl_const:Nn \c__fdu_def_cover_v_sep_ix_tl  { \stretch { 1.5 } }
%    \end{macrocode}
% \end{macro}
%
% \begin{macro}{\c__fdu_def_cover_logo_file_name_tl}
% 校名 logo 文件名。
%    \begin{macrocode}
\tl_const:Nn \c__fdu_def_cover_logo_file_name_tl { Fudan_Logo.pdf }
%    \end{macrocode}
% \end{macro}
%
% \begin{macro}{\c__fdu_def_cover_title_en_line_spread_tl}
% 英文标题的行距倍数。
%    \begin{macrocode}
\tl_const:Nn \c__fdu_def_cover_title_en_line_spread_tl { 1.2 }
%    \end{macrocode}
% \end{macro}
%
% \subsubsection{声明页}
% \begin{macro}{\c__fdu_def_decl_v_sep_i_tl,
%   \c__fdu_def_decl_v_sep_ii_tl,
%   \c__fdu_def_decl_v_sep_iii_tl,
%   \c__fdu_def_decl_v_sep_iv_tl}
% 声明页中的一些垂直间距,按自上而下的顺序排列。最后一项是标题与
% 文本、文本与签名行的间距。
%    \begin{macrocode}
\tl_const:Nn \c__fdu_def_decl_v_sep_i_tl   { \stretch { 0.2 } }
\tl_const:Nn \c__fdu_def_decl_v_sep_ii_tl  { \stretch { 2.5 } }
\tl_const:Nn \c__fdu_def_decl_v_sep_iii_tl { \stretch { 2.5 } }
\tl_const:Nn \c__fdu_def_decl_v_sep_iv_tl  { 0.8 cm }
%    \end{macrocode}
% \end{macro}
%
% \begin{macro}{\c__fdu_def_decl_text_line_spread_tl}
% 声明文本的行距倍数。
%    \begin{macrocode}
\tl_const:Nn \c__fdu_def_decl_text_line_spread_tl { 1.8 }
%    \end{macrocode}
% \end{macro}
%
% \begin{macro}{\c__fdu_def_decl_sign_width_tl,
%   \c__fdu_def_decl_date_width_tl}
% 签名栏和日期栏的宽度。
%    \begin{macrocode}
\tl_const:Nn \c__fdu_def_decl_sign_width_tl { 6 em }
\tl_const:Nn \c__fdu_def_decl_date_width_tl { 5 em }
%    \end{macrocode}
% \end{macro}
%
% \begin{macro}{\c__fdu_def_originality_decl_text_tl}
% 论文独创性声明。
% ^^A 这里切换一下句号的类别码,以使代码中显示为圆圈句号。
% ^^A 这不会影响最终论文输出(最终将始终输出圆圈句号)。
% \catcode`\。=12
%    \begin{macrocode}
\tl_const:Nn \c__fdu_def_originality_decl_text_tl
  {
    本人郑重声明:所呈交的学位论文,是本人在导师的指导下,独立进行研
    究工作所取得的成果。论文中除特别标注的内容外,不包含任何其他个人
    或机构已经发表或撰写过的研究成果。对本研究做出重要贡献的个人和集
    体,均已在论文中作了明确的声明并表示了谢意。本声明的法律结果由本
    人承担。
  }
%    \end{macrocode}
% \catcode`\。 = \active
% \newcommand{。}{.}
% \end{macro}
%
% \begin{macro}{\c__fdu_def_authorization_decl_text_tl}
% 论文使用授权声明。
% \catcode`\。=12
%    \begin{macrocode}
\tl_const:Nn \c__fdu_def_authorization_decl_text_tl
  {
    本人完全了解复旦大学有关收藏和利用博士、硕士学位论文的规定,即:
    学校有权收藏、使用并向国家有关部门或机构送交论文的印刷本和电子版
    本;允许论文被查阅和借阅;学校可以公布论文的全部或部分内容,可以
    采用影印、缩印或其它复制手段保存论文。涉密学位论文在解密后遵守此
    规定。
  }
%    \end{macrocode}
% \catcode`\。 = \active
% \newcommand{。}{.}
% \end{macro}
%
% \subsubsection{杂项}
% \begin{macro}{\c__fdu_def_secret_clist}
% 三种密级。
%    \begin{macrocode}
\clist_const:Nn \c__fdu_def_secret_clist { 秘密, 机密, 绝密 }
%    \end{macrocode}
% \end{macro}
%
% \begin{macro}{\c__fdu_def_notation_arg_tl}
% 符号表默认参数。
%    \begin{macrocode}
\tl_const:Nn \c__fdu_def_notation_arg_tl { l p { 7.5 cm } }
%    \end{macrocode}
% \end{macro}
%
% \begin{macro}{\c__fdu_def_notation_line_stretch_en_tl}
% 英文模板中符号表的行间距。
%    \begin{macrocode}
\tl_const:Nn \c__fdu_def_notation_line_stretch_en_tl { 1.3 }
%    \end{macrocode}
% \end{macro}
%
% 默认名称。注意空格是忽略掉的。
%    \begin{macrocode}
\tl_const:Nn \c__fdu_def_name_secret_level_tl     { 密 \qquad 级 }
\tl_const:Nn \c__fdu_def_name_secret_star_tl      { $ \bigstar $ }
\tl_const:Nn \c__fdu_def_name_school_id_tl        { 学校代码     }
\tl_const:Nn \c__fdu_def_name_student_id_tl       { 学 \qquad 号 }
\tl_const:Nn \c__fdu_def_name_thesis_type_tl      { 博士学位论文 }
\tl_const:Nn \c__fdu_def_name_degree_type_tl      { (学术学位) }
\tl_const:Nn \c__fdu_def_name_department_tl       { 院系         }
\tl_const:Nn \c__fdu_def_name_major_tl            { 专业         }
\tl_const:Nn \c__fdu_def_name_author_tl           { 姓名         }
\tl_const:Nn \c__fdu_def_name_supervisor_tl       { 指导教师     }
\tl_const:Nn \c__fdu_def_name_date_tl             { 完成日期     }
\tl_const:Nn \c__fdu_def_name_instructors_tl      { 指导小组成员 }
\tl_const:Nn \c__fdu_def_name_toc_tl              { 目 \quad 录  }
\tl_const:Nn \c__fdu_def_name_abstract_tl         { 摘 \quad 要  }
\tl_const:Nn \c__fdu_def_name_keywords_tl         { 关键字       }
\tl_const:Nn \c__fdu_def_name_clc_tl              { 中图分类号   }
\tl_const:Nn \c__fdu_def_name_notation_tl         { 符号表       }
\tl_const:Nn \c__fdu_def_name_toc_en_tl           { Contents     }
\tl_const:Nn \c__fdu_def_name_abstract_en_tl      { Abstract     }
\tl_const:Nn \c__fdu_def_name_keywords_en_tl      { Keywords:    }
\tl_const:Nn \c__fdu_def_name_clc_en_tl           { CLC~ number: }
\tl_const:Nn \c__fdu_def_name_notation_en_tl
  { List~ of~ Symbols }
\tl_const:Nn \c__fdu_def_name_originality_decl_tl
  { 复旦大学 \\ 学位论文独创性声明   }
\tl_const:Nn \c__fdu_def_name_authorization_decl_tl
  { 复旦大学 \\ 学位论文使用授权声明 }
\tl_const:Nn \c__fdu_def_name_author_sign_tl      { 作者签名     }
\tl_const:Nn \c__fdu_def_name_supervisor_sign_tl  { 导师签名     }
\tl_const:Nn \c__fdu_def_name_sign_date_tl        { 日期         }
%</definition>
%    \end{macrocode}
%
% \subsection{用户配置文件}
% 以下是一个示例:修改论文类型为“硕士学位论文”。
%    \begin{macrocode}
%<*user>
%%
%% \tl_set:Nn \c__fdu_def_name_thesis_type_tl { 硕士学位论文 }
%</user>
%    \end{macrocode}
%
% \Finale
