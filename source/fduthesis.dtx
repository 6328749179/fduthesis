% \iffalse meta-comment
% !TeX program  = XeLaTeX
% !TeX encoding = UTF-8
%
% Copyright (C) 2017 by Xiangdong Zeng <pssysrq@163.com>
%
% This work may be distributed and/or modified under the
% conditions of the LaTeX Project Public License, either
% version 1.3c of this license or (at your option) any later
% version. The latest version of this license is in:
%
%   http://www.latex-project.org/lppl.txt
%
% and version 1.3 or later is part of all distributions of
% LaTeX version 2005/12/01 or later.
%
% This work has the LPPL maintenance status `maintained'.
%
% The Current Maintainer of this work is Xiangdong Zeng.
%
% This work consists of the file fduthesis.dtx
%          and the derived files fduthesis.ins,
%                                fduthesis.cls,
%                                fduthesis-en.cls,
%                                fduthesis.def,
%                                fduthesis-user.def,
%                                fduthesis.pdf
%                            and README.md.
%
%<*internal>
\iffalse
%</internal>
%
%<*readme>
# fduthesis

[![Build Status](https://travis-ci.org/Stone-Zeng/fduthesis.svg?branch=master)](https://travis-ci.org/Stone-Zeng/fduthesis)
[![GitHub release](https://img.shields.io/github/release/Stone-Zeng/fduthesis/all.svg)](https://github.com/Stone-Zeng/fduthesis/releases/latest)

## 欢迎使用 fduthesis - 复旦大学论文模板!

在您使用 `fduthesis` 之前,请务必仔细阅读模板文档
[fduthesis.pdf](https://github.com/Stone-Zeng/fduthesis/releases/download/v0.3/fduthesis.pdf)。
该文档也可通过如下命令生成:

    latexmk fduthesis.dtx

若需要生成模板各部件,请执行如下命令:

    xetex fduthesis.dtx

### 模板组成

- `source/`             源代码
  - `fduthesis.dtx`       模板代码、注释以及文档
  - `ctxdoc-m.cls`        模板文档样式(修改自 `ctxdoc.cls`)
  - `.latexmkrc`           latexmk 配置文件

- `test/`               测试文件
  - `fduthesis.cls`       fduthesis 模板类
  - `fduthesis-en.cls`    fduthesis 模板类(英文版)
  - `fduthesis.def`       模板参数配置文件
  - `fduthesis-user.def`  用户配置文件
  - `test.tex`            测试论文
  - `test-en.tex`         测试论文(英文版)

- `support/`            Travis CI 支持文件
  - `texlive.sh`          安装最新版本 TeX Live
  - `texlive.profile`     TeX Live 环境变量配置
  - `local.sh`            安装本地宏包及相关字体
  - `run.sh`              执行测试命令

- `logo/`               复旦大学标识
  - `fudan-name.tex`      校名
  - `fudan-emblem.tex`    校徽
  - `fudan-motto.tex`     校训

- `.gitignore`          Git 忽略文件

- `.travis.yml`         Travis CI 配置文件

- `README.md`           自述文件(本文档)

### 许可证

本模板的发布遵守 [LaTeX Project Public License](http://www.latex-project.org/lppl.txt)
(版本 1.3c 或更高)。

<br></br>

## Welcome to fduthesis - LaTeX thesis template for Fudan University!

Before you using `fduthesis`, please read the document
[fduthesis.pdf](https://github.com/Stone-Zeng/fduthesis/releases/download/v0.3/fduthesis.pdf)
carefully. This file can be generated with the following commands
as well:

    latexmk fduthesis.dtx

If you want to get all components of `fduthesis`, please excute the
following commands:

    xetex fduthesis.dtx

### License

This work may be distributed and/or modified under the conditions of
the [LaTeX Project Public License](http://www.latex-project.org/lppl.txt),
either version 1.3c of this license or (at your option) any later
version.
%</readme>
%
%<*internal>
\fi
\begingroup
  \def\NameOfLaTeXe{LaTeX2e}
\expandafter\endgroup\ifx\NameOfLaTeXe\fmtname\else
\csname fi\endcsname
%</internal>
%
%<*install>
\input l3docstrip.tex
\keepsilent
\askforoverwritefalse

\preamble

    Copyright (C) 2017 by Xiangdong Zeng <pssysrq@163.com>

    This work may be distributed and/or modified under the
    conditions of the LaTeX Project Public License, either
    version 1.3c of this license or (at your option) any later
    version. The latest version of this license is in:

      http://www.latex-project.org/lppl.txt

    and version 1.3 or later is part of all distributions of
    LaTeX version 2005/12/01 or later.

    This work has the LPPL maintenance status `maintained'.

    The Current Maintainer of this work is Xiangdong Zeng.

    This work consists of the file fduthesis.dtx
             and the derived files fduthesis.ins,
                                   fduthesis.cls,
                                   fduthesis-en.cls,
                                   fduthesis.def,
                                   fduthesis-user.def,
                                   fduthesis.pdf
                               and README.md.

\endpreamble

\generate{
  \usedir{tex/latex/fduthesis}
  \file{\jobname.cls}       {\from{\jobname.dtx}{class}}
  \file{\jobname-en.cls}    {\from{\jobname.dtx}{class-en}}
  \file{\jobname.def}       {\from{\jobname.dtx}{definition}}
  \file{\jobname-user.def}  {\from{\jobname.dtx}{user}}
  \file{\jobname-doc.cls}   {\from{\jobname.dtx}{doc}}
%</install>
%<*internal>
  \usedir{source/latex/fduthesis}
  \file{\jobname.ins}       {\from{\jobname.dtx}{install}}
%</internal>
%<*install>
  \nopreamble\nopostamble
  \usedir{doc/latex/fduthesis}
  \file{README.md}          {\from{\jobname.dtx}{readme}}
}

\obeyspaces
\Msg{*************************************************************}
\Msg{*                                                           *}
\Msg{* To finish the installation you have to move the following *}
\Msg{* files into a directory searched by TeX:                   *}
\Msg{*                                                           *}
\Msg{* The recommended directory is TDS:tex/latex/fduthesis      *}
\Msg{*                                                           *}
\Msg{*     fduthesis.cls                                         *}
\Msg{*     fduthesis-en.cls                                      *}
\Msg{*     fduthesis.def                                         *}
\Msg{*     fduthesis-user.def                                    *}
\Msg{*                                                           *}
\Msg{* To produce the documentation run the file fduthesis.dtx   *}
\Msg{* through XeLaTeX.                                          *}
\Msg{*                                                           *}
\Msg{* Happy TeXing!                                             *}
\Msg{*                                                           *}
\Msg{*************************************************************}

\endbatchfile
%</install>
%
%<*internal>
\fi
%</internal>
%
%<class|class-en|doc>\NeedsTeXFormat{LaTeX2e}
%<class|class-en|doc>\RequirePackage{expl3}
%<*!(driver|install)>
%<!readme>\GetIdInfo $Id: fduthesis.dtx 0.3 2017-07-28 12:00:00Z Xiangdong Zeng <pssysrq@163.com> $
%<class>  {Thesis template for Fudan University}
%<class>\ProvidesExplClass{\ExplFileName}
%<class-en>  {Thesis template for Fudan University (English version)}
%<class-en>\ProvidesExplClass{\ExplFileName-en}
%<definition>  {Definition file for fduthesis}
%<definition>\ProvidesExplFile{\ExplFileName.def}
%<user>  {User definition file for fduthesis}
%<user>\ProvidesExplFile{\ExplFileName-user.def}
%<doc>  {Documentation class for fduthesis}
%<doc>\ProvidesExplClass{\ExplFileName-doc}
%<!readme>  {\ExplFileDate}{\ExplFileVersion}{\ExplFileDescription}
%</!(driver|install)>
%<*driver>
\documentclass{fduthesis-doc}
\usepackage{xpinyin}
\hypersetup
  {
    pdftitle  = {fduthesis:复旦大学论文模板},
    pdfauthor = {曾祥东}
  }
\begin{document}
  % \DisableImplementation
  \DocInput{\jobname.dtx}
  \IndexLayout
  \PrintChanges
  \PrintIndex
\end{document}
%</driver>
% \fi
%
% \changes{v0.1}{2017/02/15}{开始编写模板。}
% \changes{v0.2}{2017/02/19}{使用 Git 进行版本控制,并发布至 GitHub。}
% \changes{v0.3}{2017/02/21}{使用 \cls{doc} 和 \pkg{DocStrip}。}
% \changes{v0.3}{2017/03/04}{支持 \LuaLaTeX{}。}
% \changes{v0.3}{2017/03/20}{添加测试文件。}
% \changes{v0.3}{2017/05/26}{使用 Travis CI 进行持续集成。}
% \changes{v0.3}{2017/06/23}{添加复旦大学标识。}
% \changes{v0.3}{2017/07/10}{添加英文模板。}
% \changes{v0.3}{2017/07/19}{使用 \pkg{l3docstrip} 管理名字空间。}
% \changes{v0.3}{2017/07/28}{整理代码,编写用户文档。}
%
% \CheckSum{0}
%
% \CharacterTable
%  {Upper-case    \A\B\C\D\E\F\G\H\I\J\K\L\M\N\O\P\Q\R\S\T\U\V\W\X\Y\Z
%   Lower-case    \a\b\c\d\e\f\g\h\i\j\k\l\m\n\o\p\q\r\s\t\u\v\w\x\y\z
%   Digits        \0\1\2\3\4\5\6\7\8\9
%   Exclamation   \!     Double quote  \"     Hash (number) \#
%   Dollar        \$     Percent       \%     Ampersand     \&
%   Acute accent  \'     Left paren    \(     Right paren   \)
%   Asterisk      \*     Plus          \+     Comma         \,
%   Minus         \-     Point         \.     Solidus       \/
%   Colon         \:     Semicolon     \;     Less than     \<
%   Equals        \=     Greater than  \>     Question mark \?
%   Commercial at \@     Left bracket  \[     Backslash     \\
%   Right bracket \]     Circumflex    \^     Underscore    \_
%   Grave accent  \`     Left brace    \{     Vertical bar  \|
%   Right brace   \}     Tilde         \~}
%
%^^A 禁止使用 " 符号作为抄录文本缩略符
% \DeleteShortVerb\"
%
% \title{\textcolor{MaterialIndigo800}{^^A
%   \textbf{fduthesis:复旦大学论文^^A
%     \xpinyin[font=\sffamily]{模}{mu2}板}}}
% \author{曾祥东}
% \date{\today \quad v0.3^^A
%   \thanks{\url{https://github.com/Stone-Zeng/fduthesis}.}}
%
%^^A 封面与目录的页边距
% \newgeometry{
%   left   = 1.25 in,
%   right  = 1.25 in,
%   top    = 1.25 in,
%   bottom = 1.00 in
% }
%
% \maketitle
% \vfill
% \begin{center}
%   \includegraphics[width=8cm]{../logo/fduthesis-cover.pdf}
% \end{center}
% \vfill
% \thispagestyle{plain}
% \clearpage
%
% \tableofcontents
%
% \begin{documentation}
%
%^^A 用户手册的页边距
% \newgeometry{
%   left   = 1.75 in,
%   right  = 1.00 in,
%   top    = 1.25 in,
%   bottom = 1.00 in
% }
%
% \section{介绍}
% 目前,在网上可以找到的复旦大学 \LaTeX{} 论文模板主要有以下这些:
% \begin{itemize}
%   \item 数学科学学院 2001 级的何力同学和李湛同学在 2005 年根据
%     学校要求所设计的\cls{毕业论文格式 tex04 版},以及 2008 年
%     张越同学修改之后的\cls{毕业论文格式 tex08 版},这是专为
%     数院本科生撰写毕业论文而设计的
%     \scite{数院毕业论文格式,数院毕业论文格式更新};
%   \item Pandoxie 编写的 \cls{FDU-Thesis-Latex}
%     \scite{pandoxie2014fduthesislatex},基本满足了博士(硕士)
%     毕业论文格式要求,使用人数较多;
%   \item richarddzh 编写的硕士论文模板 \cls{fudan-thesis}
%     \scite{richard2016fudanthesis};
%   \item hmshan 编写的博士论文模板 \cls{FDU_PhD_Thesis_Template}
%     \scite{hmshan2017fduphdthesistemplate}。
% \end{itemize}
% 以上这些模板大都没有经过系统的设计,也鲜有后续维护。相比之下,
% 清华大学 \scite{thuthesis}、重庆大学 \scite{cquthesis}、
% 中国科学技术大学 \scite{zepinglee2017ustcthesis} 以及友校
% 上海交通大学 \scite{weijianwen2017sjtuthesis}等,都有成熟、
% 稳定的解决方案,值得参考。
%
% 本模板将借鉴前辈经验,重新设计,并使用 \LaTeX3
% \scite{interfaces3,source3} 编写,以适应\TeX{} 技术发展潮流;
% 同时还将构建一套简洁的接口,方便用户使用。
%
% \subsection*{\LaTeX{} 入门}
% 本文档并非是一份 \LaTeX{} 零基础教程。如果您是完完全全的新手,
% 建议先阅读相关入门文档,如刘海洋编著的《\LaTeX{} 入门》
% \scite{刘海洋2013latex入门} 第一章,或大名鼎鼎的“\pkg{lshort}”
% \scite{lshort} 及其中文翻译版 \scite{lshort-zh-cn}。当然,
% 网络上的入门教程多如牛毛,您可以自行选取。
%
% \subsection*{关于本文档}
% 本文采用不同字体表示不同内容。无衬线字体表示宏包名称,如
% \pkg{xeCJK} 宏包、\cls{fduthesis} 文档类等;等宽字体表示文件名,
% 如 \TeX{} 文档 \file{thesis.tex}、测试文件夹 \file{test} 等;
% 另一种等宽字体表示代码,如 |\fdusetup| 命令、|abstract| 环境等;
% 带有尖括号的楷体(或西文斜体)表示命令参数,如 \meta{模板选项}、
% \meta{English title} 等。在使用时,参数两侧的尖括号不必输入。
% 示例代码进行了语法高亮处理,以方便阅读。
%
% 在用户手册中,带有蓝色侧边线的为 \LaTeX{} 代码,而带有粉色侧边线
% 的则为命令行代码,请注意区分。模板提供的选项、命令、环境等,
% 均用横线框起,同时给出使用语法和相关说明。
%
% 本模板中的选项、命令或环境可以分为以下三类:
% \begin{itemize}
%   \item 名字后面带有 \rexptarget\rexpstar{} 的,表示只能在^^A
%     \emph{中文模板}中使用;
%   \item 名字后面带有 \exptarget\expstar{} 的,表示只能在^^A
%     \emph{英文模板}中使用;
%   \item 名字后面不带有特殊符号的,表示既可以在中文模板中使用,
%     也可以在英文模板中使用。
% \end{itemize}
%
% 代码实现主要面向对 \LaTeX{} 宏包开发感兴趣的用户。如果您有任何改进
% 意见或者功能需求,欢迎前往 GitHub 仓库
% \href{https://github.com/Stone-Zeng/fduthesis/issues}{提交 issue}。
%
% 文档的最后还提供了版本历史和代码索引,以供用户查阅。
%
% \section{安装}
% \subsection{获取 \cls{fduthesis}}
% 本模板目前暂未上传至 CTAN,您需要从 GitHub 上自行下载并安装。
% 具体方法如下:
% \begin{itemize}
%   \item 打开 \href{https://github.com/Stone-Zeng/fduthesis}{项目主页},
%     点击“Clone or download”,并选择“Download ZIP”,下载
%     \file{fduthesis-master.zip}。
%   \item 解压 \file{fduthesis-master.zip} 后,将 \file{test}
%     文件夹下的模板文档类文件 \file{fduthesis.cls}、
%     \file{fduthesis-en.cls},参数配置文件 \file{fduthesis.def}、
%     \file{fduthesis-user.def} 以及校名图片 \file{Fudan_Logo.pdf}
%     复制到您的工作文件夹(\TeX{}论文文件所在的文件夹)中,
%     方可完成安装。
% \end{itemize}
%
% 注:这是一条缓兵之计。
%
% \subsection{模板组成}
% 本模板主要包含核心文档类、配置文件、用户文档等几个部分,
% 其具体组成见表~\ref{tab:fduthesis-components}。
%
% \begin{table}[h]
%   \caption{\cls{fduthesis} 的组成} \label{tab:fduthesis-components}
%   \centering
%   \begin{tabular}{lp{24em}}
%     \toprule
%     \textbf{文件} & \textbf{功能说明} \\
%     \midrule
%     \file{fduthesis.cls}      & 中文模板文档类 \\
%     \file{fduthesis-en.cls}   & 英文模板文档类 \\
%     \file{fduthesis.def}      & 参数配置文件,用于设定 fduthesis
%       的初始参数,不建议您自行改动 \\
%     \file{fduthesis-user.def} & 用户配置文件,可根据您的需求进行
%       修改,默认为空 \\
%     \file{fduthesis.dtx}      & 模板源文件,包含源代码、注释以及
%       文档,同时集成有安装文件 \file{fduthesis.ins} \\
%     \file{README.md}          & 简要自述 \\
%     \file{fduthesis.pdf}      & 用户手册(本文档) \\
%     \bottomrule
%   \end{tabular}
% \end{table}
%
% 模板中的绝大多数文件,均可由 \file{fduthesis.dtx} 导出。
% 在命令行中执行
% \begin{shellexample}
%   xetex fduthesis.dtx
% \end{shellexample}
% 可生成各导出文件;而要生成用户手册 \file{fduthesis.pdf},则需执行
% \begin{shellexample}
%   xelatex fduthesis.dtx
%   makeindex -s gind.ist -o fduthesis.ind fduthesis.idx
%   makeindex -s gglo.ist -o fduthesis.gls -t fduthesis.glg fduthesis.glo
%   xelatex fduthesis.dtx
%   xelatex fduthesis.dtx
% \end{shellexample}
% 也可使用 \pkg{latexmk}:
% \begin{shellexample}
%   latexmk fduthesis.dtx
% \end{shellexample}
% 本模板已经为编译用户手册提供了 \pkg{latexmk} 配置文件
% \file{.latexmkrc}。
%
% \section{使用说明}
% \subsection{基本用法}
% 以下是一份简单的 \TeX{} 文档,它演示了 \cls{fduthesis} 的最基本用法:
% \begin{latexexample}
%   % thesis.tex
%   \documentclass{fduthesis}
%   \begin{document}
%     \chapter{您好}
%     \section{Welcome to fduthesis!}
%     你好,\LaTeX{}!
%   \end{document}
% \end{latexexample}
%
% 按照 \ref{subsec:编译方式}~小节中的方式编译该文档,您应当得到一篇
% 5 页的文章。当然,这篇文章的绝大部分都是空白的。
%
% 英文模板可以用类似的方式使用:
% \begin{latexexample}
%   % thesis-en.tex
%   \documentclass{fduthesis-en}
%   \begin{document}
%     \chapter{Hello}
%     \section{Welcome to fduthesis!}
%     Hello, \LaTeX{}!
%   \end{document}
% \end{latexexample}
% 英文模板只对正文部分进行了改动,封面、指导小组成员以及声明页仍将
% 显示为中文。
%
% \subsection{编译方式} \label{subsec:编译方式}
% 本模板不支持 \pdfTeX{} 引擎,请使用 \XeLaTeX{} 或 \LuaLaTeX{}
% 编译。推荐使用 \XeLaTeX{}。
% 为了生成正确的目录、脚注以及交叉引用,您至少需要连续编译两次。
%
% 以下代码中,假设您的 \TeX{} 论文文档名为 |thesis.tex|。
% 使用 \XeLaTeX{} 编译论文,请在命令行中执行
% \begin{shellexample}
%   xelatex thesis
%   xelatex thesis
% \end{shellexample}
% 或使用 \pkg{latexmk}:
% \begin{shellexample}
%   latexmk -xelatex thesis
% \end{shellexample}
%
% 使用 \LuaLaTeX{} 编译论文,请在命令行中执行
% \begin{shellexample}
%   lualatex thesis
%   lualatex thesis
% \end{shellexample}
% 或者
% \begin{shellexample}
%   latexmk -lualatex thesis
% \end{shellexample}
%
% 需要注意,请不要把模板中的配置文件 \file{.latexmkrc} 放置在工作
% 文件夹下。该配置文件仅适用于编译模板用户手册。
%
% \subsection{模板选项}
% 所谓“模板选项”,指需要在引入文档类的时候指定的选项:
% \begin{latexexample}
%   \documentclass(*\oarg{模板选项}*){fduthesis}
% \end{latexexample}
%
% 有些模板选项为布尔型,它们只能在 |true| 和 |false| 中取值。
% 对于这些选项,\meta{选项} |= true| 中的“|= true|”可以省略。
%
% \begin{function}{oneside,twoside}
%   指明论文的单双面模式,默认为 |twoside|。该选项会影响每章开头的
%   位置,还会影响页眉样式。
% \end{function}
%
% 在双面模式(|twoside|)下,按照通常的排版惯例,每章应只从奇数页
% (在右)开始;而在单页模式(|oneside|)下,则可以从任意页面开始。
% 本模板中,目录、摘要、符号表等均视作章,也按相同方式排版。
%
% 双面模式下,正文部分偶数页(在左)的左页眉显示章标题,奇数页
% (在右)的右页眉显示节标题;前置部分的页眉按同样格式显示,但文字
% 均为对应标题(如“{\CJKfamily{楷}目\textvisiblespace{}录}”、
% “{\CJKfamily{楷}摘\textvisiblespace{}要}”等)。
% 而在单面模式下,正文部分则页面不分奇偶,均同时显示左、右页眉,
% 文字分别为章标题和节标题;前置部分只有中间页眉,显示对应标题。
%
% \begin{function}{nofonts}
%   \begin{fdusyntax}
%     nofonts = (*<\TFF>*)
%   \end{fdusyntax}
%   选择是否禁用默认字体设置,默认关闭。
% \end{function}
%
% 禁用默认字体设置后,中文将无法正常显示,而西文则会按照 \LaTeX{}
% 的默认配置采用 Computer Modern 字体系列 \footnote{^^A
%   使用 \XeLaTeX{} 或 \LuaLaTeX{} 编译时,实际使用 Latin Modern
%   作为正文字体,数学字体仍使用 Computer Modern。本模板加载了
%   \pkg{unicode-math} 宏包,此时默认数学字体为 Latin Modern Math。}。
% 此时,您需要使用 \pkg{fontspec}、\pkg{xeCJK}、\pkg{unicode-math}
% 等宏包提供的 \cs{setmainfont}、\cs{setmainCJKfont}、
% \cs{setmathfont} 等命令来配置字体。
%
% \begin{function}{draft}
%   \begin{fdusyntax}
%     draft = (*<\TFF>*)
%   \end{fdusyntax}
%   选择是否开启草稿模式,默认关闭。
% \end{function}
%
% 草稿模式为全局选项,会影响到很多宏包的工作方式。
% 开启之后,主要的变化有:
% \begin{itemize}
%   \item 把行溢出的盒子显示为黑色方块;
%   \item 不实际插入图片,只输出一个占位方框;
%   \item 关闭超链接渲染,也不添加 PDF 书签(还未调用\pkg{hyperref},
%     目前暂时不受影响);
%   \item 显示页面边框。
% \end{itemize}
%
% \subsection{参数设置}
% 本模板提供了一系列选项,可由用户自行配置。以下所有选项均可通过
% 统一的命令 \cs{fdusetup} 来设置。
%
% \begin{function}{\fdusetup}
%   \begin{fdusyntax}
%     \fdusetup(*\marg{键值列表}*)
%   \end{fdusyntax}
%   这是本模板的通用控制命令,用来在载入文档类之后实现各种功能,
%   如修改论文格式、录入论文信息等。
% \end{function}
%
% \cs{fdusetup} 的参数是一组由(英文)逗号隔开的选项列表,列表中的
% 选项通常是 \meta{key} |=| \meta{value} 的形式。部分选项的
% \meta{value}可以省略。对于同一项,后面的设置将会覆盖前面的设置。
% 在下文的说明中,将用\textbf{粗体}表示默认值。
%
% \cs{fdusetup} 采用 \LaTeX3 风格的键值设置,支持不同类型以及多种
% 层次的选项设定。键值列表中,“|=|”左右的空格不影响设置;但需注意,
% 参数列表中不可以出现空行。
%
% 与模板选项相同,布尔型的参数可以省略 \meta{key} |= true| 中的
% “|= true|”。
%
% 另有一些选项包含子选项,如 |style| 和 |info| 等。它们可以按如下
% 两种等价方式来设定:
% \begin{latexexample}
%   \fdusetup{
%     style = {font = adobe, fontsize = -4},
%     info  = {
%       author     = {张三},
%       department = {物理系},
%       title      = {论如何使用 fduthesis 写好论文}
%     }
%   }
% \end{latexexample}
% 或者
% \begin{latexexample}
%   \fdusetup{
%     style / font       = adobe,
%     style / fontsize   = -4,
%     info  / author     = {张三},
%     info  / department = {物理系},
%     info  / title      = {论如何使用 fduthesis 写好论文}
%   }
% \end{latexexample}
%
% 在第二种方式中,“|/|” 与 “|=|” 类似,前后的空白对设置没有影响。
%
% \subsubsection{论文格式} \label{subsubsec:论文格式}
% \begin{function}{style}
%   \begin{fdusyntax}
%     style = (*\marg{键值列表}*)
%     style / (*\meta{key}*) = (*\meta{value}*)
%   \end{fdusyntax}
%   设置论文格式的通用选项,具体内容见下。
% \end{function}
%
% \begin{function}{style/font}
%   \begin{fdusyntax}
%     font = (*<libertinus|lm|palatino|(times)>*)
%   \end{fdusyntax}
%   设置西文字体(包括数学字体)。
% \end{function}
%
% \begin{function}[rEXP]{style/cjkfont}
%   \begin{fdusyntax}
%     cjkfont = (*<adobe|(fandol)|founder|linux|mac|windows>*)
%   \end{fdusyntax}
%   设置中文字体。
% \end{function}
%
% \begin{function}{style/fontsize}
%   \begin{fdusyntax}
%     fontsize = (*<(-4)|5>*)
%   \end{fdusyntax}
%   设置论文的基础字号。
% \end{function}
%
% \begin{function}[rEXP]{style/fullwidthstop}
%   \begin{fdusyntax}
%     fullwidthstop = (*<\TFF>*)
%   \end{fdusyntax}
%   选择是否把全角实心句点“\symbol{"FF0E}”作为默认的句号形状。
%   这种句号一般用于科技类文章,以便与下标 o 或 0 区分。
% \end{function}
%
% \begin{function}{style/footnotestyle}
% ^^A 这里奇怪的东西是用来控制对齐的。fdusyntax 会吃掉开头的几个
% ^^A 空格,因此这里用 X 来占位。
%   \begin{fdusyntax}
%     footnotestyle = (*<plain|\\
%       XXXXXX\mbox{}~~~~~~~~~~~~~~~~libertinus|libertinus*|libertinus-sans|\\
%       XXXXXX\mbox{}~~~~~~~~~~~~~~~~pifont|pifont*|pifont-sans|pifont-sans*|\\
%       XXXXXX\mbox{}~~~~~~~~~~~~~~~~xits|xits-sans|xits-sans*>*)
%   \end{fdusyntax}
%   设置脚注编号样式。西文字体设置会影响其默认取值(见
%   表~\ref{tab:footnote-font})。因此,要使得该选项生效,需将其放置
%   在 |font| 选项之后。带有 |sans| 的为相应的无衬线字体版本;带有
%   |*| 的为阴文样式(即黑底白字)。
% \end{function}
%
% \begin{table}[h]
% \caption{西文字体与脚注编号样式默认值的对应关系}
% \label{tab:footnote-font}
% \centering
% \begin{tabular}{ccccc}
%   \toprule
%   \textbf{西文字体设置} &
%     |libertinus| & |lm|     & |palatino| & |times| \\
%   \midrule
%   \textbf{脚注编号样式默认值} &
%     |libertinus| & |pifont| & |pifont|   & |xits|  \\
%   \bottomrule
% \end{tabular}
% \end{table}
%
% \begin{function}{style/automakecover}
%   \begin{fdusyntax}
%     automakecover = (*<\TTF>*)
%   \end{fdusyntax}
%   是否自动生成论文封面(封一)、指导小组成员名单(封二)和声明页
%   (封三)。封面中的各项信息,可通过 \cs{fdusetup} 录入,具体请参阅
%   \ref{subsubsec:信息录入}~节。
% \end{function}
%
% \begin{function}{\makecoveri,\makecoverii,\makecoveriii}
%   用于\emph{手动}生成论文封面、指导小组成员名单和声明页。这几个
%   命令不能确保页码的正确编排,因此除非必要,您应当始终使用自动生成
%   的封面。
% \end{function}
%
% \subsubsection{信息录入} \label{subsubsec:信息录入}
% \begin{function}{info}
%   \begin{fdusyntax}
%     info = (*\marg{键值列表}*)
%     info / (*\meta{key}*) = (*\meta{value}*)
%   \end{fdusyntax}
%   录入论文信息的通用选项,具体内容见下。以下各选项中,带“|*|”的为
%   对应的英文字段。
% \end{function}
%
% \begin{function}{info/title,info/title*}
%   \begin{fdusyntax}
%     title  = (*\marg{中文标题}*)
%     title* = (*\marg{English title}*)
%   \end{fdusyntax}
%   论文标题。默认会在约 20 个汉字字宽处强制断行,但为了语义的连贯
%   以及排版的美观,如果您的标题长于一行,建议使用“|\\|”手动断行。
% \end{function}
%
% \begin{function}{info/author,info/author*}
%   \begin{fdusyntax}
%     author  = (*\marg{姓名}*)
%     author* = (*\marg{English name}*)
%   \end{fdusyntax}
%   作者姓名。
% \end{function}
%
% \begin{function}{info/supervisor}
%   \begin{fdusyntax}
%     supervisor = (*\marg{姓名}*)
%   \end{fdusyntax}
%   导师姓名。
% \end{function}
%
% \begin{function}{info/department}
%   \begin{fdusyntax}
%     department = (*\marg{名称}*)
%   \end{fdusyntax}
%   院系名称。
% \end{function}
%
% \begin{function}{info/major}
%   \begin{fdusyntax}
%     major = (*\marg{名称}*)
%   \end{fdusyntax}
%   专业名称。
% \end{function}
%
% \begin{function}{info/studentid}
%   \begin{fdusyntax}
%     studentid = (*\marg{数字}*)
%   \end{fdusyntax}
%   作者学号。
% \end{function}
%
% 复旦大学学号共 11 位,前两位为入学年份,之后一位为学生类型代码
% (本科生为 1,硕士生为 2,博士生为 3),接下来的五位为专业代码,
% 最后三位为顺序号。
%
% \begin{function}{info/schoolid}
%   \begin{fdusyntax}
%     schoolid = (*\marg{数字}*)
%   \end{fdusyntax}
%   学校代码。默认值为 10246(这是复旦大学的学校代码)。
% \end{function}
%
% \begin{function}{info/date}
%   \begin{fdusyntax}
%     date = (*\marg{日期}*)
%   \end{fdusyntax}
%   论文完成日期。默认值为文档编译日期(\tn{today})。
% \end{function}
%
% \begin{function}{info/secretlevel}
%   \begin{fdusyntax}
%     secretlevel = (*<(none)|i|ii|iii>*)
%   \end{fdusyntax}
%   密级。|i|、|ii|、|iii| 分别表示秘密、机密、绝密;|none| 表示论文不
%   涉密,即不显示密级与保密年限。
% \end{function}
%
% \begin{function}{info/secretyear}
%   \begin{fdusyntax}
%     secretyear = (*\marg{年限}*)
%   \end{fdusyntax}
%   保密年限。建议您使用中文,如“五年”。此选项在 |secretlevel = none|
%   时无效。
% \end{function}
%
% \begin{function}{info/instructors}
%   \begin{fdusyntax}
%     instructors = (*\marg{成员 1, 成员 2, ...}*)
%   \end{fdusyntax}
%   指导小组成员。各成员之间需使用英文逗号隔开。为防止歧义,可以用
%   分组括号“|{...}|”把各成员字段括起来。
% \end{function}
%
% \begin{function}{info/keywords,info/keywords*}
%   \begin{fdusyntax}
%     keywords  = (*\marg{中文关键字}*)
%     keywords* = (*\marg{English keywords}*)
%   \end{fdusyntax}
%   关键字。
% \end{function}
%
% \begin{function}{info/clc}
%   \begin{fdusyntax}
%     clc = (*\marg{分类号}*)
%   \end{fdusyntax}
%   中国图书馆分类号(CLC)。
% \end{function}
%
% \subsection{正文编写}
% \begin{fduquote}[喬孟符][宋]
%   作樂府亦有法,曰鳳頭豬肚豹尾六字是也。大概起要美麗,
%   中要浩蕩,結要響亮。尤貴在首尾貫穿,意思清新。茍能若是,
%   斯可以言樂府矣。
% \end{fduquote}
%
% \subsubsection{凤头}
% \begin{function}{\frontmatter}
%   声明前置部分开始。
% \end{function}
%
% 在本模板中,前置部分包含目录、中英文摘要以及符号表等。前置部分的
% 页码采用小写罗马字母,并且与正文分开计数。
%
% \begin{function}{\tableofcontents}
%   生成目录。为了生成完整、正确的目录,您至少需要编译\emph{两次}。
% \end{function}
%
% \begin{function}{abstract}
%   \begin{fdusyntax}
%     % 中文论文                  % 英文论文
%     \begin{abstract}            \begin{abstract}
%       (*\meta{中文摘要} \hspace{2.86cm} \meta{Abstract}*)
%     \end{abstract}              \end{abstract}
%   \end{fdusyntax}
% \end{function}
% \begin{function}[rEXP]{abstract*}
%   \begin{fdusyntax}
%     % 中文论文
%     \begin{abstract*}
%       (*\meta{English abstract}*)
%     \end{abstract*}
%   \end{fdusyntax}
%   摘要。中文模板中,不带星号和带星号的版本分别用来输入中文摘要
%   和英文摘要;英文模板中没有带星号的版本,您只需输入英文摘要。
% \end{function}
%
% 摘要的最后,会显示关键字列表以及中国图书馆分类号(CLC)。
% 这两项可通过 \cs{fdusetup} 录入,具体
% 请参阅 \ref{subsubsec:信息录入}~节。
%
% \begin{function}{notation}
%   \begin{fdusyntax}
%     \begin{notation}(*\oarg{列格式说明}*)
%       (*\meta{符号 1}*)  &  (*\meta{说明}*)  \\
%       (*\meta{符号 2}*)  &  (*\meta{说明}*)  \\
%               (*$\vdots$*)
%       (*\meta{符号 $n$}*)  &  (*\meta{说明}*)
%     \end{notation}
%   \end{fdusyntax}
%   符号表。可选参数“列格式说明”与 \LaTeX 中标准表格的列格式说明
%   语法一致,默认值为“|l p{7.5 cm}|”,即第一列宽度自动调整,第二列
%   限宽 \SI{7.5}{cm},两列均为左对齐。
% \end{function}
%
%^^A \subsubsection{猪肚}
%^^A \begin{function}{\mainmatter}
%^^A   声明主体部分开始。
%^^A \end{function}
%^^A
%^^A 主体部分是论文的核心,您可以分章节撰写。如有需求,也可以采用
%^^A 多文件编译的方式。主体部分的页码采用阿拉伯数字。
%^^A
%^^A \begin{function}{\footnote}
%^^A   \begin{fdusyntax}
%^^A     \footnote(*\marg{脚注文字}*)
%^^A   \end{fdusyntax}
%^^A   插入脚注。脚注编号样式可利用 |style/footnotestyle| 选项控制,
%^^A   具体见 \ref{subsubsec:论文格式}~小节。
%^^A \end{function}
%^^A
%^^A \begin{function}{axiom,corollary,definition,example,lemma,proof,theorem}
%^^A   \begin{fdusyntax}
%^^A     \begin{proof}
%^^A       (*\meta{证明过程}*)
%^^A     \end{proof}
%^^A   \end{fdusyntax}
%^^A   一系列预定义的数学环境。具体含义见表~\ref{tab:theorem}。
%^^A \end{function}
%^^A
%^^A \begin{table}[h]
%^^A   \caption{预定义的数学环境} \label{tab:theorem}
%^^A   \begin{tabular}{cccccccc}
%^^A     \toprule
%^^A     \textbf{环境名称} &
%^^A       |axiom| & |corollary| & |definition| & |example| &
%^^A       |lemma| & |proof| & |theorem| \\
%^^A     \midrule
%^^A     \textbf{含义} &
%^^A       公理 & 推论 & 定义 & 例 & 引理 & 证明 & 定理 \\
%^^A     \bottomrule
%^^A   \end{tabular}
%^^A \end{table}
%^^A
%^^A 证明环境(|proof|)的最后会添加证毕符号“$\QED$”。要确保该符号
%^^A 正确显示,您需要按照 \ref{subsec:编译方式}~节中的有关说明
%^^A 编译\emph{两次}。
%^^A
%^^A \begin{function}{\caption}
%^^A   \begin{fdusyntax}
%^^A     \caption(*\marg{图表标题}*)
%^^A     \caption(*\oarg{短标题}\marg{长标题}*)
%^^A   \end{fdusyntax}
%^^A   插入图表标题。可选参数 \meta{短标题} 用于图表目录。在
%^^A   \meta{长标题} 中,您可以进行长达多段的叙述;但 \meta{短标题}
%^^A   和单独的 \meta{图表标题} 中则不允许分段。
%^^A   \scite{刘海洋2013latex入门}
%^^A \end{function}
%^^A
%^^A 按照排版惯例,建议您将表格的标题放置在绘制表格的命令之前,
%^^A 而将图片的标题放置在绘图或插图的命令之后。另需注意,
%^^A \tn{caption} 命令必须放置在浮动体环境(如 |table| 和 |figure|)
%^^A 中。
%^^A
%^^A \subsubsection{豹尾}
%^^A \begin{function}{\backmatter}
%^^A   声明后置部分开始。
%^^A \end{function}
%^^A
%^^A 后置部分包含声明页。目前无需开启该部分。
%^^A
%^^A \section{宏包依赖情况}
%^^A 使用不同编译方式、指定不同选项,会导致宏包依赖情况有所不同。
%^^A 具体如下:
%^^A \begin{itemize}
%^^A   \item 在任何情况下,本模板都会\emph{显式}调用以下宏包
%^^A     (或文档类):
%^^A     \begin{itemize}
%^^A       \item \pkg{expl3}、\pkg{xparse} 和 \pkg{l3keys2e} 宏包,
%^^A         用于构建 \LaTeX3 编程环境 \scite{interfaces3,source3}。
%^^A         它们分属 \pkg{l3kernel} 和 \pkg{l3packages} 宏集。
%^^A       \item \cls{book} 文档类,是 \LaTeXe{} 的标准文档类之一
%^^A         \scite{source2e}。
%^^A       \item \pkg{fontspec} 宏包,提供新一代的字体设置界面。
%^^A       \item \pkg{ctex} 宏包,提供中文排版的通用框架。属于
%^^A         \CTeX{} 宏集 \scite{CTeX}。
%^^A       \item \pkg{amsmath} 宏包,对 \LaTeX{} 的数学排版功能进行了
%^^A         全面扩展。属于 \AmSLaTeX{} 套件。
%^^A       \item \pkg{unicode-math} 宏包,负责处理 Unicode 编码的
%^^A         OpenType 数学字体。
%^^A       \item \pkg{geometry} 宏包,用于调整页面尺寸。
%^^A       \item \pkg{fancyhdr} 宏包,处理页眉页脚。
%^^A       \item \pkg{footmisc} 宏包,处理脚注。
%^^A       \item \pkg{ntheorem} 宏包,提供增强版的定理类环境。
%^^A       \item \pkg{graphicx} 宏包,提供图形插入的接口。
%^^A       \item \pkg{longtable} 宏包,长表格(允许跨页)支持。
%^^A       \item \pkg{caption} 宏包,用于设置题注。
%^^A       \item \pkg{ulem} 宏包,用于绘制下划线。
%^^A     \end{itemize}
%^^A   \item 开启 |style/footnotestyle=pifont| 选项后,会调用
%^^A     \pkg{pifont}宏包。它属于 \pkg{psnfss} 套件。
%^^A   \item 文字绕排功能需要启用 \pkg{xgalley} 宏包,它属于
%^^A     \pkg{l3experimental} 宏集。(目前暂未使用)
%^^A \end{itemize}
%^^A
%^^A 这里只列出了本模板直接调用的宏包。这些宏包自身的调用情况,
%^^A 此处不再具体展开。如有需要,请参阅相关文档。
%^^A
%^^A % \iffalse meta-comment
%
% Copyright (C) 2017, 2018 by Xiangdong Zeng <xdzeng96@gmail.com>
%
% This work may be distributed and/or modified under the
% conditions of the LaTeX Project Public License, either
% version 1.3c of this license or (at your option) any later
% version. The latest version of this license is in:
%
%   http://www.latex-project.org/lppl.txt
%
% and version 1.3 or later is part of all distributions of
% LaTeX version 2005/12/01 or later.
%
% This work has the LPPL maintenance status `maintained'.
%
% The Current Maintainer of this work is Xiangdong Zeng.
%
% \fi
%
% \EnableImplementation
%
% \begin{implementation}
%
%^^A 代码部分的页边距
% \newgeometry{
%   left      = 2.25 in,
%   right     = 1.00 in,
%   top       = 1.25 in,
%   bottom    = 1.00 in,
%   marginpar = 2.25 in
% }
%
% \subsection{模板文档样式 \cls{fdudoc}}
%
% \changes{v0.4}{2017/07/29}{在 \cls{ctxdoc} 的基础上完成
%   \cls{fdudoc} 文档类,用于模板手册的编写。}
%
% 编写 \LaTeX{} 宏包文档,传统上会采用 \pkg{doc} 宏包或
% \cls{ltxdoc} 文档类。而对于使用 \LaTeX3 开发的宏包,\cls{l3doc}
% 文档类将是一个更好的选择。\CTeX{} 宏集所附带的
% \href{https://github.com/CTeX-org/ctex-kit/blob/master/tool/ctxdoc.cls}^^A
% {\cls{ctxdoc}} 文档类,则在 \cls{l3doc} 的基础上进行了一些修正,
% 特别是重新实现了 \env{macrocode} 环境,使之能更好地应用于中文
% 文档。\cls{ctxdoc} 的主要功能如下:
%
% \begin{itemize}
%   \item 注释使用灰色、倾斜字体,以便与一般代码区分;
%   \item 模块、名字空间等使用彩色标注,并添加超链接;
%   \item 自动更新行号宽度;
%   \item 边注中的长命令使用盒子进行缩放,防止溢出;
%   \item 修订记录中将显示修改日期;
%   \item 添加中文支持。
% \end{itemize}
%
% 然而,\cls{ctxdoc} 主要供内部使用,代码较为混乱和随意。
% 本模板的文档样式(\cls{fdudoc})为适应具体需求,对其代码
% 进行了整理,添加了相关注释,并做了一些改动:
%
% \begin{itemize}
%   \item 允许模块标记 |<*|\meta{module}|>| 和
%     |</|\meta{module}|>| 出现在行号左侧;
%   \item 不再以直立、倾斜字体区分不同嵌套层次的模块;
%   \item 调整索引中版本号的排序方式;
%   \item 新增一系列实用命令;
%   \item 修改文档字体、配色等。
% \end{itemize}
%
% 本文档样式的核心代码主要来自 \cls{ctxdoc} 文档类 v2.4.10。
% 在此,本人要向原开发者
% \href{https://github.com/CTeX-org/}{CTEX.ORG}
% 团队表示由衷的感谢。
%
% 以下为 \cls{fdudoc} 的具体实现。
%
% \subsubsection{载入宏包、文档类}
%
%    \begin{macrocode}
%<*doc>
\ExplSyntaxOff
%    \end{macrocode}
%
% 无需载入 \pkg{thumbpdf}。
%    \begin{macrocode}
\@namedef{ver@thumbpdf.sty}{9999/99/99}
%    \end{macrocode}
%
% 关闭 \pkg{xparse} 中的命令声明信息。
%    \begin{macrocode}
\PassOptionsToPackage{log-declarations = false}{xparse}
%    \end{macrocode}
%
% 载入宏包和文档类。
%    \begin{macrocode}
\LoadClass[a4paper, full]{l3doc}
\RequirePackage[UTF8, heading, sub3section, fontset = none]{ctex}
%    \end{macrocode}
% 与 \pkg{hypdoc}(由 \cls{l3doc} 调用)冲突,导致脚注超链接
% 无法正常跳转。临时禁用。
%    \begin{macrocode}
% \RequirePackage[stable, bottom]{footmisc}
\RequirePackage{caption}
% \RequirePackage[showframe]{geometry}
\RequirePackage{geometry}
\RequirePackage{listings}
\RequirePackage{makecell}
\RequirePackage[toc]{multitoc}
\RequirePackage{siunitx}
\RequirePackage{tabularx}
\RequirePackage{threeparttable}
\RequirePackage{unicode-math}
\RequirePackage{xcolor}
\RequirePackage{xcolor-material}
\RequirePackage{zref-base}
%    \end{macrocode}
%
% \subsubsection{\env{macrocode} 环境}
%
% \paragraph{继承的代码}
%
% \begin{macro}[int]{\macro@code}
% 在 \pkg{doc} 宏包中,\env{macrocode} 环境的核心功能由命令
% \tn{macro@code} 负责实现,而 \tn{xmacro@code} 只用来结束
% \env{macrocode} 环境。但在 \cls{l3doc} 以及 \cls{ctxdoc} 中,
% \tn{xmacro\-@\-code} 则基本接管了 \tn{macro@code} 的功能。
% 后者此时只起辅助作用。
%    \begin{macrocode}
\def\macro@code{%
%    \end{macrocode}
% 调整前后间距,禁止 \env{macrocode} 环境前的分页。
%    \begin{macrocode}
   \topsep \MacrocodeTopsep
   \@beginparpenalty \predisplaypenalty
%    \end{macrocode}
% 将列表前后的附加垂直空白设为 0。根据 \cls{ctxdoc} 修改。
%    \begin{macrocode}
   \partopsep \z@skip
%    \if@inlabel\leavevmode\fi
%    \end{macrocode}
% 构建 \env{trivlist} 环境,设置段间距为 0。
% 之后修改字体,并调节左右间距。\tn{MacroIndent} 会根据代码行数
% 更新,具体细节见后文。
% \tn{macro@font} 用来在不同模块见切换字体。本文档类不使用
% \tn{AltMacroFont},因此这里改用 \tn{MacroFont} 代替。
%    \begin{macrocode}
   \trivlist \parskip \z@ \item[]%
%    \macro@font
   \MacroFont
   \leftskip\@totalleftmargin \advance\leftskip\MacroIndent
   \rightskip\z@ \parindent\z@ \parfillskip\@flushglue
%    \end{macrocode}
% 按照 \LaTeXe{} 中 \tn{verbatim} 环境中定义 \tn{par},使得空行
% 可以原样输出,否则空行会被吃掉。
%    \begin{macrocode}
   \blank@linefalse \def\par{\ifblank@line
                             \leavevmode\fi
                             \blank@linetrue\@@par
                             \penalty\interlinepenalty}
%    \end{macrocode}
% \tn{obeylines} 将把回车符 |^^M| 变成 \tn{par}。
% 接下来将所有特殊符号的类别码设为 12,即“其他”类。
%    \begin{macrocode}
   \obeylines
   \let\do\do@noligs \verbatim@nolig@list
   \let\do\@makeother \dospecials
%    \end{macrocode}
% 相当于退出 |\begin{list}| 和 |\begin{minipage}|。
%    \begin{macrocode}
   \global\@newlistfalse
   \global\@minipagefalse
%    \end{macrocode}
% 初始化交叉引用功能。
%    \begin{macrocode}
   \init@crossref}
%    \end{macrocode}
% \end{macro}
%
%    \begin{macrocode}
%<@@=fdudoc>
\ExplSyntaxOn
%    \end{macrocode}
%
% \begin{variable}{\l_@@_tmpa_tl,\l_@@_tmpb_tl}
% 临时变量。
%    \begin{macrocode}
\tl_new:N \l_@@_tmpa_tl
\tl_new:N \l_@@_tmpb_tl
%    \end{macrocode}
% \end{variable}
%
% \begin{macro}{\@@_patch_cmd:Nnn,\@@_preto_cmd:Nn,\@@_appto_cmd:Nn}
% 补丁工具。
%    \begin{macrocode}
\cs_new_protected:Npn \@@_patch_cmd:Nnn #1#2#3
  {
    \ctex_patch_cmd_once:NnnnTF #1 { } {#2} {#3}
      { } { \ctex_patch_failure:N #1 }
  }
\cs_new_protected:Npn \@@_preto_cmd:Nn #1#2
  {
    \ctex_preto_cmd:NnnTF #1 { } {#2}
      { } { \ctex_patch_failure:N #1 }
  }
\cs_new_protected:Npn \@@_appto_cmd:Nn #1#2
  {
    \ctex_appto_cmd:NnnTF #1 { } {#2}
      { } { \ctex_patch_failure:N #1 }
  }
%    \end{macrocode}
% \end{macro}
%
% \paragraph{代码行处理}
%
% \begin{macro}[int]{\xmacro@code,\sxmacro@code}
% 重新实现 \env{macrocode} 与 \env{macrocode*} 环境的核心功能,
% 将对代码逐行处理。后者会将空格显示为“\textvisiblespace”。
%    \begin{macrocode}
\cs_set_protected_nopar:Npn \xmacro@code
  { \@@_marco_code:w }
\cs_set_protected_nopar:Npn \sxmacro@code
  {
    \fontspec_print_visible_spaces:
    \xmacro@code
  }
%    \end{macrocode}
% \end{macro}
%
% \begin{macro}{\@@_marco_code:w}
%    \begin{macrocode}
\cs_new_protected_nopar:Npn \@@_marco_code:w
  {
%    \end{macrocode}
% 根据 \tn{codeline@index} 是否为 |true| 选择是否显示行号。
%    \begin{macrocode}
    \ifcodeline@index
      \@@_marco_code_every_par:n { \@@_code_line_no: }
    \else:
      \@@_marco_code_every_par:n { }
    \fi:
%    \end{macrocode}
% 设置代码段结束标记为“\verb*|%    \end{macrocode}^^M|”。
%    \begin{macrocode}
    \@@_make_finish_tag:x { \@currenvir }
%    \end{macrocode}
% 开始 \env{macrocode}。
%    \begin{macrocode}
    \@@_macro_code_start:w
  }
%    \end{macrocode}
% \end{macro}
%
% \begin{macro}{\@@_marco_code_every_par:n}
% 在每段之前插入内容。这里每段即相当于每行。
%    \begin{macrocode}
\cs_new_protected:Npn \@@_marco_code_every_par:n #1
  {
    \everypar
      {
        \everypar {#1}
        \if@inlabel
          \global \@inlabelfalse \@noparlistfalse
          \llap { \box \@labels \hskip \leftskip }
        \fi
        #1
      }
  }
%    \end{macrocode}
% \end{macro}
%
% 设置 \tn{endlinechar} 为 $-1$,表示行末不插入任何字符
% (实际上相当于在行尾插入注释符 |%|)。
%    \begin{macrocode}
\group_begin:
  \int_set:Nn \tex_endlinechar:D { -1 }
%    \end{macrocode}
%
% \begin{variable}{\c_@@_active_space_tl}
% 活动字符类的空格(ASCII 码为 32)。
%    \begin{macrocode}
  \use:n
    {
      \char_set_catcode_active:n { 32 }
      \tl_const:Nn \c_@@_active_space_tl
    }
    { }
\group_end:
%    \end{macrocode}
% \end{variable}
%
% ASCII 码 13 是回车符 |^^M|。将其设置为活动字符。
%    \begin{macrocode}
\group_begin:
  \char_set_catcode_active:n { 13 }
%    \end{macrocode}
%
% \begin{macro}{\@@_make_finish_tag:n,\@@_make_finish_tag:x}
% \env{macrocode} 结尾标记。展开后变成
% “\verb*|%    \end{#1}^^M|”。
%    \begin{macrocode}
  \cs_new_protected:Npx \@@_make_finish_tag:n #1
    {
      \tl_set:Nn \exp_not:N \l_@@_macro_code_finish_tl
        {
          \c_percent_str
          \prg_replicate:nn { 4 }
            { \exp_not:o { \c_@@_active_space_tl } }
          \exp_not:o { \active@escape@char } end
          \c_left_brace_str #1 \c_right_brace_str
          \exp_not:N ^^M
        }
    }
  \cs_generate_variant:Nn \@@_make_finish_tag:n { x }
%    \end{macrocode}
% \end{macro}
%
% \begin{macro}{\@@_macro_code_start:w}
% 开始代码环境。此命令主要是为了防止 |\begin{macrocode}|
% 后出现多余的空行。
%    \begin{macrocode}
  \cs_new_protected:Npn \@@_macro_code_start:w #1
    {
      \str_if_eq:nnTF {#1} { ^^M }
        { \@@_macro_code_read_line:w }
        { \@@_macro_code_read_line:w #1 }
    }
%    \end{macrocode}
% \end{macro}
%
% \begin{macro}{\@@_macro_code_read_line:w}
% 逐行读取代码,并连同行尾回车符一并存入
% \cs{l_@@_macro_code_line_tl}。如果该行与结束标记
% “\verb*|%    \end{macrocode}^^M|”相同,则结束此
% \env{macrocode};否则继续处理该行代码。
%    \begin{macrocode}
  \cs_new_protected:Npn \@@_macro_code_read_line:w #1 ^^M
    {
      \tl_set:Nn \l_@@_macro_code_line_tl { #1 ^^M }
      \tl_if_eq:NNTF
        \l_@@_macro_code_line_tl \l_@@_macro_code_finish_tl
        { \exp_args:Nx \end { \@currenvir } }
        {
          \@@_macro_code_process_line:
          \@@_macro_code_read_line:w
        }
    }
%    \end{macrocode}
% \end{macro}
%
% \changes{v0.4}{2017/08/09}{[\pkg{fdudoc}] 修复 \cls{ctxdoc}
%   文档类 v2.4.10 之前版本中行距偏小的问题,见 ctex-kit
%   \href{https://github.com/CTeX-org/ctex-kit/issues/303}{\#~303}。}
%
% \begin{macro}{\@@_swap_cr:,\@@_swap_cr_aux:w}
% 把 \cs{l_@@_macro_code_line_tl} 中的回车符 |^^M| 挪到外面。
%    \begin{macrocode}
  \cs_new_protected:Npn \@@_swap_cr:
    {
      \exp_after:wN
        \@@_swap_cr_aux:w \l_@@_macro_code_line_tl
    }
  \cs_new_protected:Npn \@@_swap_cr_aux:w #1 ^^M
    {
      \group_insert_after:N ^^M
      \tl_set:Nn \l_@@_macro_code_line_tl {#1}
    }
%    \end{macrocode}
% \end{macro}
%
% \begin{variable}{\c_@@_active_cr_tl}
% 活动字符类的回车符。
%    \begin{macrocode}
  \tl_const:Nn \c_@@_active_cr_tl { ^^M }
\group_end:
%    \end{macrocode}
% \end{variable}
%
% \begin{variable}{\l_@@_macro_code_line_tl,
%   \l_@@_macro_code_finish_tl,
%   \g_@@_macro_code_verbatim_stop_tl}
% 分别用来存储代码行、\env{macrocode} 结束标记以及抄录停止标记。
%    \begin{macrocode}
\tl_new:N \l_@@_macro_code_line_tl
\tl_new:N \l_@@_macro_code_finish_tl
\tl_new:N \g_@@_macro_code_verbatim_stop_tl
%    \end{macrocode}
% \end{variable}
%
% \begin{macro}{\@@_process_normal_line:}
% 普通代码行根据开头字符分别处理。
%    \begin{macrocode}
\cs_new_protected_nopar:Npn \@@_process_normal_line:
  {
    \str_case_x:nnF
      { \str_head:N \l_@@_macro_code_line_tl }
      {
%    \end{macrocode}
% 以 |%| 开头的行先由 \cs{tl_tail:N} 去掉 |%|,之后再检查 |<|。
%    \begin{macrocode}
        { \c_percent_str }
        {
          \@@_check_angle:x
            { \tl_tail:N \l_@@_macro_code_line_tl }
        }
%    \end{macrocode}
% 以 |#| 开头的行按注释的格式输出。
%    \begin{macrocode}
        { \c_hash_str }
        { \@@_output_comment_line: }
      }
%    \end{macrocode}
% 其余正常输出。
%    \begin{macrocode}
      { \@@_output_line: }
  }
%    \end{macrocode}
% \end{macro}
%
% \begin{macro}{\@@_process_verbatim_line:}
% 处理抄录代码行(不检查 |%| 与 |<|)。
%    \begin{macrocode}
\cs_new_protected_nopar:Npn \@@_process_verbatim_line:
  {
%    \end{macrocode}
% 将该行与抄录停止标记进行比较。
%    \begin{macrocode}
    \tl_if_eq:NNTF \l_@@_macro_code_line_tl
        \g_@@_macro_code_verbatim_stop_tl
%    \end{macrocode}
% 若相同,则结束抄录环境,清空抄录停止标记,并输出该标记;
%    \begin{macrocode}
      {
        \tl_gclear:N \g_@@_macro_code_verbatim_stop_tl
        \cs_gset_eq:NN \@@_macro_code_process_line:
          \@@_process_normal_line:
        \@@_output_module:nn
          { \color { verb@guard } }
          {
            \@@_swap_cr:
            \@@_module_pop:n { \l_@@_macro_code_line_tl }
          }
      }
%    \end{macrocode}
% 否则直接输出抄录代码。
%    \begin{macrocode}
      { \tl_use:N \l_@@_macro_code_line_tl }
  }
%    \end{macrocode}
% \end{macro}
%
% \begin{macro}{\@@_macro_code_process_line:}
% 处理代码行。该命令的作用如下:
% \begin{itemize}
%   \item 正常情况下,等同于 \cs{@@_process_normal_line:};
%   \item 在 \cs{DontCheckModules} 之后,等价于
%     \cs{@@_output_line:},即不检查模块标记,直接输出;
%   \item 在抄录环境中,等价于 \cs{@@_process_verbatim_line:},
%     此时将不再特殊处理以 |%| 开头的代码行。
% \end{itemize}
%    \begin{macrocode}
\cs_new_eq:NN \@@_macro_code_process_line:
  \@@_process_normal_line:
%    \end{macrocode}
% \end{macro}
%
% \paragraph{模块标记处理}
%
% \begin{macro}{\CheckModules,\DontCheckModules}
% 选择是否检查模块标记。这两个命令在 \pkg{doc} 宏包中已有定义,
% 此处重新声明。
%    \begin{macrocode}
\DeclareDocumentCommand \CheckModules { }
  {
    \cs_set_eq:NN \@@_macro_code_process_line:
      \@@_process_normal_line:
  }
\DeclareDocumentCommand \DontCheckModules { }
  {
    \cs_set_eq:NN \@@_macro_code_process_line:
      \@@_output_line:
  }
%    \end{macrocode}
% \end{macro}
%
% \begin{macro}{\@@_check_angle:n,\@@_check_angle:x}
% 检查第一个字符是否是 |<|。若是,则检查模块;否则立刻输出改行。
% 该函数的参数不带 |%|。
%    \begin{macrocode}
\cs_new_protected:Npn \@@_check_angle:n #1
  {
    \str_if_eq_x:nnTF { \str_head:n {#1} } { < }
      { \@@_check_module:x { \tl_tail:n {#1} } }
      { \@@_output_comment_line: }
  }
\cs_generate_variant:Nn \@@_check_angle:n { x }
%    \end{macrocode}
% \end{macro}
%
% \begin{macro}{\@@_check_module:n,\@@_check_module:x}
% 检查紧跟 |<| 之后的字符。共有四种情况:
% \begin{itemize}
%   \item |*|:模块开始(|%<*|\meta{module}|>|);
%   \item |/|:模块结束(|%</|\meta{module}|>|);
%   \item |@|:名字空间(|%<@@=|\meta{namespace}|>|);
%   \item |<|:抄录环境开始(|%<<|\meta{end-tag})。
% \end{itemize}
% 若不是这几种情况,则为单独一行的独立模块
% (|%<|\meta{module}|>|)。
%    \begin{macrocode}
\cs_new_protected:Npn \@@_check_module:n #1
  {
    \str_case_x:nnF { \str_head:n {#1} }
      {
        { * } { \@@_module_star:w }
        { / } { \@@_module_slash:w }
        { @ } { \@@_module_at:w }
        { < } { \@@_module_verb:w }
      }
      { \@@_module_pm:w }
%    \end{macrocode}
% 参数 |#1| 将被上面几个 |:w| 型函数吃掉。
%    \begin{macrocode}
    #1 \q_stop
  }
\cs_generate_variant:Nn \@@_check_module:n { x }
%    \end{macrocode}
% \end{macro}
%
% 设置 |>| 为活动字符。
%    \begin{macrocode}
\group_begin:
  \char_set_catcode_active:N \>
%    \end{macrocode}
%
% \begin{macro}{\@@_module_star:w}
% 模块开始标记。
% \begin{arguments}
%   \item |*|\meta{module}
%   \item 之后的代码
% \end{arguments}
%    \begin{macrocode}
  \cs_new_protected:Npn \@@_module_star:w #1 > #2 \q_stop
    {
%    \end{macrocode}
% 临时变量 \cs{l_@@_tmp_tl} 保存 |<*|\meta{module}|>|
% 之后的部分,即真实代码。
%    \begin{macrocode}
      \tl_set:Nn \l_@@_tmpa_tl {#2}
%    \end{macrocode}
% 判断该行是否为空(只含一个回车符 |^^M|)。
%    \begin{macrocode}
      \tl_if_eq:NNTF \l_@@_tmpa_tl \c_@@_active_cr_tl
%    \end{macrocode}
% 若是,则将 |<|\meta{module}|>| 放在行号的右侧;
%    \begin{macrocode}
        {
          \@@_output_module:nn
            { \@@_star_color: }
            {
              \@@_module_push:n
                { \@@_module_angle:n {#1} }
            }
        }
%    \end{macrocode}
% 否则放在左侧,并输出相应代码。
%    \begin{macrocode}
        {
          \@@_output_module_left:nn
            { \@@_star_color: }
            {
              \@@_module_push:n
                { \@@_module_angle:n {#1} }
            }
        }
      \@@_output_line:n {#2}
    }
%    \end{macrocode}
% \end{macro}
%
% \begin{macro}{\@@_module_slash:w}
% 模块结束标记。结构与 \cs{@@_module_star:w} 相同。
% \begin{arguments}
%   \item |/|\meta{module}
%   \item 之后的代码
% \end{arguments}
%    \begin{macrocode}
  \cs_new_protected:Npn \@@_module_slash:w #1 > #2 \q_stop
    {
      \tl_set:Nn \l_@@_tmpa_tl {#2}
      \tl_if_eq:NNTF \l_@@_tmpa_tl \c_@@_active_cr_tl
        {
          \@@_output_module:nn
            { \@@_slash_color: }
            {
              \@@_module_pop:n
                { \@@_module_angle:n {#1} }
            }
        }
        {
          \@@_output_module_left:nn
            { \@@_slash_color: }
            {
              \@@_module_pop:n
                { \@@_module_angle:n {#1} }
            }
        }
      \@@_output_line:n {#2}
    }
%    \end{macrocode}
% \end{macro}
%
% \begin{macro}{\@@_module_at:w}
% 名字空间。
% \begin{arguments}
%   \item 名字空间的名称(不含 |@@=|)
%   \item 之后的代码
% \end{arguments}
%    \begin{macrocode}
  \cs_new_protected:Npn \@@_module_at:w @ @ = #1 > #2 \q_stop
    {
      \@@_output_module:nn
        { \color { at@guard } }
        { \@@_module_angle:n { @ @ = #1 } }
%    \end{macrocode}
% 设置名字空间为 |#1|。\cls{l3doc} 中将名字空间称为
% “模块”(module),注意不要混淆。
%    \begin{macrocode}
      \tl_gset:Nn \g__codedoc_module_name_tl {#1}
      \@@_output_line:n {#2}
    }
%    \end{macrocode}
% \end{macro}
%
% \begin{macro}{\@@_module_verb:w}
% 抄录开始。|#1|: |<|\meta{end-tag},只有一个 |<|。
% \meta{end-tag} 的最后会带有一个回车符 |^^M|。
%    \begin{macrocode}
  \cs_new_protected:Npn \@@_module_verb:w #1 \q_stop
    {
%    \end{macrocode}
% 重定义 \cs{@@_macro_code_process_line:},用以输出抄录行。
%    \begin{macrocode}
      \cs_gset_eq:NN \@@_macro_code_process_line:
        \@@_process_verbatim_line:
%    \end{macrocode}
% 设置抄录停止标记。用 \cs{tl_tail:n} 去掉开头的 |<|。
%    \begin{macrocode}
      \tl_gset:Nx \g_@@_macro_code_verbatim_stop_tl
        { \c_percent_str \tl_tail:n {#1} }
%    \end{macrocode}
% 输出 |%<<|\meta{end-tag}。
%    \begin{macrocode}
      \@@_output_module:nn
        { \color { verb@guard } }
        {
          \@@_swap_cr:
          \@@_module_push:n { \l_@@_macro_code_line_tl }
        }
    }
%    \end{macrocode}
% \end{macro}
%
% \begin{macro}{\@@_module_pm:w}
% 处理单独一行的模块。|<|\meta{module}|>| 放在行号的左侧。
% \begin{arguments}
%   \item \meta{module}
%   \item 之后的代码
% \end{arguments}
%    \begin{macrocode}
  \cs_new_protected:Npn \@@_module_pm:w #1 > #2 \q_stop
    {
      \@@_output_module_left:nn
        { \@@_pm_color: }
        { \@@_module_angle:n {#1} }
      \@@_output_line:n {#2}
    }
\group_end:
%    \end{macrocode}
% \end{macro}
%
% \begin{macro}{\@@_output_line:n,\@@_output_line:}
% 输出代码行。参数将被存入 \cs{l_@@_macro_code_line_tl},
% 再由不带参数的版本调用。
%    \begin{macrocode}
\cs_new_protected:Npn \@@_output_line:n #1
  {
    \tl_set:Nn \l_@@_macro_code_line_tl {#1}
%    \end{macrocode}
% 若为空行(只含一个 |^^M|),则直接输出(换行)。
%    \begin{macrocode}
    \tl_if_eq:NNTF
      \l_@@_macro_code_line_tl \c_@@_active_cr_tl
      { \tl_use:N \l_@@_macro_code_line_tl }
      {
%    \end{macrocode}
% 检查开头是否为 |%|,据此分别处理。
%    \begin{macrocode}
        \str_if_eq_x:nnTF
          { \str_head:N \l_@@_macro_code_line_tl } { \c_percent_str }
          { \@@_output_comment_line: } { \@@_output_line: }
      }
  }
\cs_new_protected_nopar:Npn \@@_output_line:
  {
    \tex_noindent:D
%    \end{macrocode}
% 此处将把 |@@| 替换为相应的名字空间。
%    \begin{macrocode}
    \@@_replace_at_at:N \l_@@_macro_code_line_tl
    \tl_use:N \l_@@_macro_code_line_tl
  }
%    \end{macrocode}
% \end{macro}
%
% \begin{macro}{\@@_output_comment_line:}
% 输出注释代码行。用灰色、斜体显示。
%    \begin{macrocode}
\cs_new_protected:Npn \@@_output_comment_line:
  {
    \tex_noindent:D
    \group_begin:
      \__fdudoc_swap_cr:
      \color { code@gray } \slshape \@@_output_line:
    \group_end:
  }
%    \end{macrocode}
% \end{macro}
%
% \begin{macro}{\@@_replace_at_at:N,
%   \@@_replace_at_at_aux:Nn,\@@_replace_at_at_aux:No}
% 把 |@@| 替换为相应的名字空间。其名称存放在全局变量
% \cs{g__codedoc_module_name_tl} 中。
% 它为空时(|%<@@=>|),不做替换。
%    \begin{macrocode}
\cs_new_protected:Npn \@@_replace_at_at:N #1
  {
    \tl_if_empty:NF \g__codedoc_module_name_tl
      { \@@_replace_at_at_aux:No #1 \g__codedoc_module_name_tl }
  }
\cs_new_protected:Npn \@@_replace_at_at_aux:Nn #1#2
  {
%    \end{macrocode}
% 以下代码分别对应两种名字空间的替换:
% \begin{itemize}
%   \item 内部变量:|\|\meta{type}|_@@_|\meta{name} $\to$
%     |\|\meta{type}|__|\meta{namespace}|_|\meta{name};
%   \item 内部函数:|\@@_|\meta{name}” $\to$
%     |\__|\meta{namespace}|_|\meta{name}”)。
% \end{itemize}
%    \begin{macrocode}
    \tl_replace_all:Nnn #1 { _ @ @ } { _ _ #2 }
    \tl_replace_all:Nnn #1 {   @ @ } { _ _ #2 }
  }
\cs_generate_variant:Nn \@@_replace_at_at_aux:Nn { No }
%    \end{macrocode}
% \end{macro}
%
% \begin{macro}{\@@_module_push:n,
%   \@@_module_push_aux:nn,\@@_module_push_aux:on}
% 将模块名压入栈中。此处的栈主要用来处理模块名(包括抄录标记)
% 之间的超链接。
%    \begin{macrocode}
\cs_new_protected_nopar:Npn \@@_module_push:n
  { \@@_module_push_aux:on { \int_use:N \c@HD@hypercount } }
\cs_new_protected:Npn \@@_module_push_aux:nn #1
  {
    \seq_gpush:Nn \g_@@_module_dest_seq {#1}
    \hypersetup { hidelinks }
    \exp_args:Nx \hdclindex
      { \zref@extractdefault { HD.#1 } { guard@end } { 1 } } { }
  }
\cs_generate_variant:Nn \@@_module_push_aux:nn { on }
%    \end{macrocode}
% \end{macro}
%
% \begin{macro}{\@@_module_pop:n,
%   \@@_module_pop_aux:nn,\@@_module_pop_aux:on}
% 将模块名释放出栈。
%    \begin{macrocode}
\cs_new_protected_nopar:Npn \@@_module_pop:n
  {
    \seq_gpop:NNTF \g_@@_module_dest_seq \l_@@_tmpa_tl
      { \@@_module_pop_aux:on { \l_@@_tmpa_tl } }
      { \BOOM \use:n }
  }
\cs_new_protected:Npn \@@_module_pop_aux:nn #1
  {
    \zref@labelbylist { HD.#1 } { fdudoc }
    \hypersetup { hidelinks }
    \hdclindex {#1} { }
  }
\cs_generate_variant:Nn \@@_module_pop_aux:nn { on }
%    \end{macrocode}
% \end{macro}
%
% \begin{variable}{\g_@@_module_dest_seq}
% 存放模块名的序列(栈)。
%    \begin{macrocode}
\seq_new:N \g_@@_module_dest_seq
%    \end{macrocode}
% \end{variable}
%
% 处理行号超链接。使用 \pkg{zref} 宏包。
%    \begin{macrocode}
\zref@newlist { fdudoc }
\zref@newprop { guard@end } [ 1 ]
  { \int_eval:n { \c@HD@hypercount - 1 } }
\zref@addprop { fdudoc } { guard@end }
%    \end{macrocode}
%
% \paragraph{格式处理}
%
% \begin{macro}{\MacroFont}
% 代码部分的字体。
%    \begin{macrocode}
\cs_set_protected:Npn \MacroFont
  {
    \linespread { 1.05 }
    \small \ttfamily \mdseries \upshape
    \@@_verb_addon:
  }
%    \end{macrocode}
% \end{macro}
%
% \begin{macro}{\@@_output_module:nn,\@@_output_module_left:nn}
% 输出模块名(分行内和行号左侧两种)。
% \begin{arguments}
%   \item 颜色等样式
%   \item 模块名
% \end{arguments}
%    \begin{macrocode}
\cs_new_protected:Npn \@@_output_module:nn #1#2
  {
    \tex_noindent:D
    \group_begin:
      #1
      \footnotesize \normalfont \sffamily #2
    \group_end:
  }
\cs_new_protected:Npn \@@_output_module_left:nn #1#2
  {
    \tex_noindent:D
    \hbox_overlap_left:n
      {
        \@@_output_module:nn {#1} {#2}
        \skip_horizontal:n { \leftskip + \smallskipamount }
      }
  }
%    \end{macrocode}
% \end{macro}
%
% \begin{macro}{\@@_star_color:,\@@_slash_color:,\@@_pm_color:}
% 选择模块标记的颜色。模块标记的颜色会根据嵌套层次而改变。
%    \begin{macrocode}
\cs_new_protected_nopar:Npn \@@_star_color:
  {
    \seq_gpop:NNTF \g_@@_star_color_seq \current@color
      { \set@color }
      { \@@_select_color: }
    \seq_gpush:No \g_@@_slash_color_seq { \current@color }
  }
\cs_new_protected_nopar:Npn \@@_slash_color:
  {
    \seq_gpop:NNTF \g_@@_slash_color_seq \current@color
      {
        \set@color
        \seq_gpush:No \g_@@_star_color_seq { \current@color }
      }
% TODO: 需要报错:star 与 slash 没有匹配
      { \BOOM }
  }
\cs_new_protected_nopar:Npn \@@_pm_color:
  {
    \seq_get:NNTF \g_@@_star_color_seq \current@color
      { \set@color }
      {
        \@@_select_color:
        \seq_gpush:No \g_@@_star_color_seq { \current@color }
      }
  }
%    \end{macrocode}
% \end{macro}
%
% \begin{variable}{\g_@@_star_color_seq,\g_@@_slash_color_seq}
% 存放模块标记颜色的序列。
%    \begin{macrocode}
\seq_new:N \g_@@_star_color_seq
\seq_new:N \g_@@_slash_color_seq
%    \end{macrocode}
% \end{variable}
%
% \begin{macro}{\@@_select_color:}
% \begin{macro}[int]{guard@series}
% 设置模块标记的色系。
%    \begin{macrocode}
\cs_new_protected_nopar:Npn \@@_select_color:
  { \color { guard@series!!+ } }
\definecolorseries { guard@series }
  { cmyk } { last } { blue } { purple }
%    \end{macrocode}
% \end{macro}
% \end{macro}
%
% 设置色系的增量大小。可选参数 |3| 意味着基色(blue)与
% 末色(purple)之间将等分为三份。该数字比嵌套最大深度小 1。
%    \begin{macrocode}
\resetcolorseries [ 3 ] { guard@series }
%    \end{macrocode}
%
% \begin{macro}[int]{verb@guard,at@guard,code@gray}
% 设置颜色。
%    \begin{macrocode}
\definecolor { verb@guard } { named } { MaterialLime600 }
\definecolor { at@guard   } { named } { MaterialPink    }
\definecolor { code@gray  } { named } { MaterialGrey    }
%    \end{macrocode}
% \end{macro}
%
% \begin{macro}{\@@_module_angle:n}
% 输出“$\langle\cdots\rangle$”。^^A\\
%^^A 注:原来所使用的 \tn{textlangle} 和 \tn{textrangle} 在
%^^A Source Sans Pro 字体下不可用。
%    \begin{macrocode}
\cs_new_protected:Npn \@@_module_angle:n #1
  { \textlangle #1 \textrangle }
%   { < #1 > }
%   { \ensuremath \langle #1 \ensuremath \rangle }
%    \end{macrocode}
% \end{macro}
%
% \begin{macro}{\@@_code_line_no:}
% 行号。设置为阿拉伯数字。
%    \begin{macrocode}
\cs_new_protected_nopar:Npn \@@_code_line_no:
  {
    \int_gincr:N \c@CodelineNo
    \hbox_overlap_left:n
      {
        \hbox_to_wd:nn
          { \MacroIndent }
          {
            \HD@target
            \tex_hss:D \@@_code_line_no_style:
            \theCodelineNo \enspace
          }
        \tex_kern:D \@totalleftmargin
      }
  }
\tl_set:Nn \theCodelineNo { \arabic { CodelineNo } }
%    \end{macrocode}
% \end{macro}
%
% \begin{macro}{\@@_code_line_no_style:}
% 行号格式。
%    \begin{macrocode}
\cs_new_protected_nopar:Npn \@@_code_line_no_style:
  { \color { code@gray } \normalfont \sffamily \tiny }
%    \end{macrocode}
% \end{macro}
%
% \begin{macro}[int]{\HD@SetMacroIndent}
% 设置代码缩进(行号一栏的宽度)。该命令会写进 |.aux| 辅助文件,
% 以便在二次编译时确定最大行号宽度。
%    \begin{macrocode}
\cs_set_protected:Npn \HD@SetMacroIndent #1
  {
    \group_begin:
      \settowidth \MacroIndent
        {
          \@@_code_line_no_style:
          \prg_replicate:nn { \tl_count:n {#1} } { 0 }
          \enspace
        }
      \dim_gset_eq:NN \MacroIndent \MacroIndent
    \group_end:
  }
%    \end{macrocode}
% \end{macro}
%
% \subsubsection{\env{function} 环境}
%
% \begin{macro}{\@@_verb_addon:,
%   \@@_disable_ecglue:,
%   \@@_plain_punct_style:}
% \begin{macro}[int]{\meta@font@select}
% 调整文字间距,以便于让 CJK 字符占的宽度等于西文等宽字体中两个
% 空格的宽度。需要按编译情况分别定义。
%    \begin{macrocode}
\sys_if_engine_xetex:TF
  {
    \cs_set_eq:NN \@@_verb_addon: \xeCJKVerbAddon
    \cs_set:Nn \@@_plain_punct_style:
      { \xeCJKsetup { PunctStyle = plain } }
    \cs_set:Nn \@@_disable_ecglue:
      { \xeCJKsetup { CJKecglue } }
    \@@_appto_cmd:Nn \meta@font@select
      { \mode_if_inner:T { \@@_disable_ecglue: } }
  }
  {
    \cs_set_eq:NN \@@_verb_addon:        \prg_do_nothing:
    \cs_set_eq:NN \@@_plain_punct_style: \prg_do_nothing:
    \cs_set:Nn \@@_disable_ecglue:
      { \ltjsetparameter { autoxspacing = false } }
    \@@_appto_cmd:Nn \meta@font@select
      { \@@_disable_ecglue: }
  }
%    \end{macrocode}
% \end{macro}
% \end{macro}
%
% \begin{environment}{function}
% \begin{macro}{\@@_fix_previous_depth:}
% 调整 \env{function} 环境前后间距。
%    \begin{macrocode}
\BeforeBeginEnvironment { function }
  { \par \nointerlineskip }
\AtEndEnvironment { function }
  {
    \par
    \cs_gset:Nx \@@_fix_previous_depth:
      { \prevdepth = \the \prevdepth \space }
  }
\AfterEndEnvironment { function }
  { \@@_fix_previous_depth: }
%    \end{macrocode}
% \end{macro}
% \end{environment}
%
% \begin{environment}{syntax}
% \begin{environment}{fdusyntax}
% 在 \env{syntax} 和 \env{fdusyntax} 环境前设置若干活动字符。
% \texttt{\textbar} 用于分隔多个选项,无需倾斜;|<xxx>| 表示选项,
% |(xxx)| 表示默认选项。原来的括号用宏保存,并且使用直立字体。
% \env{syntax} 环境另需要额外调整行距、标点样式及字符间距。
%    \begin{macrocode}
\AtBeginEnvironment { syntax }
  {
    \linespread { 1.2 }
    \@@_plain_punct_style:
    \@@_disable_ecglue:
%     \char_set_catcode_active:N |
%     \char_set_catcode_active:N (
%     \char_set_active_eq:NN | \orbar
%     \char_set_active_eq:NN ( \defaultval@aux
  }
\AtBeginEnvironment { fdusyntax }
  {
    \cs_set:Npn \lparen { \textup { ( } }
    \cs_set:Npn \rparen { \textup { ) } }
    \char_set_catcode_active:N |
    \char_set_catcode_active:N <
    \char_set_catcode_active:N (
    \char_set_active_eq:NN | \orbar
    \char_set_active_eq:NN < \syntaxopt@aux
    \char_set_active_eq:NN ( \defaultval@aux
  }
%    \end{macrocode}
% \end{environment}
% \end{environment}
%
% \subsubsection{修订记录索引项}
%
% \begin{macro}{\@@_ltx_changes:nnn}
% 保存 \pkg{doc} 中 \tn{changes@} 的定义。
%    \begin{macrocode}
\cs_new_eq:NN \@@_ltx_changes:nnn \changes@
%    \end{macrocode}
% \end{macro}
%
% \begin{macro}[int]{\changes@}
% \changes{v0.4}{2017/07/30}{调整索引排序方式。}
% 重定义 \tn{changes@},在版本号一行显示修改日期。
%    \begin{macrocode}
\cs_set_protected:Npn \changes@ #1#2
  {
    \@@_save_version_date:nn {#1} {#2}
    \@@_ltx_changes:nnn {#1} {#2}
  }
%    \end{macrocode}
% \end{macro}
%
% \begin{variable}{\g_@@_version_date_prop}
% 存放版本号与对应的修改日期。
% key = 版本号,value = \{ 开始日期,结束日期 \}。
% 开始日期与结束日期可以相同。
%    \begin{macrocode}
\prop_new:N \g_@@_version_date_prop
%    \end{macrocode}
% \end{variable}
%
% \begin{macro}{\@@_save_version_date:nn}
% |nn| 版本最终将被 \tn{changes@} 调用。
% \begin{arguments}
%   \item 版本号
%   \item 日期
% \end{arguments}
% 它们分别对应 \tn{change} 的前两个参数(第三个是说明文字)。
%    \begin{macrocode}
\cs_new_protected:Npn \@@_save_version_date:nn #1#2
  {
    \prop_get:NnNTF \g_@@_version_date_prop
      {#1} \l_@@_tmpa_tl
      {
%    \end{macrocode}
% \cs{l_@@_tmp_tl} 相当于两个参数(开始日期、结束日期),
% 因此需要提前展开。
%    \begin{macrocode}
        \exp_after:wN
          \@@_save_version_date_aux:nnnn \l_@@_tmpa_tl
        {#2} {#1}
      }
      { \@@_save_version_date_aux:nnn {#1} {#2} {#2} }
  }
%    \end{macrocode}
% \end{macro}
%
% \begin{macro}{\@@_save_version_date_aux:nnnn}
% \begin{arguments}
%   \item 原开始日期
%   \item 原结束日期(显然应有 |#1| < |#2|)
%   \item 新读入的日期
%   \item 版本号
% \end{arguments}
% 如果 |#3| < |#1|,则读入日期 |#3|、|#2|;
% 如果 |#3| > |#2|,则读入日期 |#1|、|#3|。
%    \begin{macrocode}
\cs_new_protected:Npn \@@_save_version_date_aux:nnnn #1#2#3#4
  {
    \@@_if_date_later:nnTF {#1} {#3}
      { \@@_save_version_date_aux:nnn {#4} {#3} {#2} }
      {
        \@@_if_date_later:nnT {#3} {#2}
          { \@@_save_version_date_aux:nnn {#4} {#1} {#3} }
      }
  }
%    \end{macrocode}
% \end{macro}
%
% \begin{macro}{\@@_save_version_date_aux:nnn}
% 将版本号和日期存入 \cs{g_@@_version_date_prop}。
% \begin{arguments}
%   \item 版本号
%   \item 开始日期
%   \item 结束日期
% \end{arguments}
%    \begin{macrocode}
\cs_new_protected:Npn \@@_save_version_date_aux:nnn #1#2#3
  { \prop_gput:Nnn \g_@@_version_date_prop {#1} { {#2} {#3} } }
%    \end{macrocode}
% \end{macro}
%
% \begin{macro}[TF]{\@@_if_date_later:nn}
% \begin{macro}{\@@_parse_date:w}
% 比较两个日期。如果 |#1| 在 |#2| 之后,则为 |true|;反之为 |false|。
% 日期的格式为 YYYY/MM/DD。比较方法是直接将日期化成 8 位数字,
% 所以月、日前的 0 不可以省略。
%    \begin{macrocode}
\prg_new_conditional:Npnn \@@_if_date_later:nn #1#2 { TF, T }
  {
    \if_int_compare:w
        \@@_parse_date:w #1 / / / 0 \q_stop >
        \@@_parse_date:w #2 / / / 0 \q_stop \exp_stop_f:
      \prg_return_true:
    \else:
      \prg_return_false:
    \fi:
  }
\cs_new:Npn \@@_parse_date:w #1/#2/#3/ #4 \q_stop
  { #1#2#3 }
%    \end{macrocode}
% \end{macro}
% \end{macro}
%
% \begin{macro}[int]{\CTEX@versionitem}
% 版本条目标签。如果版本号不在 \cs{g_@@_version_date_prop} 的
% key 里面,则利用未定义的 \cs{BOOM} 报错。
%    \begin{macrocode}
\cs_new_protected:Npn \CTEX@versionitem #1 \efill
  {
    \@idxitem
    \prop_get:NnNTF \g_@@_version_date_prop
      {#1} \l_@@_tmpa_tl
      {
        \exp_after:wN
          \@@_print_version_date:nnn \l_@@_tmpa_tl
        {#1}
      }
      { \BOOM }
  }
%    \end{macrocode}
% \end{macro}
%
% \begin{macro}{\@@_print_version_date:nnn}
% 输出版本号和日期。如果开始日期和结束日期相同,则只输出一项。
% \begin{arguments}
%   \item 开始日期
%   \item 结束日期
%   \item 版本号
% \end{arguments}
%    \begin{macrocode}
\cs_new_protected:Npn \@@_print_version_date:nnn #1#2#3
  {
    \noindent
    \Hy@raisedlink { \belowpdfbookmark {#3} { HD.#3 } }
    \textbf {#3} \hfill
    \hbox:n
      {
        \footnotesize
        \str_if_eq:nnTF {#1} {#2}
          { ( #1 ) } { ( #1 ~ -- ~ #2 ) }
      }
    \par \nopagebreak
  }
%    \end{macrocode}
% \end{macro}
%
% \begin{macro}[int]{\HDorg@theglossary}
% 该命令由 \pkg{hypdoc} 宏包定义,用于存放标准文档类 \cls{book}
% 中定义的 \tn{theindex} 命令。
% 此处的补丁将在版本号一行最后加上修改日期。
%    \begin{macrocode}
\ctex_patch_cmd:Nnn \HDorg@theglossary
  { \let \item \@idxitem }
  { \let \item \CTEX@versionitem }
%    \end{macrocode}
% \end{macro}
%
% \begin{macro}[int]{\@wrglossary}
% 该命令由 \LaTeXe{} 内核定义,又由 \pkg{hypdoc} 宏包作了修改。
% 此处的补丁使得修订记录条目的页码能够指向对应行。
%    \begin{macrocode}
\ctex_patch_cmd:Nnn \@wrglossary
  { hdpindex }
  {
    \ifnum \c@HD@hypercount = \z@
      hdpindex
    \else
      hdclindex { \the \c@HD@hypercount }
    \fi
  }
%    \end{macrocode}
% \end{macro}
%
% \subsubsection{命令补丁}
%
% \paragraph{\LaTeXe{} 补丁}
%
% \begin{macro}[int]{\@addtocurcol}
% 调整浮动体、代码等与文字的间距。
% 见 \url{http://tex.stackexchange.com/a/40896}。
%    \begin{macrocode}
\ctex_patch_cmd:Nnn \@addtocurcol
  { \vskip \intextsep }
  {
    \edef \save@first@penalty { \the \lastpenalty } \unpenalty
    \ifnum \lastpenalty = \@M
      \unpenalty
    \else
      \penalty \save@first@penalty \relax
    \fi
    \ifnum \outputpenalty < -\@Mii
      \addvspace\intextsep
      \vskip\parskip
    \else
      \addvspace\intextsep
    \fi
  }
\ctex_patch_cmd:Nnn \@addtocurcol
  {
    \vskip \intextsep
    \ifnum \outputpenalty < -\@Mii
      \vskip -\parskip
    \fi
  }
  {
    \ifnum \outputpenalty < -\@Mii
      \aftergroup \vskip \aftergroup \intextsep
      \aftergroup \nointerlineskip
    \else
      \vskip \intextsep
    \fi
  }
%    \end{macrocode}
% \end{macro}
%
% \begin{macro}[int]{\@getpen}
% 将换行或换页的最大罚值由 \num{10000} 改为 \num{10001}。
%    \begin{macrocode}
\ctex_patch_cmd:Nnn \@getpen { \@M } { \@Mi }
%    \end{macrocode}
% \end{macro}
%
% \begin{macro}[int]{\l@section,\l@subsection}
% 修改目录条目的缩进。
%    \begin{macrocode}
\ctex_patch_cmd:Nnn \l@section    { 2.5em } { 1.5em }
\ctex_patch_cmd:Nnn \l@subsection { 2.5em } { 1.5em }
%    \end{macrocode}
% \end{macro}
%
% \begin{macro}[int]{\@thehead}
% 修改页眉,禁用 \tn{MakeUppercase}。
%    \begin{macrocode}
\@@_preto_cmd:Nn \@thehead
  { \cs_set_eq:cN { MakeUppercase \space } \@iden }
%    \end{macrocode}
% \end{macro}
%
% \begin{macro}{\thebibliography}
% \begin{macro}[int]{\HDorg@thebibliography}
% 参考文献一节需要编号。
%    \begin{macrocode}
\ctex_patch_cmd:Nnn \HDorg@thebibliography
  { \section* } { \section }
\cs_set_eq:NN \thebibliography \HDorg@thebibliography
%    \end{macrocode}
% \end{macro}
% \end{macro}
%
% \begin{macro}{\GlossaryParms}
% 修改修订记录中的一些缩进。
%    \begin{macrocode}
\@@_appto_cmd:Nn \GlossaryParms
  {
    \raggedcolumns
    \cs_set_eq:NN \Hy@writebookmark \HDorg@writebookmark
    \cs_set:Npn \@idxitem   { \par \hangindent 2em }
    \cs_set:Npn \subitem    { \@idxitem \hspace* { 1em } }
    \cs_set:Npn \subsubitem { \@idxitem \hspace* { 2em } }
  }
%    \end{macrocode}
% \end{macro}
%
% \begin{macro}[int]{\HoLogo@LaTeXe}
% 由于使用了 \pkg{unicode-math},需要额外修改 \pkg{hologo} 中的
% \tn{LaTeXe} 命令,以使粗体正常显示。
%    \begin{macrocode}
\ctex_patch_cmd:Nnn \HoLogo@LaTeXe
  { \hbox { \HOLOGO@MathSetup 2 $ _{ \textstyle \varepsilon } $ } }
  {
    \hbox
      {
        \mathsurround 0pt \relax
        2
        \if b \expandafter \@car \f@series \@nil
          $ _{ \textstyle \symbf { \varepsilon } } $
        \else
          $ _{ \textstyle \varepsilon } $
        \fi
      }
  }
%    \end{macrocode}
% \end{macro}
%
% \begin{macro}{\SpecialMainEnvIndex}
% 索引汉化。
%    \begin{macrocode}
\ctex_patch_cmd:Nnn \SpecialMainEnvIndex
  { (environment) } { ~ 环境 }
\ctex_patch_cmd:Nnn \SpecialMainEnvIndex
  { environments: } { 环境: }
%    \end{macrocode}
% \end{macro}
%
% \begin{macro}[int]{\HDorg@SpecialEnvIndex}
% \pkg{hypdoc} 宏包重新定义了 \tn{SpecialEnvIndex} 命令,因此需要
% 修改内部定义。
%    \begin{macrocode}
\ctex_patch_cmd:Nnn \HDorg@SpecialEnvIndex
  { (environment) } { ~ 环境 }
\ctex_patch_cmd:Nnn \HDorg@SpecialEnvIndex
  { environments: } { 环境: }
%    \end{macrocode}
% \end{macro}
%
% \paragraph{\cls{l3doc} 补丁}
%
%    \begin{macrocode}
%<@@=codedoc>
%    \end{macrocode}
%
% \begin{macro}{\list}
% \cls{l3doc} 会设置列表环境中 \tn{listparindent} |=| \tn{z@},
% 这里将其恢复。
%    \begin{macrocode}
\cs_set_eq:NN \list \@@_oldlist:nn
%    \end{macrocode}
% \end{macro}
%
% \begin{macro}{\@@_function_descr_start:w}
% 抑制首段的 \tn{parskip}。
%    \begin{macrocode}
\__fdudoc_patch_cmd:Nnn \@@_function_descr_start:w
  { \noindent }
  { \skip_vertical:n { -\parskip } \noindent }
%    \end{macrocode}
% \end{macro}
%
% \begin{macro}{\@@_function_assemble:}
% 在 \cls{l3doc} 中,\env{function} 环境里的 \env{syntax} 和
% \env{descr} 盒子中间存在一段 \tn{med\-skip\-amount} 的距离。
% 但是如果 \env{syntax} 盒子为空(未使用 \env{syntax} 环境),
% 就会显得不好看。此时通过把 \tn{medskipamount} 设置为零来修正。
% 若盒子非空,则把 \tn{parskip} 还回去。
%    \begin{macrocode}
\__fdudoc_preto_cmd:Nn \@@_function_assemble:
  {
    \box_if_empty:NTF \g_@@_syntax_box
      { \skip_zero:N \medskipamount }
      { \skip_add:Nn \medskipamount { \parskip } }
  }
%    \end{macrocode}
% \end{macro}
%
% \begin{macro}{\@@_typeset_functions:}
% 调整 \env{function} 环境的字体。
%    \begin{macrocode}
\__fdudoc_patch_cmd:Nnn \@@_typeset_functions:
  { \small \ttfamily } { \footnotesize \CodeFont }
%    \end{macrocode}
% \end{macro}
%
% \begin{macro}{\@@_typeset_functions:,\@@_macro_init:,
%   \@@_macro_dump:}
% 左侧边注的函数列表采用单倍行距。
%    \begin{macrocode}
\__fdudoc_preto_cmd:Nn \@@_typeset_functions:
  { \MacroFont }
\__fdudoc_patch_cmd:Nnn \@@_macro_init:
  { \hbox:n } { \MacroFont \hbox:n }
\__fdudoc_patch_cmd:Nnn \@@_macro_dump:
  { \hbox_unpack_clear:N } { \MacroFont \hbox_unpack_clear:N }
%    \end{macrocode}
% \end{macro}
%
% \begin{macro}{\@@_macro_end_style:n}
% 禁止显示 \env{macro} 环境最后的 “(\emph{End definition for ...})”。
%    \begin{macrocode}
\cs_set_eq:NN \@@_macro_end_style:n \use_none:n
%    \end{macrocode}
% \end{macro}
%
% \begin{macro}{\@@_macro_typeset_one:nN}
% 在 \env{macro} 环境的侧边栏中,\cls{l3doc} 根据命令的长短,分别用
% 普通字体和紧缩字体输出。然而很长的命令还是会超出页边。这里用缩放
% 盒子的手段使得长命令也可正常显示。
%    \begin{macrocode}
\cs_set_protected:Npn \@@_macro_typeset_one:nN #1#2
  {
    \vbox_set:Nn \l_@@_macro_box
      {
        \MacroFont
        \vbox_unpack_clear:N \l_@@_macro_box
        \hbox_set:Nn \l_tmpa_box
          { \@@_print_macroname:nN {#1} #2 }
%    \end{macrocode}
% \tn{marginparwidth} 和 \tn{marginparsep} 分别是边注的宽度及其到
% 版心的距离,\tn{la\-bel\-sep} 则是编号盒子右端与条目首行文本之间
% 的距离。
%    \begin{macrocode}
        \dim_set:Nn \l_tmpa_dim
          { \marginparwidth - \labelsep - \marginparsep }
        \dim_compare:nNnT { \box_wd:N \l_tmpa_box } > \l_tmpa_dim
          {
            \box_resize_to_wd_and_ht:Nnn \l_tmpa_box
              { \l_tmpa_dim } { \box_ht:N \l_tmpa_box }
          }
        \hbox_overlap_left:n
          {
            \box_use:N \l_tmpa_box
            \skip_horizontal:n { \marginparsep - \labelsep }
          }
      }
    \int_incr:N \l_@@_macro_int
  }
%    \end{macrocode}
% \end{macro}
%
% \begin{macro}{\@@_print_macroname:nN}
% 该函数不再需要根据命令的长短切换字体。
%    \begin{macrocode}
\cs_set_protected:Npn \@@_print_macroname:nN #1#2
  {
    \strut
    \@@_get_hyper_target:xN
      {
        \exp_not:n {#1}
        \bool_if:NT #2 { \tl_to_str:n {TF} }
      }
      \l_@@_tmpa_tl
    \cs_if_exist:cTF { r@ \l_@@_tmpa_tl }
      { \exp_args:NNo \label@hyperref [ \l_@@_tmpa_tl ] }
      { \use:n }
      {
        \tl_set:Nn \l_@@_tmpa_tl {#1}
%    \end{macrocode}
% 命令中的空格改用“\textvisiblespace”显示。
%    \begin{macrocode}
        \tl_replace_all:Non \l_@@_tmpa_tl
          { \c_catcode_other_space_tl }
          { \fontspec_visible_space: }
        \@@_macroname_prefix:o \l_@@_tmpa_tl
        \@@_macroname_suffix:N #2
      }
  }
%    \end{macrocode}
% \end{macro}
%
% \begin{macro}{\@@_special_index_module:nnnnN}
% 索引汉化。
%    \begin{macrocode}
\cs_set_protected:Npn \@@_special_index_module:nnnnN #1#2#3#4#5
  {
    \use:x
      {
        \exp_not:n { \@@_special_index_aux:nnnnnnn {#1} {#2} }
          \tl_if_empty:nTF {#3}
            { { } { } { } { } }
            {
              \str_if_eq_x:nnTF {#3} { TeX }
                {
                  { TeX~ and~ LaTeX2e }
                  { \string \TeX{}~ 和~ \string \LaTeXe{} }
                }
                { {#3} { \string \pkg {#3} } }
              \bool_if:NTF #5
                { { commands~ internal } { ~ 内部命令: } }
                { { commands           } { ~ 命令:     } }
            }
      }
    {#4}
  }
%    \end{macrocode}
% \end{macro}
%
% \begin{macro}{\@@_special_index_aux:nnnnnnn}
% 该函数在 \cls{l3doc} 中本来只有 6 个参数。这里增加了一个,用来辅助
% 排序。
% \begin{arguments}
%   \item 键(即宏名称字符串,用来排序)
%   \item 宏名称
%   \item 索引头排序字符串(排序)
%   \item 索引头文字
%   \item 索引头后缀字符串(排序,新增)
%   \item 索引头后缀文字
%   \item 索引类型(\opt{main}/\opt{usage} 等)
% \end{arguments}
%    \begin{macrocode}
\cs_new_protected:Npn \@@_special_index_aux:nnnnnnn #1#2#3#4#5#6#7
  {
    \tl_set:Nn \l_@@_index_escaped_key_tl {#1}
    \@@_quote_special_char:N \l_@@_index_escaped_key_tl
    \@@_special_index_set:Nn
      \l_@@_index_escaped_macro_tl {#2}
    \str_if_eq:onTF { \@currenvir } { macrocode }
      { \codeline@wrindex }
      {
        \str_case:nnF {#7}
          {
            { main  } { \codeline@wrindex }
            { usage } { \index }
          }
          { \HD@target \index }
      }
      {
        \tl_if_empty:nF { #3 #4 #5 #6 }
          { #3 #5 \actualchar #4 #6 \levelchar }
        \l_@@_index_escaped_key_tl
        \actualchar
        {
          \token_to_str:N \verbatim@font \c_space_tl
          \l_@@_index_escaped_macro_tl
        }
        \encapchar
        hdclindex { \the \c@HD@hypercount } {#7}
      }
  }
%    \end{macrocode}
% \end{macro}
%
% \subsubsection{杂项}
%
% \begin{macro}{\StopSpecialIndexModule,
%   \@@_special_index_module:nnnnN}
% 不对 \cs{cs} 和 \cs{tn} 等编索引。用于目录、索引等。
%    \begin{macrocode}
\DeclareDocumentCommand \StopSpecialIndexModule { }
  {
    \cs_set_eq:NN
      \@@_special_index_module:nnnnN \use_none:nnnnn
  }
\tl_map_inline:nn { \actualchar \encapchar \levelchar }
  { \exp_args:Nx \DoNotIndex { \bslash \tl_to_str:N #1 } }
%    \end{macrocode}
% \end{macro}
%
% \begin{macro}{\meta}
% 重定义 \cs{meta} 命令,需要禁用中文、西文之间的空格,
% 并确保为罗马字体族。
%    \begin{macrocode}
\RenewDocumentCommand \meta { m }
  {
    \group_begin:
      \sys_if_engine_xetex:T { \xeCJKsetup { CJKecglue = { } } }
      \rmfamily \@@_meta:n {#1}
    \group_end:
  }
%    \end{macrocode}
% \end{macro}
%
% 关闭 \cls{l3doc} 中的一些提示信息。
%    \begin{macrocode}
% \msg_redirect_name:nnn { l3doc } { foreign-internal    } { none }
% \msg_redirect_name:nnn { l3doc } { print-changes-howto } { none }
% \msg_redirect_name:nnn { l3doc } { print-index-howto   } { none }
%    \end{macrocode}
%
%    \begin{macrocode}
%<@@=>
\ExplSyntaxOff
%    \end{macrocode}
%
% \subsubsection{排版样式设置}
%
% 目录中禁止对 \cs{cs} 和 \cs{tn} 等的索引。
%    \begin{macrocode}
\AtBeginDocument{%
  \addtocontents{toc}{\StopSpecialIndexModule}}
%    \end{macrocode}
%
% \begin{macro}{\path,\opt,\kaishu}
% 设置 PDF 字符串中的命令替换。
%    \begin{macrocode}
\pdfstringdefDisableCommands{%
  \let\path\meta
  \let\opt\@firstofone
  \let\kaishu\relax}
%    \end{macrocode}
% \end{macro}
%
% \begin{macro}[int]{\@makefntext}
% 调整脚注文本缩进。
%    \begin{macrocode}
\renewcommand\@makefntext[1]{\parindent 0em\noindent\@makefnmark~#1}
%    \end{macrocode}
% \end{macro}
%
% \begin{macro}{\IndexLayout,\indexname}
% 调整索引外观。
%    \begin{macrocode}
\IndexPrologue{%
  \section{\indexname}%
  \textit{无衬线字体的数字表示对应索引项出现的页码;
    带下划线的数字表示定义对应索引项的代码行号;
    其他普通数字则表示使用对应索引项的代码行号.}}
\def\IndexLayout{%
  \newgeometry{%
    left   = 0.90 in,
    right  = 0.90 in,
    top    = 1.25 in,
    bottom = 1.00 in}%
  \setlength\IndexMin{0.5\textheight}%
  \ctexset{section/numbering=false}%
  \StopSpecialIndexModule}
\def\indexname{代码索引}
%    \end{macrocode}
% \end{macro}
%
% \begin{macro}{\usage}
% 控制“描述对应索引项的页码”样式。在 \pkg{doc} 宏包中的定义为
% \tn{textit}。
%    \begin{macrocode}
\def\usage#1{\textsf{#1}}
%    \end{macrocode}
% \end{macro}
%
% \begin{macro}{\glossaryname}
% 调整修订记录外观。
%    \begin{macrocode}
\GlossaryPrologue{\section{\glossaryname}}
\def\glossaryname{修订记录}
%    \end{macrocode}
% \end{macro}
%
% 西文字体。
%    \begin{macrocode}
\setmainfont{TeX Gyre Pagella}
\setsansfont{TeX Gyre Heros}
\setmathfont{TeX Gyre Pagella Math}
%    \end{macrocode}
%
% \begin{macro}{\kaishu,\fangsong}
% 中文字体。
%    \begin{macrocode}
\setCJKmainfont{FZShuSong-Z01}%
  [
    BoldFont       = FZHei-B01,
    ItalicFont     = FZKai-Z03,
    BoldItalicFont = FZKai-Z03
  ]
\setCJKsansfont{FZHei-B01}%
  [
    BoldFont       = FZHei-B01,
    ItalicFont     = FZKai-Z03,
    BoldItalicFont = FZKai-Z03
  ]
\setCJKmonofont{FZFangSong-Z02}%
  [
    BoldFont       = FZHei-B01,
    ItalicFont     = FZKai-Z03,
    BoldItalicFont = FZKai-Z03
  ]
\newCJKfontfamily\kaishu{FZKai-Z03}%
  [BoldFont = *, ItalicFont = *, BoldItalicFont = *]
\newCJKfontfamily\fangsong{FZFangSong-Z02}%
  [BoldFont = *, ItalicFont = *, BoldItalicFont = *]
%    \end{macrocode}
% \end{macro}
%
% \begin{macro}{\CodeFont}
% \begin{macro}[int]{\fdudoc@code@font,\fdudoc@cjk@code@font}
% 代码部分的字体,这里用了两种系列。
%    \begin{macrocode}
\setmonofont{CMU Typewriter Text}%
  [
    UprightFont = * Light,
    BoldFont    = * Bold,
    SlantedFont = * Light Oblique,
    HyphenChar  = None
  ]
\newfontfamily\fdudoc@code@font{Source Code Pro}%
  [BoldFont = * Semibold]
\newCJKfontfamily\fdudoc@cjk@code@font{Source Han Sans SC}%
  [BoldFont = * Bold, AutoFakeSlant = 0.22]
\newcommand*\CodeFont{\fdudoc@code@font\fdudoc@cjk@code@font}
%    \end{macrocode}
% \end{macro}
% \end{macro}
%
% 中文排版格式(\pkg{ctex} 宏包)。
%    \begin{macrocode}
\ctexset
  {
    section = {name = {第,节}, format+ = \raggedright},
    subsubsection/tocline = {\kaishu\CTEXnumberline{#1}#2},
    paragraph = {runin = false, numbering = false}
  }
%    \end{macrocode}
%
% 设定章节标题、目录深度。
%    \begin{macrocode}
\setcounter{secnumdepth}{4}
\setcounter{tocdepth}{3}
%    \end{macrocode}
%
% 单位设置(\pkg{siunitx} 宏包)。
%    \begin{macrocode}
\sisetup
  {
    number-math-rm       = \ensuremath,
    inter-unit-product   = \ensuremath{{}\cdot{}},
    group-digits         = true,
    group-minimum-digits = 4,
    group-separator      = \text{~},
    range-phrase         = \symbol{"FF5E},
    separate-uncertainty = true
  }
%    \end{macrocode}
%
% 超链接设置(\pkg{hyperref} 宏包)。
%    \begin{macrocode}
\hypersetup
  {
    bookmarksdepth    = 4,
    bookmarksnumbered = true,
    colorlinks        = true,
    citecolor         = MaterialGreen,
    linkcolor         = MaterialPink,
    urlcolor          = MaterialIndigo
  }
%    \end{macrocode}
%
% 浮动体标题设置(\pkg{caption} 宏包)。
%    \begin{macrocode}
\captionsetup{labelsep = quad, labelfont+ = bf}
%    \end{macrocode}
%
% 设置标准列表环境样式。
%    \begin{macrocode}
\setlist{noitemsep, topsep=\smallskipamount}
\setlist[1]{labelindent=\parindent}
\setlist[enumerate]{leftmargin=*}
\setlist[itemize]{leftmargin=*}
%    \end{macrocode}
%
% \begin{environment}{optdesc}
% 用于描述各选项。设置条目间距为 \tn{marginparsep},与
% \cls{l3doc} 一致。
%    \begin{macrocode}
\newlist{optdesc}{description}{3}
\setlist[optdesc]{%
  font=\mdseries\small\ttfamily, align=right,
  listparindent=\parindent,
  labelsep=\marginparsep, labelindent=-\marginparsep,
  leftmargin=*}
%    \end{macrocode}
% \end{environment}
%
% \begin{environment}{tablenotes}
% \begin{variable}{tpt@id}
% 重新定义 \pkg{threeparttable} 包的 \env{tablenotes} 环境,
% 用于表格的注释。
%    \begin{macrocode}
\renewlist{tablenotes}{description}{1}
\setlist[tablenotes]{%
  format=\normalfont\tnote@item, align=right,
  listparindent=\parindent, labelindent=\tabcolsep,
  leftmargin=*, rightmargin=\tabcolsep,
  after=\@noparlisttrue}
\AtBeginEnvironment{tablenotes}{%
  \setlength\parindent{2\ccwd}%
  \normalfont\footnotesize}
\AtBeginEnvironment{threeparttable}{%
  \stepcounter{tpt@id}%
  \edef\curr@tpt@id{tpt@\arabic{tpt@id}}}
\newcounter{tpt@id}
%    \end{macrocode}
% \end{variable}
% \end{environment}
%
% \begin{macro}[int]{\tnote@item,\TPTtagStyle}
% 为 \tn{tnote} 增加超链接。
%    \begin{macrocode}
\def\tnote@item#1{%
  \Hy@raisedlink{\hyper@anchor{\curr@tpt@id-#1}}#1}
\def\TPTtagStyle#1{\hyperlink{\curr@tpt@id-#1}{#1}}
%    \end{macrocode}
% \end{macro}
%
% \begin{macro}{\UrlAlphabet,\UrlDigits}
% 网址断行。\tn{UrlOrds}、\tn{UrlAlphabet} 和 \tn{UrlDigits}
% 分别记录了特殊符号、字母和数字,把它们依次附加在 \pkg{url} 宏包
% 提供的命令 \tn{UrlBreaks} 之后,即可允许在这些位置处断行。与
% \cs{fdu_allow_url_break:} 的原理是相同的。
%    \begin{macrocode}
\def\UrlAlphabet{%
  \do\a\do\b\do\c\do\d\do\e\do\f\do\g\do\h\do\i\do\j%
  \do\k\do\l\do\m\do\n\do\o\do\p\do\q\do\r\do\s\do\t%
  \do\u\do\v\do\w\do\x\do\y\do\z\do\A\do\B\do\C\do\D%
  \do\E\do\F\do\G\do\H\do\I\do\J\do\K\do\L\do\M\do\N%
  \do\O\do\P\do\Q\do\R\do\S\do\T\do\U\do\V\do\W\do\X%
  \do\Y\do\Z}
\def\UrlDigits{%
  \do\1\do\2\do\3\do\4\do\5\do\6\do\7\do\8\do\9\do\0}
\g@addto@macro\UrlBreaks{\UrlOrds}
\g@addto@macro\UrlBreaks{\UrlAlphabet}
\g@addto@macro\UrlBreaks{\UrlDigits}
%    \end{macrocode}
% \end{macro}
%
% 不对下列各项添加索引。
%    \begin{macrocode}
\DoNotIndex{\begin,\end,
  \a,\b,\c,\d,\e,\f,\g,\h,\i,\j,\k,\l,\m,
  \n,\o,\p,\q,\r,\s,\t,\u,\v,\w,\x,\y,\z,
  \A,\B,\C,\D,\E,\F,\G,\H,\I,\J,\K,\L,\M,
  \N,\O,\P,\Q,\R,\S,\T,\U,\V,\W,\X,\Y,\Z,
  \0,\1,\2,\3,\4,\5,\6,\7,\8,\9}
%    \end{macrocode}
%
% 启用索引、交叉引用、历史记录。
%    \begin{macrocode}
\EnableCrossrefs
\CodelineIndex
\RecordChanges
%    \end{macrocode}
%
% \subsubsection{文档层命令}
%
% \begin{macro}{\exptarget,\rexptarget,\expstar,\rexpstar,
%   \__codedoc_typeset_exp:,\__codedoc_typeset_rexp:}
% 部分命令之后的星号($\star$ 或 \ding{73}),表明其不同用法。
% 这里的“exp”和“rexp”分别源自 \LaTeX3 中的“expandable”
% 和“restricted-expandable”。
%    \begin{macrocode}
\newcommand*\exptarget{\Hy@raisedlink{\hypertarget{expstar}{}}}
\newcommand*\rexptarget{\Hy@raisedlink{\hypertarget{rexpstar}{}}}
\newcommand*\expstar{\hyperlink{expstar}{$\star$}}
\newcommand*\rexpstar{\hyperlink{rexpstar}{\ding{73}}}
% TODO: (2017/08/12) 允许修改 exptstar 和 rexpstar
\renewcommand*\expstar{\hyperlink{expstar}{$\mfrake$}}
\renewcommand*\rexpstar{\hyperlink{rexpstar}{$\mfrakc$}}
\ExplSyntaxOn
\cs_set_eq:NN \__codedoc_typeset_exp:  \expstar
\cs_set_eq:NN \__codedoc_typeset_rexp: \rexpstar
\ExplSyntaxOff
%    \end{macrocode}
% \end{macro}
%
% \begin{macro}{\marg,\oarg,\parg}
% 几种命令参数:
% \begin{itemize}
%   \item 必选参数:|{|\meta{arg}|}|;
%   \item 可选参数:|[|\meta{arg}|]|;
%   \item 图形参数:|(|\meta{arg}|)|。
% \end{itemize}
%    \begin{macrocode}
\renewcommand*\marg[1]{\{\meta{#1}\}}
\renewcommand*\oarg[1]{[\meta{#1}]}
\renewcommand*\parg[1]{(\meta{#1})}
%    \end{macrocode}
% \end{macro}
%
% \begin{macro}{\opt}
% 选项名。
%    \begin{macrocode}
\newcommand*\opt[1]{\texttt{#1}}
%    \end{macrocode}
% \end{macro}
%
% \begin{macro}{\kvopt}
% \meta{key} |=| \meta{value} 型选项。带星号的版本不会在等号两侧
% 加上空格。
%    \begin{macrocode}
\DeclareDocumentCommand\kvopt{smm}
  {\IfBooleanTF{#1}{\texttt{#2=#3}}{\texttt{#2 = #3}}}
%    \end{macrocode}
% \end{macro}
%
% \begin{macro}{\syntaxopt,\defaultval}
% \begin{macro}[int]{\syntaxopt@aux,\defaultval@aux}
% \env{syntax} 中的选项及命令选项。
% |aux| 结尾的两个命令用于定义利用 |<>| 和 |()| 的简写形式。
%    \begin{macrocode}
\def\syntaxopt#1{\textit{#1}}
\def\defaultval#1{\textbf{\textup{#1}}}
\def\syntaxopt@aux#1>{\syntaxopt{#1}}
\def\defaultval@aux#1){\defaultval{#1}}
%    \end{macrocode}
% \end{macro}
% \end{macro}
%
% \begin{macro}{\orbar,\TF,\TTF,\TFF}
% \env{syntax} 中的选项分隔符,以及 \opt{true} 或 \opt{false}
% 选项的几种快捷方式。
%    \begin{macrocode}
\def\orbar{\textup{\textbar}}
\def\TF{true\orbar false}
\def\TTF{\defaultval{true}\orbar false}
\def\TFF{true\orbar\defaultval{false}}
%    \end{macrocode}
% \end{macro}
%
% \begin{environment}{arguments}
% 放在 \env{macro} 环境中,用于描述对应命令的参数。
% \cls{l3doc} 中的定义 \opt{labelsep} 设置不太合理,会使标签被覆盖,
% 这里重新定义。
%    \begin{macrocode}
\DeclareDocumentEnvironment{arguments}{}
  {\enumerate[%
    label={\texttt{\#\arabic*:~}}, labelsep=0pt, nolistsep]}%
  {\endenumerate}
%    \end{macrocode}
% \end{environment}
%
% \begin{macro}{\TeX,\LaTeX,\LaTeXe,
%   \pdfTeX,\pdfLaTeX,\XeTeX,\XeLaTeX,\LuaTeX,\LuaLaTeX,
%   \AmSLaTeX,\TeXLive,\MiKTeX,\BibTeX,\TikZ}
% \TeX{} 相关标志。
%    \begin{macrocode}
\def\TeX{\hologo{TeX}}
\def\LaTeX{\hologo{LaTeX}}
\def\LaTeXe{\hologo{LaTeXe}}
\def\pdfTeX{\hologo{pdfTeX}}
\def\pdfLaTeX{\hologo{pdfLaTeX}}
\def\XeTeX{\hologo{XeTeX}}
\def\XeLaTeX{\hologo{XeLaTeX}}
\def\LuaTeX{\hologo{LuaTeX}}
\def\LuaLaTeX{\hologo{LuaLaTeX}}
\def\AmSLaTeX{\hologo{AmSLaTeX}}
\def\TeXLive{\TeX\ Live}
\def\MiKTeX{\hologo{MiKTeX}}
\def\BibTeX{\hologo{BibTeX}}
%    \end{macrocode}
% 该定义来自 \file{pgfmanual-en-macros.tex}。
%    \begin{macrocode}
\def\TikZ{Ti\emph{k}Z}
%    \end{macrocode}
% \end{macro}
%
% \begin{macro}{\cs,\tn}
% \begin{macro}[int]{\codedoc@cs,\codedoc@tn}
% 控制序列。
%^^A Colors are used for debug.
%    \begin{macrocode}
% \let\codedoc@cs=\cs
% \let\codedoc@tn=\tn
% \renewcommand*\cs[2][]{%
%   \textcolor{MaterialIndigo}{\codedoc@cs[#1]{#2}}}
% \renewcommand*\tn[2][]{%
%   \textcolor{MaterialPink}{\codedoc@tn[#1]{#2}}}
%    \end{macrocode}
% \end{macro}
% \end{macro}
%
% \begin{macro}{\file,\env,\pkg,\cls}
% 文件、环境、宏包、文档类。
%    \begin{macrocode}
% \renewcommand*\file[1]{%
%   \textcolor{MaterialGrey900}{\texttt{#1}}}
\renewcommand*\env[1]{\textbf{\texttt{#1}}}
% \renewcommand*\pkg[1]{\textsf{#1}}
% \renewcommand*\cls[1]{\textit{\textsf{#1}}}
%    \end{macrocode}
% \end{macro}
%
% \begin{macro}{\bashcmd}
% Bash 中的命令。
%    \begin{macrocode}
\newcommand*\bashcmd[1]{\texttt{#1}}
%    \end{macrocode}
% \end{macro}
%
% \begin{macro}{\scite}
% 位于上标的文献引用。
%    \begin{macrocode}
\newcommand*\scite[1]{\textsuperscript{\cite{#1}}}
%    \end{macrocode}
% \end{macro}
%
% \begin{environment}{quote*}
% 引述环境。
% \begin{arguments}
%   \item 作者
%   \item 朝代
% \end{arguments}
%    \begin{macrocode}
\DeclareDocumentEnvironment{quote*}{oo}
  {\quote\fangsong\qquad}%
  {\endquote\IfNoValueF{#1}{%
    \hfill —— \IfNoValueF{#2}{〔#2〕}#1}}
%    \end{macrocode}
% \end{environment}
%
% \paragraph{示例代码环境}
%
% \pkg{listings} 宏包中连字符 |-| 原本以数学模式输出,
% 此处改为普通文本。
% 见 \url{https://tex.stackexchange.com/a/33188/136923}。
%    \begin{macrocode}
\lst@CCPutMacro\lst@ProcessOther {"2D}{\lst@ttfamily{-{}}{-{}}}
\@empty\z@\@empty
%    \end{macrocode}
%
% \changes{v0.5}{2017/09/05}{[\pkg{fdudoc}] 移除 \pkg{listings}
%   关键字定义文件。}
%
% 定义几种代码样式。
% \begin{macro}[int]{style@base}
%    \begin{macrocode}
\lstdefinestyle{style@base}
  {
    extendedchars   = true,
    gobble          = 3,
    lineskip        = 2 pt,
    frame           = l,
    framerule       = 1 pt,
    framesep        = 0 pt,
    escapeinside    = {(*}{*)},
    basicstyle      = \small\CodeFont\color{MaterialGrey900},
    keywordstyle    = \bfseries\color{MaterialIndigo},
    commentstyle    = \itshape\color{MaterialGrey600},
    stringstyle     = \color{MaterialRed},
    backgroundcolor = \color{MaterialGrey50}
  }
%    \end{macrocode}
% \end{macro}
%
% \begin{macro}[int]{style@shell}
%    \begin{macrocode}
\lstdefinestyle{style@shell}
  {
    style      = style@base,
    rulecolor  = \color{MaterialPink},
    language   = bash,
    alsoletter = {-},
    emphstyle  = \color{MaterialGreen800}
  }
%    \end{macrocode}
% \end{macro}
%
% \begin{macro}[int]{style@latex}
%    \begin{macrocode}
\lstdefinestyle{style@latex}
  {
    style      = style@base,
    rulecolor  = \color{MaterialIndigo},
    language   = [LaTeX]TeX,
    alsoletter = {*},
    texcsstyle = *\color{MaterialDeepOrange},
    emphstyle  = [1]\color{MaterialGreen800},
    emphstyle  = [2]\color{MaterialTeal}
  }
%    \end{macrocode}
% \end{macro}
%
% \begin{macro}[int]{style@syntax}
%    \begin{macrocode}
\lstdefinestyle{style@syntax}
  {
    extendedchars = true,
    gobble        = 6,
    escapeinside  = {(*}{*)},
    language      = [LaTeX]TeX,
    alsoletter    = {*},
    basicstyle    = \footnotesize\CodeFont\color{MaterialGrey900},
    keywordstyle  = \bfseries\color{MaterialIndigo},
    commentstyle  = \itshape\color{MaterialGrey600},
    texcsstyle    = *\color{MaterialDeepOrange},
    emphstyle     = [1]\color{MaterialGreen800},
    emphstyle     = [2]\color{MaterialTeal}
  }
%    \end{macrocode}
% \end{macro}
%
% \begin{environment}{shellexample}
% \begin{environment}{latexexample}
% Shell 和 \LaTeX{} 示例代码。
%    \begin{macrocode}
\lstnewenvironment{shellexample}[1][]{%
  \lstset{style=style@shell, #1}}{}
\lstnewenvironment{latexexample}[1][]{%
  \lstset{style=style@latex, #1}}{}
%    \end{macrocode}
% \end{environment}
% \end{environment}
%
% \begin{environment}{fdusyntax}
% 语法说明。用于代替 \cls{l3doc} 中的 \env{syntax} 环境。
%    \begin{macrocode}
\lstnewenvironment{fdusyntax}[1][]{%
  \lstset{style=style@syntax, #1}\vspace{-1.8ex}}{}
%</doc>
%    \end{macrocode}
% \end{environment}
%
% \subsubsection{\pkg{latexmk} 配置文件}
%
%    \begin{macrocode}
%<*latexmk|latexmk-en>
# Latexmk configuration file.

# Use XeLaTeX to compile.
$pdf_mode = 5;

# Process index.
$makeindex = 'zhmakeindex -s gind.ist %O -o %D %S';

# Show CPU time used.
$show_time = 1;
%</latexmk|latexmk-en>
%<*latexmk>

# Process glossary (change history).
add_cus_dep('glo', 'gls', 0, 'makeglo2gls');
sub makeglo2gls {
    system("zhmakeindex -s gglo.ist -o \"$_[0].gls\"
        -t \"$_[0].glg\" \"$_[0].glo\"");
}
%</latexmk>
%    \end{macrocode}
%
% \clearpage
%
% \end{implementation}
%

%
% \end{documentation}
%
% \StopEventually{}
%
%^^A \DisableImplementation
%
% \begin{implementation}
%
%^^A 代码部分的页边距
% \newgeometry{
%   left   = 2.50 in,
%   right  = 1.00 in,
%   top    = 1.25 in,
%   bottom = 1.00 in
% }
%
% \section{实现细节}
% 本模板使用 \LaTeX3 语法编写,依赖 \pkg{expl3} 环境,
% 并需调用 \pkg{l3packages} 中的相关宏包。
%
% 按照 \LaTeX3 语法,代码中的空格、换行、回车与制表符完全忽略,
% 而下划线“|_|”和冒号“|:|”则可作为一般字母使用。
% 正常的空格可以使用“|~|”代替;至于 |~| 原来所表示的“带子”,
% 则要用 \LaTeXe{} 的原始命令 \tn{nobreakspace} 代替。
%
% 以下代码中有一些形如
% \textcolor[HTML]{2E3191}{\textsf{\textlangle *class\textrangle}}
% 的标记,这是 \pkg{DocStrip} 中的“guard”,用来选择性地提取文件。
% “\textsf{*}”和“\textsf{/}”分别表示该部分的开始和结束。不含
% “\textsf{*}”和“\textsf{/}”的 guard 出现在行号右侧,它们用来确定
% 单独一行代码的归属。这些 guard 的颜色深浅不一,用以明确嵌套关系。
%
% 另有若干形如
% \textcolor{MaterialPink}{\textsf{\textlangle @@=fdu\textrangle}}
% 的 guard ,它们由 \pkg{l3docstrip} 定义,用来指示名字空间(模块)。
%
%    \begin{macrocode}
%<@@=fdu>
%<*class|class-en>
\RequirePackage { xparse, l3keys2e }
%    \end{macrocode}
%
% 载入参数配置文件。
%    \begin{macrocode}
\file_input:n { fduthesis.def }
\file_input:n { fduthesis-user.def }
%    \end{macrocode}
%
% \subsection{内部变量声明}
% \begin{macro}{\l_@@_tmpa_box,
%   \l_@@_tmpa_dim,\l_@@_tmpb_dim,
%   \l_@@_tmpa_tl,\l_@@_tmpb_tl,
%^^A   \l_@@_tmpa_int,
%   \l_@@_tmpa_clist,\l_@@_tmpb_clist}
% 临时变量。
%    \begin{macrocode}
\box_new:N   \l_@@_tmpa_box
\dim_new:N   \l_@@_tmpa_dim
\dim_new:N   \l_@@_tmpb_dim
\tl_new:N    \l_@@_tmpa_tl
\tl_new:N    \l_@@_tmpb_tl
% \int_new:N   \l_@@_tmpa_int
\clist_new:N \l_@@_tmpa_clist
\clist_new:N \l_@@_tmpb_clist
%    \end{macrocode}
% \end{macro}
%
% \begin{macro}{\g_@@_to_book_clist}
% 保存由 \cls{fduthesis} 传入 \cls{book} 文档类的选项列表。
%    \begin{macrocode}
\clist_new:N \g_@@_to_book_clist
%    \end{macrocode}
% \end{macro}
%
% \begin{macro}{\g_@@_twoside_bool}
% 是否开启双页模式(默认打开)。
%    \begin{macrocode}
\bool_new:N \g_@@_twoside_bool
\bool_set_true:N \g_@@_twoside_bool
%    \end{macrocode}
% \end{macro}
%
% \begin{macro}{\g_@@_no_fonts_bool}
% 是否禁用默认字体设置。
%    \begin{macrocode}
\bool_new:N \g_@@_no_fonts_bool
%    \end{macrocode}
% \end{macro}
%
% \begin{macro}{\g_@@_draft_bool}
% 是否开启草稿模式。
%    \begin{macrocode}
\bool_new:N \g_@@_draft_bool
%    \end{macrocode}
% \end{macro}
%
% \subsection{选项处理}
% 定义 |fdu/option| 键值类。
%    \begin{macrocode}
\keys_define:nn { fdu / option }
  {
%    \end{macrocode}
%
% \begin{macro}{oneside,twoside}
% 设置页面类型为单面或双面。
%    \begin{macrocode}
    oneside .value_forbidden:n = true,
    twoside .value_forbidden:n = true,
    oneside .code:n = {
      \clist_gput_right:Nn \g_@@_to_book_clist { oneside }
      \bool_set_false:N    \g_@@_twoside_bool
    },
    twoside .code:n = {
      \clist_gput_right:Nn \g_@@_to_book_clist { twoside }
      \bool_set_true:N     \g_@@_twoside_bool
    },
%    \end{macrocode}
% \end{macro}
%
% \begin{macro}{nofonts}
% 是否禁用默认字体设置(默认关闭)。
%    \begin{macrocode}
    nofonts .choice:,
    nofonts .bool_set:N = \g_@@_no_fonts_bool,
    nofonts .default:n = true,
    nofonts .initial:n = false,
%    \end{macrocode}
% \end{macro}
%
% \begin{macro}{draft}
% 是否开启草稿模式(默认关闭)。
%    \begin{macrocode}
    draft .choice:,
    draft / true  .code:n = {
      \bool_set_true:N     \g_@@_draft_bool
      \clist_gput_right:Nn \g_@@_to_book_clist { draft }
    },
    draft / false .code:n = {
      \bool_set_false:N    \g_@@_draft_bool
    },
    draft .default:n = true,
    draft .initial:n = false
  }
%    \end{macrocode}
% \end{macro}
%
% 将文档类选项传给 |fdu/option|。
%    \begin{macrocode}
\ProcessKeysOptions { fdu / option }
%    \end{macrocode}
%
% \subsection{载入宏包、文档类}
% 载入 \cls{book} 标准文档类,并传入相应的选项。
%    \begin{macrocode}
\PassOptionsToClass { \g_@@_to_book_clist } { book }
\LoadClass { book }
%    \end{macrocode}
%
% \XeLaTeX{} \LuaLaTeX{} 下的字体选取。|no-math| 选项保证该宏包不参与
% 数学字体的设置。
%    \begin{macrocode}
\RequirePackage [ no-math ] { fontspec }
%    \end{macrocode}
%
% 中文排版支持。使用 \XeLaTeX{} 编译时,底层将调用 \pkg{xeCJK} 宏包;
% 使用 \LuaLaTeX{} 编译时,底层则将调用 \pkg{luatexja} 宏包。
% TODO(20170722): 英文模板中文字体的处理。
%    \begin{macrocode}
\RequirePackage [
    UTF8,
%<class-en>    scheme     = plain,
    heading    = true,
%<class>    fontset    = none,
%<class-en>    fontset    = fandol,
    zihao      = \c_@@_def_font_size_tl,
%<class>    linespread = \c_@@_def_line_spread_fp
  ] { ctex }
%    \end{macrocode}
%
% 本模板使用 Unicode 编码的 OpenType 数学字体,此功能由
% \pkg{unicode-math} 宏包实现。为防止冲突,\pkg{amsmath} 必须在它
% 之前引入。
%    \begin{macrocode}
\RequirePackage { amsmath }
\RequirePackage { unicode-math }
%    \end{macrocode}
%
% 设置页面尺寸与页眉页脚。
%    \begin{macrocode}
\RequirePackage { geometry, fancyhdr }
%    \end{macrocode}
%
% 处理脚注。|perpage| 选项将使脚注编号每页清零。
%    \begin{macrocode}
\RequirePackage [ perpage ] { footmisc }
%    \end{macrocode}
%
% 定理环境。
%    \begin{macrocode}
\RequirePackage [ amsmath, thmmarks ] { ntheorem }
%    \end{macrocode}
%
% 插图、表格与浮动体控制。
%    \begin{macrocode}
\RequirePackage { graphicx }
\RequirePackage { longtable }
\RequirePackage { caption }
%    \end{macrocode}
%
% 下划线。
%    \begin{macrocode}
\RequirePackage { ulem }
%    \end{macrocode}
%
% \subsection{页面布局}
% 利用 \pkg{geometry} 宏包设置纸张大小、页面边距以及页眉高度。
%    \begin{macrocode}
\geometry
  {
    paper      = \c_@@_def_paper_size_tl,
    top        = \c_@@_def_page_margin_top_dim,
    bottom     = \c_@@_def_page_margin_bottom_dim,
    left       = \c_@@_def_page_margin_left_dim,
    right      = \c_@@_def_page_margin_right_dim,
    headheight = \c_@@_def_header_height_dim
  }
%    \end{macrocode}
%
% 草稿模式下显示页面边框及页眉、页脚线 。
%    \begin{macrocode}
\bool_if:NT \g_@@_draft_bool
  { \geometry { showframe } }
%    \end{macrocode}
%
% \subsection{字体}
% 根据相关规定,数学表达式中,表示变量的拉丁字母和希腊字母均应当
% 使用斜体。
%    \begin{macrocode}
\unimathsetup { math-style = ISO, bold-style = ISO }
%    \end{macrocode}
%
% \subsubsection{西文字体、数学字体配置}
% 定义 |fdu/style| 键值类。
%    \begin{macrocode}
\keys_define:nn { fdu / style }
  {
%    \end{macrocode}
%
% \begin{macro}{style/font}
% 预定义西文字体。等宽字体使用 |Ligatures = NoCommon| 选项以禁用连字。
%    \begin{macrocode}
    font .choice:,
    font .value_required:n = true,
%    \end{macrocode}
% Libertinus 系列。
%    \begin{macrocode}
    font / libertinus .code:n = {
      \setmainfont { Libertinus~ Serif }
      \setsansfont { Libertinus~ Sans  }
      \setmonofont { TeX~ Gyre~ Cursor }
        [ Ligatures = NoCommon ]
      \setmathfont { Libertinus~ Math  }
      \keys_set:nn { fdu / style } { footnotestyle = libertinus }
    },
%    \end{macrocode}
% Latin Modern 系列。
%    \begin{macrocode}
    font / lm .code:n = {
      \setmainfont { Latin~ Modern~ Roman }
      \setsansfont { Latin~ Modern~ Sans  }
      \setmonofont { Latin~ Modern~ Mono  }
      \setmathfont { Latin~ Modern~ Math  }
      \keys_set:nn { fdu / style } { footnotestyle = pifont }
    },
%    \end{macrocode}
% Palatino 系列。
%    \begin{macrocode}
    font / palatino .code:n = {
      \setmainfont { TeX~ Gyre~ Pagella       }
      \setsansfont { TeX~ Gyre~ Heros         }
      \setmonofont { TeX~ Gyre~ Cursor        }
        [ Ligatures = NoCommon ]
      \setmathfont { TeX~ Gyre~ Pagella~ Math }
      \keys_set:nn { fdu / style } { footnotestyle = pifont }
    },
%    \end{macrocode}
% Times Roman 系列。
%    \begin{macrocode}
    font / times .code:n = {
      \setmainfont { XITS              }
      \setsansfont { TeX~ Gyre~ Heros  }
      \setmonofont { TeX~ Gyre~ Cursor }
        [ Ligatures = NoCommon ]
      \setmathfont { XITS~ Math        }
      \keys_set:nn { fdu / style } { footnotestyle = xits }
    },
%    \end{macrocode}
% \end{macro}
%
% \subsubsection{中文字体配置}
% \begin{macro}{style/cjkfont}
% 预定义中文(CJK)字体。对于没有粗体的字体,利用
% |AutoFakeBold = true| 获得伪粗体。
%    \begin{macrocode}
%<*class>
    cjkfont .choice:,
    cjkfont .value_required:n = true,
%    \end{macrocode}
% Adobe 系列。
%    \begin{macrocode}
    cjkfont / adobe .code:n = {
      \setCJKmainfont { Adobe~ Song~ Std~ L     }
        [
          ItalicFont   = Adobe~ Kaiti~ Std~ R,
          AutoFakeBold = true
        ]
      \setCJKsansfont { Adobe~ Heiti~ Std~ R    }
        [
          ItalicFont   = Adobe~ Heiti~ Std~ R,
          AutoFakeBold = true
        ]
      \setCJKmonofont { Adobe~ Fangsong~ Std~ R }
        [
          ItalicFont   = Adobe~ Fangsong~ Std~ R,
          AutoFakeBold = true
        ]
      \setCJKfamilyfont { song } { Adobe~ Song~     Std~ L }
      \setCJKfamilyfont { hei  } { Adobe~ Heiti~    Std~ R }
      \setCJKfamilyfont { fang } { Adobe~ Fangsong~ Std~ R }
      \setCJKfamilyfont { kai  } { Adobe~ Kaiti~    Std~ R }
    },
%    \end{macrocode}
% Fandol 系列。
%    \begin{macrocode}
    cjkfont / fandol .code:n = {
      \setCJKmainfont { FandolSong }
        [
          ItalicFont   = FandolKai
        ]
      \setCJKsansfont { FandolHei  }
        [
          ItalicFont   = FandolHei
        ]
      \setCJKmonofont { FandolFang }
        [
          ItalicFont   = FandolFang,
          AutoFakeBold = true
        ]
      \setCJKfamilyfont { song } { FandolSong }
      \setCJKfamilyfont { hei  } { FandolHei  }
      \setCJKfamilyfont { fang } { FandolFang }
      \setCJKfamilyfont { kai  } { FandolKai  }
    },
%    \end{macrocode}
% 方正系列。
%    \begin{macrocode}
    cjkfont / founder .code:n = {
      \setCJKmainfont { FZShuSong-Z01  }
        [
          BoldFont     = FZXiaoBiaoSong-B05,
          ItalicFont   = FZKai-Z03
        ]
      \setCJKsansfont { FZHei-B01      }
        [
          ItalicFont   = FZHei-B01,
          AutoFakeBold = true
        ]
      \setCJKmonofont { FZFangSong-Z02 }
        [
          ItalicFont   = FZFangSong-Z02,
          AutoFakeBold = true
        ]
      \setCJKfamilyfont { song } { FZShuSong -Z01 }
      \setCJKfamilyfont { hei  } { FZHei     -B01 }
      \setCJKfamilyfont { fang } { FZFangSong-Z02 }
      \setCJKfamilyfont { kai  } { FZKai     -Z03 }
    },
%    \end{macrocode}
% Linux 系列。(没做)
%    \begin{macrocode}
%    cjkfont / linux .code:n = {
%      \setCJKmainfont{ SimSun }
%        [ BoldFont = SimHei, ItalicFont   = KaiTi ]
%    },
%    \end{macrocode}
% Mac (华文)系列。
%    \begin{macrocode}
    cjkfont / mac .code:n = {
      \setCJKmainfont { STSong     }
        [
          BoldFont     = STZhongsong,
          ItalicFont   = STKaiti
        ]
      \setCJKsansfont { STHeiti    }
        [
          ItalicFont   = STHeiti,
          AutoFakeBold = true
        ]
      \setCJKmonofont { STFangsong }
        [
          ItalicFont   = STFangsong,
          AutoFakeBold = true
        ]
      \setCJKfamilyfont { song } { STSong     }
      \setCJKfamilyfont { hei  } { STHeiti    }
      \setCJKfamilyfont { fang } { STFangsong }
      \setCJKfamilyfont { kai  } { STKaiti    }
    },
%    \end{macrocode}
% Windows (中易)系列。
%    \begin{macrocode}
    cjkfont / windows .code:n = {
      \setCJKmainfont { SimSun   }
        [
          ItalicFont   = KaiTi,
          AutoFakeBold = true
        ]
      \setCJKsansfont { SimHei   }
        [
          ItalicFont   = SimHei,
          AutoFakeBold = true
        ]
      \setCJKmonofont { FangSong }
        [
          ItalicFont   = FangSong,
          AutoFakeBold = true
        ]
      \setCJKfamilyfont { song } { SimSun   }
      \setCJKfamilyfont { hei  } { SimHei   }
      \setCJKfamilyfont { fang } { FangSong }
      \setCJKfamilyfont { kai  } { KaiTi    }
    }
  }
%    \end{macrocode}
% \end{macro}
%
% \begin{macro}[TF]{\fdu_family_if_exist:n}
% 判断字体族是否存在。
%    \begin{macrocode}
\prg_new_conditional:Npnn \fdu_family_if_exist:n #1 { TF }
  {
%    \end{macrocode}
% \XeTeX{} 引擎下直接利用 \pkg{xeCJK} 宏包提供的函数。
%    \begin{macrocode}
    \sys_if_engine_xetex:TF
      {
        \xeCJK_family_if_exist:nTF { #1 }
          { \prg_return_true: } { \prg_return_false: }
      }
      {
%    \end{macrocode}
% \LuaTeX{} 引擎下需要利用 \pkg{ctex} 宏包提供的函数。这两个函数
% 的参数略有不同。
%
% TODO(2017/07/22): \cs{ctex_ltj_family_if_exist:xNTF} 将在下一个版本
% 中被改为 |nNTF| 型。需要有最低版本限制。
%    \begin{macrocode}
        \sys_if_engine_luatex:T
          {
            \ctex_ltj_family_if_exist:xNTF { #1 } \l_@@_tmpa_tl
              { \prg_return_true: } { \prg_return_false: }
            \tl_clear:N \l_@@_tmpa_tl
          }
      }
  }
%    \end{macrocode}
% \end{macro}
%
% 以下为宋、黑、仿、楷四种中文字体各定义了一些内部函数。函数
% \cs{xeCJK_family_if_exist:nTF} 是不能展开的,因此此处使用了
% \cs{cs_new_protected:Nn} 禁止展开。
% 见 \url{https://tex.stackexchange.com/q/380612} 和
% \url{https://tex.stackexchange.com/q/380556}。
%
% \begin{macro}{\fdu_cjk_font_song:}
% 宋体,相当于普通罗马字体。
%    \begin{macrocode}
\cs_new_protected:Nn \fdu_cjk_font_song:
  {
    \fdu_family_if_exist:nTF { song }
      { \CJKfamily { song } } { \rmfamily }
  }
%    \end{macrocode}
% \end{macro}
%
% \begin{macro}{\fdu_cjk_font_hei:}
% 黑体,相当于无衬线体。
%    \begin{macrocode}
\cs_new_protected:Nn \fdu_cjk_font_hei:
  {
    \fdu_family_if_exist:nTF { hei  }
      { \CJKfamily { hei  } } { \sffamily }
  }
%    \end{macrocode}
% \end{macro}
%
% \begin{macro}{\fdu_cjk_font_fang:}
% 仿宋,相当于打字机字体(或等宽字体)。
%    \begin{macrocode}
\cs_new_protected:Nn \fdu_cjk_font_fang:
  {
    \fdu_family_if_exist:nTF { fang }
      { \CJKfamily { fang } } { \ttfamily }
  }
%    \end{macrocode}
% \end{macro}
%
% \begin{macro}{\fdu_cjk_font_kai:}
% 楷体,相当于罗马字体的意大利形状(斜体)。
%    \begin{macrocode}
\cs_new_protected:Nn \fdu_cjk_font_kai:
  {
    \fdu_family_if_exist:nTF { kai  }
      { \CJKfamily { kai  } } { \rmfamily \itshape }
  }
%    \end{macrocode}
% \end{macro}
%
% \begin{macro}{\fdu@song,\fdu@hei,\fdu@fang,\fdu@kai}
% \cs{fdu_cjk_font_kai:} 等一些 \LaTeX3 风格的命令在 |toc| 文件里面
% 无法正常使用,因此重新定义为 \LaTeXe{} 风格的命令。
%    \begin{macrocode}
\NewDocumentCommand \fdu@song { } { \fdu_cjk_font_song: }
\NewDocumentCommand \fdu@hei  { } { \fdu_cjk_font_hei:  }
\NewDocumentCommand \fdu@fang { } { \fdu_cjk_font_fang: }
\NewDocumentCommand \fdu@kai  { } { \fdu_cjk_font_kai:  }
%    \end{macrocode}
% \end{macro}
%
% \subsubsection{字号}
%    \begin{macrocode}
\keys_define:nn { fdu / style }
  {
%</class>
%    \end{macrocode}
%
% \begin{macro}{style/fontsize}
% |fontsize| 不是文档类选项,不能传给 \pkg{ctex} 宏包
% 或者 \cls{book} 文档类,因此只能手动重定义字号命令。
%    \begin{macrocode}
    fontsize .choice:,
    fontsize .value_required:n = true,
    fontsize / -4 .code:n = { },
%    \end{macrocode}
% \end{macro}
%
% \begin{macro}{\tiny,\scriptsize,\footnotesize,\small,
%   \normalsize,\large,\Large,\LARGE,\huge,\Huge}
% 默认使用小四号字,所以只有五号字需要重新设置。
%    \begin{macrocode}
    fontsize /  5 .code:n = {
      \RenewDocumentCommand \tiny         { } { \zihao {  7 } }
      \RenewDocumentCommand \scriptsize   { } { \zihao { -6 } }
      \RenewDocumentCommand \footnotesize { } { \zihao {  6 } }
      \RenewDocumentCommand \small        { } { \zihao { -5 } }
      \RenewDocumentCommand \normalsize   { } { \zihao {  5 } }
      \RenewDocumentCommand \large        { } { \zihao { -4 } }
      \RenewDocumentCommand \Large        { } { \zihao { -3 } }
      \RenewDocumentCommand \LARGE        { } { \zihao { -2 } }
      \RenewDocumentCommand \huge         { } { \zihao {  2 } }
      \RenewDocumentCommand \Huge         { } { \zihao {  1 } }
%<class>    },
%<class-en>    }
%    \end{macrocode}
% \end{macro}
%
% \subsubsection{句号}
% \begin{macro}{style/fullwidthstop}
% 设置句号形状(圆圈或是圆点)。
% 本模板采用的实现方法是将“\symbol{"3002}”设置为活动符,
% 并定义为句点“\symbol{"FF0E}”。
%
% \pkg{xeCJK} 宏包提供了 |Mapping = fullwidth-stop| 和 |full-stop|
% 选项,也能实现两种句号的切换。此种方法是基于字体映射实现的,
% 当需要同时使用两种句号的时候将会带来不便。另外通过 \LuaLaTeX{}
% 编译时,底层使用 \pkg{luatexja} 而非\pkg{xeCJK},也必须采取
% \tn{catcode} 的手段来切换。
%    \begin{macrocode}
%<*class>
    fullwidthstop .choice:,
    fullwidthstop / true  .code:n = {
      \char_set_active_eq:nN { "3002 }
        \c_@@_full_stop_fullwidth_tl
      \char_set_catcode_active:n { "3002 }
    },
    fullwidthstop / false .code:n = { },
    fullwidthstop .default:n  = true
%</class>
  }
%    \end{macrocode}
% \end{macro}
%
% \subsection{章节标题结构}
% |\keys_set:nn {ctex}| 实际相当于 \cs{ctexset}。
%    \begin{macrocode}
\keys_set:nn { ctex }
  {
%    \end{macrocode}
%
% 设置章(chapter)、节(section)与小节(sub-section)标题样式。
% 此处使用 |fixskip = true| 选项来抑制前后的多余间距。
%    \begin{macrocode}
    chapter = {
%<class>      format      = \c_@@_def_chapter_format_tl,
%<*class-en>
      format      = \c_@@_def_chapter_format_en_tl,
      nameformat  = \c_@@_def_chapter_name_format_en_tl,
      titleformat = \c_@@_def_chapter_title_format_en_tl,
      aftername   = \c_@@_def_chapter_after_name_en_tl,
%</class-en>
      beforeskip  = \c_@@_def_chapter_before_sep_tl,
      afterskip   = \c_@@_def_chapter_after_sep_tl,
      number      = { \arabic { chapter } },
      fixskip     = true
    },
    section = {
%<class>      format      = \c_@@_def_section_format_tl,
%<class-en>      format      = \c_@@_def_section_format_en_tl,
      beforeskip  = \c_@@_def_section_before_sep_tl,
      afterskip   = \c_@@_def_section_after_sep_tl,
      fixskip     = true
    },
    subsection = {
%<class>      format      = \c_@@_def_subsection_format_tl,
%<class-en>      format      = \c_@@_def_subsection_format_en_tl,
      beforeskip  = \c_@@_def_subsection_before_sep_tl,
      afterskip   = \c_@@_def_subsection_after_sep_tl,
      fixskip     = true
    }
  }
%    \end{macrocode}
%
% \subsection{页眉页脚}
% 清除默认页眉页脚格式。
%    \begin{macrocode}
\fancyhf { }
%    \end{macrocode}
%
% \begin{macro}{\l_@@_header_center_mark_tl}
% 保存中间页眉的文字。正文中设置为空,目录、摘要、符号表等设置为
% 相应标题。
%    \begin{macrocode}
\tl_new:N \l_@@_header_center_mark_tl
%    \end{macrocode}
% \end{macro}
%
% 构建页眉,要在单面或双面下分别设置。
% \cs{fancyhead} 的选项中,|E| 和 |O| 分别表示偶数(even)和奇数
% (odd), 而 |L|、|R| 和 |C| 则分别表示左(left)、右(right)
% 和中间(center)。按照通常的排版规则,在双面模式下,偶数页的中间
% 页眉文字在左,奇数页则在右。单面模式下,左右页眉都要显示。
%    \begin{macrocode}
\bool_if:NTF \g_@@_twoside_bool
%<*class>
  {
    \fancyhead [ EL ] { \small \nouppercase { \fdu@kai \leftmark  } }
    \fancyhead [ OR ] { \small \nouppercase { \fdu@kai \rightmark } }
  }
  {
    \fancyhead [ L ] { \small \nouppercase { \fdu@kai \leftmark  } }
    \fancyhead [ R ] { \small \nouppercase { \fdu@kai \rightmark } }
    \fancyhead [ C ]
      {
        \small \nouppercase
          { \fdu@kai \l_@@_header_center_mark_tl }
      }
  }
%</class>
%<*class-en>
  {
    \fancyhead [ EL ] { \small \nouppercase { \itshape \leftmark  } }
    \fancyhead [ OR ] { \small \nouppercase { \itshape \rightmark } }
  }
  {
    \fancyhead [ L ] { \small \nouppercase { \itshape \leftmark  } }
    \fancyhead [ R ] { \small \nouppercase { \itshape \rightmark } }
    \fancyhead [ C ]
      {
        \small \nouppercase
          { \itshape \l_@@_header_center_mark_tl }
      }
  }
%</class-en>
%    \end{macrocode}
%
% 构建页脚,用来显示页码。选项 |C| 表示居中(center)。
%    \begin{macrocode}
\fancyfoot [ C ] { \small \thepage }
%    \end{macrocode}
%
% 关闭横线显示(未启用)。
%    \begin{macrocode}
% \RenewDocumentCommand \headrulewidth { } { 0 pt }
%    \end{macrocode}
%
% \begin{macro}{\fdu_front_matter_header:n}
% 在单页模式下,设置前导部分(包括目录、摘要、符号表等)的页眉中间
% 为相应标题,左右为空。
%    \begin{macrocode}
\cs_new:Npn \fdu_front_matter_header:n #1
  {
    \bool_if:NTF \g_@@_twoside_bool
      { \markboth { #1 } { #1 } }
      {
        \markboth { } { }
        \tl_gset:Nn \l_@@_header_center_mark_tl { #1 }
      }
  }
%    \end{macrocode}
% \end{macro}
%
% \begin{macro}{\cleardoublepage}
% 重定义 \tn{cleardoublepage},使得偶数页面在没有内容时也不显示
% 页眉页脚。\\
% 见 http://tex.stackexchange.com/q/1681 \\
% 最后清空中间页眉,确保正文部分页眉显示正确。
%    \begin{macrocode}
\RenewDocumentCommand \cleardoublepage { }
  {
    \clearpage
    \bool_if:NT \g_@@_twoside_bool
      {
        \int_if_odd:nF \c@page
          { \hbox:n { } \thispagestyle { empty } \newpage }
      }
    \tl_gset:Nn \l_@@_header_center_mark_tl { }
  }
%    \end{macrocode}
% \end{macro}
%
% \pkg{ctex} 宏包使用 |heading| 选项后,会把页面格式设置为
% |headings|。因此必须在 \pkg{ctex} 调用之后重新设置 \cs{pagestyle}
% 为 |fancy|。
%    \begin{macrocode}
\pagestyle { fancy }
%    \end{macrocode}
%
% \begin{macro}{\sectionmark}
% 重定义右侧页眉格式(否则貌似少了一个空格)。
%    \begin{macrocode}
\RenewDocumentCommand \sectionmark { m }
  { \markright { \CTEXthesection \negthinspace \quad #1 } }
%    \end{macrocode}
% \end{macro}
%
% \subsection{脚注}
% \changes{v0.3}{2017/02/21}{支持脚注。}
% \subsubsection{编号样式}
% \begin{macro}{\c_@@_fn_style_plain_tl,
%   \c_@@_fn_style_libertinus_tl,
%   \c_@@_fn_style_libertinus_negative_tl,
%   \c_@@_fn_style_libertinus_sans_tl,
%   \c_@@_fn_style_pifont_tl,
%   \c_@@_fn_style_pifont_negative_tl,
%   \c_@@_fn_style_pifont_sans_tl,
%   \c_@@_fn_style_pifont_sans_negative_tl,
%   \c_@@_fn_style_xits_tl,
%   \c_@@_fn_style_xits_sans_tl,
%   \c_@@_fn_style_xits_sans_negative_tl}
% 各种脚注编号样式的名称。
%    \begin{macrocode}
\tl_const:Nn \c_@@_fn_style_plain_tl           { plain           }
\tl_const:Nn \c_@@_fn_style_libertinus_tl      { libertinus      }
\tl_const:Nn \c_@@_fn_style_libertinus_negative_tl
  { libertinus* }
\tl_const:Nn \c_@@_fn_style_libertinus_sans_tl { libertinus-sans }
\tl_const:Nn \c_@@_fn_style_pifont_tl          { pifont          }
\tl_const:Nn \c_@@_fn_style_pifont_negative_tl { pifont*         }
\tl_const:Nn \c_@@_fn_style_pifont_sans_tl     { pifont-sans     }
\tl_const:Nn \c_@@_fn_style_pifont_sans_negative_tl
  { pifont-sans* }
\tl_const:Nn \c_@@_fn_style_xits_tl            { xits            }
\tl_const:Nn \c_@@_fn_style_xits_sans_tl       { xits-sans       }
\tl_const:Nn \c_@@_fn_style_xits_sans_negative_tl
  { xits-sans* }
%    \end{macrocode}
% \end{macro}
%
% \begin{macro}{\l_@@_fn_style_tl}
% 保存当前使用的脚注编号样式。
%    \begin{macrocode}
\tl_new:N \l_@@_fn_style_tl
%    \end{macrocode}
% \end{macro}
%
% 在 |nofonts=true| 的情况下设置脚注样式为 |plain|。
%    \begin{macrocode}
\bool_if:NT \g_@@_no_fonts_bool
  { \tl_set_eq:NN \l_@@_fn_style_tl \c_@@_fn_style_plain_tl }
%    \end{macrocode}
%
% 脚注样式也归入 |fdu/style| 键值类。
%    \begin{macrocode}
\keys_define:nn { fdu / style }
  {
%    \end{macrocode}
%
% \begin{macro}{style/footnotestyle}
% 脚注类型共分四大类:
% |plain|:使用当前字体;
% |libertinus|:取自 Libertinus Serif 和 Libertinus Sans 字体;
% |pifont|:使用 \pkg{pifont} 宏包;
% |xits|:取自 XITS-Math 字体。
%
% 不带任何修饰的为衬线阳文符号,带“|sans|”的为无衬线符号,带“|*|”的
% 为阴文版本。
%    \begin{macrocode}
    footnotestyle .choices:nn = {
      plain,
      libertinus, libertinus*, libertinus-sans,
      pifont,     pifont*,     pifont-sans,     pifont-sans*,
      xits,                    xits-sans,       xits-sans*
    }
%    \end{macrocode}
%
% 若使用 |pifont| 类型,则需引入 \pkg{pifont} 宏包;
% 若使用 |xits| 类型,则需调用 XITS Math 字体。
%    \begin{macrocode}
    {
      \tl_gset_eq:NN \l_@@_fn_style_tl \l_keys_choice_tl
      \int_compare:nTF
        { 5 <= \l_keys_choice_int <= 8 }
        { \RequirePackage { pifont } }
        {
          \int_compare:nT
            { 9 <= \l_keys_choice_int <= 11 }
            { \setmathfont { XITS~ Math } [ version = fn-XITS ] }
        }
    },
    footnotestyle .value_required:n = true
  }
%    \end{macrocode}
% \end{macro}
%
% \begin{macro}{\@@_fn_symbol_libertinus:n}
% |libertinus| 普通版。\numrange{1}{20} 为数字,\numrange{21}{46}
% 为小写英文字母,\numrange{47}{72} 为大写英文字母。
%    \begin{macrocode}
\cs_new:Npn \@@_fn_symbol_libertinus:n #1
  {
    \int_compare:nTF { #1 >= 21 }
      {
        \int_compare:nTF { #1 >= 47 }
          { \symbol { \int_eval:n { "24B6 - 47 + #1 } } }
          { \symbol { \int_eval:n { "24D0 - 21 + #1 } } }
      }
      { \symbol { \int_eval:n { "2460 - 1 + #1 } } }
  }
%    \end{macrocode}
% \end{macro}
%
% \begin{macro}{\@@_fn_symbol_libertinus_negative:n}
% |libertinus| 阴文衬线版。只含 \numrange{1}{20}。
%    \begin{macrocode}
\cs_new:Npn \@@_fn_symbol_libertinus_negative:n #1
  {
    \int_compare:nTF { #1 >= 11 }
      { \symbol { \int_eval:n { "24EB - 11 + #1 } } }
      { \symbol { \int_eval:n { "2776 -  1 + #1 } } }
  }
%    \end{macrocode}
% \end{macro}
%
% \begin{macro}{\@@_fn_symbol_libertinus_sans:n}
% |libertinus| 阳文无衬线版。符号排列与普通版相同。
%    \begin{macrocode}
\cs_new_eq:NN \@@_fn_symbol_libertinus_sans:n
  \@@_fn_symbol_libertinus:n
%    \end{macrocode}
% \end{macro}
%
% \begin{macro}{\@@_fn_symbol_pifont:n}
% |pifont| 普通版。以下四种都只包含 \numrange{1}{10}。
%    \begin{macrocode}
\cs_new:Npn \@@_fn_symbol_pifont:n #1
  { \ding { \int_eval:n { 171 + #1 } } }
%    \end{macrocode}
% \end{macro}
%
% \begin{macro}{\@@_fn_symbol_pifont_negative:n}
% |pifont| 阴文衬线版。
%    \begin{macrocode}
\cs_new:Npn \@@_fn_symbol_pifont_negative:n #1
  { \ding { \int_eval:n { 181 + #1 } } }
%    \end{macrocode}
% \end{macro}
%
% \begin{macro}{\@@_fn_symbol_pifont_sans:n}
% |pifont| 阳文无衬线版。
%    \begin{macrocode}
\cs_new:Npn \@@_fn_symbol_pifont_sans:n #1
  { \ding { \int_eval:n { 191 + #1 } } }
%    \end{macrocode}
% \end{macro}
%
% \begin{macro}{\@@_fn_symbol_pifont_sans_negative:n}
% |pifont| 阴文无衬线版。
%    \begin{macrocode}
\cs_new:Npn \@@_fn_symbol_pifont_sans_negative:n #1
  { \ding { \int_eval:n { 201 + #1 } } }
%    \end{macrocode}
% \end{macro}
%
% \begin{macro}{\@@_fn_symbol_xits:n}
% |xits| 普通版。\numrange{1}{9} 为数字,\numrange{10}{35} 为小写
% 英文字母,\numrange{36}{61} 为大写英文字母。
%    \begin{macrocode}
\cs_new:Npn \@@_fn_symbol_xits:n #1
  {
    \int_compare:nTF { #1 >= 10 }
      {
        \int_compare:nTF { #1 >= 36 }
          { \symbol { \int_eval:n { "24B6 - 36 + #1 } } }
          { \symbol { \int_eval:n { "24D0 - 10 + #1 } } }
      }
      { \symbol { \int_eval:n { "2460 - 1 + #1 } } }
  }
%    \end{macrocode}
% \end{macro}
%
% \begin{macro}{\@@_fn_symbol_xits_sans:n}
% |xits| 阳文无衬线版。只包含 \numrange{1}{10}。
%    \begin{macrocode}
\cs_new:Npn \@@_fn_symbol_xits_sans:n #1
  { \symbol { \int_eval:n { "2780 - 1 + #1 } } }
%    \end{macrocode}
% \end{macro}
%
% \begin{macro}{\@@_fn_symbol_xits_sans_negative:n}
% |xits| 阴文无衬线版。也只包含 \numrange{1}{10}。
%    \begin{macrocode}
\cs_new:Npn \@@_fn_symbol_xits_sans_negative:n #1
  { \symbol { \int_eval:n { "278A - 1 + #1 } } }
%    \end{macrocode}
% \end{macro}
%
% \begin{macro}{\thefootnote,\fdu_footnote_number:n}
% 重定义脚注编号。
%    \begin{macrocode}
\RenewDocumentCommand \thefootnote { }
  { \fdu_footnote_number:n { \value { footnote } } }
\cs_new:Npn \fdu_footnote_number:n #1
  {
    \tl_case:NnF \l_@@_fn_style_tl
      {
%    \end{macrocode}
%
% |plain| 类型直接使用计数器 |footnote| 的值。
%    \begin{macrocode}
        \c_@@_fn_style_plain_tl
          { \int_use:N #1 }
%    \end{macrocode}
%
% |libertinus| 类型需要使用 Libertinus Serif 或
% Libertinus Sans 字体。
%    \begin{macrocode}
        \c_@@_fn_style_libertinus_tl
          {
            \fontspec { Libertinus~ Serif }
            \@@_fn_symbol_libertinus:n { #1 }
          }
        \c_@@_fn_style_libertinus_negative_tl
          {
            \fontspec { Libertinus~ Serif }
            \@@_fn_symbol_libertinus_negative:n { #1 }
          }
        \c_@@_fn_style_libertinus_sans_tl
          {
            \fontspec { Libertinus~ Sans }
            \@@_fn_symbol_libertinus_sans:n { #1 }
          }
%    \end{macrocode}
%
% |pifont| 类型无需进行额外的操作。
%    \begin{macrocode}
        \c_@@_fn_style_pifont_tl
          { \@@_fn_symbol_pifont:n { #1 } }
        \c_@@_fn_style_pifont_negative_tl
          { \@@_fn_symbol_pifont_negative:n { #1 } }
        \c_@@_fn_style_pifont_sans_tl
          { \@@_fn_symbol_pifont_sans:n { #1 } }
        \c_@@_fn_style_pifont_sans_negative_tl
          { \@@_fn_symbol_pifont_sans_negative:n { #1 } }
%    \end{macrocode}
%
% |xits| 类型需要临时切换数学字体。
%    \begin{macrocode}
        \c_@@_fn_style_xits_tl
          {
            \mathversion { fn-XITS }
            $ \@@_fn_symbol_xits:n { #1 } $
          }
        \c_@@_fn_style_xits_sans_tl
          {
            \mathversion { fn-XITS }
            $ \@@_fn_symbol_xits_sans:n { #1 } $
          }
        \c_@@_fn_style_xits_sans_negative_tl
          {
            \mathversion { fn-XITS }
            $ \@@_fn_symbol_xits_sans_negative:n { #1 } $
          }
      }
%    \end{macrocode}
%
% 变量 \cs{l_@@_fn_style_tl} 保存的类型未知时,默认使用 |plain|
% 类型。
%    \begin{macrocode}
      { \int_use:N #1 }
  }
%    \end{macrocode}
% \end{macro}
%
% \subsubsection{整体样式}
% \begin{macro}{\@makefntext}
% 重定义内部脚注文字命令。
%    \begin{macrocode}
\RenewDocumentCommand \@makefntext { +m }
  {
%    \end{macrocode}
%
% 脚注编号不使用上标,宽度为 \SI{1.5}{em}。
%
% 见 http://tex.stackexchange.com/q/19844
%    \begin{macrocode}
    \dim_set:Nn \l_@@_tmpa_dim { \textwidth - 1.5 em }
    \makebox [ 1.5 em ] [ l ] { \@thefnmark }
%    \end{macrocode}
%
% 脚注文字用 |parbox| 封装。首段无缩进,第二段起缩进 \SI{2}{em}。
%    \begin{macrocode}
    \parbox [ t ] { \l_@@_tmpa_dim }
      {
        \everypar { \hspace* { 2 em } }
        \hspace* { -2 em } #1
      }
  }
%    \end{macrocode}
% \end{macro}
%
% \subsection{定理环境}
% \changes{v0.3}{2017/05/07}{新增定理环境。}
% \begin{macro}{\c_@@_thm_style_plain_clist,
%   \c_@@_thm_style_break_clist}
% 保存 |plain|、|break| 两种类型的定理样式名称。
%    \begin{macrocode}
\clist_const:Nn \c_@@_thm_style_plain_clist
  { plain, margin, change }
\clist_const:Nn \c_@@_thm_style_break_clist
  { break, marginbreak, changebreak }
%    \end{macrocode}
% \end{macro}
%
% \begin{macro}{\l_@@_thm_style_tl,
%   \l_@@_thm_header_font_tl,
%   \l_@@_thm_body_font_tl,
%   \l_@@_thm_qed_tl,
%   \l_@@_thm_counter_tl}
% 定理所需的一些字段。
%    \begin{macrocode}
\tl_new:N \l_@@_thm_style_tl
\tl_new:N \l_@@_thm_header_font_tl
\tl_new:N \l_@@_thm_body_font_tl
\tl_new:N \l_@@_thm_qed_tl
\tl_new:N \l_@@_thm_counter_tl
%    \end{macrocode}
% \end{macro}
%
% \begin{macro}{theorem/style,
%   theorem/headerfont,
%   theorem/bodyfont,
%   theorem/qed,
%   theorem/counter}
% 定义 |fdu/theorem| 键值类。
%    \begin{macrocode}
\keys_define:nn { fdu / theorem }
  {
    style      .tl_set:N  = \l_@@_thm_style_tl,
    headerfont .tl_set:N  = \l_@@_thm_header_font_tl,
    bodyfont   .tl_set:N  = \l_@@_thm_body_font_tl,
    qed        .tl_set:N  = \l_@@_thm_qed_tl,
    counter    .tl_set:N  = \l_@@_thm_counter_tl
  }
%    \end{macrocode}
% \end{macro}
%
% \begin{macro}{\fdu_thm_new:nnnn,\fdu_thm_new:Vnnn}
% 带编号的定理环境。|#1| = 样式, |#2| = 计数器,|#3| = 定理环境名称,
% |#4| = 定理头文字。
%    \begin{macrocode}
\cs_new:Npn \fdu_thm_new:nnnn #1#2#3#4
  {
    \theoremstyle { #1 }
    \newtheorem { #3 } { #4 } [ #2 ]
  }
\cs_generate_variant:Nn \fdu_thm_new:nnnn { Vnnn }
%    \end{macrocode}
% \end{macro}
%
% \begin{macro}{\fdu_thm_new_no_number:nnn,
%   \fdu_thm_new_no_number:Vnn}
% 不带编号的定理环境。|#1| = 样式, |#2| = 定理环境名称,
% |#3| = 定理头文字。
%    \begin{macrocode}
\cs_new:Npn \fdu_thm_new_no_number:nnn #1#2#3
  {
    \theoremstyle { #1 }
    \newtheorem { #2 } { #3 }
  }
\cs_generate_variant:Nn \fdu_thm_new_no_number:nnn { Vnn }
%    \end{macrocode}
% \end{macro}
%
% \begin{macro}{\fdu_thm_set_qed:n,
%   \fdu_thm_set_header_font:n,\fdu_thm_set_body_font:n}
% 封装 \pkg{ntheorem} 宏包提供的若干命令,分别用以设置证毕符号、
% 定理头字体和定理正文字体。
%    \begin{macrocode}
\cs_new:Npn \fdu_thm_set_qed:n         #1
  { \theoremsymbol     { #1 } }
\cs_new:Npn \fdu_thm_set_header_font:n #1
  { \theoremheaderfont { #1 } }
\cs_new:Npn \fdu_thm_set_body_font:n   #1
  { \theorembodyfont   { #1 } }
%    \end{macrocode}
% \end{macro}
%
% \begin{macro}{\fdu_thm_set_qed:V,
%   \fdu_thm_set_header_font:V,\fdu_thm_set_body_font:V}
% 生成以上三个函数的变体。
%    \begin{macrocode}
\cs_generate_variant:Nn \fdu_thm_set_qed:n         { V }
\cs_generate_variant:Nn \fdu_thm_set_header_font:n { V }
\cs_generate_variant:Nn \fdu_thm_set_body_font:n   { V }
%    \end{macrocode}
% \end{macro}
%
% \begin{macro}{\fdunewtheorem,\fdunewtheorem*}
% 创建新的定理环境。
%    \begin{macrocode}
\NewDocumentCommand \fdunewtheorem { s o m m }
  {
%    \end{macrocode}
%
% 默认情况下,由 \cs{fdunewtheorem*} 创建的定理其证毕符号为
% \cs{QED},而由 \cs{fdunewtheorem} 创建的则不带证毕符号。符号
% \cs{QED} 由 \pkg{unicode-math} 宏包提供。
%    \begin{macrocode}
    \IfBooleanTF #1
      { \tl_set:Nn \l_@@_thm_qed_tl { \ensuremath { \QED } } }
      { \tl_set:Nn \l_@@_thm_qed_tl { } }
%    \end{macrocode}
%
% 设置默认样式为 |plain|。
%    \begin{macrocode}
    \tl_set:Nn \l_@@_thm_style_tl { plain }
%    \end{macrocode}
%
% 处理可选参数。利用 |fdu/theorem| 键值对设置,并按此修改证毕符号、
% 定理头字体和定理正文字体。
%    \begin{macrocode}
    \IfValueT { #2 }
      { \keys_set:nn { fdu / theorem } { #2 } }
    \fdu_thm_set_header_font:V \l_@@_thm_header_font_tl
    \fdu_thm_set_body_font:V   \l_@@_thm_body_font_tl
    \fdu_thm_set_qed:V         \l_@@_thm_qed_tl
%    \end{macrocode}
%
% \cs{fdunewtheorem} 负责创建编号定理,而 \cs{fdunewtheorem*}
% 则负责创建无编号定理。以下分这两种情况处理。
%    \begin{macrocode}
    \IfBooleanTF #1
      {
%    \end{macrocode}
%
% 带 |*| 的版本原则上只接受 |plain| 和 |break| 两种样式,其余样式
% 将被转换成这两者其中之一。
%
% TODO(20170602): 给出重定义样式的警告。
%    \begin{macrocode}
        \clist_if_in:NVTF
          \c_@@_thm_style_plain_clist
          \l_@@_thm_style_tl
          { \tl_set:Nn \l_@@_thm_style_tl { plain } }
          {
            \clist_if_in:NVTF
              \c_@@_thm_style_break_clist
              \l_@@_thm_style_tl
              { \tl_set:Nn \l_@@_thm_style_tl { break } }
% TODO(20170602): 给出样式未定义错误。
              { }
          }
%    \end{macrocode}
%
% \pkg{ntheorem} 宏包提供的无编号定理带有 |nonumber| 前缀,
% 这里将其加上。
%    \begin{macrocode}
        \tl_put_left:Nn \l_@@_thm_style_tl { nonumber }
        \fdu_thm_new_no_number:Vnn \l_@@_thm_style_tl
          { #3 } { #4 }
      }
      {
%    \end{macrocode}
%
% 不带 |*| 的版本支持不含“|nonumber|”的所有定理样式。
%    \begin{macrocode}
        \clist_clear:N \l_@@_tmpa_clist
        \clist_concat:NNN \l_@@_tmpa_clist
          \c_@@_thm_style_plain_clist \c_@@_thm_style_break_clist
        \clist_if_in:NVF
          \l_@@_tmpa_clist \l_@@_thm_style_tl
% TODO(20170602): 给出样式未定义错误。
          { }
        \fdu_thm_new:Vnnn \l_@@_thm_style_tl
          { \l_@@_thm_counter_tl } { #3 } { #4 }
      }
  }
%    \end{macrocode}
% \end{macro}
%
% \subsection{图表绘制;浮动体}
% \changes{v0.3}{2017/07/09}{支持浮动体。}
% 分别设置浮动体 |figure| 和 |table| 的标题样式。
%    \begin{macrocode}
\captionsetup [ figure ]
  {
    font     = small,
    labelsep = quad
  }
\captionsetup [ table  ]
  {
    font     = { small, sf },
    labelsep = quad
  }
%    \end{macrocode}
%
% \begin{macro}{\thefigure,\thetable}
% 重定义图表编号。
%    \begin{macrocode}
\RenewDocumentCommand \thefigure { }
  { \arabic { chapter } - \arabic { figure } }
\RenewDocumentCommand \thetable  { }
  { \arabic { chapter } - \arabic { table  } }
%    \end{macrocode}
% \end{macro}
%
% \subsection{封面}
% \subsubsection{信息录入}
% \begin{macro}{\l_@@_info_title_tl,
%   \l_@@_info_date_tl,
%   \l_@@_info_author_tl,
%   \l_@@_info_supervisor_tl,
%   \l_@@_info_instructors_clist,
%   \l_@@_info_department_tl,
%   \l_@@_info_major_tl,
%   \l_@@_info_student_id_tl,
%   \l_@@_info_school_id_tl,
%   \l_@@_info_keywords_clist,
%   \l_@@_info_clc_tl}
% 封面所需的一些字段。
%    \begin{macrocode}
\tl_new:N    \l_@@_info_title_tl
\tl_new:N    \l_@@_info_date_tl
\tl_new:N    \l_@@_info_author_tl
\tl_new:N    \l_@@_info_supervisor_tl
\clist_new:N \l_@@_info_instructors_clist
\tl_new:N    \l_@@_info_department_tl
\tl_new:N    \l_@@_info_major_tl
\tl_new:N    \l_@@_info_student_id_tl
\tl_new:N    \l_@@_info_school_id_tl
\clist_new:N \l_@@_info_keywords_clist
\tl_new:N    \l_@@_info_clc_tl
%    \end{macrocode}
% \end{macro}
%
% \begin{macro}{\l_@@_info_title_en_tl,
%   \l_@@_info_author_en_tl,
%   \l_@@_info_supervisor_en_tl,
%   \l_@@_info_department_en_tl,
%   \l_@@_info_major_en_tl,
%   \l_@@_info_keywords_en_clist}
% 对应的英文字段。
%    \begin{macrocode}
\tl_new:N    \l_@@_info_title_en_tl
\tl_new:N    \l_@@_info_author_en_tl
\tl_new:N    \l_@@_info_supervisor_en_tl
\tl_new:N    \l_@@_info_department_en_tl
\tl_new:N    \l_@@_info_major_en_tl
\clist_new:N \l_@@_info_keywords_en_clist
%    \end{macrocode}
% \end{macro}
%
% 定义 |fdu/info| 键值类。
%    \begin{macrocode}
\keys_define:nn { fdu / info }
  {
%    \end{macrocode}
%
% \begin{macro}{info/title,info/title}
% 论文题目。以下带星号的项目均表示相应的英文字段。
%    \begin{macrocode}
    title       .tl_set:N    = \l_@@_info_title_tl,
    title*      .tl_set:N    = \l_@@_info_title_en_tl,
%    \end{macrocode}
% \end{macro}
%
% \begin{macro}{info/date}
% 论文完成日期。
%    \begin{macrocode}
    date        .tl_set:N    = \l_@@_info_date_tl,
%    \end{macrocode}
% \end{macro}
%
% \begin{macro}{info/author,info/author*}
% 作者姓名。
%    \begin{macrocode}
    author      .tl_set:N    = \l_@@_info_author_tl,
    author*     .tl_set:N    = \l_@@_info_author_en_tl,
%    \end{macrocode}
% \end{macro}
%
% \begin{macro}{info/supervisor,info/supervisor*}
% 导师姓名。
%    \begin{macrocode}
    supervisor  .tl_set:N    = \l_@@_info_supervisor_tl,
    supervisor* .tl_set:N    = \l_@@_info_supervisor_en_tl,
%    \end{macrocode}
% \end{macro}
%
% \begin{macro}{info/instructors}
% 指导小组成员。
%    \begin{macrocode}
    instructors .clist_set:N = \l_@@_info_instructors_clist,
%    \end{macrocode}
% \end{macro}
%
% \begin{macro}{info/department,info/department*}
% 院系。
%    \begin{macrocode}
    department  .tl_set:N    = \l_@@_info_department_tl,
    department* .tl_set:N    = \l_@@_info_department_en_tl,
%    \end{macrocode}
% \end{macro}
%
% \begin{macro}{info/major,info/major*}
% 专业。
%    \begin{macrocode}
    major       .tl_set:N    = \l_@@_info_major_tl,
    major*      .tl_set:N    = \l_@@_info_major_en_tl,
%    \end{macrocode}
% \end{macro}
%
% \begin{macro}{info/studentid}
% 学号。
%    \begin{macrocode}
    studentid   .tl_set:N    = \l_@@_info_student_id_tl,
%    \end{macrocode}
% \end{macro}
%
% \begin{macro}{info/schoolid}
% 学校代码。
%    \begin{macrocode}
    schoolid    .tl_set:N    = \l_@@_info_school_id_tl,
%    \end{macrocode}
% \end{macro}
%
% \begin{macro}{info/keywords,info/keywords*}
% 论文关键字。
%    \begin{macrocode}
    keywords    .clist_set:N = \l_@@_info_keywords_clist,
    keywords*   .clist_set:N = \l_@@_info_keywords_en_clist,
%    \end{macrocode}
% \end{macro}
%
% \begin{macro}{info/clc}
% 中国图书馆分类号。
%    \begin{macrocode}
    clc         .tl_set:N    = \l_@@_info_clc_tl
  }
%    \end{macrocode}
% \end{macro}
%
% \subsubsection{密级}
% \changes{v0.3}{2017/07/04}{新增 |secretlevel| 选项。}
% \begin{macro}{\l_@@_secret_bool}
% 是否显示密级。
%    \begin{macrocode}
\bool_new:N \l_@@_secret_bool
%    \end{macrocode}
% \end{macro}
%
% \begin{macro}{\l_@@_info_secret_level_tl}
% 保存当前的密级。
%    \begin{macrocode}
\tl_new:N \l_@@_info_secret_level_tl
%    \end{macrocode}
% \end{macro}
%
% 密级也放在 |fdu/info| 键值类中。
%    \begin{macrocode}
\keys_define:nn { fdu / info }
  {
%    \end{macrocode}
%
% \begin{macro}{info/secretlevel}
% 密级。|none| 表示不涉密,|i|、|ii|、|iii| 分别为秘密、机密、绝密。
% 密级与保密年限中间的五角星符号需要利用 XITS-Math 字体。
%    \begin{macrocode}
    secretlevel .choices:nn  = {
      none, i, ii, iii
    }
    {
      \int_compare:nTF
        { \l_keys_choice_int >= 2 }
        {
          \bool_set_true:N \l_@@_secret_bool
          \setmathfont { XITS~ Math } [ version = secret-XITS ]
          \tl_set:Nn \l_@@_info_secret_level_tl
            {
              \clist_item:Nn \c_@@_def_secret_clist
                { \l_keys_choice_int - 1 }
            }
        }
        { \bool_set_false:N \l_@@_secret_bool }
    },
    secretlevel .value_required:n = true,
%    \end{macrocode}
% \end{macro}
%
% \begin{macro}{info/secretyear}
% 保密年限。
%    \begin{macrocode}
    secretyear .tl_set:N = \l_@@_info_secret_year_tl
  }
%    \end{macrocode}
% \end{macro}
%
% \subsubsection{定义内部函数}
% \begin{macro}{\fdu_spread_box:Nn,\fdu_spread_box:NV,
%   \fdu_spread_box:Nx}
% 分散对齐的盒子。|#1| = 长度, |#2| = 内容。
%
% 利用 \cs{tl_map_inline:nn} 在字符间插入 \tn{hfil};
% 紧随其后的 \tn{unskip} 将会去掉最后一个 \tn{hfil}。\\
% 见 http://tex.stackexchange.com/q/169689
%
% |NV| 以及 |Nx| 版本的命令可以避免展开 token list 时出现坏盒子。
%    \begin{macrocode}
\cs_new:Npn \fdu_spread_box:Nn #1#2
  {
    \makebox [ #1 ] [ s ]
      { \tl_map_inline:nn { #2 } { ##1 \hfil } \unskip }
  }
\cs_generate_variant:Nn \fdu_spread_box:Nn { NV }
\cs_generate_variant:Nn \fdu_spread_box:Nn { Nx }
%    \end{macrocode}
% \end{macro}
%
% \begin{macro}{\fdu_center_box:Nn,\fdu_center_box:NV}
% 居中对齐的盒子。|#1| = 长度, |#2| = 内容。
%    \begin{macrocode}
\cs_new:Npn \fdu_center_box:Nn #1#2
  { \makebox [ #1 ] [ c ] { #2 } }
\cs_generate_variant:Nn \fdu_center_box:Nn { NV }
%    \end{macrocode}
% \end{macro}
%
% \begin{macro}{\fdu_fixed_width_box:Nn}
% 限宽盒子。|#1| = 长度, |#2| = 内容。
%    \begin{macrocode}
\cs_new:Npn \fdu_fixed_width_box:Nn #1#2
  { \parbox { #1 } { #2 } }
%    \end{macrocode}
% \end{macro}
%
% \begin{macro}{\fdu_fixed_width_center_box:Nn,
%   \fdu_fixed_width_center_box:NV}
% 居中对齐的限宽盒子。|#1| = 长度, |#2| = 内容。
%    \begin{macrocode}
\cs_new:Npn \fdu_fixed_width_center_box:Nn #1#2
  { \fdu_fixed_width_box:Nn { #1 } { \centering #2 } }
\cs_generate_variant:Nn \fdu_fixed_width_center_box:Nn { NV }
%    \end{macrocode}
% \end{macro}
%
% \begin{macro}{\fdu_get_text_width:Nn}
% 获取文本宽度,并存入 |dim| 型变量。|#1| = |dim| 型变量,
% |#2| = 内容。
%    \begin{macrocode}
\cs_new:Npn \fdu_get_text_width:Nn #1#2
  {
    \hbox_set:Nn \l_@@_tmpa_box { #2 }
    \dim_set:Nn #1
      { \box_wd:N \l_@@_tmpa_box }
  }
%    \end{macrocode}
% \end{macro}
%
% \begin{macro}{\fdu_get_max_text_width:NN}
% 获取多个文本中的最大宽度,并存入 |dim| 型变量。
% |#1| = |dim| 型变量,|#2| = 文本 |clist|。
%
% 当 \cs{l_@@_tmpa_clist} 非空时,弹出最后一个元素
% 赋给 \cs{l_@@_tmpa_tl},获取其长度后与 |#1| 进行比较,
% 二者中较大的那一个将成为 |#1| 的新值。
% 不断循环,直至 \cs{l_@@_tmpa_clist} 为空。
%    \begin{macrocode}
\cs_new:Npn \fdu_get_max_text_width:NN #1#2
  {
    \group_begin:
    \clist_set_eq:NN \l_@@_tmpa_clist #2
    \bool_until_do:nn { \clist_if_empty_p:N \l_@@_tmpa_clist }
      {
        \clist_pop:NN \l_@@_tmpa_clist \l_@@_tmpa_tl
        \fdu_get_text_width:Nn \l_@@_tmpa_dim
          { \large \l_@@_tmpa_tl }
        \dim_gset:Nn #1
          { \dim_max:nn { #1 } { \l_@@_tmpa_dim } }
      }
    \group_end:
  }
%    \end{macrocode}
% \end{macro}
%
% \begin{macro}{\fdu_blank_underline:N}
% 下划线占位符。|#1| = 长度。
%    \begin{macrocode}
\cs_new:Npn \fdu_blank_underline:N #1
  { \uline { \hbox_to_wd:nn { #1 } { } } }
%    \end{macrocode}
% \end{macro}
%
% \begin{macro}{\fdu_line_spread:N,\fdu_line_spread:n}
% 设置行距。|#1| = 行距倍数。
%    \begin{macrocode}
\cs_new:Npn \fdu_line_spread:N #1
  { \linespread { #1 } \selectfont }
\cs_generate_variant:Nn \fdu_line_spread:N { n }
%    \end{macrocode}
% \end{macro}
%
% \subsubsection{封面各部件}
% \begin{macro}{\@@_cover_id:}
% 右上角的学校代码和学号。
%    \begin{macrocode}
\cs_new:Nn \@@_cover_id:
  {
    \begin{flushright}
      \dim_set:Nn \rightskip { \c_@@_def_cover_id_margin_sep_tl }
      \fdu_fixed_width_box:Nn \c_@@_def_cover_id_width_tl
        {
          \@@_cover_font_size_small:
          \bool_if:NT \l_@@_secret_bool
            {
              \group_begin:
                \sffamily \mathversion { secret-XITS }
                \c_@@_def_name_secret_level_tl
                \c_@@_colon_fullwidth_tl
                \l_@@_info_secret_level_tl
                \c_@@_def_name_secret_star_tl
                \l_@@_info_secret_year_tl
              \group_end:
              \par
            }
          \c_@@_def_name_school_id_tl
          \c_@@_colon_fullwidth_tl
          \l_@@_info_school_id_tl
          \par
          \c_@@_def_name_student_id_tl
          \c_@@_colon_fullwidth_tl
          \l_@@_info_student_id_tl
        }
    \end{flushright}
  }
%    \end{macrocode}
% \end{macro}
%
% \begin{macro}{\@@_cover_logo:}
% 插入校名(毛体“復旦大學”)。
%    \begin{macrocode}
\cs_new:Nn \@@_cover_logo:
  {
    \begin{center}
      \includegraphics [
        width = \c_@@_def_cover_logo_width_tl
      ] { \c_@@_def_cover_logo_file_name_tl }
    \end{center}
  }
%    \end{macrocode}
% \end{macro}
%
% \begin{macro}{\@@_cover_title:}
% 标题,共有四行。第一行是论文类型,第二行是学位类型,三、四两行分别
% 是中英文题目。
%    \begin{macrocode}
\cs_new:Nn \@@_cover_title:
  {
    \begin{center}
      {
        \@@_cover_font_size_huge:
        \fdu_spread_box:NV
          \c_@@_def_cover_type_width_tl
          \c_@@_def_name_thesis_type_tl
      }
      \par \vspace { \c_@@_def_cover_v_sep_iii_tl }
      {
        \@@_cover_font_size_normal:
        \c_@@_def_name_degree_type_tl
      }
      \par \vspace { \c_@@_def_cover_v_sep_iv_tl }
      {
        \@@_cover_font_size_large:  \sffamily
        \fdu_fixed_width_center_box:NV
          \c_@@_def_cover_title_width_tl
          \l_@@_info_title_tl
      }
      \par \vspace { \c_@@_def_cover_v_sep_v_tl }
      {
        \@@_cover_font_size_normal: \bfseries
        \fdu_fixed_width_center_box:Nn
          \c_@@_def_cover_title_en_width_tl
          {
            \fdu_line_spread:N
              \c_@@_def_cover_title_en_line_spread_tl
            \l_@@_info_title_en_tl
          }
      }
    \end{center}
  }
%    \end{macrocode}
% \end{macro}
%
% \begin{macro}{\@@_cover_info:}
% 信息栏。
%    \begin{macrocode}
\cs_new:Nn \@@_cover_info:
  {
    \begin{center}
%    \end{macrocode}
%
% 读取左侧名称字段。
%    \begin{macrocode}
      \clist_set:Nn \l_@@_tmpa_clist
        {
          \c_@@_def_name_department_tl,
          \c_@@_def_name_major_tl,
          \c_@@_def_name_author_tl,
          \c_@@_def_name_supervisor_tl,
          \c_@@_def_name_date_tl,
        }
%    \end{macrocode}
%
% 设置信息栏右侧宽度。读取各字段,并将最宽者的宽度赋给
% \cs{l_@@_tmpb_dim}。
%    \begin{macrocode}
      \clist_set:Nn \l_@@_tmpb_clist
        {
          \l_@@_info_department_tl,
          \l_@@_info_major_tl,
          \l_@@_info_author_tl,
          \l_@@_info_supervisor_tl,
          \l_@@_info_date_tl
        }
      \fdu_get_max_text_width:NN
        \l_@@_tmpb_dim \l_@@_tmpb_clist
%    \end{macrocode}
%
% 在 |minipage| 环境中输出各字段。用循环实现。
%    \begin{macrocode}
      \begin{minipage} [ c ] { \textwidth }
        \centering \@@_cover_font_size_normal:
        \bool_until_do:nn
          { \clist_if_empty_p:N \l_@@_tmpa_clist }
          {
            \clist_pop:NN \l_@@_tmpa_clist \l_@@_tmpa_tl
            \clist_pop:NN \l_@@_tmpb_clist \l_@@_tmpb_tl
            \fdu_spread_box:Nx
              \c_@@_def_cover_info_left_width_tl \l_@@_tmpa_tl
            \c_@@_colon_fullwidth_tl
            \fdu_center_box:NV
              \l_@@_tmpb_dim \l_@@_tmpb_tl
            \par \vspace { \c_@@_def_cover_v_sep_vii_tl }
          }
      \end{minipage}
    \end{center}
  }
%    \end{macrocode}
% \end{macro}
%
% \begin{macro}{\@@_decl_text:NNn}
% 构建声明文本。|#1| = 标题,|#2| = 内容,|#3| = 签名行。段前空格
% 需要用 \tn{qquad} 手动生成。
%    \begin{macrocode}
\cs_new:Npn \@@_decl_text:NNn #1#2#3
  {
    \begin{center}
%<class-en>      \fdu_line_spread:n { \fp_use:N \c_@@_def_line_spread_fp }
      \@@_cover_font_size_large: \sffamily #1
    \end{center}
    \vspace { \c_@@_def_decl_v_sep_iv_tl }
    \begin{center}
      \fdu_fixed_width_box:Nn \textwidth
        {
          \fdu_line_spread:N \c_@@_def_decl_text_line_spread_tl
          \qquad #2
        }
    \end{center}
    \vspace { \c_@@_def_decl_v_sep_iv_tl }
%    \end{macrocode}
%
% \tn{hfill} 用来确保签名行靠右对齐。
%    \begin{macrocode}
    { \hfill #3 }
  }
%    \end{macrocode}
% \end{macro}
%
% \subsubsection{绘制封面}
% \begin{macro}{\makecoveri}
% 生成封一,即真正的封面。各部件之间用橡皮长度隔开。
%    \begin{macrocode}
\NewDocumentCommand \makecoveri { }
  {
    \group_begin:
%<class-en>      \fdu_line_spread:n { \fp_use:N \c_@@_def_line_spread_fp }
      \@@_cover_id:
      \vspace { \c_@@_def_cover_v_sep_i_tl  }
      \@@_cover_logo:
      \vspace { \c_@@_def_cover_v_sep_ii_tl }
      \@@_cover_title:
      \vspace { \c_@@_def_cover_v_sep_vi_tl }
      \@@_cover_info:
      \vspace { \c_@@_def_cover_v_sep_ix_tl }
    \group_end:
  }
%    \end{macrocode}
% \end{macro}
%
% \begin{macro}{\makecoverii}
% 生成封二,即指导小组成员名单。
%    \begin{macrocode}
\NewDocumentCommand \makecoverii { }
  {
    \group_begin:
%    \end{macrocode}
%
% 临时禁用 \tn{cleardoublepage} 带来的分页。
%    \begin{macrocode}
      \cs_set_eq:NN \cleardoublepage \relax
      \thispagestyle { empty }
%    \end{macrocode}
%
% 保持英文模板与中文模板的一致。
%    \begin{macrocode}
%<*class-en>
      \keys_set:nn { ctex }
        { chapter / titleformat = \c_@@_def_chapter_format_tl }
      \fdu_line_spread:n { \fp_use:N \c_@@_def_line_spread_fp }
%</class-en>
%    \end{macrocode}
%
% 为了关闭页眉页脚,此处使用了不编号章节的原始命令 \tn{@schapter}。
%    \begin{macrocode}
      \@schapter
        {
          \fdu_spread_box:NV
            \c_@@_def_cover_instructors_width_tl
            \c_@@_def_name_instructors_tl
        }
      \begin{center}
        \large
        \clist_use:Nn \l_@@_info_instructors_clist { \par }
      \end{center}
    \group_end:
  }
%    \end{macrocode}
% \end{macro}
%
% \changes{v0.3}{2017/07/05}{新增声明页。}
%
% \begin{macro}{\makecoveriii}
% 生成封三,即声明页。该页也需要关闭页眉、页脚显示。
%    \begin{macrocode}
\NewDocumentCommand \makecoveriii { }
  {
    \cleardoublepage
    \thispagestyle { empty }
    \vspace* { \c_@@_def_decl_v_sep_i_tl }
%    \end{macrocode}
%
% 独创性声明。
%    \begin{macrocode}
    \@@_decl_text:NNn
      \c_@@_def_name_originality_decl_tl
      \c_@@_def_originality_decl_text_tl
      {
        \c_@@_def_name_author_sign_tl
        \c_@@_colon_fullwidth_tl
        \fdu_blank_underline:N \c_@@_def_decl_sign_width_tl
        \quad
        \c_@@_def_name_sign_date_tl
        \c_@@_colon_fullwidth_tl
        \fdu_blank_underline:N \c_@@_def_decl_date_width_tl
      }
    \vspace { \c_@@_def_decl_v_sep_ii_tl }
%    \end{macrocode}
%
% 使用授权声明。
%    \begin{macrocode}
    \@@_decl_text:NNn
      \c_@@_def_name_authorization_decl_tl
      \c_@@_def_authorization_decl_text_tl
      {
        \c_@@_def_name_author_sign_tl
        \c_@@_colon_fullwidth_tl
        \fdu_blank_underline:N \c_@@_def_decl_sign_width_tl
        \quad
        \c_@@_def_name_supervisor_sign_tl
        \c_@@_colon_fullwidth_tl
        \fdu_blank_underline:N \c_@@_def_decl_sign_width_tl
        \quad
        \c_@@_def_name_sign_date_tl
        \c_@@_colon_fullwidth_tl
        \fdu_blank_underline:N \c_@@_def_decl_date_width_tl
      }
    \vspace { \c_@@_def_decl_v_sep_iii_tl }
  }
%    \end{macrocode}
% \end{macro}
%
% \begin{macro}{\l_@@_auto_make_cover_bool,style/automakecover}
% 是否自动生成封面。
%    \begin{macrocode}
\bool_new:N \l_@@_auto_make_cover_bool
\keys_define:nn { fdu / style }
  {
    automakecover .bool_set:N = \l_@@_auto_make_cover_bool,
    automakecover .default:n  = true
  }
%    \end{macrocode}
% \end{macro}
%
% 在 |document| 开始位置添加封面以及指导小组成员名单。
%    \begin{macrocode}
\AtBeginDocument
  {
    \bool_if:NT \l_@@_auto_make_cover_bool
      {
        \begin{titlepage}
          \makecoveri \newpage \makecoverii
        \end{titlepage}
      }
  }
%    \end{macrocode}
%
% 在 |document| 结束位置添加声明页。
%    \begin{macrocode}
\AtEndDocument
  { \bool_if:NT \l_@@_auto_make_cover_bool { \makecoveriii } }
%    \end{macrocode}
%
% \subsection{目录}
% 设置目录标题。
%    \begin{macrocode}
\keys_set:nn { ctex }
  {
%<class>    contentsname = { \c_@@_def_name_toc_tl },
%<class-en>    contentsname = { \c_@@_def_name_toc_en_tl },
%    \end{macrocode}
%
% 设置目录中章节标题的样式。
%    \begin{macrocode}
    chapter    / tocline = {
%<class>      \c_@@_def_chapter_toc_format_tl    \CTEXnumberline { #1 } #2
%<class-en>      \c_@@_def_chapter_toc_format_en_tl \CTEXnumberline { #1 } #2
    },
    section    / tocline = {
%<class>      \c_@@_def_section_toc_format_tl    \CTEXnumberline { #1 } #2
%<class-en>      \c_@@_def_section_toc_format_en_tl \CTEXnumberline { #1 } #2
    },
    subsection / tocline = {
%<class>      \c_@@_def_subsection_toc_format_tl \CTEXnumberline { #1 } #2
%<class-en>      \c_@@_def_subsection_toc_format_en_tl
%<class-en>        \CTEXnumberline { #1 } #2
    }
  }
%    \end{macrocode}
%
% \begin{macro}{\tableofcontents}
% 修改 \tn{tableofcontents} 的定义,使得页眉正确显示。第二个参数中的
% 代码来源于 \LaTeXe{} 标准文档类 \file{book.cls}。
%    \begin{macrocode}
\ctex_patch_cmd_once:NnnnTF \tableofcontents
  { }
  {
    \chapter*{\contentsname
      \@mkboth{%
        \MakeUppercase\contentsname}{\MakeUppercase\contentsname}}%
  }
  {
    \chapter* { \contentsname }
%<class>    \fdu_front_matter_header:n { \c_@@_def_name_toc_tl }
%<class-en>    \fdu_front_matter_header:n { \c_@@_def_name_toc_en_tl }
  }
  { } { \ctex_patch_failure:N \tableofcontents }
%    \end{macrocode}
% \end{macro}
%
% \begin{macro}{\@starttoc}
% 修改 \tn{@starttoc} 的定义以调整英文模板中的目录行距。
%    \begin{macrocode}
%<*class-en>
\ctex_patch_cmd_once:NnnnTF \@starttoc
  { }
  { \begingroup }
  {
    \begingroup
      \fdu_line_spread:n { \fp_use:N \c_@@_def_line_spread_fp }
  }
  { } { \ctex_patch_failure:N \@starttoc }
%</class-en>
%    \end{macrocode}
% \end{macro}
%
% \subsection{摘要}
% \subsubsection{中文摘要}
% \begin{macro}{abstract}
% 中文摘要及关键字。
%    \begin{macrocode}
%<*class>
\NewDocumentEnvironment { abstract } { }
  {
%    \end{macrocode}
%
% 中文摘要标题为“摘 \quad 要”,需要修改页眉,并添加到目录。
%    \begin{macrocode}
    \chapter* { \c_@@_def_name_abstract_tl }
    \fdu_front_matter_header:n { \c_@@_def_name_abstract_tl }
    \addcontentsline { toc } { chapter }
      {
        \c_@@_def_chapter_toc_format_tl
        \c_@@_def_name_abstract_tl
      }
  }
  {
%    \end{macrocode}
%
% 摘要正文完成后,空行,输出关键字列表,之间用分号隔开。
%    \begin{macrocode}
    \par \mbox{} \par
    \noindent \hangindent = 4 em  \hangafter = 1
    {
      \normalfont \sffamily
      \c_@@_def_name_keywords_tl \c_@@_colon_fullwidth_tl
    }
    \clist_use:Nn \l_@@_info_keywords_clist
      { \c_@@_semicolon_fullwidth_tl }
    \par
%    \end{macrocode}
%
% 下一行输出中图分类号(CLC)。
%    \begin{macrocode}
    \noindent
    {
      \normalfont \sffamily
      \c_@@_def_name_clc_tl \c_@@_colon_fullwidth_tl
    }
    \l_@@_info_clc_tl
  }
%</class>
%    \end{macrocode}
% \end{macro}
%
% \subsubsection{英文摘要}
% \begin{macro}{abstract*,abstract}
% 英文摘要及关键字。注意英文模板中的 |abstract| 环境与中文模板中的
% |abstract*| 环境是相同的,后者在英文模板中没有定义。
%    \begin{macrocode}
%<class>\NewDocumentEnvironment { abstract* } { }
%<class-en>\NewDocumentEnvironment { abstract } { }
  {
%    \end{macrocode}
%
% 英文摘要标题为“Abstract”,也要修改页眉并添加到目录。
%    \begin{macrocode}
    \chapter* { \c_@@_def_name_abstract_en_tl }
    \fdu_front_matter_header:n { \c_@@_def_name_abstract_en_tl }
    \addcontentsline { toc } { chapter }
      {
%<class>        \c_@@_def_chapter_toc_format_tl
%<class-en>        \c_@@_def_chapter_toc_format_en_tl
        \c_@@_def_name_abstract_en_tl
      }
  }
  {
%    \end{macrocode}
%
% 空行,输出关键字,之间为全角空格。
%    \begin{macrocode}
    \par \mbox{} \par
    \noindent \hangindent = 4 em \hangafter = 1
    \textbf{\c_@@_def_name_keywords_en_tl} \quad
    \clist_use:Nn \l_@@_info_keywords_en_clist { \quad }
    \par
%    \end{macrocode}
%
% 下一行输出中图分类号(CLC)。
%    \begin{macrocode}
    \noindent
    \textbf{\c_@@_def_name_clc_en_tl} \quad
    \l_@@_info_clc_tl
  }
%    \end{macrocode}
% \end{macro}
%
% \subsection{符号表}
% \begin{macro}{notation}
% 符号表环境,利用 |longtable| 封装。可选参数为表格列格式说明符。
% 与摘要类似,符号表页需要修改页眉,并添加到目录。另外需要调整
% \cs{LTpre} 和 \cs{LTpost},以删去 |longtable| 前后的空白。
%    \begin{macrocode}
\NewDocumentEnvironment { notation }
  { O { \c_@@_def_notation_arg_tl } }
  {
%<*class>
    \chapter* { \c_@@_def_name_notation_tl }
    \fdu_front_matter_header:n { \c_@@_def_name_notation_tl }
    \addcontentsline { toc } { chapter }
      {
        \c_@@_def_chapter_toc_format_tl
        \c_@@_def_name_notation_tl
      }
    \group_begin:
%</class>
%<*class-en>
    \chapter* { \c_@@_def_name_notation_en_tl }
    \fdu_front_matter_header:n { \c_@@_def_name_notation_en_tl }
    \addcontentsline { toc } { chapter }
      {
        \c_@@_def_chapter_toc_format_en_tl
        \c_@@_def_name_notation_en_tl
      }
    \group_begin:
      \cs_set_eq:NN \arraystretch
        \c_@@_def_notation_line_stretch_en_tl
%</class-en>
      \dim_set_eq:NN \LTpre  \c_zero_dim
      \dim_set_eq:NN \LTpost \c_zero_dim
      \begin{longtable} { #1 }
  }
  {
      \end{longtable}
    \group_end:
  }
%    \end{macrocode}
% \end{macro}
%
% \subsection{文字绕排}
% \changes{v0.3}{2017/02/26}{尝试利用 \pkg{l3galley} 进行文字绕排。}
% WARNING:严重冲突,暂时不启用。
%    \begin{macrocode}
% \RequirePackage{xgalley}
%
%
% \box_new:N \l_@@_tmpb_box
%
% \dim_new:N \l_@@_wrap_width_dim
% \dim_new:N \l_@@_wrap_height_dim
%
% \clist_new:N \l_@@_wrap_indent_clist
%
% \int_new:N \l_@@_tmpa_int
% \int_new:N \l_@@_wrap_lines_int
%
% \fp_new:N \l_@@_tmpa_fp
%
%
% \keys_define:nn { xwrapfig }
% {
%   cutout .code:n = {
%     \keys_set:nn { xwrapfig / cutout } { #1 }
%   }
% }
%
% \keys_define:nn { fdu / wrap / cutout }
% {
%   % 环境前不改变的行数
%   top~ lines    .int_set:N = \l_@@_wrap_top_lines_int,
%   % 左右边距
%   left~  margin .dim_set:N = \l_@@_wrap_L_margin_dim,
%   right~ margin .dim_set:N = \l_@@_wrap_R_margin_dim,
%   % 上下行距
%   before~ lines .int_set:N = \l_@@_wrap_before_lines_int,
%   after~  lines .int_set:N = \l_@@_wrap_after_lines_int,
%   %
%   top~ lines    .initial:n = { 2 },
%   left~  margin .initial:n = { 0.5 em },
%   right~ margin .initial:n = { 0.5 em },
%   before~ lines .initial:n = { 1 },
%   after~  lines .initial:n = { 1 }
% }
%
%
% \cs_generate_variant:Nn \galley_cutout_right:nn { nV }
% \cs_generate_variant:Nn \galley_cutout_left:nn  { nV }
%
%
% % 预先准备
% % 参数:内容
% \cs_new_protected:Nn \fdu_wrap_prewrap:n
% {
%   % 清除列表,初始化
%   \clist_clear:N \l_@@_wrap_indent_clist
%
%   % 装到 hbox
%   \hbox_set:Nn \l_@@_tmpa_box { #1 }
%   % 总宽度 = 盒子宽 + 调整距离
%   \dim_set:Nn \l_@@_wrap_width_dim
%     { \box_wd:N \l_@@_tmpa_box }
%   \dim_add:Nn \l_@@_wrap_width_dim
%     { \l_@@_wrap_L_margin_dim + \l_@@_wrap_R_margin_dim }
%
%   % 内容装到 vbox
%   \vbox_set:Nn \l_@@_tmpb_box { #1 }
%   % 总高度 = 盒子高 + 盒子深
%   \dim_set:Nn \l_@@_wrap_height_dim
%     { \box_ht:N \l_@@_tmpb_box + \box_dp:N \l_@@_tmpb_box }
%   % 总占据行数 = 总高度 / 行距 + 调整行数
%   \int_set:Nn \l_@@_wrap_lines_int
%     {
%       ( \l_@@_wrap_height_dim / \baselineskip )
%       + \l_@@_wrap_before_lines_int
%       + \l_@@_wrap_after_lines_int
%     }
%
%   % 循环:构建 clist,共 {行数} 个元素,每个元素均为 {总宽度}
%   \int_zero:N \l_@@_tmpa_int
%   \int_do_while:nn
%     { \l_@@_tmpa_int < \l_@@_wrap_lines_int }
%     {
%       \int_incr:N \l_@@_tmpa_int
%       \clist_put_right:Nn \l_@@_wrap_indent_clist
%         { \l_@@_wrap_width_dim }
%     }
% }
%
% % 右边插入内容
% % 参数1:不动的行数,参数2:内容
% \cs_new_protected:Nn \fdu_wrap_put_right:nn
% {
%   \fdu_wrap_prewrap:n { #2 }
%
%   % 开窗
%   \galley_cutout_right:nV { #1 } \l_@@_wrap_indent_clist
%
%   % 内容存入盒子
%   \vbox_set:Nn \l_@@_tmpa_box
%     {
%       % 垂直移动距离 = (不动的行数 + 0.5 * 调整行数) * 行距
%       \fp_set:Nn \l_@@_tmpa_fp
%         {
%           ( #1 + \l_@@_wrap_before_lines_int )
%           * \baselineskip
%         }
%       \skip_vertical:n  { \fp_to_dim:N \l_@@_tmpa_fp }
%
%       % 插入盒子
%       % 宽度:行宽
%       % 内容:跳一个距离(行宽 - 内容总宽 + 左调整宽度)
%       %      内容
%       %      再跳一个距离(右调整宽度)
%       \hbox_to_wd:nn { \linewidth }
%         {
%           \skip_horizontal:n
%             {
%               \linewidth
%               - \l_@@_wrap_width_dim
%               + \l_@@_wrap_L_margin_dim
%             }
%           #2
%           % \skip_horizontal:n { \l_@@_wrap_R_margin_dim }
%         }
%     }
%
%   \box_set_ht:Nn \l_@@_tmpa_box { 0pt }
%   \box_set_dp:Nn \l_@@_tmpa_box { 0pt }
%   \skip_vertical:n { -\baselineskip }
%   \box_use:N \l_@@_tmpa_box
% }
%
% % 左边插入内容
% % 参数1:不动的行数,参数2:内容
% \cs_new_protected:Nn \fdu_wrap_put_left:nn
% {
%   \fdu_wrap_prewrap:n { #2 }
%
%   % 开窗
%   \galley_cutout_left:nV { #1 } \l_@@_wrap_indent_clist
%
%   % 内容存入盒子
%   \vbox_set:Nn \l_@@_tmpa_box
%     {
%       % 垂直移动距离 = (不动的行数 + 0.5 * 调整行数) * 行距
%       \fp_set:Nn \l_@@_tmpa_fp
%         { ( #1 + \l_@@_wrap_before_lines_int ) * \baselineskip }
%       \skip_vertical:n  { \fp_to_dim:N \l_@@_tmpa_fp }
%
%       % 插入盒子
%       % 宽度:行宽
%       % 内容:跳一个距离(左调整宽度)
%       %      内容
%       \hbox_to_wd:nn { \linewidth }
%         {
%           \skip_horizontal:n {  \l_@@_wrap_L_margin_dim }
%           #2
%         }
%     }
%
%   \box_set_ht:Nn \l_@@_tmpa_box { 0pt }
%   \box_set_dp:Nn \l_@@_tmpa_box { 0pt }
%   \skip_vertical:n { -\baselineskip }
%   \box_use:N \l_@@_tmpa_box
% }
%
% \cs_generate_variant:Nn \fdu_wrap_put_right:nn { Vn }
% \cs_generate_variant:Nn \fdu_wrap_put_left:nn { Vn }
%
%
% % 参数1:选项,参数2:内容
% \NewDocumentCommand\putright { O { } +m }
% {
%   \keys_set:nn { fdu / wrap / cutout } { #1 }
%   \fdu_wrap_put_right:Vn \l_@@_wrap_top_lines_int { #2 }
% }
% \NewDocumentCommand\putleft { O { } +m }
% {
%   \keys_set:nn { fdu / wrap / cutout } { #1 }
%   \fdu_wrap_put_left:Vn \l_@@_wrap_top_lines_int { #2 }
% }
%
%
% \NewDocumentCommand\resetindents { }
% {
%   \galley_parshape_set_multi:nnnN
%     { 0 } { 0pt } { 0pt } \c_true_bool
% }
%    \end{macrocode}
%
% \subsection{用户接口}
% \begin{macro}{info,style}
% 定义元(meta)键值对。
%    \begin{macrocode}
\keys_define:nn { fdu }
  {
    info  .meta:nn = { fdu / info  } { #1 },
    style .meta:nn = { fdu / style } { #1 }
  }
%    \end{macrocode}
% \end{macro}
%
% 文档类初始设置。
%    \begin{macrocode}
\keys_set:nn { fdu }
  {
    style   / fontsize      =  -4,
%<class>    style   / fullwidthstop =  false,
    style   / automakecover =  true,
    info    / secretlevel   =  none,
    info    / date          =  \zhtoday,
    info    / schoolid      =  10246,
%<class>    theorem / headerfont    = { \sffamily },
%<class-en>    theorem / headerfont    = { \bfseries \upshape },
%<class>    theorem / bodyfont      = { \fdu@kai },
%<class-en>    theorem / bodyfont      = { \itshape },
    theorem / counter       = { chapter }
  }
%    \end{macrocode}
%
% 在 |nofonts=false| 的情况下设置默认字体。
%    \begin{macrocode}
\bool_if:NF \g_@@_no_fonts_bool
%<*class>
  {
    \keys_set:nn { fdu }
      {
        style / font    = times,
        style / cjkfont = fandol
      }
  }
%</class>
%<class-en>  { \keys_set:nn { fdu } { style / font = times } }
%    \end{macrocode}
%
% \begin{macro}{\fdusetup}
% 用户设置接口。
%    \begin{macrocode}
\NewDocumentCommand \fdusetup { m }
  { \keys_set:nn { fdu } { #1 } }
%    \end{macrocode}
% \end{macro}
%
% \begin{macro}{proof,
%   axiom,corollary,definition,example,lemma,theorem}
% 模板预定义的常用数学环境。
% 其中的“证明”比较特殊,它不编号,但会添加证毕符号。
%    \begin{macrocode}
%<*class>
\fdunewtheorem* { proof       } { \c_@@_def_name_proof_tl      }
\fdunewtheorem  { axiom       } { \c_@@_def_name_axiom_tl      }
\fdunewtheorem  { corollary   } { \c_@@_def_name_corollary_tl  }
\fdunewtheorem  { definition  } { \c_@@_def_name_definition_tl }
\fdunewtheorem  { example     } { \c_@@_def_name_example_tl    }
\fdunewtheorem  { lemma       } { \c_@@_def_name_lemma_tl      }
\fdunewtheorem  { theorem     } { \c_@@_def_name_theorem_tl    }
%</class>
%<*class-en>
\fdunewtheorem* { proof       } { \c_@@_def_name_proof_en_tl      }
\fdunewtheorem  { axiom       } { \c_@@_def_name_axiom_en_tl      }
\fdunewtheorem  { corollary   } { \c_@@_def_name_corollary_en_tl  }
\fdunewtheorem  { definition  } { \c_@@_def_name_definition_en_tl }
\fdunewtheorem  { example     } { \c_@@_def_name_example_en_tl    }
\fdunewtheorem  { lemma       } { \c_@@_def_name_lemma_en_tl      }
\fdunewtheorem  { theorem     } { \c_@@_def_name_theorem_en_tl    }
%</class-en>
%</class|class-en>
%    \end{macrocode}
% \end{macro}
%
% \subsection{模板参数配置文件}
% \changes{v0.3}{2017/06/27}{分离文档类与参数配置文件。}
%    \begin{macrocode}
%<*definition>
%    \end{macrocode}
%
% \subsubsection{通用配置}
% \begin{macro}{\c_@@_full_stop_ideographic_tl,
%   \c_@@_full_stop_fullwidth_tl,
%   \c_@@_colon_fullwidth_tl,
%   \c_@@_semicolon_fullwidth_tl}
% 一些标点符号:
% U+3002 是圆圈句号“\symbol{"3002}”(ideographic full stop),
% U+FF0E 是全角实心句点“\symbol{"FF0E}”(fullwidth full stop),
% U+FF1A 是全角冒号“\symbol{"FF1A}”(fullwidth colon),
% U+FF1B 是全角分号“\symbol{"FF1B}”(fullwidth semicolon)。
%    \begin{macrocode}
\tl_const:Nn \c_@@_full_stop_ideographic_tl { \symbol { "3002 } }
\tl_const:Nn \c_@@_full_stop_fullwidth_tl   { \symbol { "FF0E } }
\tl_const:Nn \c_@@_colon_fullwidth_tl       { \symbol { "FF1A } }
\tl_const:Nn \c_@@_semicolon_fullwidth_tl   { \symbol { "FF1B } }
%    \end{macrocode}
% \end{macro}
%
% \begin{macro}{\c_@@_def_paper_size_tl}
% 纸张大小(A4)。
%    \begin{macrocode}
\tl_const:Nn \c_@@_def_paper_size_tl { a4paper }
%    \end{macrocode}
% \end{macro}
%
% \begin{macro}{\c_@@_def_page_margin_top_dim,
%   \c_@@_def_page_margin_bottom_dim,
%   \c_@@_def_page_margin_left_dim,
%   \c_@@_def_page_margin_right_dim}
% 页面边距。这里,$\SI{2.54}{\centi\meter}=\SI{1}{in}$,
% $\SI{3.18}{\centi\meter}=\SI{1.25}{in}$。
%    \begin{macrocode}
\dim_const:Nn \c_@@_def_page_margin_top_dim    { 2.54 cm }
\dim_const:Nn \c_@@_def_page_margin_bottom_dim { 2.54 cm }
\dim_const:Nn \c_@@_def_page_margin_left_dim   { 3.18 cm }
\dim_const:Nn \c_@@_def_page_margin_right_dim  { 3.18 cm }
%    \end{macrocode}
% \end{macro}
%
% \begin{macro}{\c_@@_def_header_height_dim}
% 页眉高度。此高度与五号字大致相配。
%    \begin{macrocode}
\dim_const:Nn \c_@@_def_header_height_dim { 15 pt }
%    \end{macrocode}
% \end{macro}
%
% \begin{macro}{\c_@@_def_font_size_tl}
% 字号(小四)。
%    \begin{macrocode}
\tl_const:Nn \c_@@_def_font_size_tl { -4 }
%    \end{macrocode}
% \end{macro}
%
% \begin{macro}{\c_@@_def_line_spread_fp}
% 行距倍数。行距倍数 $k$ 由下式确定:
% \begin{equation*}
%   \num{1.2} \times k \times \SI{12}{bp} = \SI{20}{pt}。
% \end{equation*}
% 式中,\num{1.2} 是基本行距与文字大小之比,\SI{12}{bp} 是小四号字
% 的大小,\SI{20}{pt} 是行距固定值。
%    \begin{macrocode}
\fp_const:Nn \c_@@_def_line_spread_fp
  { ( 20 pt ) / ( 12 bp ) / 1.2 }
%    \end{macrocode}
% \end{macro}
%
% \subsubsection{章节标题}
% \begin{macro}{\c_@@_def_chapter_format_tl,
%   \c_@@_def_section_format_tl,
%   \c_@@_def_subsection_format_tl}
% 中文模板章节标题样式。均使用黑体。章标题居中,节与小节标题左对齐
% (但需要使用 \tn{raggedright})。
%    \begin{macrocode}
\tl_const:Nn \c_@@_def_chapter_format_tl
  { \huge  \normalfont \sffamily \centering   }
\tl_const:Nn \c_@@_def_section_format_tl
  { \Large \normalfont \sffamily \raggedright }
\tl_const:Nn \c_@@_def_subsection_format_tl
  { \large \normalfont \sffamily \raggedright }
%    \end{macrocode}
% \end{macro}
%
% \begin{macro}{\c_@@_def_chapter_format_en_tl,
%   \c_@@_def_chapter_name_format_en_tl,
%   \c_@@_def_chapter_title_format_en_tl,
%   \c_@@_def_chapter_after_name_en_tl,
%   \c_@@_def_section_format_en_tl,
%   \c_@@_def_subsection_format_en_tl}
% 英文模板章节标题样式。均使用粗体。
%    \begin{macrocode}
\tl_const:Nn \c_@@_def_chapter_format_en_tl { \centering }
\tl_const:Nn \c_@@_def_chapter_name_format_en_tl
  { \LARGE \bfseries }
\tl_const:Nn \c_@@_def_chapter_title_format_en_tl
  { \huge  \bfseries }
\tl_const:Nn \c_@@_def_chapter_after_name_en_tl
  { \par \nobreak \vskip 10 pt }
\tl_const:Nn \c_@@_def_section_format_en_tl
  { \Large \bfseries \raggedright }
\tl_const:Nn \c_@@_def_subsection_format_en_tl
  { \large \bfseries \raggedright }
%    \end{macrocode}
% \end{macro}
%
% \begin{macro}{\c_@@_def_chapter_before_sep_tl,
%   \c_@@_def_chapter_after_sep_tl,
%   \c_@@_def_section_before_sep_tl,
%   \c_@@_def_section_after_sep_tl,
%   \c_@@_def_subsection_before_sep_tl,
%   \c_@@_def_subsection_after_sep_tl}
% 章节标题前后间距。使用 |tl| 而非 |skip|,是为了防止在没有上下文的
% 时候 |ex| 被展开成 0。之后的不少间距也是这样定义的。
%    \begin{macrocode}
\tl_const:Nn \c_@@_def_chapter_before_sep_tl { 50 pt }
\tl_const:Nn \c_@@_def_chapter_after_sep_tl  { 40 pt }
\tl_const:Nn \c_@@_def_section_before_sep_tl
  { 3.5  ex plus 1   ex minus 0.2 ex }
\tl_const:Nn \c_@@_def_section_after_sep_tl
  { 2.7  ex plus 0.5 ex }
\tl_const:Nn \c_@@_def_subsection_before_sep_tl
  { 3.25 ex plus 1   ex minus 0.2 ex }
\tl_const:Nn \c_@@_def_subsection_after_sep_tl
  { 2.5  ex plus 0.3 ex }
%    \end{macrocode}
% \end{macro}
%
% \begin{macro}{\c_@@_def_chapter_toc_format_tl,
%   \c_@@_def_section_toc_format_tl,
%   \c_@@_def_subsection_toc_format_tl,
%   \c_@@_def_chapter_toc_format_en_tl,
%   \c_@@_def_section_toc_format_en_tl,
%   \c_@@_def_subsection_toc_format_en_tl}
% 章节目录在目录中的样式。
%    \begin{macrocode}
\tl_const:Nn \c_@@_def_chapter_toc_format_tl
  { \normalfont \sffamily }
\tl_const:Nn \c_@@_def_section_toc_format_tl
  { }
\tl_const:Nn \c_@@_def_subsection_toc_format_tl    { \fdu@kai  }
\tl_const:Nn \c_@@_def_chapter_toc_format_en_tl    { \bfseries }
\tl_const:Nn \c_@@_def_section_toc_format_en_tl    { \bfseries }
\tl_const:Nn \c_@@_def_subsection_toc_format_en_tl { }
%    \end{macrocode}
% \end{macro}
%
% \subsubsection{封面}
% \begin{macro}{\@@_cover_font_size_small:,
%   \@@_cover_font_size_normal:,
%   \@@_cover_font_size_large:,
%   \@@_cover_font_size_huge:}
% 字号,使用固定值。这里的定义与正文字号有所不同。
%    \begin{macrocode}
\cs_new:Nn \@@_cover_font_size_small:  { \zihao { -5 } }
\cs_new:Nn \@@_cover_font_size_normal: { \zihao {  4 } }
\cs_new:Nn \@@_cover_font_size_large:  { \zihao { -2 } }
\cs_new:Nn \@@_cover_font_size_huge:   { \zihao {  2 } }
%    \end{macrocode}
% \end{macro}
%
% \begin{macro}{\c_@@_def_cover_id_width_tl,
%   \c_@@_def_cover_id_margin_sep_tl,
%   \c_@@_def_cover_logo_width_tl,
%   \c_@@_def_cover_type_width_tl,
%   \c_@@_def_cover_title_width_tl,
%   \c_@@_def_cover_title_en_width_tl,
%   \c_@@_def_cover_info_left_width_tl,
%   \c_@@_def_cover_instructors_width_tl}
% 封面中的一些长度。
%    \begin{macrocode}
\tl_const:Nn \c_@@_def_cover_id_width_tl          { 10 em }
\tl_const:Nn \c_@@_def_cover_id_margin_sep_tl     { -2 em }
\tl_const:Nn \c_@@_def_cover_logo_width_tl
  { 0.5  \textwidth }
\tl_const:Nn \c_@@_def_cover_type_width_tl
  { 0.45 \textwidth }
\tl_const:Nn \c_@@_def_cover_title_width_tl
  { 0.9  \textwidth }
\tl_const:Nn \c_@@_def_cover_title_en_width_tl
  { 0.9  \textwidth }
\tl_const:Nn \c_@@_def_cover_info_left_width_tl   { 6 em }
\tl_const:Nn \c_@@_def_cover_instructors_width_tl { 7 em }
%    \end{macrocode}
% \end{macro}
%
% \begin{macro}{\c_@@_def_cover_v_sep_i_tl,
%   \c_@@_def_cover_v_sep_ii_tl,
%   \c_@@_def_cover_v_sep_iii_tl,
%   \c_@@_def_cover_v_sep_iv_tl,
%   \c_@@_def_cover_v_sep_v_tl,
%   \c_@@_def_cover_v_sep_vi_tl,
%   \c_@@_def_cover_v_sep_vii_tl,
%   \c_@@_def_cover_v_sep_ix_tl}
% 封面中的一些垂直间距,按自上而下的顺序排列。
%    \begin{macrocode}
\tl_const:Nn \c_@@_def_cover_v_sep_i_tl   { \stretch { 1.5 } }
\tl_const:Nn \c_@@_def_cover_v_sep_ii_tl  { \stretch { 0.8 } }
\tl_const:Nn \c_@@_def_cover_v_sep_iii_tl { 0.4 cm }
\tl_const:Nn \c_@@_def_cover_v_sep_iv_tl  { \stretch { 2   } }
\tl_const:Nn \c_@@_def_cover_v_sep_v_tl   { 0.8 cm }
\tl_const:Nn \c_@@_def_cover_v_sep_vi_tl  { \stretch { 2.5 } }
\tl_const:Nn \c_@@_def_cover_v_sep_vii_tl { 1 ex }
\tl_const:Nn \c_@@_def_cover_v_sep_ix_tl  { \stretch { 1.5 } }
%    \end{macrocode}
% \end{macro}
%
% \begin{macro}{\c_@@_def_cover_logo_file_name_tl}
% 校名 logo 文件名。
%    \begin{macrocode}
\tl_const:Nn \c_@@_def_cover_logo_file_name_tl { Fudan_Logo.pdf }
%    \end{macrocode}
% \end{macro}
%
% \begin{macro}{\c_@@_def_cover_title_en_line_spread_tl}
% 英文标题的行距倍数。
%    \begin{macrocode}
\tl_const:Nn \c_@@_def_cover_title_en_line_spread_tl { 1.2 }
%    \end{macrocode}
% \end{macro}
%
% \subsubsection{声明页}
% \begin{macro}{\c_@@_def_decl_v_sep_i_tl,
%   \c_@@_def_decl_v_sep_ii_tl,
%   \c_@@_def_decl_v_sep_iii_tl,
%   \c_@@_def_decl_v_sep_iv_tl}
% 声明页中的一些垂直间距,按自上而下的顺序排列。最后一项是标题与
% 文本、文本与签名行的间距。
%    \begin{macrocode}
\tl_const:Nn \c_@@_def_decl_v_sep_i_tl   { \stretch { 0.2 } }
\tl_const:Nn \c_@@_def_decl_v_sep_ii_tl  { \stretch { 2.5 } }
\tl_const:Nn \c_@@_def_decl_v_sep_iii_tl { \stretch { 2.5 } }
\tl_const:Nn \c_@@_def_decl_v_sep_iv_tl  { 0.8 cm }
%    \end{macrocode}
% \end{macro}
%
% \begin{macro}{\c_@@_def_decl_text_line_spread_tl}
% 声明文本的行距倍数。
%    \begin{macrocode}
\tl_const:Nn \c_@@_def_decl_text_line_spread_tl { 1.8 }
%    \end{macrocode}
% \end{macro}
%
% \begin{macro}{\c_@@_def_decl_sign_width_tl,
%   \c_@@_def_decl_date_width_tl}
% 签名栏和日期栏的宽度。
%    \begin{macrocode}
\tl_const:Nn \c_@@_def_decl_sign_width_tl { 6 em }
\tl_const:Nn \c_@@_def_decl_date_width_tl { 5 em }
%    \end{macrocode}
% \end{macro}
%
% \begin{macro}{\c_@@_def_originality_decl_text_tl}
% 论文独创性声明。
% ^^A 这里切换一下句号的类别码,以使代码中显示为圆圈句号。
% ^^A 这不会影响最终论文输出(最终将始终输出圆圈句号)。
% \catcode`\。=12
%    \begin{macrocode}
\tl_const:Nn \c_@@_def_originality_decl_text_tl
  {
    本人郑重声明:所呈交的学位论文,是本人在导师的指导下,独立进行研
    究工作所取得的成果。论文中除特别标注的内容外,不包含任何其他个人
    或机构已经发表或撰写过的研究成果。对本研究做出重要贡献的个人和集
    体,均已在论文中作了明确的声明并表示了谢意。本声明的法律结果由本
    人承担。
  }
%    \end{macrocode}
% \catcode`\。 = \active
% \newcommand{。}{.}
% \end{macro}
%
% \begin{macro}{\c_@@_def_authorization_decl_text_tl}
% 论文使用授权声明。
% \catcode`\。=12
%    \begin{macrocode}
\tl_const:Nn \c_@@_def_authorization_decl_text_tl
  {
    本人完全了解复旦大学有关收藏和利用博士、硕士学位论文的规定,即:
    学校有权收藏、使用并向国家有关部门或机构送交论文的印刷本和电子版
    本;允许论文被查阅和借阅;学校可以公布论文的全部或部分内容,可以
    采用影印、缩印或其它复制手段保存论文。涉密学位论文在解密后遵守此
    规定。
  }
%    \end{macrocode}
% \catcode`\。 = \active
% \newcommand{。}{.}
% \end{macro}
%
% \subsubsection{杂项}
% \begin{macro}{\c_@@_def_secret_clist}
% 三种密级。
%    \begin{macrocode}
\clist_const:Nn \c_@@_def_secret_clist { 秘密, 机密, 绝密 }
%    \end{macrocode}
% \end{macro}
%
% \begin{macro}{\c_@@_def_notation_arg_tl}
% 符号表默认参数。
%    \begin{macrocode}
\tl_const:Nn \c_@@_def_notation_arg_tl { l p { 7.5 cm } }
%    \end{macrocode}
% \end{macro}
%
% \begin{macro}{\c_@@_def_notation_line_stretch_en_tl}
% 英文模板中符号表的行间距。
%    \begin{macrocode}
\tl_const:Nn \c_@@_def_notation_line_stretch_en_tl { 1.3 }
%    \end{macrocode}
% \end{macro}
%
% 默认名称。注意空格是忽略掉的。
%    \begin{macrocode}
\tl_const:Nn \c_@@_def_name_secret_level_tl     { 密 \qquad 级 }
\tl_const:Nn \c_@@_def_name_secret_star_tl      { $ \bigstar $ }
\tl_const:Nn \c_@@_def_name_school_id_tl        { 学校代码     }
\tl_const:Nn \c_@@_def_name_student_id_tl       { 学 \qquad 号 }
\tl_const:Nn \c_@@_def_name_thesis_type_tl      { 博士学位论文 }
\tl_const:Nn \c_@@_def_name_degree_type_tl      { (学术学位) }
\tl_const:Nn \c_@@_def_name_department_tl       { 院系         }
\tl_const:Nn \c_@@_def_name_major_tl            { 专业         }
\tl_const:Nn \c_@@_def_name_author_tl           { 姓名         }
\tl_const:Nn \c_@@_def_name_supervisor_tl       { 指导教师     }
\tl_const:Nn \c_@@_def_name_date_tl             { 完成日期     }
\tl_const:Nn \c_@@_def_name_instructors_tl      { 指导小组成员 }
\tl_const:Nn \c_@@_def_name_toc_tl              { 目 \quad 录  }
\tl_const:Nn \c_@@_def_name_abstract_tl         { 摘 \quad 要  }
\tl_const:Nn \c_@@_def_name_keywords_tl         { 关键字       }
\tl_const:Nn \c_@@_def_name_clc_tl              { 中图分类号   }
\tl_const:Nn \c_@@_def_name_notation_tl         { 符号表       }
\tl_const:Nn \c_@@_def_name_toc_en_tl           { Contents     }
\tl_const:Nn \c_@@_def_name_abstract_en_tl      { Abstract     }
\tl_const:Nn \c_@@_def_name_keywords_en_tl
  { Keywords \c_colon_str }
\tl_const:Nn \c_@@_def_name_clc_en_tl
  { CLC~ number \c_colon_str }
\tl_const:Nn \c_@@_def_name_notation_en_tl
  { List~ of~ Symbols }
\tl_const:Nn \c_@@_def_name_originality_decl_tl
  { 复旦大学 \\ 学位论文独创性声明   }
\tl_const:Nn \c_@@_def_name_authorization_decl_tl
  { 复旦大学 \\ 学位论文使用授权声明 }
\tl_const:Nn \c_@@_def_name_author_sign_tl      { 作者签名     }
\tl_const:Nn \c_@@_def_name_supervisor_sign_tl  { 导师签名     }
\tl_const:Nn \c_@@_def_name_sign_date_tl        { 日期         }
%    \end{macrocode}
%
% 默认定理头名称。
%    \begin{macrocode}
\tl_const:Nn \c_@@_def_name_proof_tl         { 证明 }
\tl_const:Nn \c_@@_def_name_axiom_tl         { 公理 }
\tl_const:Nn \c_@@_def_name_corollary_tl     { 推论 }
\tl_const:Nn \c_@@_def_name_definition_tl    { 定义 }
\tl_const:Nn \c_@@_def_name_example_tl       { 例   }
\tl_const:Nn \c_@@_def_name_lemma_tl         { 引理 }
\tl_const:Nn \c_@@_def_name_theorem_tl       { 定理 }
\tl_const:Nn \c_@@_def_name_proof_en_tl      { Proof      }
\tl_const:Nn \c_@@_def_name_axiom_en_tl      { Axiom      }
\tl_const:Nn \c_@@_def_name_corollary_en_tl  { Corollary  }
\tl_const:Nn \c_@@_def_name_definition_en_tl { Definition }
\tl_const:Nn \c_@@_def_name_example_en_tl    { Example    }
\tl_const:Nn \c_@@_def_name_lemma_en_tl      { Lemma      }
\tl_const:Nn \c_@@_def_name_theorem_en_tl    { Theorem    }
%</definition>
%<@@=>
%    \end{macrocode}
%
% \subsection{用户配置文件}
% 以下是一个示例:修改论文类型为“硕士学位论文”。
%    \begin{macrocode}
%<*user>
%%
%% \tl_set:Nn \c__fdu_def_name_thesis_type_tl { 硕士学位论文 }
%</user>
%    \end{macrocode}
%
% \end{implementation}
%
% \EnableImplementation
%
% \begin{implementation}
%
%^^A 代码部分的页边距
% \newgeometry{
%   left   = 2.50 in,
%   right  = 1.00 in,
%   top    = 1.25 in,
%   bottom = 1.00 in
% }
%
% \subsection{附:\cls{fduthesis} 模板文档样式}
%    \begin{macrocode}
%<*doc>
\ExplSyntaxOff
\let\pdfmdfivesum\mdfivesum
%    \end{macrocode}
%
% 无需载入 \pkg{thumbpdf}。
%    \begin{macrocode}
\@namedef{ver@thumbpdf.sty}{9999/99/99}
%    \end{macrocode}
%
% 载入宏包和文档类。
%    \begin{macrocode}
\LoadClass[a4paper, full]{l3doc}
\RequirePackage[UTF8, heading, fontset=none]{ctex}
\RequirePackage[stable, perpage, bottom]{footmisc}
\RequirePackage[toc]{multitoc}
\RequirePackage{caption}
\RequirePackage[showframe]{geometry}
\RequirePackage{listings}
\RequirePackage{makecell}
\RequirePackage{siunitx}
\RequirePackage{tabularx}
\RequirePackage{threeparttable}
\RequirePackage{unicode-math}
\RequirePackage{xcolor}
\RequirePackage{xcolor-material}
\RequirePackage{zref-base}
%    \end{macrocode}
%
% 调整浮动体、代码等与文字的间距。
% 见 \url{http://tex.stackexchange.com/a/40896}。
% \begin{macro}[int]{\@addtocurcol}
%    \begin{macrocode}
\patchcmd{\@addtocurcol}%
  {\vskip \intextsep}%
  {\edef\save@first@penalty{\the\lastpenalty}\unpenalty
    \ifnum \lastpenalty = \@M  % hopefully the OR penalty
       \unpenalty
    \else
       \penalty \save@first@penalty \relax % put it back
    \fi
    \ifnum\outputpenalty <-\@Mii
                       \addvspace\intextsep
                       \vskip\parskip
    \else
                       \addvspace\intextsep
    \fi}%
  {\typeout{Info: Command `@addtocurcol' patched successfully.}}
  {\typeout{Warning: Command `@addtocurcol' patched failed.}}
\patchcmd{\@addtocurcol}%
  {\vskip\intextsep
    \ifnum\outputpenalty <-\@Mii \vskip -\parskip\fi}%
  {\ifnum\outputpenalty <-\@Mii
      \aftergroup\vskip\aftergroup\intextsep
      \aftergroup\nointerlineskip
    \else
      \vskip\intextsep
    \fi}%
  {\typeout{Info: Command `@addtocurcol' patched successfully.}}
  {\typeout{Warning: Command `@addtocurcol' patched failed.}}
%    \end{macrocode}
% \end{macro}
%
% \begin{macro}[int]{\@getpen}
%    \begin{macrocode}
\patchcmd{\@getpen}{\@M}{\@Mi}
  {\typeout{Info: Command `@getpen' patched successfully.}}
  {\typeout{Warning: Command `@getpen' patched failed.}}
%    \end{macrocode}
% \end{macro}
%
% 不对代码实现的 \tn{section} 以下标题编目录。
%    \begin{macrocode}
% \AtBeginEnvironment{implementation}{%
%   \ifnum\value{tocdepth}>\@ne
%     \addtocontents{toc}{\protect\value{tocdepth}=1\relax}%
%   \fi}
%    \end{macrocode}
%
% \begin{macro}[int]{\ctexdocverbaddon,\ctexdisableecglue,
%   \ctexplainps}
% 调整文字间距,以便于让 CJK 字符占的宽度等于西文等宽字体中两个
% 空格的宽度。需要按编译情况分别定义。
%    \begin{macrocode}
\ifxetex
  \let\ctexdocverbaddon\xeCJKVerbAddon
  \def\ctexdisableecglue{\xeCJKsetup{CJKecglue}}
  \def\ctexplainps{\xeCJKsetup{PunctStyle=plain}}
  \appto\meta@font@select{\ifinner\ctexdisableecglue\fi}
\else
  \let\ctexdocverbaddon\relax
  \def\ctexdisableecglue{\ltjsetparameter{autoxspacing=false}}
  \let\ctexplainps\relax
  \appto\meta@font@select{\ctexdisableecglue}
\fi
%    \end{macrocode}
% \end{macro}
%
% 设置标准列表环境样式。
%    \begin{macrocode}
\setlist{noitemsep, topsep=\smallskipamount}
\setlist[1]{labelindent=\parindent}
\setlist[enumerate]{leftmargin=*}
\setlist[itemize]{leftmargin=*}
%    \end{macrocode}
%
% \begin{macro}{optdesc}
% 用于描述各选项。设置条目间距为 \tn{marginparsep},与
% \cls{l3doc} 一致。
%    \begin{macrocode}
\newlist{optdesc}{description}{3}
\setlist[optdesc]{%
  font=\mdseries\small\ttfamily, align=right,
  listparindent=\parindent,
  labelsep=\marginparsep, labelindent=-\marginparsep,
  leftmargin=*}
%    \end{macrocode}
% \end{macro}
%
% \begin{macro}{tablenotes}
% 重新定义 \pkg{threeparttable} 包的 |tablenotes| 环境,
% 用于表格的注释。
%    \begin{macrocode}
\renewlist{tablenotes}{description}{1}
\setlist[tablenotes]{%
  format=\normalfont\tnote@item, align=right,
  listparindent=\parindent, labelindent=\tabcolsep,
  leftmargin=*, rightmargin=\tabcolsep,
  after=\@noparlisttrue}
\AtBeginEnvironment{tablenotes}{%
  \setlength\parindent{2\ccwd}%
  \normalfont\footnotesize}
\AtBeginEnvironment{threeparttable}{%
  \stepcounter{tpt@id}%
  \edef\curr@tpt@id{tpt@\arabic{tpt@id}}}
\newcounter{tpt@id}
%    \end{macrocode}
% \end{macro}
%
% \begin{macro}[aux]{\tnote@item,\TPTtagStyle}
%    \begin{macrocode}
\def\tnote@item#1{%
  \Hy@raisedlink{\hyper@anchor{\curr@tpt@id-#1}}#1}
\def\TPTtagStyle#1{\hyperlink{\curr@tpt@id-#1}{#1}}
%    \end{macrocode}
% \end{macro}
%
% 调整 |function| 环境前后间距。
%    \begin{macrocode}
\BeforeBeginEnvironment{function}{\par\nointerlineskip}
\AtEndEnvironment{function}{%
  \par\xdef\ctexfixprevdepth{\prevdepth=\the\prevdepth\space}}
\AfterEndEnvironment{function}{\ctexfixprevdepth}
%    \end{macrocode}
%
% 调整 |syntax| 环境的样式。
%    \begin{macrocode}
\AtBeginEnvironment{syntax}{%
  \linespread{1.2}\ctexplainps\ctexdisableecglue}
%    \end{macrocode}
%
% \begin{macro}{\exptarget,\rexptarget,\expstar,\rexpstar}
% 部分命令之后的星号($\star$ 或 \ding{73}),表明其不同用法。
% 这里的“exp”和“rexp”分别源自 \LaTeX3 中的“expandable”
% 和“restricted-expandable”。
%    \begin{macrocode}
\newcommand*\exptarget{\Hy@raisedlink{\hypertarget{expstar}{}}}
\newcommand*\rexptarget{\Hy@raisedlink{\hypertarget{rexpstar}{}}}
\newcommand*\expstar{\hyperlink{expstar}{$\star$}}
\newcommand*\rexpstar{\hyperlink{rexpstar}{\ding{73}}}
%    \end{macrocode}
% \end{macro}
%
%    \begin{macrocode}
\ExplSyntaxOn
%<@@=codedoc>
%    \end{macrocode}
%
% \begin{macro}{\list}
% l3doc 会设置列表环境中 \tn{listparindent} |=| \tn{z@},
% 这里将其恢复。
%    \begin{macrocode}
\cs_set_eq:NN \list \@@_oldlist:nn
%    \end{macrocode}
% \end{macro}
%
% \begin{macro}[aux]{\@@_function_descr_start:w}
% 抑制首段的 \tn{parskip}。
%    \begin{macrocode}
\ctex_patch_cmd_once:NnnnTF \@@_function_descr_start:w
  { }
  { \noindent }
  { \skip_vertical:n { -\parskip } \noindent }
  { \iow_term:n { *** ~ SUCCESS ~ *** } }
  { \iow_term:n { *** ~ FAIL ~ *** } }
%    \end{macrocode}
% \end{macro}
%
% \begin{macro}[aux]{\@@_function_assemble:}
%    \begin{macrocode}
\ctex_preto_cmd:NnnTF \@@_function_assemble:
  { }
  { \ctxdoc_fix_yoffset: }
  { \iow_term:n { *** ~ SUCCESS ~ *** } }
  { \iow_term:n { *** ~ FAIL ~ *** } }
%    \end{macrocode}
% \end{macro}
%
% \begin{macro}[int]{\ctxdoc_fix_yoffset:}
% \cls{l3doc} 会在 |function| 环境的 |syntax| 和 |descr| 盒子
% 中间加上 \tn{medskipamount} 的距离。但是若 |syntax| 盒子为空
% (未使用 |syntax| 环境),就会显得不好看。此时通过将
% \tn{medskipamount} 设置为零来修正。若盒子非空,则把
% \tn{parskip} 还回去。
%    \begin{macrocode}
\cs_new_protected_nopar:Npn \ctxdoc_fix_yoffset:
  {
    \box_if_empty:NTF \g_@@_syntax_box
      { \skip_zero:N \medskipamount }
      { \skip_add:Nn \medskipamount { \parskip } }
  }
%    \end{macrocode}
% \end{macro}
%
% \begin{macro}[aux]{\@@_typeset_functions:,\@@_macro_init:,
%   \@@_macro_dump:}
% 左侧边注的函数列表采用单倍行距。
%    \begin{macrocode}
\ctex_preto_cmd:NnnTF \@@_typeset_functions:
  { }
  { \MacroFont }
  { \iow_term:n { *** ~ SUCCESS ~ *** } }
  { \iow_term:n { *** ~ FAIL ~ *** } }
\ctex_patch_cmd_once:NnnnTF \@@_macro_init:
  { }
  { \hbox:n }
  { \MacroFont \hbox:n }
  { \iow_term:n { *** ~ SUCCESS ~ *** } }
  { \iow_term:n { *** ~ FAIL ~ *** } }
\ctex_patch_cmd_once:NnnnTF \@@_macro_dump:
  { }
  { \hbox_unpack_clear:N }
  { \MacroFont \hbox_unpack_clear:N }
  { \iow_term:n { *** ~ SUCCESS ~ *** } }
  { \iow_term:n { *** ~ FAIL ~ *** } }
%    \end{macrocode}
% \end{macro}
%
% \begin{macro}[aux]{\@@_macro_end_style:n}
% 禁止显示 |macro| 环境最后的“(\emph{End definition for ...})”。
%    \begin{macrocode}
\cs_set_eq:NN \@@_macro_end_style:n \use_none:n
%    \end{macrocode}
% \end{macro}
%
% \begin{macro}[aux]{\@@_macro_typeset_one:nN}
% 在 |macro| 环境的侧边栏中,\cls{l3doc} 根据命令的长短,
% 分别用普通字体和紧缩字体输出。然而很长的命令还是会超出页边。
% 这里用缩放盒子的手段使得长命令也可正常显示。
%    \begin{macrocode}
\cs_set_protected:Npn \@@_macro_typeset_one:nN #1#2
  {
    \vbox_set:Nn \l_@@_macro_box
      {
        \MacroFont
        \vbox_unpack_clear:N \l_@@_macro_box
        \hbox_set:Nn \l_tmpa_box
          { \@@_print_macroname:nN {#1} #2 }
        \dim_set:Nn \l_tmpa_dim { \marginparwidth - \labelsep }
        \dim_compare:nNnT { \box_wd:N \l_tmpa_box } > \l_tmpa_dim
          {
            \box_resize_to_wd_and_ht:Nnn \l_tmpa_box
              { \l_tmpa_dim } { \box_ht:N \l_tmpa_box }
          }
        \hbox_overlap_left:n
          {
            \box_use:N \l_tmpa_box
            \skip_horizontal:n { \marginparsep - \labelsep }
          }
      }
    \int_incr:N \l_@@_macro_int
  }
%    \end{macrocode}
% \end{macro}
%
% \begin{macro}[aux]{\@@_print_macro name:nN,\@@_print_macro~ name:nN}
% \cs{MacroFont} 已经放在了 \cs{@@_macro_typeset_one:nN}
% 中处理,此处也不再需要使用 \cs{MacroLongFont}。
% 命令中的空格改用“\textvisiblespace”显示。
%    \begin{macrocode}
\cs_set_protected:Npn \@@_print_macroname:nN #1#2
  {
    \strut
    \@@_get_hyper_target:xN
      {
        \exp_not:n {#1}
        \bool_if:NT #2 { \tl_to_str:n {TF} }
      }
      \l_@@_tmpa_tl
    \cs_if_exist:cTF { r@ \l_@@_tmpa_tl }
      { \exp_args:NNo \label@hyperref [ \l_@@_tmpa_tl ] }
      { \use:n }
      {
        \tl_set:Nn \l_@@_tmpa_tl { #1 }
        \tl_replace_all:Non \l_@@_tmpa_tl
          { \c_catcode_other_space_tl }
          { \fontspec_visible_space: }
        \@@_macroname_prefix:o \l_@@_tmpa_tl
        \@@_macroname_suffix:N #2
      }
  }
%    \end{macrocode}
% \end{macro}
%
% 在 |fdusyntax| 环境前设置若干活动字符。
% \texttt{\textbar} 分隔多个选项,无需倾斜;
% |<xxx>| 表示选项,|(xxx)| 表示默认选项。
%    \begin{macrocode}
\AtBeginEnvironment { fdusyntax }
  {
    \char_set_catcode_active:N \|
    \char_set_active_eq:NN \| \orbar
    \char_set_catcode_active:N \<
    \char_set_active_eq:NN \< \fduoptionsaux
    \char_set_catcode_active:N \(
    \char_set_active_eq:NN \( \defaultvalaux
  }
% \AtBeginEnvironment { syntax }
%   {
%     \char_set_catcode_active:N \|
%     \char_set_active_eq:NN \| \orbar
%     \char_set_catcode_active:N \(
%     \char_set_active_eq:NN \( \defaultvalaux
%   }
%    \end{macrocode}
%
% \begin{macro}{\StopSpecialIndexModule}
% \begin{macro}[aux]{\@@_special_index_module:nnnnN}
% 不对 \cs{cs} 和 \cs{tn} 等编索引。用于目录、索引等。
%    \begin{macrocode}
\DeclareDocumentCommand \StopSpecialIndexModule { }
  {
    \cs_set_eq:NN
      \@@_special_index_module:nnnnN \use_none:nnnnn
  }
\tl_map_inline:nn { \actualchar \encapchar \levelchar }
  { \exp_args:Nx \DoNotIndex { \bslash \tl_to_str:N #1 } }
%    \end{macrocode}
% \end{macro}
% \end{macro}
%
% \subsubsection{\cls{ctxdoc} 开始}
%    \begin{macrocode}
%<@@=ctxdoc>
%    \end{macrocode}
%
% \begin{macro}{\package}
% 带 CTAN 链接的 \cs{pkg} 命令。
%    \begin{macrocode}
% \DeclareDocumentCommand \package { o m }
%   {
%     \exp_args:Nx \href
%       {
%         http \c_colon_str //www.ctan.org/pkg/
%         \IfNoValueTF { #1 } { \str_fold_case:n {#2} } { #1 }
%       }
%       { \pkg {#2} }
%   }
%    \end{macrocode}
% \end{macro}
%
% \begin{macro}{\GetFileId}
% \begin{variable}[aux]{\g_@@_id_ior}
% 获取文件 ID。
%    \begin{macrocode}
% \DeclareDocumentCommand \GetFileId { m }
%   {
%     \GetFileInfo { #1 }
%     \ior_open:NnTF \g_@@_id_ior { \c_sys_jobname_str .id }
%       {
%         \ior_get:NN \g_@@_id_ior \l_@@_tmp_tl
%         \ior_close:N \g_@@_id_ior
%         \exp_after:wN \GetIdInfo \l_@@_tmp_tl
%       }
%       { \GetIdInfo $Id$ }
%       { \fileinfo }
%   }
% \ior_new:N \g_@@_id_ior
%    \end{macrocode}
% \end{variable}
% \end{macro}
%
% \subsubsection{版本历史}
% \begin{macro}[aux]{\@@_ltx_changes:nnn}
% 保存 \pkg{doc} 中 \tn{changes@} 的定义。
%    \begin{macrocode}
\cs_new_eq:NN \@@_ltx_changes:nnn \changes@
%    \end{macrocode}
% \end{macro}
%
% \begin{macro}[int]{\changes@}
% 重定义 \tn{changes@},在版本号一行显示修改日期。
% 注释掉的部分是 \cls{ctxdoc} 为改变版本号排序方式而引入的。
%    \begin{macrocode}
\cs_set_protected:Npn \changes@ #1#2
  {
    \@@_save_version_date:nn { #1 } { #2 }
    \@@_ltx_changes:nnn { #1 } { #2 }
%     \tl_if_empty:nTF { #1 }
%       { \@@_ltx_changes:nnn }
%       { \@@_version_zfill:wnnn #1 \q_stop }
%       { #1 } { #2 }
  }
%    \end{macrocode}
% \end{macro}
%
% \begin{variable}[int]{\l_@@_tmp_tl}
% 临时变量。
%    \begin{macrocode}
\tl_new:N \l_@@_tmp_tl
%    \end{macrocode}
% \end{variable}

% \begin{macro}[aux]{\@@_version_zfill:wnnn,
%   \@@_version_zfill:nnnn,\@@_version_zfill:n}
% \cls{ctxdoc} 中对版本号排序方式进行了调整,但会使得诸如“v1.0a”
% 这样的版本出现在最前。因此这里全部注释掉。
%    \begin{macrocode}
% \cs_new_protected:Npn \@@_version_zfill:wnnn #1#2 \q_stop
%   {
%     \str_if_eq:nnTF { #1 } { v }
%       { \@@_version_zfill:nnnn { #2 } }
%       { \@@_ltx_changes:nnn }
%   }
% \cs_new_protected:Npn \@@_version_zfill:nnnn #1#2
%   {
%     \tl_clear:N \l_@@_tmp_tl
%     \int_zero:N \l_tmpa_int
%     \seq_set_split:Nnn \l_tmpa_seq { . } { #1 }
%     \seq_map_function:NN \l_tmpa_seq \@@_version_zfill:n
%     \int_compare:nNnF \l_tmpa_int > \c_two
%       {
%         \tl_put_right:Nx \l_@@_tmp_tl
%           {
%             \prg_replicate:nn
%               { \c_three - \l_tmpa_int } { 00000 }
%           }
%       }
%     \@@_ltx_changes:nnn { \l_@@_tmp_tl \actualchar #2 }
%   }
% \cs_new_protected:Npn \@@_version_zfill:n #1
%   {
%     \int_incr:N \l_tmpa_int
%     \tl_put_right:Nx \l_@@_tmp_tl
%       {
%         \prg_replicate:nn
%           { \int_max:nn { 0 } { 5 - \tl_count:n { #1 } } } { 0 }
%         \exp_not:n { #1 }
%       }
%   }
%    \end{macrocode}
% \end{macro}
%
% \begin{variable}[int]{\g_@@_version_date_prop}
% 存放版本号与对应的修改日期。
% key = 版本号,value = \{ 开始日期,结束日期 \}。
% 开始日期与结束日期可以相同。
%    \begin{macrocode}
\prop_new:N \g_@@_version_date_prop
%    \end{macrocode}
% \end{variable}
%
% \begin{macro}[int]{\@@_save_version_date:nn}
% 两个 |n| 的版本最终将被 \tn{changes@} 调用。
% |#1| = 版本号,|#2| = 日期。
% 它们分别对应 \tn{change} 的前两个参数(第三个是说明文字)。
%    \begin{macrocode}
\cs_new_protected:Npn \@@_save_version_date:nn #1#2
  {
    \prop_get:NnNTF \g_@@_version_date_prop
      { #1 } \l_@@_tmp_tl
      {
%    \end{macrocode}
% \cs{l_@@_tmp_tl} 相当于两个参数(开始日期、结束日期),
% 因此需要提前展开。
%    \begin{macrocode}
        \exp_after:wN
          \@@_save_version_date_aux:nnnn \l_@@_tmp_tl
        { #2 } { #1 }
      }
      { \@@_save_version_date:nnn { #1 } { #2 } { #2 } }
  }
%    \end{macrocode}
% \end{macro}
%
% \begin{macro}[aux]{\@@_save_version_date_aux:nnnn}
% |#1| = 原开始日期,|#2| = 原结束日期,|#3| = 新读入的日期,
% |#4| = 版本号。显然应有 |#1| < |#2|。\\
% 如果 |#3| < |#1|,则读入日期 |#3|、|#2|;
% 如果 |#3| > |#2|,则读入日期 |#1|、|#3|。
%    \begin{macrocode}
\cs_new_protected:Npn \@@_save_version_date_aux:nnnn #1#2#3#4
  {
    \@@_if_date_later:nnTF { #1 } { #3 }
      { \@@_save_version_date:nnn { #4 } { #3 } { #2 } }
      {
        \@@_if_date_later:nnT { #3 } { #2 }
          { \@@_save_version_date:nnn { #4 } { #1 } { #3 } }
      }
  }
%    \end{macrocode}
% \end{macro}
%
% \begin{macro}[aux]{\@@_save_version_date:nnn}
% 将版本号和日期存入 \cs{g_@@_version_date_prop}。
% |#1| = 版本号,|#2| = 开始日期,|#3| = 结束日期。
%    \begin{macrocode}
\cs_new_protected:Npn \@@_save_version_date:nnn #1#2#3
  {
    \prop_gput:Nnn \g_@@_version_date_prop
      { #1 } { { #2 } { #3 } }
  }
%    \end{macrocode}
% \end{macro}
%
% \begin{macro}[int,TF]{\@@_if_date_later:nn}
% \begin{macro}[aux]{\@@_parse_date:w}
% 比较两个日期。如果 |#1| 在 |#2| 之后,则为 true;反之为 false。
% 日期的格式为 YYYY/MM/DD。比较方法是直接将日期化成 8 位数字,
% 所以月、日前的 0 不可以省略。
%    \begin{macrocode}
\prg_new_conditional:Npnn \@@_if_date_later:nn #1#2 { TF , T }
  {
    \if_int_compare:w
        \@@_parse_date:w #1 / / / 0 \q_stop >
        \@@_parse_date:w #2 / / / 0 \q_stop \exp_stop_f:
      \prg_return_true:
    \else:
      \prg_return_false:
    \fi:
  }
\cs_new:Npn \@@_parse_date:w #1/#2/#3/ #4 \q_stop
  { #1#2#3 }
%    \end{macrocode}
% \end{macro}
% \end{macro}
%
% \begin{macro}[int]{\CTEX@versionitem}
% 版本条目标签。如果版本号不在 \cs{g_@@_version_date_prop} 的
% key 里面,则利用未定义的 \cs{BOOM} 报错。
%    \begin{macrocode}
\cs_new_protected:Npn \CTEX@versionitem #1 \efill
  {
    \@idxitem
    \prop_get:NnNTF \g_@@_version_date_prop
      { #1 } \l_@@_tmp_tl
      {
        \exp_after:wN
          \@@_print_version_date:nnn \l_@@_tmp_tl
        { #1 }
      }
      { \BOOM }
  }
%    \end{macrocode}
% \end{macro}
%
% \begin{macro}[aux]{\@@_print_version_date:nnn}
% 输出版本号和日期。如果开始日期和结束日期相同,则只输出一项。
% |#1| = 开始日期,|#2| = 结束日期,|#3| = 版本号。
%    \begin{macrocode}
\cs_new_protected:Npn \@@_print_version_date:nnn #1#2#3
  {
    \noindent
    \Hy@raisedlink { \belowpdfbookmark { #3 } { HD.#3 } }
    \textbf { #3 } \hfill
    \hbox:n
      {
        \footnotesize
        \str_if_eq:nnTF { #1 } { #2 }
          { ( #1 ) }
          { ( #1 ~ -- ~ #2 ) }
      }
    \par \nopagebreak
  }
%    \end{macrocode}
% \end{macro}
%
% \begin{macro}[int]{\HDorg@theglossary}
% 该命令由 \pkg{hypdoc} 宏包定义,用于存放标准文档类 \cls{book}
% 中定义的 \tn{theindex} 命令。
% 此处的补丁将在版本号一行最后加上修改日期。
%    \begin{macrocode}
\ctex_patch_cmd:Nnn \HDorg@theglossary
  { \let \item \@idxitem }
  { \let \item \CTEX@versionitem }
%    \end{macrocode}
% \end{macro}
%
% \begin{macro}[int]{\@wrglossary}
% 该命令由 \LaTeXe{} 内核定义,又由 \pkg{hypdoc} 宏包作了修改。
% 此处的补丁使得版本历史条目的页码能够指向对应行。
%    \begin{macrocode}
\ctex_patch_cmd:Nnn \@wrglossary
  { hdpindex }
  {
    \ifnum \c@HD@hypercount = \z@
      hdpindex
    \else
      hdclindex { \the \c@HD@hypercount }
    \fi
  }
%    \end{macrocode}
% \end{macro}
%
% \subsubsection{目录条目缩进}
% \begin{macro}[int]{\l@section,\l@subsection}
% 修正目录条目的缩进。
%    \begin{macrocode}
\ctex_patch_cmd:Nnn \l@section    { 2.5em } { 1.5em }
\ctex_patch_cmd:Nnn \l@subsection { 2.5em } { 1.5em }
%    \end{macrocode}
% \end{macro}
%
% \subsubsection{\texttt{macrocode} 环境}
% \begin{macro}[int]{\macro@code}
% 在 \pkg{doc} 宏包中,\env{macrocode} 环境的核心功能由
% \tn{macro@code} 负责实现,而 \tn{xmacro@code} 只用来结束
% \env{macrocode} 环境。但在 \cls{l3doc} 以及 \cls{ctxdoc} 中,
% \tn{xmacro@code} 则基本接管了 \tn{macro@code} 的功能。
% 后者此时只起辅助作用。
%    \begin{macrocode}
\ExplSyntaxOff
\def\macro@code{%
%    \end{macrocode}
% 调整前后间距,禁止 \env{macrocode} 环境前的分页。
%    \begin{macrocode}
   \topsep \MacrocodeTopsep
   \@beginparpenalty \predisplaypenalty
%    \end{macrocode}
% 将列表前后的附加垂直空白设为 0。根据 \cls{ctxdoc} 修改。
%    \begin{macrocode}
   \partopsep \z@skip
%    \if@inlabel\leavevmode\fi
%    \end{macrocode}
% 构建 \env{trivlist} 环境,设置段间距为 0。
% 之后修改字体,并调节左右间距。\tn{MacroIndent} 会根据代码行数
% 更新,具体细节见后文。
%    \begin{macrocode}
   \trivlist \parskip \z@ \item[]%
   \macro@font
   \leftskip\@totalleftmargin \advance\leftskip\MacroIndent
   \rightskip\z@ \parindent\z@ \parfillskip\@flushglue
%    \end{macrocode}
% 按照 \LaTeXe{} 中 \tn{verbatim} 环境中定义 \tn{par},使得空行
% 可以原样输出。
% TODO(20170731): 不很确定,但似乎已经没有用了。注释掉。
%    \begin{macrocode}
%    \blank@linefalse \def\par{\ifblank@line
%                              \leavevmode\fi
%                              \blank@linetrue\@@@@par
%                              \penalty\interlinepenalty}
%    \end{macrocode}
% \tn{obeylines} 将把回车符 |^^M| 变成 \tn{par}。
% 接下来将所有特殊符号的类别码设为 12,即“其他”类。
%    \begin{macrocode}
   \obeylines
   \let\do\do@noligs \verbatim@nolig@list
   \let\do\@makeother \dospecials
%    \end{macrocode}
% 相当于退出 |\begin{list}| 和 |\begin{minipage}|。
%    \begin{macrocode}
   \global\@newlistfalse
   \global\@minipagefalse
%    \end{macrocode}
% 检查 \pkg{DocStrip} 模块,输出行号。
% 此功能已被 \cls{ctxdoc} 接管,具体细节见后文。
%    \begin{macrocode}
%    \ifcodeline@index
%      \everypar{\global\advance\c@CodelineNo\@ne
%                \llap{\theCodelineNo\ \hskip\@totalleftmargin}%
%                \check@module}%
%    \else \everypar{\check@module}%
%    \fi
%    \end{macrocode}
% 初始化交叉引用功能。
%    \begin{macrocode}
   \init@crossref}
\ExplSyntaxOn
% \ctex_patch_cmd:Nnn \macro@code
%   { \if@inlabel \leavevmode \fi }
%   { \partopsep \z@skip }
%    \end{macrocode}
% \end{macro}
%
% \begin{macro}[int]{\xmacro@code,\sxmacro@code}
% 重新实现 \env{macrocode} 与 \env{macrocode*} 环境的核心功能,
% 将对代码逐行处理。后者会将空格显示为“\textvisiblespace”。
%    \begin{macrocode}
\cs_set_protected_nopar:Npn \xmacro@code
  { \@@_marco_code:w }
\cs_set_protected_nopar:Npn \sxmacro@code
  {
    \fontspec_print_visible_spaces:
    \xmacro@code
  }
%    \end{macrocode}
% \end{macro}
%
% \begin{macro}[int]{\@@_marco_code:w}
% 处理代码。
%    \begin{macrocode}
\cs_new_protected_nopar:Npn \@@_marco_code:w
  {
    \ifcodeline@index
      \@@_marco_every_par:n { \@@_code_line_no: }
    \else:
      \@@_marco_every_par:n { }
    \fi:
%    \end{macrocode}
% 设置代码段结束标记为“\verb*|%    \end{macrocode}^^M|”。
%    \begin{macrocode}
    \exp_args:Nx \@@_make_finish_tag:n { \@currenvir }
    \@@_verbatim_start:w
  }
%    \end{macrocode}
% \end{macro}
%
% \begin{macro}[aux]{\@@_marco_every_par:n}
% 在每段之前插入内容。这里每段即相当于每行。
%    \begin{macrocode}
\cs_new_protected:Npn \@@_marco_every_par:n #1
  {
    \everypar
      {
        \everypar { #1 }
        \if@inlabel
          \global \@inlabelfalse
          \@noparlistfalse
          \llap { \box \@labels \hskip \leftskip }
        \fi
        #1
      }
  }
\group_begin:
%    \end{macrocode}
% \end{macro}
%
% 设置 \tn{endlinechar} 为 -1,表示行末不插入任何字符
% (实际上相当于在行尾插入注释符 |%|)。
%    \begin{macrocode}
  \int_set_eq:NN \tex_endlinechar:D \c_minus_one
%    \end{macrocode}
%
% \begin{variable}[aux]{\c_@@_active_space_tl}
% 活动字符类的空格(ASCII 码为 32)。
%    \begin{macrocode}
  \use:n
    {
      \char_set_catcode_active:n { 32 }
      \tl_const:Nn \c_@@_active_space_tl
    }
    { }
%    \end{macrocode}
% \end{variable}
%
%    \begin{macrocode}
\group_end:
%    \end{macrocode}
%
% ASCII 码 13 是回车符 |^^M|。将其设置为活动字符。
%    \begin{macrocode}
\group_begin:
  \char_set_catcode_active:n { 13 }
%    \end{macrocode}
%
% \begin{macro}[aux]{\@@_make_finish_tag:n}
% \env{macrocode} 结尾标记。展开后变成
% “\verb*|%    \end{#1}^^M|”。
%    \begin{macrocode}
  \cs_new_protected:Npx \@@_make_finish_tag:n #1
    {
      \tl_set:Nn \exp_not:N \l_@@_verbatim_finish_tl
        {
          \c_percent_str
          \prg_replicate:nn { 4 }
            { \exp_not:o { \c_@@_active_space_tl } }
          \exp_not:o { \active@escape@char } end
          \c_left_brace_str #1 \c_right_brace_str
          \exp_not:N ^^M
        }
    }
%    \end{macrocode}
% \end{macro}
%
% \begin{macro}[aux]{\@@_verbatim_start:w}
% 开始代码抄录环境。此命令主要是为了防止 |\begin{macrocode}|
% 后出现多余的空行。
%    \begin{macrocode}
  \cs_new_protected:Npn \@@_verbatim_start:w #1
    {
      \str_if_eq:nnTF { #1 } { ^^M }
        { \@@_verbatim_read_line:w }
        { \@@_verbatim_read_line:w #1 }
    }
%    \end{macrocode}
% \end{macro}
%
% \begin{macro}[aux]{\@@_verbatim_read_line:w}
% 逐行读取代码,并连同行尾回车符一并存入
% \cs{\l_@@_verbatim_line_tl}。如果该行与结束标记
% “\verb*|%    \end{macrocode}^^M|”相同,则结束此
% \env{macrocode};否则继续处理该行代码。
%    \begin{macrocode}
  \cs_new_protected:Npn \@@_verbatim_read_line:w #1 ^^M
    {
      \tl_set:Nn \l_@@_verbatim_line_tl { #1 ^^M }
      \tl_if_eq:NNTF
        \l_@@_verbatim_line_tl \l_@@_verbatim_finish_tl
        { \exp_args:Nx \end { \@currenvir } }
        {
          \@@_verbatim_process_line:
          \@@_verbatim_read_line:w
        }
    }
%    \end{macrocode}
% \end{macro}
%
% \begin{variable}[aux]{\c_@@_active_cr_tl}
% 活动字符类的回车符。
%    \begin{macrocode}
  \tl_const:Nn \c_@@_active_cr_tl { ^^M }
\group_end:
%    \end{macrocode}
% \end{variable}
%
% \begin{variable}[aux]{\l_@@_verbatim_line_tl,
%   \l_@@_verbatim_finish_tl,
%   \g_@@_verbatim_verb_stop_tl}
% 分别用来存储代码行、结束标记、
%    \begin{macrocode}
\tl_new:N \l_@@_verbatim_line_tl
\tl_new:N \l_@@_verbatim_finish_tl
\tl_new:N \g_@@_verbatim_verb_stop_tl
%    \end{macrocode}
% \end{variable}
%
% \begin{macro}[aux]{\@@_process_normal_line:}
% 如果代码行开头是 |%|,则检查 guard 后输出;否则正常输出。
%    \begin{macrocode}
% \tl_new:N \l__mytest_tl
\cs_new_protected_nopar:Npn \@@_process_normal_line:
  {
    \str_if_eq_x:nnTF
      { \str_head:N \l_@@_verbatim_line_tl } { \c_percent_str }
      {
% \tl_set_eq:NN \l__mytest_tl \l_@@_verbatim_line_tl
% \tl_show:N \l__mytest_tl
% \tl_set:Nx \l__mytest_tl { \tl_tail:N \l_@@_verbatim_line_tl }
% \tl_show:N \l__mytest_tl
        % \@@_check_angle:x 会去掉 <xxx>,
        % 但 <xxx> some code 这种则不会去掉
        \@@_check_angle:x
          { \tl_tail:N \l_@@_verbatim_line_tl }
% \tl_set_eq:NN \l__mytest_tl \l_@@_verbatim_line_tl
% \tl_show:N \l__mytest_tl
% \tl_set:Nx \l__mytest_tl { \tl_tail:N \l_@@_verbatim_line_tl }
% \tl_show:N \l__mytest_tl
      }
      { \@@_output_line: }
  }
%    \end{macrocode}
% \end{macro}
%
% \begin{macro}[aux]{\@@_process_verb_line:}
% 
%    \begin{macrocode}
\cs_new_protected_nopar:Npn \@@_process_verb_line:
  {
    \tl_if_eq:NNTF \l_@@_verbatim_line_tl
        \g_@@_verbatim_verb_stop_tl
      {
        \tl_gclear:N \g_@@_verbatim_verb_stop_tl
        \cs_gset_eq:NN \@@_verbatim_process_line:
          \@@_process_normal_line:
        \@@_output_module:nn
          { \color { verb@guard } }
          { \@@_module_pop:n { \l_@@_verbatim_line_tl } }
      }
      { \tl_use:N \l_@@_verbatim_line_tl }
  }
%    \end{macrocode}
% \end{macro}
%
% \begin{macro}[aux]{\@@_verbatim_process_line:}
%    \begin{macrocode}
\cs_new_eq:NN \@@_verbatim_process_line:
  \@@_process_normal_line:
%    \end{macrocode}
% \end{macro}
%
% \begin{macro}{\CheckModules,\DontCheckModules}
%    \begin{macrocode}
\DeclareDocumentCommand \CheckModules { }
  {
    \cs_set_eq:NN \@@_verbatim_process_line:
      \@@_process_normal_line:
  }
\DeclareDocumentCommand \DontCheckModules { }
  {
    \cs_set_eq:NN \@@_verbatim_process_line:
      \@@_output_line:
  }
%    \end{macrocode}
% \end{macro}
%
% \begin{macro}[aux]{\@@_check_angle:n,\@@_check_angle:x}
%    \begin{macrocode}
\cs_new_protected:Npn \@@_check_angle:n #1
  {
    \str_if_eq_x:nnTF { \str_head:n { #1 } } { < }
      { \@@_check_module:x { \tl_tail:n { #1 } } }
      { \@@_output_percent_line: }
  }
\cs_generate_variant:Nn \@@_check_angle:n { x }
%    \end{macrocode}
% \end{macro}
%
% \begin{macro}[aux]{\@@_check_module:n,\@@_check_module:x}
%    \begin{macrocode}
\cs_new_protected:Npn \@@_check_module:n #1
  {
    \exp_args:Nx \str_case:nnF { \str_head:n { #1 } }
      {
        { * } { \@@_module_star:w }
        { / } { \@@_module_slash:w }
        { @ } { \@@_module_at:w }
        { < } { \@@_module_verb:w }
      }
      { \@@_module_pm:w }
    #1 \q_stop
  }
\cs_generate_variant:Nn \@@_check_module:n { x }
%    \end{macrocode}
% \end{macro}
%
%    \begin{macrocode}
\group_begin:
  \char_set_catcode_active:N \>
  \cs_new_protected:Npn \@@_module_star:w #1 > #2 \q_stop
    {
      \@@_output_module:nn
        { \@@_star_color: }
        { \@@_module_push:n { \@@_module_angle:n { #1 } } }
      \@@_output_line:n {#2}
      \@@_star_format:
    }
  \cs_new_protected:Npn \@@_module_slash:w #1 > #2 \q_stop
    {
      \@@_output_module:nn
        { \@@_slash_color: }
        { \@@_module_pop:n { \@@_module_angle:n { #1 } } }
      \@@_output_line:n {#2}
      \@@_slash_format:
    }
  \cs_new_protected:Npn \@@_module_at:w @ @ = #1 > #2 \q_stop
    {
      \@@_output_module:nn
        { \color { at@guard } }
        { \@@_module_angle:n { @ @ = #1 } }
      \tl_gset:Nn \g__codedoc_module_name_tl { #1 }
      \@@_output_line:n {#2}
    }
  \cs_new_protected:Npn \@@_module_pm:w #1 > #2 \q_stop
    {
      \tex_noindent:D
      \hbox_overlap_left:n
        {
          \@@_output_module:nn
            { \@@_pm_color: }
            { \@@_module_angle:n { #1 } }
          \skip_horizontal:n { \leftskip + \smallskipamount }
        }
      \group_begin:
        \@@_pm_format:
        \@@_output_line:n {#2}
      \group_end:
    }
  \cs_new_protected:Npn \@@_module_verb:w #1 \q_stop
    {
      \cs_gset_eq:NN \@@_verbatim_process_line: \@@_process_verb_line:
      \tl_gset:Nx \g_@@_verbatim_verb_stop_tl
        { \c_percent_str \tl_tail:n { #1 } }
      \@@_output_module:nn
        { \color { verb@guard } }
        { \@@_module_push:n { \l_@@_verbatim_line_tl } }
    }
\group_end:
\cs_new_protected_nopar:Npn \@@_output_line:
  {
    \tex_noindent:D
    \@@_replace_at_at:N \l_@@_verbatim_line_tl
    \tl_use:N \l_@@_verbatim_line_tl
  }
\cs_new_protected:Npn \@@_replace_at_at:N #1
  {
    \tl_if_empty:NF \g__codedoc_module_name_tl
      {
        \exp_args:NNo \@@_replace_at_at_aux:Nn
          #1 \g__codedoc_module_name_tl
      }
  }
\cs_new_protected:Npn \@@_replace_at_at_aux:Nn #1#2
  {
    \tl_replace_all:Nnn #1 { _ @ @ } { _ _ #2 }
    \tl_replace_all:Nnn #1 {   @ @ } { _ _ #2 }
  }
\cs_new_protected:Npn \@@_output_line:n #1
  {
    \tl_set:Nn \l_@@_verbatim_line_tl { #1 }
    \tl_if_eq:NNTF \l_@@_verbatim_line_tl \c_@@_active_cr_tl
      { \tl_use:N \l_@@_verbatim_line_tl }
      {
        \str_if_eq_x:nnTF
          { \str_head:N \l_@@_verbatim_line_tl }
          { \c_percent_str }
          { \@@_output_percent_line: }
          { \@@_output_line: }
      }
  }
\cs_new_protected:Npn \@@_output_percent_line:
  {
    \tex_noindent:D
    \group_begin:
      \color { code@gray }
      \str_if_eq_x:nnTF { \f@shape } { \updefault }
        { \slshape }
        { \upshape }
      \@@_output_line:
    \group_end:
  }
\cs_new_protected_nopar:Npn \@@_module_push:n
  { \exp_args:No \@@_module_push_aux:nn { \int_use:N \c@HD@hypercount } }
\cs_new_protected:Npn \@@_module_push_aux:nn #1
  {
    \seq_gpush:Nn \g_@@_module_dest_seq { #1 }
    \hypersetup { hidelinks }
    \exp_args:Nx \hdclindex
      { \zref@extractdefault { HD.#1 } { guard@end } { 1 } } { }
  }
\cs_new_protected_nopar:Npn \@@_module_pop:n
  {
    \seq_gpop:NNTF \g_@@_module_dest_seq \l_@@_tmp_tl
      { \exp_args:No \@@_module_pop_aux:nn { \l_@@_tmp_tl } }
      { \BOOM \use:n }
  }
\cs_new_protected:Npn \@@_module_pop_aux:nn #1
  {
    \zref@labelbylist { HD.#1 } { ctxdoc }
    \hypersetup { hidelinks }
    \hdclindex { #1 } { }
  }
\seq_new:N \g_@@_module_dest_seq
\zref@newlist { ctxdoc }
\zref@newprop { guard@end } [ 1 ]
  { \int_eval:n { \c@HD@hypercount - 1 } }
\zref@addprop { ctxdoc } { guard@end }
\cs_new_protected_nopar:Npn \@@_star_format:
  {
    \seq_gpush:No \g_@@_slash_format_seq { \macro@font }
    \seq_gpop:NNF \g_@@_star_format_seq \l_@@_format_tl
      { \@@_pop_format: }
    \@@_select_format:
  }
\cs_new_protected_nopar:Npn \@@_slash_format:
  {
    \seq_gpop:NNTF \g_@@_slash_format_seq \l_@@_format_tl
      {
        \seq_gpush:No \g_@@_star_format_seq { \macro@font }
        \@@_select_format:
      }
      { \BOOM }
  }
\cs_new_protected_nopar:Npn \@@_pm_format:
  {
    \seq_get:NNF \g_@@_star_format_seq \l_@@_format_tl
      {
        \@@_pop_format:
        \seq_gpush:No \g_@@_star_format_seq { \l_@@_format_tl }
      }
    \cs_if_eq:NNF \macro@font \l_@@_format_tl
      { \l_@@_format_tl }
  }
\cs_new_protected_nopar:Npn \@@_pop_format:
  {
    \seq_gpop_left:NN \g_@@_format_seq \l_@@_format_tl
    \seq_gput_right:No \g_@@_format_seq { \l_@@_format_tl }
  }
\cs_new_protected_nopar:Npn \@@_select_format:
  {
    \cs_if_eq:NNF \macro@font \l_@@_format_tl
      {
        \cs_gset_eq:NN \macro@font \l_@@_format_tl
        \macro@font
      }
  }
\tl_new:N \l_@@_format_tl
\seq_new:N \g_@@_format_seq
\seq_new:N \g_@@_star_format_seq
\seq_new:N \g_@@_slash_format_seq
\seq_gput_right:Nn \g_@@_format_seq { \MacroFont }
\seq_gput_right:Nn \g_@@_format_seq { \AltMacroFont }
\cs_set_protected:Npn \MacroFont
  {
%% Modified by Xiangdong Zeng, 2017-06-25.
    \linespread { 1.05 }
%%%%    \linespread { 1 }
    \small
    \fontseries { \mddefault }
    \fontshape  { \updefault }
    \ttfamily
    \ctexdocverbaddon
  }
%% Modified by Xiangdong Zeng, 2017-07-11.
\cs_set_eq:NN \AltMacroFont \MacroFont
%%%%\cs_set_protected:Npn \AltMacroFont
%%%%  {
%%%%    \linespread { 1 }
%%%%    \small
%%%%    \fontseries { \mddefault }
%%%%    \fontshape  { \sldefault }
%%%%    \ttfamily
%%%%    \ctexdocverbaddon
%%%%  }
\AtBeginDocument
  {
    \tl_gset:Nx \macro@font
      { \seq_item:Nn \g_@@_format_seq { 1 } }
  }
\cs_new_protected:Npn \@@_output_module:nn #1#2
  {
    \tex_noindent:D
    \group_begin:
      #1
      \footnotesize \normalfont \sffamily #2
    \group_end:
  }
\cs_new_protected_nopar:Npn \@@_star_color:
  {
    \seq_gpop:NNTF \g_@@_star_color_seq \current@color
      { \set@color }
      { \@@_select_color: }
    \seq_gpush:No \g_@@_slash_color_seq { \current@color }
  }
\cs_new_protected_nopar:Npn \@@_slash_color:
  {
    \seq_gpop:NNTF \g_@@_slash_color_seq \current@color
      {
        \set@color
        \seq_gpush:No \g_@@_star_color_seq { \current@color }
      }
      { \BOOM }
  }
\cs_new_protected_nopar:Npn \@@_pm_color:
  {
    \seq_get:NNTF \g_@@_star_color_seq \current@color
      { \set@color }
      {
        \@@_select_color:
        \seq_gpush:No \g_@@_star_color_seq { \current@color }
      }
  }
\seq_new:N \g_@@_star_color_seq
\seq_new:N \g_@@_slash_color_seq
\cs_new_protected_nopar:Npn \@@_select_color:
  { \color { guard@series!!+ } }
\definecolorseries { guard@series }
  { cmyk } { last } { blue } { purple }
\resetcolorseries [ 3 ] { guard@series }
%% Modified by Xiangdong Zeng, 2017-06-26.
\definecolor { verb@guard } { named } { MaterialLime600 }
\definecolor { at@guard }   { named } { MaterialPink    }
\definecolor { code@gray }  { named } { MaterialGrey    }
%%%%\definecolor { verb@guard } { rgb }  { 0.5  , 0.5 , 0 }
%%%%\definecolor { at@guard }   { rgb }  { 0.5  , 0   , 0.5 }
%%%%\definecolor { code@gray }  { gray } { 0.5 }
\cs_new_protected:Npn \@@_module_angle:n #1
  { \textlangle #1 \textrangle }
\cs_new_protected_nopar:Npn \@@_code_line_no:
  {
    \int_gincr:N \c@CodelineNo
    \hbox_overlap_left:n
      {
        \hbox_to_wd:nn
          { \MacroIndent }
          {
            \HD@target
            \tex_hss:D
            \@@_code_line_no_style:
            \theCodelineNo \enspace
          }
        \tex_kern:D \@totalleftmargin
      }
  }
%    \end{macrocode}
%
% \begin{macro}{\theCodelineNo}
% 行号设置为阿拉伯数字。
%    \begin{macrocode}
\tl_set:Nn \theCodelineNo { \arabic { CodelineNo } }
%    \end{macrocode}
% \end{macro}
%
% \begin{macro}[aux]{\@@_code_line_no_style:}
% 行号格式。
%    \begin{macrocode}
\cs_new_protected_nopar:Npn \@@_code_line_no_style:
  { \color { code@gray } \normalfont \sffamily \tiny }
%    \end{macrocode}
% \end{macro}
%
%    \begin{macrocode}
\cs_set_protected:Npn \HD@SetMacroIndent #1
  {
    \group_begin:
      \settowidth \MacroIndent
        {
          \@@_code_line_no_style:
          \prg_replicate:nn { \tl_count:n { #1 } } { 0 }
          \enspace
        }
      \dim_gset_eq:NN \MacroIndent \MacroIndent
    \group_end:
  }
%% Added by Xiangdong Zeng, 2017-07-26.
% 禁用中文、西文之间的空格
\RenewDocumentCommand \meta { m }
  {
    \group_begin:
      \sys_if_engine_xetex:T { \xeCJKsetup{CJKecglue={}} }
      \__codedoc_meta:n { #1 }
    \group_end:
  }
% 调整 function 字体
\ctex_patch_cmd_once:NnnnTF \__codedoc_typeset_functions: { }
  { \small \ttfamily }
  { \footnotesize \CodeFont \CJKCodeFont }
  { } { \ctex_patch_failure:N \__codedoc_meta_original:n }
%%%% End patch by Xiangdong Zeng
\ExplSyntaxOff
\AtBeginDocument{\addtocontents{toc}{\StopSpecialIndexModule}}
\pdfstringdefDisableCommands{%
  \let\path\meta
  \let\opt\@firstofone}
\preto\@thehead{\cslet{MakeUppercase\space}{\@iden}}
\def\orbar{\textup{\textbar}}
\def\defaultval#1{\textbf{\textup{#1}}}
\def\defaultvalaux#1){\defaultval{#1}}
\def\TF{true\orbar false}
\def\TTF{\defaultval{true}\orbar false}
\def\TFF{true\orbar\defaultval{false}}
\protected\def\opt{\texttt}
\def\pdfTeX{\hologo{pdfTeX}}
\def\XeTeX{\hologo{XeTeX}}
\def\XeLaTeX{\hologo{XeLaTeX}}
\def\LuaLaTeX{\hologo{LuaLaTeX}}
\def\pdfLaTeX{\hologo{pdfLaTeX}}
\def\LaTeX{\hologo{LaTeX}}
\def\LaTeXe{\hologo{LaTeX2e}}
\def\LaTeXiii{\hologo{LaTeX3}}
\def\AmSLaTeX{\hologo{AmSLaTeX}}
\def\dvipdfmx{DVIPDFM\textit{x}}
\def\TeX{\hologo{TeX}}
\def\ApTeX{Ap\TeX}
\def\ApLaTeX{Ap\LaTeX}
\def\upTeX{up\TeX}
\def\upLaTeX{up\LaTeX}
\def\bashcmd{\texttt}
\def\TeXLive{\TeX\ Live}
\def\MiKTeX{\hologo{MiKTeX}}
\def\BSTACK{\begin{tabular}[t]{@{}l@{}}}
\def\ESTACK{\end{tabular}}
\newenvironment{defaultcapconfig}{%
  \MakePercentComment
  \input{ctex-name-utf8.cfg}%
  \ExplSyntaxOff
  \MakePercentIgnore}{}
%% Modified by Xiangdong Zeng, 2017-04-08.
%%%%\def\ctexkit{\href{https://github.com/CTeX-org/ctex-kit/}{\texttt{ctex-kit}}}
%%%%\def\ctexkitrev#1{%
%%%%  \href{https://github.com/CTeX-org/ctex-kit/commit/#1}{\texttt{ctex-kit} rev. #1}}
\appto\GlossaryParms{%
  \raggedcolumns
  \let\Hy@writebookmark\HDorg@writebookmark
  \def\@idxitem{\par\hangindent 2em }%
  \def\subitem{\@idxitem\hspace*{1em}}%
  \def\subsubitem{\@idxitem\hspace*{2em}}}
\def\glossaryname{版本历史}
\GlossaryPrologue{\section{\glossaryname}}
\IndexPrologue{%
  \section{\indexname}
%% Modified by Xiangdong Zeng, 2017-07-19.
%%%%  \textit{意大利体的数字表示描述对应索引项的页码;
%%%%  带下划线的数字表示定义对应索引项的代码行号;
%%%%  罗马字体的数字表示使用对应索引项的代码行号。}}
  \textit{无衬线字体的数字表示描述对应索引项的页码;
  带下划线的数字表示定义对应索引项的代码行号;
  罗马字体的数字表示使用对应索引项的代码行号(或页码).}}
%% Modified by Xiangdong Zeng, 2017-07-26.
\def\IndexLayout{%
%%%%  \newgeometry{hmargin=15mm,vmargin={25mm,15mm},footskip=7mm}%
  \newgeometry{
    left   = 1.00 in,
    right  = 1.00 in,
    top    = 1.25 in,
    bottom = 1.00 in}%
  \setlength\IndexMin{.5\textheight}%
  \ctexset{section/numbering=false}%
  \StopSpecialIndexModule}
%% Added by Xiangdong Zeng, 2017-07-18.
% 修改“描述对应索引项的页码”样式。
\def\usage#1{\textsf{#1}}

%%%%%%%%%%%%%%%%%%%%%%%%%%%%%%%%%%%%%%%%%%%%%%%%%%%%%%%%%%%%%%%%%%%%%%
%%%%%%%%%%%%%%%%%%%%%%%%%%%%%%%%%%%%%%%%%%%%%%%%%%%%%%%%%%%%%%%%%%%%%%

%% Added by Xiangdong Zeng, 2017-06-24.
\renewenvironment{thebibliography}[1]
     {\section{\refname}%
      \@mkboth{\MakeUppercase\refname}{\MakeUppercase\refname}%
      \list{\@biblabel{\@arabic\c@enumiv}}%
           {\settowidth\labelwidth{\@biblabel{#1}}%
            \leftmargin\labelwidth
            \advance\leftmargin\labelsep
            \@openbib@code
            \usecounter{enumiv}%
            \let\p@enumiv\@empty
            \renewcommand\theenumiv{\@arabic\c@enumiv}}%
      \sloppy
      \clubpenalty4000
      \@clubpenalty \clubpenalty
      \widowpenalty4000%
      \sfcode`\.\@m}
     {\def\@noitemerr
       {\@latex@warning{Empty `thebibliography' environment}}%
      \endlist}
%    \end{macrocode}
%
% 设置字体。
%    \begin{macrocode}
\setmainfont{TeX Gyre Pagella}
\setsansfont{TeX Gyre Heros}
\setmonofont{CMU Typewriter Text}[%
  UprightFont = * Light,
  BoldFont    = * Bold,
  SlantedFont = * Light Oblique,
  HyphenChar  = None]
\setmathfont{TeX Gyre Pagella Math}
\setCJKmainfont{FZShuSong-Z01}%
  [BoldFont = FZHei-B01, ItalicFont = FZKai-Z03]
\setCJKsansfont{FZHei-B01}%
  [BoldFont = FZHei-B01, ItalicFont = FZKai-Z03]
\setCJKmonofont{FZFangSong-Z02}%
  [BoldFont = FZHei-B01, ItalicFont = FZKai-Z03]
% \setCJKfamilyfont{宋}{FZShuSong-Z01}
% \setCJKfamilyfont{黑}{FZHei-B01}
% \setCJKfamilyfont{仿}{FZFangSong-Z02}
\setCJKfamilyfont{楷}{FZKai-Z03}
\newfontfamily\CodeFont{Source Code Pro}%
  [BoldFont = Source Code Pro Semibold]
\newCJKfontfamily\CJKCodeFont{Source Han Sans SC}%
  [ItalicFont = FZKai-Z03]
%    \end{macrocode}
%
% 中文排版格式。
%    \begin{macrocode}
\ctexset
  {
    section = {name = {第,节}, format += \raggedright },
    subsubsection / tocline = {\CJKfamily{楷}\CTEXnumberline{#1}#2},
    indexname = {代码索引}
  }
%    \end{macrocode}
%
% 单位设置。
%    \begin{macrocode}
\sisetup
  {
    number-math-rm       = \ensuremath,
    inter-unit-product   = \ensuremath{{}\cdot{}},
    group-digits         = true,
    group-minimum-digits = 4,
    group-separator      = \text{~},
    range-phrase         = \symbol{"FF5E},
    separate-uncertainty = true
  }
%    \end{macrocode}
%
% 超链接设置。
%    \begin{macrocode}
\hypersetup
  {
    bookmarksopen     = true,
    bookmarksnumbered = true,
    colorlinks        = true,
    linkcolor         = MaterialPink,
    citecolor         = MaterialGreen,
    urlcolor          = MaterialIndigo
  }
%    \end{macrocode}
%
% 浮动体标题设置。
%    \begin{macrocode}
% \captionsetup[figure]%
%   {labelsep = quad, justification = centering}
\captionsetup[table]%
  {labelsep = quad, font = sf, justification = centering}
%    \end{macrocode}
%
% \pkg{listings} 代码样式。
%    \begin{macrocode}
\lst@CCPutMacro\lst@ProcessOther {"2D}{\lst@ttfamily{-{}}{-{}}}
\@empty\z@\@empty

\lstdefinestyle{lstStyleBase}{%
  extendedchars   = true,
  gobble          = 3,
  lineskip        = 2 pt,
  frame           = l,
  framerule       = 1 pt,
  framesep        = 0 pt,
  escapeinside    = {(*}{*)},
  %basicstyle      = \small\ttfamily\color{MaterialGrey900},
  basicstyle      = \small\CodeFont\CJKCodeFont%
    \color{MaterialGrey900},
  keywordstyle    = \bfseries\color{MaterialIndigo},
  commentstyle    = \itshape\color{MaterialGrey600},
  stringstyle     = \color{MaterialDeepOrange},
  backgroundcolor = \color{MaterialGrey50}
}

\lstdefinestyle{lstStyleShell}{%
  style           = lstStyleBase,
  rulecolor       = \color{MaterialPink},
  language        = bash,
  alsoletter      = {-},
  morekeywords    = {mkdir,cp,git,fc,%
    tex,pdftex,xetex,luatex,%
    latex,pdflatex,xelatex,lualatex,%
    bibtex,makeindex,latexmk},
  emph            = {-o,-s,-t,-xelatex,-lualatex},
  emphstyle       = \color{MaterialGreen800}
}

\lstdefinestyle{lstStyleLaTeX}{%
  style           = lstStyleBase,
  rulecolor       = \color{MaterialIndigo},
  language        = [LaTeX]TeX,
  texcsstyle      = *\bfseries\color{MaterialDeepOrange},
  deletetexcs     = {\documentclass},
  moretexcs       = {\chapter},
  morekeywords    = {\documentclass,\fdusetup},
  emph            = {style,info,%
    font,fontsize,%
    author,department,title},
  emphstyle       = \color{MaterialGreen800}
}

\lstdefinestyle{lstStyleSyntax}{
  extendedchars   = true,
  gobble          = 6,
  % lineskip        = 2 pt,
  escapeinside    = {(*}{*)},
  language        = [LaTeX]TeX,
  alsoletter      = {*},
  basicstyle      = \footnotesize\CodeFont\CJKCodeFont%
    \color{MaterialGrey900},
  keywordstyle    = \bfseries\color{MaterialIndigo},
  commentstyle    = \itshape\color{MaterialGrey600},
  stringstyle     = \color{MaterialRed},
  texcsstyle      = *\bfseries\color{MaterialDeepOrange},
  deletetexcs     = {\documentclass},
  moretexcs       = {\chapter},
  morekeywords    = {\documentclass,\fdusetup},
  emph            = [1]{oneside,twoside,nofonts,draft,%模板选项
      info,style,%meta 选项
      automakecover,cjkfont,font,fontsize,footnotestyle,fullwidthstop,%style
      author,author*,clc,date,department,department*,instructors,%
      keywords,keywords*,major,schoolid,secretlevel,studentid,%
      supervisor,title,title*%info
    },
  emphstyle       = [1]\color{MaterialGreen800},
  emph            = [2]{abstract,abstract*,notation,%
      axiom,corollary,definition,example,lemma,proof,theorem%
    },
  emphstyle       = [2]\color{MaterialBlue900},
}

\lstnewenvironment{shellexample}{\lstset{style=lstStyleShell}}{}
\lstnewenvironment{latexexample}{\lstset{style=lstStyleLaTeX}}{}
\lstnewenvironment{fdusyntax}{\lstset{style=lstStyleSyntax}\vspace{-1.8ex}}{}
%    \end{macrocode}
%
% 引用环境。
%    \begin{macrocode}
\ExplSyntaxOn
\NewDocumentEnvironment { fduquote } { o o }
  { \quote \ttfamily \qquad }
  {
    \endquote
    \IfNoValueF { #1 } { \hfill —— \IfNoValueF { #2 } { 〔#2〕 } #1 }
  }
\ExplSyntaxOff
%    \end{macrocode}
%
% 脚注。与 \cls{fduthesis} 中的设置一致。
%    \begin{macrocode}
\renewcommand{\thefootnote}{\ding{\numexpr191+\value{footnote}}}
\newlength{\fnBreite}
\renewcommand{\@makefntext}[1]{%
  \settowidth{\fnBreite}{\footnotesize\@thefnmark.i}%
  \protect\footnotesize\upshape%
  \setlength{\@tempdima}{\columnwidth}%
  \addtolength{\@tempdima}{-\fnBreite}%
  \makebox[\fnBreite][l]{\@thefnmark\phantom{  }}%
  \parbox[t]{\@tempdima}%
  {\everypar{\hspace*{1em}}\hspace*{-1em}\upshape#1}%
}
%    \end{macrocode}
%
% 网址断行。
%    \begin{macrocode}
% 见 https://liam0205.me/2017/05/17/help-the-url-command-from-hyperref-to-break-at-line-wrapping-point/
\def\UrlAlphabet{%
      \do\a\do\b\do\c\do\d\do\e\do\f\do\g\do\h\do\i\do\j%
      \do\k\do\l\do\m\do\n\do\o\do\p\do\q\do\r\do\s\do\t%
      \do\u\do\v\do\w\do\x\do\y\do\z\do\A\do\B\do\C\do\D%
      \do\E\do\F\do\G\do\H\do\I\do\J\do\K\do\L\do\M\do\N%
      \do\O\do\P\do\Q\do\R\do\S\do\T\do\U\do\V\do\W\do\X%
      \do\Y\do\Z}
\def\UrlDigits{\do\1\do\2\do\3\do\4\do\5\do\6\do\7\do\8\do\9\do\0}
\g@addto@macro{\UrlBreaks}{\UrlOrds}
\g@addto@macro{\UrlBreaks}{\UrlAlphabet}
\g@addto@macro{\UrlBreaks}{\UrlDigits}
%    \end{macrocode}
%
% 命令。
%    \begin{macrocode}
% \let\DocCs=\cs
% \let\DocTn=\tn
% \renewcommand\cs[2][]{\textcolor{MaterialIndigo}{\DocCs[#1]{#2}}}
% \renewcommand\tn[2][]{\textcolor{MaterialPink}{\DocTn[#1]{#2}}}

% \renewcommand{\pkg}[1]{\textsf{#1}}
% \renewcommand{\cls}[1]{\textsf{#1}}

\let\OldMeta=\meta
\renewcommand\meta[1]{{\rmfamily\OldMeta{#1}}}
\renewcommand\marg[1]{\{\meta{#1}\}}
\renewcommand\oarg[1]{[\meta{#1}]}
\renewcommand\parg[1]{(\meta{#1})}

% 与之前 \defaultvalaux 定义相同
\def\fduoptions#1{\textit{#1}}
\def\fduoptionsaux#1>{\fduoptions{#1}}

\newcommand*\email{\nolinkurl}

\renewcommand\file[1]{{%
  \ttfamily%
  \textcolor{MaterialGrey900}{#1}%
}}

\newcommand\scite[1]{\textsuperscript{\cite{#1}}}

\hyphenation{fdu-the-sis}
\hyphenation{clear-dou-ble-page}
\hyphenation{set-main-font}
\hyphenation{set-main-CJK-font}
\hyphenation{set-math-font}

\DoNotIndex{\begin,\end}

\EnableCrossrefs
\CodelineIndex
\RecordChanges
%</doc>
%    \end{macrocode}
%
% \end{implementation}
%
% \Finale
