% \subsection{字体}
%
% \subsubsection{预定义字体配置}
%
% \changes{v0.7c}{2019/03/05}{重构字体配置,并兼容 macOS。}
%
% \begin{variable}{\g_@@_font_set_tl, \g_@@_cjk_font_set_tl}
% 存放字体选项值。
%    \begin{macrocode}
\tl_new:N \g_@@_font_set_tl
%<class>\tl_new:N \g_@@_cjk_font_set_tl
%    \end{macrocode}
% \end{variable}
%
% \begin{macro}{style/font}
% \changes{v0.7c}{2019/03/05}{新增 \opt{garamond} 和 \opt{times*} 样式。}
% 预定义西文字体。
%    \begin{macrocode}
\keys_define:nn { fdu / style }
  {
    font .choices:nn =
      { garamond, libertinus, lm, palatino, times, times*, none }
      { \tl_set_eq:NN \g_@@_font_set_tl \l_keys_choice_tl }
  }
%    \end{macrocode}
% \end{macro}
%
% \begin{macro}{style/cjk-font}
% \changes{v0.7c}{2019/03/05}{新增 \opt{sinotype} 和 \opt{sourcehan} 样式。}
% 预定义中文字体。
%    \begin{macrocode}
%<*class>
\keys_define:nn { fdu / style }
  {
    cjk-font .choices:nn =
      { adobe, fandol, founder, mac, sinotype, sourcehan, windows, none }
      { \tl_set_eq:NN \g_@@_cjk_font_set_tl \l_keys_choice_tl }
  }
%    \end{macrocode}
%</class>
% \end{macro}
%
% \begin{macro}{
%   \@@_setmainfont:nn,
%   \@@_setsansfont:nn,
%   \@@_setmonofont:nn,
%   \@@_setmathfont:nn}
% 用于设置西文字体的辅助函数,来源于 \pkg{fontspec} 和 \pkg{unicode-math}。
% \begin{arguments}
%   \item 字体名
%   \item 选项
% \end{arguments}
%    \begin{macrocode}
\cs_new_protected:Npn \@@_setmainfont:nn #1#2
  { \__fontspec_main_setmainfont:nn {#2} {#1} }
\cs_new_protected:Npn \@@_setsansfont:nn #1#2
  { \__fontspec_main_setsansfont:nn {#2} {#1} }
\cs_new_protected:Npn \@@_setmonofont:nn #1#2
  { \__fontspec_main_setmonofont:nn {#2} {#1} }
\cs_new_protected:Npn \@@_setmathfont:nn #1#2
  { \__um_setmathfont:nn {#2} {#1} }
%    \end{macrocode}
% \end{macro}
%
% \begin{macro}{
%   \@@_setCJKmainfont:nn,
%   \@@_setCJKsansfont:nn,
%   \@@_setCJKmonofont:nn}
% 用于设置中文字体的辅助函数,来源于 \pkg{xeCJK} 和 \pkg{ctex} 宏包。
%    \begin{macrocode}
%<*class>
\cs_new_protected:Npn \@@_setCJKmainfont:nn #1#2
  { \@@_set_family:nnn { \CJKrmdefault } {#2} {#1} }
\cs_new_protected:Npn \@@_setCJKsansfont:nn #1#2
  { \@@_set_family:nnn { \CJKsfdefault } {#2} {#1} }
\cs_new_protected:Npn \@@_setCJKmonofont:nn #1#2
  { \@@_set_family:nnn { \CJKttdefault } {#2} {#1} }
%    \end{macrocode}
% \end{macro}
%
% \begin{macro}{\@@_set_cjk_font_kai:nn,\fdu@kai}
% 楷体需要单独设置。
%    \begin{macrocode}
\cs_new_protected:Npn \@@_set_cjk_font_kai:nn #1#2
  { \@@_set_family:nnn { fdu@kai } {#2} {#1} }
\cs_new_protected:Npn \fdu@kai
  { \@@_switch_family:n { fdu@kai } }
%    \end{macrocode}
% \end{macro}
%
% \begin{macro}{
%   \@@_cjk_font_options:,
%   \@@_setCJKmainfont:n,
%   \@@_setCJKsansfont:n,
%   \@@_setCJKmonofont:n,
%   \@@_set_cjk_font_kai:n}
% 将 bold、italic 和 bold italic 统一按照 roman 设置。
%    \begin{macrocode}
\tl_const:Nn \@@_cjk_font_options:
 { UprightFont = *, BoldFont = *, ItalicFont = *, BoldItalicFont = * }
\cs_new_protected:Npx \@@_setCJKmainfont:n   #1
  { \@@_setCJKmainfont:nn   {#1} { \@@_cjk_font_options: } }
\cs_new_protected:Npx \@@_setCJKsansfont:n   #1
  { \@@_setCJKsansfont:nn   {#1} { \@@_cjk_font_options: } }
\cs_new_protected:Npx \@@_setCJKmonofont:n   #1
  { \@@_setCJKmonofont:nn   {#1} { \@@_cjk_font_options: } }
\cs_new_protected:Npx \@@_set_cjk_font_kai:n #1
  { \@@_set_cjk_font_kai:nn {#1} { \@@_cjk_font_options: } }
%    \end{macrocode}
% \end{macro}
%
% \begin{macro}{\@@_set_family:nnn,\@@_switch_family:n}
% 封装 CJK 字体族的设定和切换命令。
%    \begin{macrocode}
\sys_if_engine_xetex:TF
  {
    \cs_new_eq:NN \@@_set_family:nnn  \xeCJK_set_family:nnn
    \cs_new_eq:NN \@@_switch_family:n \xeCJK_switch_family:n
  }
  {
    \cs_new_eq:NN \@@_set_family:nnn  \ctex_ltj_set_family:nnn
    \cs_new_eq:NN \@@_switch_family:n \ctex_ltj_switch_family:n
  }
%</class>
%    \end{macrocode}
% \end{macro}
%
% \begin{macro}{
%   \setmainfont,
%   \setsansfont,
%   \setmonofont,
%   \setmathfont,
%   \setCJKmainfont,
%   \setCJKsansfont,
%   \setCJKmonofont,
%   \@@_set_font:n}
% 重新定义以上宏包提供的字体选择命令。我们把它放在导言区末尾,使得用户配置不被
% 模板配置覆盖。
%    \begin{macrocode}
\cs_new_protected:Npn \@@_set_font_helper:n #1
  {
    \exp_args:Nc \RenewDocumentCommand { set #1 font } { O { } m O { } }
      {
        \ctex_at_end_preamble:n
          { \use:c { @@_set #1 font:nn } {##2} { ##1, ##3 } }
      }
  }
\clist_map_inline:nn { main, sans, mono, math    } { \@@_set_font_helper:n {#1} }
%<class>\clist_map_inline:nn { CJKmain, CJKsans, CJKmono } { \@@_set_font_helper:n {#1} }
%    \end{macrocode}
% \end{macro}
%
% \begin{macro}{\@@_load_font_garamond:}
% EB Garamond 系列。
%    \begin{macrocode}
\cs_new_protected:Npn \@@_load_font_garamond:
  {
    \@@_setmainfont:nn { EBGaramond }
      {
        Extension      = .otf,
        UprightFont    = *-Regular,
        BoldFont       = *-Bold,
        ItalicFont     = *-Italic,
        BoldItalicFont = *-BoldItalic
      }
    \@@_setsansfont:nn { libertinussans }
      {
        Extension      = .otf,
        UprightFont    = *-regular,
        BoldFont       = *-bold,
        ItalicFont     = *-italic,
        BoldItalicFont = *-bolditalic
      }
    \@@_setmonofont:nn { lmmonolt10 }
      {
        Extension      = .otf,
        UprightFont    = *-regular,
        BoldFont       = *-bold,
        ItalicFont     = *-oblique,
        BoldItalicFont = *-boldoblique
      }
    \@@_setmathfont:nn { Garamond-Math.otf } { }
  }
%    \end{macrocode}
% \end{macro}
%
% \begin{macro}{\@@_load_font_libertinus:}
% Libertinus 系列。
%    \begin{macrocode}
\cs_new_protected:Npn \@@_load_font_libertinus:
  {
    \@@_setmainfont:nn { libertinusserif }
      {
        Extension      = .otf,
        UprightFont    = *-regular,
        BoldFont       = *-bold,
        ItalicFont     = *-italic,
        BoldItalicFont = *-bolditalic
      }
    \@@_setsansfont:nn { libertinussans }
      {
        Extension      = .otf,
        UprightFont    = *-regular,
        BoldFont       = *-bold,
        ItalicFont     = *-italic,
        BoldItalicFont = *-bolditalic
      }
    \@@_setmonofont:nn { lmmonolt10 }
      {
        Extension      = .otf,
        UprightFont    = *-regular,
        BoldFont       = *-bold,
        ItalicFont     = *-oblique,
        BoldItalicFont = *-boldoblique
      }
    \@@_setmathfont:nn { libertinusmath-regular.otf } { }
  }
%    \end{macrocode}
% \end{macro}
%
% \begin{macro}{\@@_load_font_lm:}
% Latin Modern 系列。在 \XeLaTeX{} 和 \LuaLaTeX{} 中已作为默认字体,所以仅需
% 额外处理数学部分。
%    \begin{macrocode}
\cs_new_protected:Npn \@@_load_font_lm:
  { \@@_setmathfont:nn { latinmodern-math.otf } { } }
%    \end{macrocode}
% \end{macro}
%
% \begin{macro}{\@@_load_font_palatino:}
% Palatino 系列。
%    \begin{macrocode}
\cs_new_protected:Npn \@@_load_font_palatino:
  {
    \@@_setmainfont:nn { texgyrepagella }
      {
        Extension      = .otf,
        UprightFont    = *-regular,
        BoldFont       = *-bold,
        ItalicFont     = *-italic,
        BoldItalicFont = *-bolditalic
      }
    \@@_setsansfont:nn { libertinussans }
      {
        Extension      = .otf,
        UprightFont    = *-regular,
        BoldFont       = *-bold,
        ItalicFont     = *-italic,
        BoldItalicFont = *-bolditalic,
        Scale          = MatchUppercase
      }
    \@@_setmonofont:nn { lmmonolt10 }
      {
        Extension      = .otf,
        UprightFont    = *-regular,
        BoldFont       = *-bold,
        ItalicFont     = *-oblique,
        BoldItalicFont = *-boldoblique
      }
    \@@_setmathfont:nn { texgyrepagella-math.otf } { }
  }
%    \end{macrocode}
% \end{macro}
%
% \begin{macro}{\@@_load_font_times:}
% Times 系列。
%    \begin{macrocode}
\cs_new_protected:Npn \@@_load_font_times:
  {
    \@@_setmainfont:nn { XITS }
      {
        Extension          = .otf,
        UprightFont        = *-Regular,
        BoldFont           = *-Bold,
        ItalicFont         = *-Italic,
        BoldItalicFont     = *-BoldItalic,
        UprightFeatures    = { SmallCapsFont = *-Regular    },
        BoldFeatures       = { SmallCapsFont = *-Bold       },
        ItalicFeatures     = { SmallCapsFont = *-Italic     },
        BoldItalicFeatures = { SmallCapsFont = *-BoldItalic }
      }
    \@@_setsansfont:nn { texgyreheros }
      {
        Extension      = .otf,
        UprightFont    = *-regular,
        BoldFont       = *-bold,
        ItalicFont     = *-italic,
        BoldItalicFont = *-bolditalic
      }
    \@@_setmonofont:nn { texgyrecursor }
      {
        Extension      = .otf,
        UprightFont    = *-regular,
        BoldFont       = *-bold,
        ItalicFont     = *-italic,
        BoldItalicFont = *-bolditalic,
        Ligatures      = CommonOff
      }
    \@@_setmathfont:nn { XITSMath-Regular.otf }{ BoldFont = XITSMath-Bold.otf }
  }
%    \end{macrocode}
% \end{macro}
%
% \begin{macro}{\@@_load_font_times*:}
% Times* 系列,出数学部分外采用系统字体。
%    \begin{macrocode}
\cs_new_protected:cpn { @@_load_font_ times* : }
  {
    \@@_setmainfont:nn { Times~ New~ Roman    } { }
    \@@_setsansfont:nn { Arial                } { }
    \@@_setmonofont:nn { Courier~ New         } { }
    \@@_setmathfont:nn { XITSMath-Regular.otf }{ BoldFont = XITSMath-Bold.otf }
  }
%    \end{macrocode}
% \end{macro}
%
% \begin{macro}{\@@_load_cjk_font_adobe:}
% Adobe 字库。
%    \begin{macrocode}
\cs_new_protected:Npn \@@_load_cjk_font_adobe:
  {
    \@@_setCJKmainfont:n   { AdobeSongStd-Light       }
    \@@_setCJKsansfont:n   { AdobeHeitiStd-Regular    }
    \@@_setCJKmonofont:n   { AdobeFangsongStd-Regular }
    \@@_set_cjk_font_kai:n { AdobeKaitiStd-Regular    }
  }
%    \end{macrocode}
% \end{macro}
%
% \begin{macro}{\@@_load_cjk_font_fandol:}
% Fandol 字库。注意它是安装在 TeX 发行版中的,所以使用文件名调用。
%    \begin{macrocode}
\cs_new_protected:Npn \@@_load_cjk_font_fandol:
  {
    \@@_setCJKmainfont:nn   { FandolSong }
      {
        Extension      = .otf,
        UprightFont    = *-Regular,
        BoldFont       = *-Bold,
        ItalicFont     = *-Regular,
        BoldItalicFont = *-Bold
      }
    \@@_setCJKsansfont:nn   { FandolHei  }
      {
        Extension      = .otf,
        UprightFont    = *-Regular,
        BoldFont       = *-Bold,
        ItalicFont     = *-Regular,
        BoldItalicFont = *-Bold
      }
    \@@_setCJKmonofont:nn   { FandolFang }
      {
        Extension      = .otf,
        UprightFont    = *-Regular,
        BoldFont       = *-Regular,
        ItalicFont     = *-Regular,
        BoldItalicFont = *-Regular
      }
    \@@_set_cjk_font_kai:nn { FandolKai  }
      {
        Extension      = .otf,
        UprightFont    = *-Regular,
        BoldFont       = *-Regular,
        ItalicFont     = *-Regular,
        BoldItalicFont = *-Regular
      }
  }
%    \end{macrocode}
% \end{macro}
%
% \begin{macro}{\@@_load_cjk_font_founder:}
% 方正字库。虽然有粗体(方正小标宋)等,但并非免费,故这里不做处理。
%    \begin{macrocode}
\cs_new_protected:Npn \@@_load_cjk_font_founder:
  {
    \@@_setCJKmainfont:n   { FZShuSong-Z01  }
    \@@_setCJKsansfont:n   { FZHei-B01      }
    \@@_setCJKmonofont:n   { FZFangSong-Z02 }
    \@@_set_cjk_font_kai:n { FZKai-Z03      }
  }
%    \end{macrocode}
% \end{macro}
%
% \begin{macro}{\@@_load_cjk_font_mac:}
% macOS 自带中文字体。
%    \begin{macrocode}
\cs_new_protected:Npn \@@_load_cjk_font_mac:
  {
    \@@_setCJKmainfont:nn   { STSongti-SC }
      {
        UprightFont    = *-Light,
        BoldFont       = *-Bold,
        ItalicFont     = *-Light,
        BoldItalicFont = *-Bold
      }
    \@@_setCJKsansfont:nn   { STHeitiSC   }
      {
        UprightFont    = *-Medium,
        BoldFont       = *-Medium,
        ItalicFont     = *-Medium,
        BoldItalicFont = *-Medium
      }
    \@@_setCJKmonofont:n    { STFangsong  }
    \@@_set_cjk_font_kai:nn { STKaitiSC   }
      {
        UprightFont    = *-Regular,
        BoldFont       = *-Bold,
        ItalicFont     = *-Regular,
        BoldItalicFont = *-Bold
      }
  }
%    \end{macrocode}
% \end{macro}
%
% \begin{macro}{\@@_load_cjk_font_sinotype:}
% 华文字库。
%    \begin{macrocode}
\cs_new_protected:Npn \@@_load_cjk_font_sinotype:
  {
    \@@_setCJKmainfont:n   { STSong     }
    \@@_setCJKsansfont:n   { STHeiti    }
    \@@_setCJKmonofont:n   { STFangsong }
    \@@_set_cjk_font_kai:n { STKaiti    }
  }
%    \end{macrocode}
% \end{macro}
%
% \begin{macro}{\@@_load_cjk_font_sourcehan:}
% 思源宋体、思源黑体。由于没有对应的楷体和仿宋,这里直接给出警告。
%    \begin{macrocode}
\cs_new_protected:Npn \@@_load_cjk_font_sourcehan:
  {
    \@@_setCJKmainfont:nn { SourceHanSerifSC }
      {
        UprightFont    = *-Regular,
        BoldFont       = *-Bold,
        ItalicFont     = *-Regular,
        BoldItalicFont = *-Bold
      }
    \@@_setCJKsansfont:nn { SourceHanSansSC  }
      {
        UprightFont    = *-Regular,
        BoldFont       = *-Bold,
        ItalicFont     = *-Regular,
        BoldItalicFont = *-Bold
      }
    \@@_warning:n { source-han }
  }
\@@_msg_new:nn { source-han }
  { Font~ set~ `sourcehan'~ does~ not~ contain~ kaiti~ and~ fangsong. }
%    \end{macrocode}
% \end{macro}
%
% \begin{macro}{\@@_load_cjk_font_windows:}
% Windows 自带中文字体。
%    \begin{macrocode}
\cs_new_protected:Npn \@@_load_cjk_font_windows:
  {
    \@@_setCJKmainfont:n   { SimSun   }
    \@@_setCJKsansfont:n   { SimHei   }
    \@@_setCJKmonofont:n   { FangSong }
    \@@_set_cjk_font_kai:n { KaiTi    }
  }
%    \end{macrocode}
% \end{macro}
%
% \begin{macro}{\@@_load_font:}
% 字体加载命令。
%    \begin{macrocode}
\cs_new_protected:Npn \@@_load_font:
  {
    \use:c { @@_load_font_     \g_@@_font_set_tl     : }
%<class>    \use:c { @@_load_cjk_font_ \g_@@_cjk_font_set_tl : }
  }
\ctex_at_end_preamble:n { \@@_load_font: }
%    \end{macrocode}
% \end{macro}
%
%^^A 以下相关代码已在 v0.7c 中移除。
% \changes{v0.5}{2017/09/09}{重新实现字体调用。核心内容是分离字体的声明与设定,
%   并按照宋、黑、仿、楷划分中文字体。}
% \changes{v0.6}{2017/11/11}{额外处理 XITS 字体的小型大写字母。}
% \changes{v0.5}{2017/09/23}{新增 \kvopt{font}{none} 选项。}
% \changes{v0.5}{2017/09/23}{新增 \kvopt{cjk-font}{none} 选项。}
% \changes{v0.5}{2017/09/23}{同步 \pkg{fontspec} v2.6e。}
% \changes{v0.6}{2017/10/11}{同步 \pkg{unicode-math} v0.8h。}
%
% \subsubsection{数学字体设置}
%
% 根据 GB 3102.11--93 以及 ISO 80000-2:2009 的规定,数学表达式中
% 表示变量的拉丁字母和希腊字母均应当使用斜体。
% 这里的 |\keys_set:nn{unicode-math}| 实际相当于 \cs{unimathsetup}。
%    \begin{macrocode}
% TODO: (2018-01-19) Do we need `mathrm=sym`?
\keys_set:nn { unicode-math }
  {
    math-style = ISO,
    bold-style = ISO,
%   mathrm     = sym
  }
%    \end{macrocode}
%
% \subsubsection{字号}
%
%    \begin{macrocode}
\keys_define:nn { fdu / style }
  {
%    \end{macrocode}
%
% \begin{macro}{style/font-size}
% |font-size| 不是文档类选项,不能传给 \cls{ctexbook} 文档类,因此
% 只能手动重定义字号命令。
%    \begin{macrocode}
    font-size .choice:,
    font-size .value_required:n = true,
    font-size / -4 .code:n = { },
%    \end{macrocode}
% \end{macro}
%
% \begin{macro}{\tiny,\scriptsize,\footnotesize,\small,
%   \normalsize,\large,\Large,\LARGE,\huge,\Huge}
% 默认使用小四号字,所以只有五号字需要重新设置。
%    \begin{macrocode}
    font-size /  5 .code:n =
      {
        \RenewDocumentCommand \tiny         { } { \zihao {  7 } }
        \RenewDocumentCommand \scriptsize   { } { \zihao { -6 } }
        \RenewDocumentCommand \footnotesize { } { \zihao {  6 } }
        \RenewDocumentCommand \small        { } { \zihao { -5 } }
        \RenewDocumentCommand \normalsize   { } { \zihao {  5 } }
        \RenewDocumentCommand \large        { } { \zihao { -4 } }
        \RenewDocumentCommand \Large        { } { \zihao { -3 } }
        \RenewDocumentCommand \LARGE        { } { \zihao { -2 } }
        \RenewDocumentCommand \huge         { } { \zihao {  2 } }
        \RenewDocumentCommand \Huge         { } { \zihao {  1 } }
%<class-en>      }
%<class-en>  }
%<*class>
      },
%    \end{macrocode}
% \end{macro}
%
% \subsubsection{句号}
%
% \begin{macro}{style/fullwidth-stop}
% \changes{v0.6}{2017/10/14}{支持类别码和 TECKit 映射两种机制。}
% 设置句号形状(圆圈或是圆点)。
%    \begin{macrocode}
    fullwidth-stop .choice:,
    fullwidth-stop .value_required:n = true,
%    \end{macrocode}
% 利用类别码机制切换,只有显式的\FSID 会被替换。
%    \begin{macrocode}
    fullwidth-stop / catcode .code:n =
      { \@@_set_fullwidth_stop_catcode: },
%    \end{macrocode}
% 利用 TECKit 映射机制切换,相当于设置了 \tn{defaultCJKfontfeatures}
% |{|\kvopt{Map\-ping}{fullwidth-stop}|}|。这种手段会替换所有出现的\FSID,
% 并且将影响所有字体。只在 \XeTeX{} 下可用。
%    \begin{macrocode}
    fullwidth-stop / mapping .code:n =
      {
        \sys_if_engine_xetex:TF
          {
            \clist_gset:Nn \g__xeCJK_default_features_clist
              { Mapping = fullwidth-stop }
          }
          {
%    \end{macrocode}
% \LuaTeX{} 下改用类别码机制代替,并给出警告。
%    \begin{macrocode}
            \sys_if_engine_luatex:T
              {
                \@@_warning:n { mapping-not-available }
                \@@_set_fullwidth_stop_catcode:
              }
          }
      },
    fullwidth-stop / false .code:n = { }
  }
%    \end{macrocode}
% \end{macro}
%
% 提示信息。
%    \begin{macrocode}
\@@_msg_new:nn { mapping-not-available }
  {
    Option~ "fullwidth-stop = mapping"~ is~ not~ available~ in~ LuaTeX. \\
    "fullwidth-stop = catcode"~ will~ be~ set~ instead.
  }
%    \end{macrocode}
%
% \begin{macro}{\@@_set_fullwidth_stop_catcode:}
% 将\FSID 设置为活动符,并定义为句点\FSFW。
%    \begin{macrocode}
\cs_new:Npn \@@_set_fullwidth_stop_catcode:
  {
    \char_set_active_eq:nN { "3002 } \c_@@_fwid_full_stop_tl
    \char_set_catcode_active:n { "3002 }
  }
%</class>
%    \end{macrocode}
% \end{macro}
%
