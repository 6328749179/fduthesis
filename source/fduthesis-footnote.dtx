% \subsection{脚注}
%
% \changes{v0.3}{2017/02/21}{支持脚注。}
%
% \subsubsection{编号样式}
%
% 各种脚注编号样式的名称。
%    \begin{macrocode}
\clist_map_inline:nn
  {
    { plain           } { plain           },
    { libertinus      } { libertinus      },
    { libertinus_neg  } { libertinus*     },
    { libertinus_sans } { libertinus-sans },
    { pifont          } { pifont          },
    { pifont_neg      } { pifont*         },
    { pifont_sans     } { pifont-sans     },
    { pifont_sans_neg } { pifont-sans*    },
    { xits            } { xits            },
    { xits_sans       } { xits-sans       },
    { xits_sans_neg   } { xits-sans*      }
  }
  { \@@_define_fn_style:nn #1 }
%    \end{macrocode}
%
% \begin{variable}{\l_@@_fn_style_tl}
% 保存当前使用的脚注编号样式。
%    \begin{macrocode}
\tl_new:N \l_@@_fn_style_tl
%    \end{macrocode}
% \end{variable}
%
%    \begin{macrocode}
\keys_define:nn { fdu / style }
  {
%    \end{macrocode}
%
% \begin{macro}{style/footnote-style}
% 脚注类型共分四大类:
% \begin{itemize}
%   \item \opt{plain}:使用当前字体;
%   \item \opt{libertinus}:取自 Libertinus Serif 和 Libertinus Sans
%     字体;
%   \item \opt{pifont}:使用 \pkg{pifont} 宏包;
%   \item \opt{xits}:取自 XITS 字体。
% \end{itemize}
% 不带任何修饰的为衬线阳文符号,带“|sans|”的为无衬线符号,带“|*|”的
% 为阴文版本。
%    \begin{macrocode}
    footnote-style .choices:nn =
      {
        plain,
        libertinus, libertinus*, libertinus-sans,
        pifont,     pifont*,     pifont-sans,     pifont-sans*,
        xits,                    xits-sans,       xits-sans*
      }
%    \end{macrocode}
% \changes{v0.6}{2017/11/12}{不再依赖 XITS-Math 字体。}
% 若使用 \opt{pifont} 类型,则需引入 \pkg{pifont} 宏包。
%    \begin{macrocode}
      {
        \tl_gset_eq:NN \l_@@_fn_style_tl \l_keys_choice_tl
        \int_compare:nT { 5 <= \l_keys_choice_int <= 8 }
          { \RequirePackage { pifont } }
      },
    footnote-style .value_required:n = true
  }
%    \end{macrocode}
% \end{macro}
%
% \begin{macro}{\@@_fn_symbol_libertinus:n}
% \opt{libertinus} 普通版。\numrange{1}{20} 为数字,\numrange{21}{46}
% 为小写英文字母,\numrange{47}{72} 为大写英文字母。
%    \begin{macrocode}
\cs_new:Npn \@@_fn_symbol_libertinus:n #1
  {
    \int_compare:nTF { #1 >= 21 }
      {
        \int_compare:nTF { #1 >= 47 }
          { \@@_symbol:n { \int_eval:n { "24B6 - 47 + #1 } } }
          { \@@_symbol:n { \int_eval:n { "24D0 - 21 + #1 } } }
      }
      { \@@_symbol:n { \int_eval:n { "2460 - 1 + #1 } } }
  }
%    \end{macrocode}
% \end{macro}
%
% \begin{macro}{\@@_fn_symbol_libertinus_neg:n}
% \opt{libertinus} 阴文衬线版。只含 \numrange{1}{20}。
%    \begin{macrocode}
\cs_new:Npn \@@_fn_symbol_libertinus_neg:n #1
  {
    \int_compare:nTF { #1 >= 11 }
      { \@@_symbol:n { \int_eval:n { "24EB - 11 + #1 } } }
      { \@@_symbol:n { \int_eval:n { "2776 -  1 + #1 } } }
  }
%    \end{macrocode}
% \end{macro}
%
% \begin{macro}{\@@_fn_symbol_libertinus_sans:n}
% \opt{libertinus} 阳文无衬线版。符号排列与普通版相同。
%    \begin{macrocode}
\cs_new_eq:NN \@@_fn_symbol_libertinus_sans:n
  \@@_fn_symbol_libertinus:n
%    \end{macrocode}
% \end{macro}
%
% \begin{macro}{\@@_fn_symbol_pifont:n}
% \opt{pifont} 普通版。以下四种都只包含 \numrange{1}{10}。
%    \begin{macrocode}
\cs_new:Npn \@@_fn_symbol_pifont:n #1
  { \ding { \int_eval:n { 171 + #1 } } }
%    \end{macrocode}
% \end{macro}
%
% \begin{macro}{\@@_fn_symbol_pifont_neg:n}
% \opt{pifont} 阴文衬线版。
%    \begin{macrocode}
\cs_new:Npn \@@_fn_symbol_pifont_neg:n #1
  { \ding { \int_eval:n { 181 + #1 } } }
%    \end{macrocode}
% \end{macro}
%
% \begin{macro}{\@@_fn_symbol_pifont_sans:n}
% \opt{pifont} 阳文无衬线版。
%    \begin{macrocode}
\cs_new:Npn \@@_fn_symbol_pifont_sans:n #1
  { \ding { \int_eval:n { 191 + #1 } } }
%    \end{macrocode}
% \end{macro}
%
% \begin{macro}{\@@_fn_symbol_pifont_sans_neg:n}
% \opt{pifont} 阴文无衬线版。
%    \begin{macrocode}
\cs_new:Npn \@@_fn_symbol_pifont_sans_neg:n #1
  { \ding { \int_eval:n { 201 + #1 } } }
%    \end{macrocode}
% \end{macro}
%
% \begin{macro}{\@@_fn_symbol_xits:n}
% \opt{xits} 普通版。\numrange{1}{9} 为数字,\numrange{10}{35}
% 为小写英文字母,\numrange{36}{61} 为大写英文字母。
%    \begin{macrocode}
\cs_new:Npn \@@_fn_symbol_xits:n #1
  {
    \int_compare:nTF { #1 >= 10 }
      {
        \int_compare:nTF { #1 >= 36 }
          { \@@_symbol:n { \int_eval:n { "24B6 - 36 + #1 } } }
          { \@@_symbol:n { \int_eval:n { "24D0 - 10 + #1 } } }
      }
      { \@@_symbol:n { \int_eval:n { "2460 - 1 + #1 } } }
  }
%    \end{macrocode}
% \end{macro}
%
% \begin{macro}{\@@_fn_symbol_xits_sans:n}
% \opt{xits} 阳文无衬线版。只包含 \numrange{1}{10}。
%    \begin{macrocode}
\cs_new:Npn \@@_fn_symbol_xits_sans:n #1
  { \@@_symbol:n { \int_eval:n { "2780 - 1 + #1 } } }
%    \end{macrocode}
% \end{macro}
%
% \begin{macro}{\@@_fn_symbol_xits_sans_neg:n}
% \opt{xits} 阴文无衬线版。也只包含 \numrange{1}{10}。
%    \begin{macrocode}
\cs_new:Npn \@@_fn_symbol_xits_sans_neg:n #1
  { \@@_symbol:n { \int_eval:n { "278A - 1 + #1 } } }
%    \end{macrocode}
% \end{macro}
%
% \begin{macro}{\thefootnote}
% \changes{v0.7}{2018/01/17}{改为可完全展开的命令。}
% 重定义脚注编号。
%    \begin{macrocode}
\cs_set:Npn \thefootnote { \fdu_footnote_number:N \c@footnote }
%    \end{macrocode}
% \end{macro}
%
% \begin{macro}{\fdu_footnote_number:N}
% 脚注编号样式。
%    \begin{macrocode}
\cs_new:Npn \fdu_footnote_number:N #1
  {
    \tl_case:NnF \l_@@_fn_style_tl
      {
%    \end{macrocode}
% \opt{plain} 类型直接使用计数器 |footnote| 的值。
%    \begin{macrocode}
        \c_@@_fn_style_plain_tl
          { \int_use:N #1 }
%    \end{macrocode}
% \opt{libertinus} 类型需要使用 Libertinus Serif 或
% Libertinus Sans 字体。
%    \begin{macrocode}
        \c_@@_fn_style_libertinus_tl
          {
            \fontspec { libertinusserif-regular.otf }
            \@@_fn_symbol_libertinus:n {#1}
          }
        \c_@@_fn_style_libertinus_neg_tl
          {
            \fontspec { libertinusserif-regular.otf }
            \@@_fn_symbol_libertinus_neg:n {#1}
          }
        \c_@@_fn_style_libertinus_sans_tl
          {
            \fontspec { libertinussans-regular.otf }
            \@@_fn_symbol_libertinus_sans:n {#1}
          }
%    \end{macrocode}
% \opt{pifont} 类型无需进行额外的操作。
%    \begin{macrocode}
        \c_@@_fn_style_pifont_tl
          { \@@_fn_symbol_pifont:n {#1} }
        \c_@@_fn_style_pifont_neg_tl
          { \@@_fn_symbol_pifont_neg:n {#1} }
        \c_@@_fn_style_pifont_sans_tl
          { \@@_fn_symbol_pifont_sans:n {#1} }
        \c_@@_fn_style_pifont_sans_neg_tl
          { \@@_fn_symbol_pifont_sans_neg:n {#1} }
%    \end{macrocode}
% \opt{xits} 类型需要临时切换数学字体。
%    \begin{macrocode}
        \c_@@_fn_style_xits_tl
          {
            \fontspec { XITS-Regular.otf }
            \@@_fn_symbol_xits:n {#1}
          }
        \c_@@_fn_style_xits_sans_tl
          {
            \fontspec { XITS-Regular.otf }
            \@@_fn_symbol_xits_sans:n {#1}
          }
        \c_@@_fn_style_xits_sans_neg_tl
          {
            \fontspec { XITS-Regular.otf }
            \@@_fn_symbol_xits_sans_neg:n {#1}
          }
      }
%    \end{macrocode}
% 变量 \cs{l_@@_fn_style_tl} 保存的类型未知时,默认使用 \opt{plain}
% 类型。
%    \begin{macrocode}
      { \int_use:N #1 }
  }
%    \end{macrocode}
% \end{macro}
%
% \subsubsection{整体样式}
%
% \begin{macro}[int]{\@makefntext}
% \changes{v0.7}{2018/01/18}{简化实现,兼容 \pkg{fancyvrb} 宏包。
%   不再使用悬挂缩进。}
% 重定义内部脚注文字命令,使脚注编号不使用上标,宽度为 \SI{1.5}{em}。
% 见 \url{http://tex.stackexchange.com/q/19844} 和
% \url{https://www.zhihu.com/question/53030087}。
%    \begin{macrocode}
\cs_set:Npn \@makefntext #1
  {
    \mode_leave_vertical:
    \hbox_to_wd:nn { 1.5 em } { \@thefnmark \hfil }
    #1
  }
%    \end{macrocode}
% \end{macro}
%
