% \EnableImplementation
%^^A \DisableImplementation
%
% \begin{implementation}
%
%^^A 代码部分的页边距
% \newgeometry{
%   left   = 2.50 in,
%   right  = 1.00 in,
%   top    = 1.25 in,
%   bottom = 1.00 in
% }
%
% \subsection{附:\cls{fduthesis} 模板文档样式}
%    \begin{macrocode}
%<*doc>
\ExplSyntaxOff
\let\pdfmdfivesum\mdfivesum
%    \end{macrocode}
%
% 无需载入 \pkg{thumbpdf}。
%    \begin{macrocode}
\@namedef{ver@thumbpdf.sty}{9999/99/99}
%    \end{macrocode}
%
% 载入宏包和文档类。
%    \begin{macrocode}
\LoadClass[a4paper, full]{l3doc}
\RequirePackage[UTF8, heading, fontset=none]{ctex}
%    \end{macrocode}
% 与 \pkg{hypdoc} (\cls{l3doc} 调用)冲突,导致脚注超链接
% 无法正常跳转。临时禁用。
%    \begin{macrocode}
% \RequirePackage[stable, bottom]{footmisc}
\RequirePackage[toc]{multitoc}
\RequirePackage{caption}
\RequirePackage{geometry}
\RequirePackage{listings}
\RequirePackage{makecell}
\RequirePackage{siunitx}
\RequirePackage{tabularx}
\RequirePackage{threeparttable}
\RequirePackage{unicode-math}
\RequirePackage{xcolor}
\RequirePackage{xcolor-material}
\RequirePackage{zref-base}
%    \end{macrocode}
%
%    \begin{macrocode}
\ExplSyntaxOn
%<@@=codedoc>
%    \end{macrocode}
%
% \begin{macro}{\list}
% l3doc 会设置列表环境中 \tn{listparindent} |=| \tn{z@},
% 这里将其恢复。
%    \begin{macrocode}
\cs_set_eq:NN \list \@@_oldlist:nn
%    \end{macrocode}
% \end{macro}
%
% \begin{macro}[aux]{\@@_function_descr_start:w}
% 抑制首段的 \tn{parskip}。
%    \begin{macrocode}
\ctex_patch_cmd_once:NnnnTF \@@_function_descr_start:w
  { }
  { \noindent }
  { \skip_vertical:n { -\parskip } \noindent }
  { \iow_term:n { *** ~ SUCCESS ~ *** } }
  { \iow_term:n { *** ~ FAIL ~ *** } }
%    \end{macrocode}
% \end{macro}
%
% \begin{macro}[aux]{\@@_function_assemble:}
%    \begin{macrocode}
\ctex_preto_cmd:NnnTF \@@_function_assemble:
  { }
  { \ctxdoc_fix_yoffset: }
  { \iow_term:n { *** ~ SUCCESS ~ *** } }
  { \iow_term:n { *** ~ FAIL ~ *** } }
%    \end{macrocode}
% \end{macro}
%
% \begin{macro}[int]{\ctxdoc_fix_yoffset:}
% \cls{l3doc} 会在 |function| 环境的 |syntax| 和 |descr| 盒子
% 中间加上 \tn{medskipamount} 的距离。但是若 |syntax| 盒子为空
% (未使用 |syntax| 环境),就会显得不好看。此时通过将
% \tn{medskipamount} 设置为零来修正。若盒子非空,则把
% \tn{parskip} 还回去。
%    \begin{macrocode}
\cs_new_protected_nopar:Npn \ctxdoc_fix_yoffset:
  {
    \box_if_empty:NTF \g_@@_syntax_box
      { \skip_zero:N \medskipamount }
      { \skip_add:Nn \medskipamount { \parskip } }
  }
%    \end{macrocode}
% \end{macro}
%
% \begin{macro}[aux]{\@@_typeset_functions:,\@@_macro_init:,
%   \@@_macro_dump:}
% 左侧边注的函数列表采用单倍行距。
%    \begin{macrocode}
\ctex_preto_cmd:NnnTF \@@_typeset_functions:
  { }
  { \MacroFont }
  { \iow_term:n { *** ~ SUCCESS ~ *** } }
  { \iow_term:n { *** ~ FAIL ~ *** } }
\ctex_patch_cmd_once:NnnnTF \@@_macro_init:
  { }
  { \hbox:n }
  { \MacroFont \hbox:n }
  { \iow_term:n { *** ~ SUCCESS ~ *** } }
  { \iow_term:n { *** ~ FAIL ~ *** } }
\ctex_patch_cmd_once:NnnnTF \@@_macro_dump:
  { }
  { \hbox_unpack_clear:N }
  { \MacroFont \hbox_unpack_clear:N }
  { \iow_term:n { *** ~ SUCCESS ~ *** } }
  { \iow_term:n { *** ~ FAIL ~ *** } }
%    \end{macrocode}
% \end{macro}
%
% \begin{macro}[aux]{\@@_macro_end_style:n}
% 禁止显示 |macro| 环境最后的“(\emph{End definition for ...})”。
%    \begin{macrocode}
\cs_set_eq:NN \@@_macro_end_style:n \use_none:n
%    \end{macrocode}
% \end{macro}
%
% \begin{macro}[aux]{\@@_macro_typeset_one:nN}
% 在 |macro| 环境的侧边栏中,\cls{l3doc} 根据命令的长短,
% 分别用普通字体和紧缩字体输出。然而很长的命令还是会超出页边。
% 这里用缩放盒子的手段使得长命令也可正常显示。
%    \begin{macrocode}
\cs_set_protected:Npn \@@_macro_typeset_one:nN #1#2
  {
    \vbox_set:Nn \l_@@_macro_box
      {
        \MacroFont
        \vbox_unpack_clear:N \l_@@_macro_box
        \hbox_set:Nn \l_tmpa_box
          { \@@_print_macroname:nN {#1} #2 }
        \dim_set:Nn \l_tmpa_dim { \marginparwidth - \labelsep }
        \dim_compare:nNnT { \box_wd:N \l_tmpa_box } > \l_tmpa_dim
          {
            \box_resize_to_wd_and_ht:Nnn \l_tmpa_box
              { \l_tmpa_dim } { \box_ht:N \l_tmpa_box }
          }
        \hbox_overlap_left:n
          {
            \box_use:N \l_tmpa_box
            \skip_horizontal:n { \marginparsep - \labelsep }
          }
      }
    \int_incr:N \l_@@_macro_int
  }
%    \end{macrocode}
% \end{macro}
%
% \begin{macro}[aux]{\@@_print_macro name:nN,\@@_print_macro~ name:nN}
% \cs{MacroFont} 已经放在了 \cs{@@_macro_typeset_one:nN}
% 中处理,此处也不再需要使用 \cs{MacroLongFont}。
% 命令中的空格改用“\textvisiblespace”显示。
%    \begin{macrocode}
\cs_set_protected:Npn \@@_print_macroname:nN #1#2
  {
    \strut
    \@@_get_hyper_target:xN
      {
        \exp_not:n {#1}
        \bool_if:NT #2 { \tl_to_str:n {TF} }
      }
      \l_@@_tmpa_tl
    \cs_if_exist:cTF { r@ \l_@@_tmpa_tl }
      { \exp_args:NNo \label@hyperref [ \l_@@_tmpa_tl ] }
      { \use:n }
      {
        \tl_set:Nn \l_@@_tmpa_tl { #1 }
        \tl_replace_all:Non \l_@@_tmpa_tl
          { \c_catcode_other_space_tl }
          { \fontspec_visible_space: }
        \@@_macroname_prefix:o \l_@@_tmpa_tl
        \@@_macroname_suffix:N #2
      }
  }
%    \end{macrocode}
% \end{macro}
%
% 在 |fdusyntax| 环境前设置若干活动字符。
% \texttt{\textbar} 分隔多个选项,无需倾斜;
% |<xxx>| 表示选项,|(xxx)| 表示默认选项。
%    \begin{macrocode}
\AtBeginEnvironment { fdusyntax }
  {
    \char_set_catcode_active:N \|
    \char_set_active_eq:NN \| \orbar
    \char_set_catcode_active:N \<
    \char_set_active_eq:NN \< \fduoptionsaux
    \char_set_catcode_active:N \(
    \char_set_active_eq:NN \( \defaultvalaux
  }
% \AtBeginEnvironment { syntax }
%   {
%     \char_set_catcode_active:N \|
%     \char_set_active_eq:NN \| \orbar
%     \char_set_catcode_active:N \(
%     \char_set_active_eq:NN \( \defaultvalaux
%   }
%    \end{macrocode}
%
% \begin{macro}{\StopSpecialIndexModule}
% \begin{macro}[aux]{\@@_special_index_module:nnnnN}
% 不对 \cs{cs} 和 \cs{tn} 等编索引。用于目录、索引等。
%    \begin{macrocode}
\DeclareDocumentCommand \StopSpecialIndexModule { }
  {
    \cs_set_eq:NN
      \@@_special_index_module:nnnnN \use_none:nnnnn
  }
\tl_map_inline:nn { \actualchar \encapchar \levelchar }
  { \exp_args:Nx \DoNotIndex { \bslash \tl_to_str:N #1 } }
%    \end{macrocode}
% \end{macro}
% \end{macro}
%
% \subsubsection{版本历史}
%    \begin{macrocode}
%<@@=ctxdoc>
%    \end{macrocode}
%
% \begin{macro}[aux]{\@@_ltx_changes:nnn}
% 保存 \pkg{doc} 中 \tn{changes@} 的定义。
%    \begin{macrocode}
\cs_new_eq:NN \@@_ltx_changes:nnn \changes@
%    \end{macrocode}
% \end{macro}
%
% \begin{macro}[int]{\changes@}
% \changes{v0.4}{2017/07/30}{修改索引排序方式。}
% 重定义 \tn{changes@},在版本号一行显示修改日期。
%    \begin{macrocode}
\cs_set_protected:Npn \changes@ #1#2
  {
    \@@_save_version_date:nn { #1 } { #2 }
    \@@_ltx_changes:nnn { #1 } { #2 }
  }
%    \end{macrocode}
% \end{macro}
%
% \begin{variable}[int]{\l_@@_tmpa_tl}
% 临时变量。
%    \begin{macrocode}
\tl_new:N \l_@@_tmpa_tl
\tl_new:N \l_@@_tmpb_tl
%    \end{macrocode}
% \end{variable}
%
% \begin{variable}[int]{\g_@@_version_date_prop}
% 存放版本号与对应的修改日期。
% key = 版本号,value = \{ 开始日期,结束日期 \}。
% 开始日期与结束日期可以相同。
%    \begin{macrocode}
\prop_new:N \g_@@_version_date_prop
%    \end{macrocode}
% \end{variable}
%
% \begin{macro}[int]{\@@_save_version_date:nn}
% 两个 |n| 的版本最终将被 \tn{changes@} 调用。
% |#1| = 版本号,|#2| = 日期。
% 它们分别对应 \tn{change} 的前两个参数(第三个是说明文字)。
%    \begin{macrocode}
\cs_new_protected:Npn \@@_save_version_date:nn #1#2
  {
    \prop_get:NnNTF \g_@@_version_date_prop
      { #1 } \l_@@_tmpa_tl
      {
%    \end{macrocode}
% \cs{l_@@_tmp_tl} 相当于两个参数(开始日期、结束日期),
% 因此需要提前展开。
%    \begin{macrocode}
        \exp_after:wN
          \@@_save_version_date_aux:nnnn \l_@@_tmpa_tl
        { #2 } { #1 }
      }
      { \@@_save_version_date:nnn { #1 } { #2 } { #2 } }
  }
%    \end{macrocode}
% \end{macro}
%
% \begin{macro}[aux]{\@@_save_version_date_aux:nnnn}
% |#1| = 原开始日期,|#2| = 原结束日期,|#3| = 新读入的日期,
% |#4| = 版本号。显然应有 |#1| < |#2|。\\
% 如果 |#3| < |#1|,则读入日期 |#3|、|#2|;
% 如果 |#3| > |#2|,则读入日期 |#1|、|#3|。
%    \begin{macrocode}
\cs_new_protected:Npn \@@_save_version_date_aux:nnnn #1#2#3#4
  {
    \@@_if_date_later:nnTF { #1 } { #3 }
      { \@@_save_version_date:nnn { #4 } { #3 } { #2 } }
      {
        \@@_if_date_later:nnT { #3 } { #2 }
          { \@@_save_version_date:nnn { #4 } { #1 } { #3 } }
      }
  }
%    \end{macrocode}
% \end{macro}
%
% \begin{macro}[aux]{\@@_save_version_date:nnn}
% 将版本号和日期存入 \cs{g_@@_version_date_prop}。
% |#1| = 版本号,|#2| = 开始日期,|#3| = 结束日期。
%    \begin{macrocode}
\cs_new_protected:Npn \@@_save_version_date:nnn #1#2#3
  {
    \prop_gput:Nnn \g_@@_version_date_prop
      { #1 } { { #2 } { #3 } }
  }
%    \end{macrocode}
% \end{macro}
%
% \begin{macro}[int,TF]{\@@_if_date_later:nn}
% \begin{macro}[aux]{\@@_parse_date:w}
% 比较两个日期。如果 |#1| 在 |#2| 之后,则为 true;反之为 false。
% 日期的格式为 YYYY/MM/DD。比较方法是直接将日期化成 8 位数字,
% 所以月、日前的 0 不可以省略。
%    \begin{macrocode}
\prg_new_conditional:Npnn \@@_if_date_later:nn #1#2 { TF , T }
  {
    \if_int_compare:w
        \@@_parse_date:w #1 / / / 0 \q_stop >
        \@@_parse_date:w #2 / / / 0 \q_stop \exp_stop_f:
      \prg_return_true:
    \else:
      \prg_return_false:
    \fi:
  }
\cs_new:Npn \@@_parse_date:w #1/#2/#3/ #4 \q_stop
  { #1#2#3 }
%    \end{macrocode}
% \end{macro}
% \end{macro}
%
% \begin{macro}[int]{\CTEX@versionitem}
% 版本条目标签。如果版本号不在 \cs{g_@@_version_date_prop} 的
% key 里面,则利用未定义的 \cs{BOOM} 报错。
%    \begin{macrocode}
\cs_new_protected:Npn \CTEX@versionitem #1 \efill
  {
    \@idxitem
    \prop_get:NnNTF \g_@@_version_date_prop
      { #1 } \l_@@_tmpa_tl
      {
        \exp_after:wN
          \@@_print_version_date:nnn \l_@@_tmpa_tl
        { #1 }
      }
      { \BOOM }
  }
%    \end{macrocode}
% \end{macro}
%
% \begin{macro}[aux]{\@@_print_version_date:nnn}
% 输出版本号和日期。如果开始日期和结束日期相同,则只输出一项。
% |#1| = 开始日期,|#2| = 结束日期,|#3| = 版本号。
%    \begin{macrocode}
\cs_new_protected:Npn \@@_print_version_date:nnn #1#2#3
  {
    \noindent
    \Hy@raisedlink { \belowpdfbookmark { #3 } { HD.#3 } }
    \textbf { #3 } \hfill
    \hbox:n
      {
        \footnotesize
        \str_if_eq:nnTF { #1 } { #2 }
          { ( #1 ) }
          { ( #1 ~ -- ~ #2 ) }
      }
    \par \nopagebreak
  }
%    \end{macrocode}
% \end{macro}
%
% \begin{macro}[int]{\HDorg@theglossary}
% 该命令由 \pkg{hypdoc} 宏包定义,用于存放标准文档类 \cls{book}
% 中定义的 \tn{theindex} 命令。
% 此处的补丁将在版本号一行最后加上修改日期。
%    \begin{macrocode}
\ctex_patch_cmd:Nnn \HDorg@theglossary
  { \let \item \@idxitem }
  { \let \item \CTEX@versionitem }
%    \end{macrocode}
% \end{macro}
%
% \begin{macro}[int]{\@wrglossary}
% 该命令由 \LaTeXe{} 内核定义,又由 \pkg{hypdoc} 宏包作了修改。
% 此处的补丁使得版本历史条目的页码能够指向对应行。
%    \begin{macrocode}
\ctex_patch_cmd:Nnn \@wrglossary
  { hdpindex }
  {
    \ifnum \c@HD@hypercount = \z@
      hdpindex
    \else
      hdclindex { \the \c@HD@hypercount }
    \fi
  }
%    \end{macrocode}
% \end{macro}
%
% \subsubsection{目录条目缩进}
% \begin{macro}[int]{\l@section,\l@subsection}
% 修正目录条目的缩进。
%    \begin{macrocode}
\ctex_patch_cmd:Nnn \l@section    { 2.5em } { 1.5em }
\ctex_patch_cmd:Nnn \l@subsection { 2.5em } { 1.5em }
%    \end{macrocode}
% \end{macro}
%
% \subsubsection{\texttt{macrocode} 环境}
% \paragraph{继承的代码}
% \begin{macro}[int]{\macro@code}
% 在 \pkg{doc} 宏包中,\env{macrocode} 环境的核心功能由
% \tn{macro@code} 负责实现,而 \tn{xmacro@code} 只用来结束
% \env{macrocode} 环境。但在 \cls{l3doc} 以及 \cls{ctxdoc} 中,
% \tn{xmacro@code} 则基本接管了 \tn{macro@code} 的功能。
% 后者此时只起辅助作用。
%    \begin{macrocode}
\ExplSyntaxOff
\def\macro@code{%
%    \end{macrocode}
% 调整前后间距,禁止 \env{macrocode} 环境前的分页。
%    \begin{macrocode}
   \topsep \MacrocodeTopsep
   \@beginparpenalty \predisplaypenalty
%    \end{macrocode}
% 将列表前后的附加垂直空白设为 0。根据 \cls{ctxdoc} 修改。
%    \begin{macrocode}
   \partopsep \z@skip
%    \if@inlabel\leavevmode\fi
%    \end{macrocode}
% 构建 \env{trivlist} 环境,设置段间距为 0。
% 之后修改字体,并调节左右间距。\tn{MacroIndent} 会根据代码行数
% 更新,具体细节见后文。
% \tn{macro@font} 用来在不同模块见切换字体。本文档类不使用
% \tn{AltMacroFont},因此这里改用 \tn{MacroFont} 代替。
%    \begin{macrocode}
   \trivlist \parskip \z@ \item[]%
%    \macro@font
   \MacroFont
   \leftskip\@totalleftmargin \advance\leftskip\MacroIndent
   \rightskip\z@ \parindent\z@ \parfillskip\@flushglue
%    \end{macrocode}
% 按照 \LaTeXe{} 中 \tn{verbatim} 环境中定义 \tn{par},使得空行
% 可以原样输出,否则空行会被吃掉。
%    \begin{macrocode}
   \blank@linefalse \def\par{\ifblank@line
                             \leavevmode\fi
                             \blank@linetrue\@@@@par
                             \penalty\interlinepenalty}
%    \end{macrocode}
% \tn{obeylines} 将把回车符 |^^M| 变成 \tn{par}。
% 接下来将所有特殊符号的类别码设为 12,即“其他”类。
%    \begin{macrocode}
   \obeylines
   \let\do\do@noligs \verbatim@nolig@list
   \let\do\@makeother \dospecials
%    \end{macrocode}
% 相当于退出 |\begin{list}| 和 |\begin{minipage}|。
%    \begin{macrocode}
   \global\@newlistfalse
   \global\@minipagefalse
%    \end{macrocode}
% 初始化交叉引用功能。
%    \begin{macrocode}
   \init@crossref}
\ExplSyntaxOn
%    \end{macrocode}
% \end{macro}
%
% \paragraph{处理代码行}
% \begin{macro}[int]{\xmacro@code,\sxmacro@code}
% 重新实现 \env{macrocode} 与 \env{macrocode*} 环境的核心功能,
% 将对代码逐行处理。后者会将空格显示为“\textvisiblespace”。
%    \begin{macrocode}
\cs_set_protected_nopar:Npn \xmacro@code
  { \@@_marco_code:w }
\cs_set_protected_nopar:Npn \sxmacro@code
  {
    \fontspec_print_visible_spaces:
    \xmacro@code
  }
%    \end{macrocode}
% \end{macro}
%
% \begin{macro}[int]{\@@_marco_code:w}
% 处理代码。
%    \begin{macrocode}
\cs_new_protected_nopar:Npn \@@_marco_code:w
  {
    \ifcodeline@index
      \@@_marco_code_every_par:n { \@@_code_line_no: }
    \else:
      \@@_marco_code_every_par:n { }
    \fi:
%    \end{macrocode}
% 设置代码段结束标记为“\verb*|%    \end{macrocode}^^M|”。
%    \begin{macrocode}
    \exp_args:Nx \@@_make_finish_tag:n { \@currenvir }
    \@@_macro_code_start:w
  }
%    \end{macrocode}
% \end{macro}
%
% \begin{macro}[int]{\@@_marco_code_every_par:n}
% 在每段之前插入内容。这里每段即相当于每行。
%    \begin{macrocode}
\cs_new_protected:Npn \@@_marco_code_every_par:n #1
  {
    \everypar
      {
        \everypar { #1 }
        \if@inlabel
          \global \@inlabelfalse
          \@noparlistfalse
          \llap { \box \@labels \hskip \leftskip }
        \fi
        #1
      }
  }
\group_begin:
%    \end{macrocode}
% \end{macro}
%
% 设置 \tn{endlinechar} 为 $-1$,表示行末不插入任何字符
% (实际上相当于在行尾插入注释符 |%|)。
%    \begin{macrocode}
  \int_set_eq:NN \tex_endlinechar:D \c_minus_one
%    \end{macrocode}
%
% \begin{variable}[int]{\c_@@_active_space_tl}
% 活动字符类的空格(ASCII 码为 32)。
%    \begin{macrocode}
  \use:n
    {
      \char_set_catcode_active:n { 32 }
      \tl_const:Nn \c_@@_active_space_tl
    }
    { }
%    \end{macrocode}
% \end{variable}
%
%    \begin{macrocode}
\group_end:
%    \end{macrocode}
%
% ASCII 码 13 是回车符 |^^M|。将其设置为活动字符。
%    \begin{macrocode}
\group_begin:
  \char_set_catcode_active:n { 13 }
%    \end{macrocode}
%
% \begin{macro}[aux]{\@@_make_finish_tag:n}
% \env{macrocode} 结尾标记。展开后变成
% “\verb*|%    \end{#1}^^M|”。
%    \begin{macrocode}
  \cs_new_protected:Npx \@@_make_finish_tag:n #1
    {
      \tl_set:Nn \exp_not:N \l_@@_macro_code_finish_tl
        {
          \c_percent_str
          \prg_replicate:nn { 4 }
            { \exp_not:o { \c_@@_active_space_tl } }
          \exp_not:o { \active@escape@char } end
          \c_left_brace_str #1 \c_right_brace_str
          \exp_not:N ^^M
        }
    }
%    \end{macrocode}
% \end{macro}
%
% \begin{macro}[aux]{\@@_macro_code_start:w}
% 开始代码环境。此命令主要是为了防止 |\begin{macrocode}|
% 后出现多余的空行。
%    \begin{macrocode}
  \cs_new_protected:Npn \@@_macro_code_start:w #1
    {
      \str_if_eq:nnTF { #1 } { ^^M }
        { \@@_macro_code_read_line:w }
        { \@@_macro_code_read_line:w #1 }
    }
%    \end{macrocode}
% \end{macro}
%
% \begin{macro}[aux]{\@@_macro_code_read_line:w}
% 逐行读取代码,并连同行尾回车符一并存入
% \cs{l_@@_macro_code_line_tl}。如果该行与结束标记
% “\verb*|%    \end{macrocode}^^M|”相同,则结束此
% \env{macrocode};否则继续处理该行代码。
%    \begin{macrocode}
  \cs_new_protected:Npn \@@_macro_code_read_line:w #1 ^^M
    {
      \tl_set:Nn \l_@@_macro_code_line_tl { #1 ^^M }
      \tl_if_eq:NNTF
        \l_@@_macro_code_line_tl \l_@@_macro_code_finish_tl
        { \exp_args:Nx \end { \@currenvir } }
        {
          \@@_macro_code_process_line:
          \@@_macro_code_read_line:w
        }
    }
%    \end{macrocode}
% \end{macro}
%
% \begin{variable}[aux]{\c_@@_active_cr_tl}
% 活动字符类的回车符。
%    \begin{macrocode}
  \tl_const:Nn \c_@@_active_cr_tl { ^^M }
\group_end:
%    \end{macrocode}
% \end{variable}
%
% \begin{variable}[int]{\l_@@_macro_code_line_tl,
%   \l_@@_macro_code_finish_tl,
%   \g_@@_macro_code_verbatim_stop_tl}
% 分别用来存储代码行、\env{macrocode} 结束标记以及抄录停止标记。
%    \begin{macrocode}
\tl_new:N \l_@@_macro_code_line_tl
\tl_new:N \l_@@_macro_code_finish_tl
\tl_new:N \g_@@_macro_code_verbatim_stop_tl
%    \end{macrocode}
% \end{variable}
%
% \begin{macro}[aux]{\@@_process_normal_line:}
% 处理普通代码行。
% 如果代码行开头是 |%|,则检查 |<| 后输出;否则正常输出。
%    \begin{macrocode}
\cs_new_protected_nopar:Npn \@@_process_normal_line:
  {
    \str_if_eq_x:nnTF
      { \str_head:N \l_@@_macro_code_line_tl }
      { \c_percent_str }
      {
%    \end{macrocode}
% 命令 \cs{tl_tail:N} 将去掉代码行开头的 |%|。之后检查 |<|。
%    \begin{macrocode}
        \@@_check_angle:x
          { \tl_tail:N \l_@@_macro_code_line_tl }
      }
      { \@@_output_line: }
  }
%    \end{macrocode}
% \end{macro}
%
% \begin{macro}[aux]{\@@_process_verbatim_line:}
% 处理抄录代码行(不检查 |%| 与 |<|)。
%    \begin{macrocode}
\cs_new_protected_nopar:Npn \@@_process_verbatim_line:
  {
%    \end{macrocode}
% 将该行与抄录停止标记进行比较。
%    \begin{macrocode}
    \tl_if_eq:NNTF \l_@@_macro_code_line_tl
        \g_@@_macro_code_verbatim_stop_tl
%    \end{macrocode}
% 若相同,则结束抄录环境,清空停止标记,并输出该标记;
%    \begin{macrocode}
      {
        \tl_gclear:N \g_@@_macro_code_verbatim_stop_tl
        \cs_gset_eq:NN \@@_macro_code_process_line:
          \@@_process_normal_line:
        \@@_output_verbatim_tag:nn
          { \color { verb@guard } } { end }
      }
%    \end{macrocode}
% 否则直接输出抄录代码。
%    \begin{macrocode}
      { \tl_use:N \l_@@_macro_code_line_tl }
  }
%    \end{macrocode}
% \end{macro}
%
% \begin{macro}[aux]{\@@_macro_code_process_line:}
% 处理代码行。
% 正常情况下,该命令等同于 \cs{@@_process_normal_line:};
% 在 \cs{DontCheckModules} 之后,等价于 \cs{@@_output_line:},
% 即不检查模块标记,直接输出;
% 在抄录环境中,等价于 \cs{@@_process_verbatim_line:},
% 此时将不再特殊处理 |%| 开头的代码行。
%    \begin{macrocode}
\cs_new_eq:NN \@@_macro_code_process_line:
  \@@_process_normal_line:
%    \end{macrocode}
% \end{macro}
%
% \paragraph{处理模块标记}
% \begin{macro}{\CheckModules,\DontCheckModules}
% 选择是否检查模块标记。这两个命令在 \pkg{doc} 宏包中已有定义,
% 此处重新声明。
%    \begin{macrocode}
\DeclareDocumentCommand \CheckModules { }
  {
    \cs_set_eq:NN \@@_macro_code_process_line:
      \@@_process_normal_line:
  }
\DeclareDocumentCommand \DontCheckModules { }
  {
    \cs_set_eq:NN \@@_macro_code_process_line:
      \@@_output_line:
  }
%    \end{macrocode}
% \end{macro}
%
% \begin{macro}[aux]{\@@_check_angle:n,\@@_check_angle:x}
% 该函数的参数不带 |%|。检查第一个字符是否是 |<|。
% 若是,则检查模块;否则立刻输出改行。
% 注意不带参数的函数 \cs{@@_output_percent_line:} 将会输出
% \cs{l_@@_macro_code_line_tl}。
%    \begin{macrocode}
\cs_new_protected:Npn \@@_check_angle:n #1
  {
    \str_if_eq_x:nnTF { \str_head:n { #1 } } { < }
      { \@@_check_module:x { \tl_tail:n { #1 } } }
      { \@@_output_percent_line: }
  }
\cs_generate_variant:Nn \@@_check_angle:n { x }
%    \end{macrocode}
% \end{macro}
%
% \begin{macro}[aux]{\@@_check_module:n,\@@_check_module:x}
% 检查紧跟 |<| 之后的字符。共有四种情况:
% \begin{itemize}
%   \item |*|:模块开始(|%<*|\meta{module}|>|);
%   \item |/|:模块结束(|%</|\meta{module}|>|);
%   \item |@|:名字空间(|%<@@=|\meta{namespace}|>|);
%   \item |<|:抄录环境开始(|%<<|\meta{end-tag})。
% \end{itemize}
% 若不是这几种情况,则为单独一行的独立模块
% (|%<|\meta{module}|>|)。
%    \begin{macrocode}
\cs_new_protected:Npn \@@_check_module:n #1
  {
    \exp_args:Nx \str_case:nnF { \str_head:n { #1 } }
      {
        { * } { \@@_module_star:w }
        { / } { \@@_module_slash:w }
        { @ } { \@@_module_at:w }
        { < } { \@@_module_verb:w }
      }
      { \@@_module_pm:w }
%    \end{macrocode}
% 参数 |#1| 将被上面几个 |:w| 型函数吃掉。
%    \begin{macrocode}
    #1 \q_stop
  }
\cs_generate_variant:Nn \@@_check_module:n { x }
%    \end{macrocode}
% \end{macro}
%
% 设置 |>| 为活动字符。
%    \begin{macrocode}
\group_begin:
  \char_set_catcode_active:N \>
%    \end{macrocode}
%
% \begin{macro}[int]{\@@_module_star:w}
% 处理模块开始。
% |#1| = |<*|\meta{module}|>|,|#2| 为之后的代码。
%    \begin{macrocode}
  \cs_new_protected:Npn \@@_module_star:w #1 > #2 \q_stop
    {
%    \end{macrocode}
% 临时变量 \cs{l_@@_tmp_tl} 保存 |<*|\meta{module}|>|
% 之后的部分,即真实代码。
%    \begin{macrocode}
      \tl_set:Nn \l_@@_tmpa_tl {#2}
%    \end{macrocode}
% 判断该行是否为空(只含一个回车符 |^^M|)。
% 若是,则将 |<|\meta{module}|>| 放在行号的右侧;
% 否则放在左侧,并输出代码。
%    \begin{macrocode}
      \tl_if_eq:NNTF \l_@@_tmpa_tl \c_@@_active_cr_tl
        {
          \@@_output_module:nn
            { \@@_star_color: }
            {
              \@@_module_push:n
                { \@@_module_angle:n {#1} }
            }
        }
        {
          \@@_output_module_left:nn
            { \@@_star_color: }
            {
              \@@_module_push:n
                { \@@_module_angle:n {#1} }
            }
        }
      \@@_output_line:n {#2}
      %\@@_star_format:
    }
%    \end{macrocode}
% \end{macro}
%
% \begin{macro}[int]{\@@_module_slash:w}
% 处理模块结束。结构与 \cs{@@_module_star:w} 相同。
% |#1| = |/<|\meta{module}|>|,|#2| 为之后的代码。
%    \begin{macrocode}
  \cs_new_protected:Npn \@@_module_slash:w #1 > #2 \q_stop
    {
      \tl_set:Nn \l_@@_tmpa_tl {#2}
      \tl_if_eq:NNTF \l_@@_tmpa_tl \c_@@_active_cr_tl
        {
          \@@_output_module:nn
            { \@@_slash_color: }
            {
              \@@_module_pop:n
                { \@@_module_angle:n {#1} }
            }
        }
        {
          \@@_output_module_left:nn
            { \@@_slash_color: }
            {
              \@@_module_pop:n
                { \@@_module_angle:n {#1} }
            }
        }
      \@@_output_line:n {#2}
      %\@@_slash_format:
    }
%    \end{macrocode}
% \end{macro}
%
% \begin{macro}[int]{\@@_module_at:w}
% 名字空间。|#1| 为名字空间的名称(不含 |@@=|),
% |#2| 为之后的代码。
%    \begin{macrocode}
  \cs_new_protected:Npn \@@_module_at:w @ @ = #1 > #2 \q_stop
    {
      \@@_output_module:nn
        { \color { at@guard } }
        { \@@_module_angle:n { @ @ = #1 } }
%    \end{macrocode}
% 设置名字空间为 |#1|。\cls{l3doc} 中将名字空间称为
% “模块”(module),注意不要混淆。
%    \begin{macrocode}
      \tl_gset:Nn \g__codedoc_module_name_tl {#1}
      \@@_output_line:n {#2}
    }
%    \end{macrocode}
% \end{macro}
%
% \begin{macro}[int]{\@@_module_verb:w}
% 抄录开始。|#1| = |<|\meta{end-tag},只有一个 |<|。
%    \begin{macrocode}
  \cs_new_protected:Npn \@@_module_verb:w #1 \q_stop
    {
%    \end{macrocode}
% 重定义 \cs{@@_macro_code_process_line:},用以输出抄录行。
%    \begin{macrocode}
      \cs_gset_eq:NN \@@_macro_code_process_line:
        \@@_process_verbatim_line:
%    \end{macrocode}
% 设置抄录停止标记。用 \cs{tl_tail:n} 去掉开头的 |<|。
%    \begin{macrocode}
      \tl_gset:Nx \g_@@_macro_code_verbatim_stop_tl
        { \c_percent_str \tl_tail:n {#1} }
%    \end{macrocode}
% 输出 |%<<|\meta{end-tag}。
%    \begin{macrocode}
      \@@_output_verbatim_tag:nn
        { \color { verb@guard } } { begin }
    }
%    \end{macrocode}
% \end{macro}
%
% \begin{macro}[int]{\@@_module_pm:w}
% 处理单独一行的模块。|<|\meta{module}|>| 放在行号的左侧。
% |#1| = 模块名,|#2| 为之后的代码。
%    \begin{macrocode}
  \cs_new_protected:Npn \@@_module_pm:w #1 > #2 \q_stop
    {
      \@@_output_module_left:nn
        { \@@_pm_color: }
        { \@@_module_angle:n { #1 } }
      \group_begin:
        %\@@_pm_format:
        \@@_output_line:n {#2}
      \group_end:
    }
\group_end:
%    \end{macrocode}
% \end{macro}
%
% \begin{macro}[int]{\@@_output_line:n}
% \begin{macro}[aux]{\@@_output_line:,\@@_output_percent_line:}
% 输出代码行。参数将被存入 \cs{l_@@_macro_code_line_tl},
% 再由不带参数的版本调用。
%    \begin{macrocode}
\cs_new_protected:Npn \@@_output_line:n #1
  {
    \tl_set:Nn \l_@@_macro_code_line_tl { #1 }
%    \end{macrocode}
% 若为空行(只含一个 |^^M|),则直接输出(换行)。
%    \begin{macrocode}
    \tl_if_eq:NNTF
      \l_@@_macro_code_line_tl \c_@@_active_cr_tl
      { \tl_use:N \l_@@_macro_code_line_tl }
      {
%    \end{macrocode}
% 检查开头是否为 |%|,据此分别处理。
%    \begin{macrocode}
        \str_if_eq_x:nnTF
          { \str_head:N \l_@@_macro_code_line_tl }
          { \c_percent_str }
          { \@@_output_percent_line: }
          { \@@_output_line: }
      }
  }
\cs_new_protected_nopar:Npn \@@_output_line:
  {
    \tex_noindent:D
%    \end{macrocode}
% 此处将把 |@@| 替换为相应的名字空间。
%    \begin{macrocode}
    \@@_replace_at_at:N \l_@@_macro_code_line_tl
    \tl_use:N \l_@@_macro_code_line_tl
  }
\cs_new_protected:Npn \@@_output_percent_line:
  {
    \tex_noindent:D
    \group_begin:
      \color { code@gray }
      \str_if_eq_x:nnTF { \f@shape } { \updefault }
        { \slshape }
        { \upshape }
      \@@_output_line:
    \group_end:
  }
%    \end{macrocode}
% \end{macro}
% \end{macro}
%
% \begin{macro}[int]{\@@_replace_at_at:N}
% \begin{macro}[aux]{\@@_replace_at_at_aux:Nn}
% 把 |@@| 替换为相应的名字空间。其名称存放在全局变量
% \cs{g__codedoc_module_name_tl} 中。
% 它为空时(|%<@@=>|),不做替换。
%    \begin{macrocode}
\cs_new_protected:Npn \@@_replace_at_at:N #1
  {
    \tl_if_empty:NF \g__codedoc_module_name_tl
      {
        \exp_args:NNo \@@_replace_at_at_aux:Nn
          #1 \g__codedoc_module_name_tl
      }
  }
\cs_new_protected:Npn \@@_replace_at_at_aux:Nn #1#2
  {
%    \end{macrocode}
% 两项分别对应内部变量(“|\|\meta{type}|_@@_|\meta{name}” $\to$
% “|\|\meta{type}|__|\meta{namespace}|_|\meta{name}”)
% 和函数(“|\@@_|\meta{name}” $\to$
% “|\__|\meta{namespace}|_|\meta{name}”)。
%    \begin{macrocode}
    \tl_replace_all:Nnn #1 { _ @ @ } { _ _ #2 }
    \tl_replace_all:Nnn #1 {   @ @ } { _ _ #2 }
  }
%    \end{macrocode}
% \end{macro}
% \end{macro}
%
% \begin{macro}[int]{\@@_module_push:n}
% \begin{macro}[aux]{\@@_module_push_aux:nn}
% 将模块名压入栈中。
%    \begin{macrocode}
\cs_new_protected_nopar:Npn \@@_module_push:n
  {
    \exp_args:No \@@_module_push_aux:nn
      { \int_use:N \c@HD@hypercount }
  }
\cs_new_protected:Npn \@@_module_push_aux:nn #1
  {
    \seq_gpush:Nn \g_@@_module_dest_seq { #1 }
    \hypersetup { hidelinks }
    \exp_args:Nx \hdclindex
      { \zref@extractdefault { HD.#1 } { guard@end } { 1 } } { }
  }
%    \end{macrocode}
% \end{macro}
% \end{macro}
%
% \begin{macro}[int]{\@@_module_pop:n}
% \begin{macro}[aux]{\@@_module_pop_aux:nn}
% 将模块名释放出栈。
%    \begin{macrocode}
\cs_new_protected_nopar:Npn \@@_module_pop:n
  {
    \seq_gpop:NNTF \g_@@_module_dest_seq \l_@@_tmpa_tl
      {
        \exp_args:No \@@_module_pop_aux:nn
          { \l_@@_tmpa_tl }
      }
      { \BOOM \use:n }
  }
\cs_new_protected:Npn \@@_module_pop_aux:nn #1
  {
    \zref@labelbylist { HD.#1 } { ctxdoc }
    \hypersetup { hidelinks }
    \hdclindex { #1 } { }
  }
%    \end{macrocode}
% \end{macro}
% \end{macro}
%
% \begin{variable}{\g_@@_module_dest_seq}
% 存放模块名的序列(栈)。
%    \begin{macrocode}
\seq_new:N \g_@@_module_dest_seq
%    \end{macrocode}
% \end{variable}
%
% 处理行号超链接。使用 \pkg{zref} 宏包。
%    \begin{macrocode}
\zref@newlist { ctxdoc }
\zref@newprop { guard@end } [ 1 ]
  { \int_eval:n { \c@HD@hypercount - 1 } }
\zref@addprop { ctxdoc } { guard@end }
%    \end{macrocode}
%
% \paragraph{格式处理}
% \begin{macro}{\MacroFont}
% 代码部分的字体。
%    \begin{macrocode}
\cs_set_protected:Npn \MacroFont
  {
    \linespread { 1.05 }
    \small \ttfamily \mdseries \upshape
    \ctexdocverbaddon
  }
%    \end{macrocode}
% \end{macro}
%
% \begin{macro}[int]{\@@_output_module:nn,
%   \@@_output_module_left:nn}
% \changes{v0.4}{2017/08/01}{允许模块名在行号左侧显示。}
% 输出模块名(分行内和行号左侧两种)。
% |#1| = 颜色等样式,|#2| = 模块名。
%    \begin{macrocode}
\cs_new_protected:Npn \@@_output_module:nn #1#2
  {
    \tex_noindent:D
    \group_begin:
      #1
      \footnotesize \normalfont \sffamily #2
    \group_end:
  }
\cs_new_protected:Npn \@@_output_module_left:nn #1#2
  {
    \tex_noindent:D
    \hbox_overlap_left:n
      {
        \@@_output_module:nn {#1} {#2}
        \skip_horizontal:n { \leftskip + \smallskipamount }
      }
  }
%    \end{macrocode}
% \end{macro}
%
% \begin{macro}[int]{\@@_output_verbatim_tag:nn}
% \changes{v0.4}{2017/08/01}{修复原 \cls{ctxdoc} 文档类中行距
%   偏小的问题。}
% 输出抄录标记。
% |#1| = 颜色等样式,|#2| = |begin| 或 |end|。
% 在 \cls{ctxdoc} 中,抄录标记使用 |footnotesize|,
% 使得该行行距偏小
% (见 \url{https://github.com/CTeX-org/ctex-kit/issues/303})。
% 这里的解决方案是先去掉 \cs{l_@@_macro_code_line_tl} 最后的
% 回车符 |^^M|,接着正常输出(做一个判断),最后补上 \tn{par}
% 并换行。
%    \begin{macrocode}
\cs_new_protected:Npn \@@_output_verbatim_tag:nn #1#2
  {
    \tl_reverse:N \l_@@_macro_code_line_tl
%    \end{macrocode}
% \cs{l_@@_tmpb_tl} 被 \cs{@@_module_pop:n} 使用了。
%    \begin{macrocode}
    \tl_set:Nx \l_@@_tmpb_tl
      { \tl_tail:N \l_@@_macro_code_line_tl }
    \tl_reverse:N \l_@@_tmpb_tl
%    \end{macrocode}
% 根据 |#2| 是 |begin| 还是 |end| 来判断应该进栈(push)
% 还是出栈(pop)。
%    \begin{macrocode}
    \tl_if_eq:nnTF {#2} { begin }
      {
        \@@_output_module:nn {#1}
          { \@@_module_push:n { \l_@@_tmpb_tl } }
      }
      {
        \tl_if_eq:nnTF {#2} { end }
          {
            \@@_output_module:nn {#1}
              { \@@_module_pop:n { \l_@@_tmpb_tl } }
          }
          { \BOOM }
      }
    \par
  }
%    \end{macrocode}
% \end{macro}
%
% \begin{macro}[int]{\@@_star_color:,
%   \@@_slash_color:,
%   \@@_pm_color:}
% 选择模块标记的颜色。模块标记的颜色会根据嵌套层次而改变。
%    \begin{macrocode}
\cs_new_protected_nopar:Npn \@@_star_color:
  {
    \seq_gpop:NNTF \g_@@_star_color_seq \current@color
      { \set@color }
      { \@@_select_color: }
    \seq_gpush:No \g_@@_slash_color_seq { \current@color }
  }
\cs_new_protected_nopar:Npn \@@_slash_color:
  {
    \seq_gpop:NNTF \g_@@_slash_color_seq \current@color
      {
        \set@color
        \seq_gpush:No \g_@@_star_color_seq { \current@color }
      }
      %TODO: 报错:star 与 slash 没有匹配
      { \BOOM }
  }
\cs_new_protected_nopar:Npn \@@_pm_color:
  {
    \seq_get:NNTF \g_@@_star_color_seq \current@color
      { \set@color }
      {
        \@@_select_color:
        \seq_gpush:No \g_@@_star_color_seq { \current@color }
      }
  }
%    \end{macrocode}
% \end{macro}
%
% \begin{variable}[int]{\g_@@_star_color_seq,
%   \g_@@_slash_color_seq}
%    \begin{macrocode}
\seq_new:N \g_@@_star_color_seq
\seq_new:N \g_@@_slash_color_seq
%    \end{macrocode}
% \end{variable}
%
% \begin{macro}[int]{\@@_select_color:,guard@series}
% 设置模块标记的颜色序列。\tn{resetcolorseries} 的参数 |3|
% 比嵌套最大深度少 1。
%    \begin{macrocode}
\cs_new_protected_nopar:Npn \@@_select_color:
  { \color { guard@series!!+ } }
\definecolorseries { guard@series }
  { cmyk } { last } { blue } { purple }
\resetcolorseries [ 3 ] { guard@series }
%    \end{macrocode}
% \end{macro}
%
% \begin{macro}[int]{verb@guard,at@guard,code@gray}
% 设置颜色。
%    \begin{macrocode}
\definecolor { verb@guard } { named } { MaterialLime600 }
\definecolor { at@guard }   { named } { MaterialPink    }
\definecolor { code@gray }  { named } { MaterialGrey    }
%    \end{macrocode}
% \end{macro}
%
% \begin{macro}[int]{\@@_module_angle:n}
% 输出“\textlangle\ldots\textrangle”。
%    \begin{macrocode}
\cs_new_protected:Npn \@@_module_angle:n #1
  { \textlangle #1 \textrangle }
%    \end{macrocode}
% \end{macro}
%
% \begin{macro}[int]{\@@_code_line_no:}
% 行号。设置为阿拉伯数字。
%    \begin{macrocode}
\cs_new_protected_nopar:Npn \@@_code_line_no:
  {
    \int_gincr:N \c@CodelineNo
    \hbox_overlap_left:n
      {
        \hbox_to_wd:nn
          { \MacroIndent }
          {
            \HD@target
            \tex_hss:D
            \@@_code_line_no_style:
            \theCodelineNo \enspace
          }
        \tex_kern:D \@totalleftmargin
      }
  }
\tl_set:Nn \theCodelineNo { \arabic { CodelineNo } }
%    \end{macrocode}
% \end{macro}
%
% \begin{macro}[aux]{\@@_code_line_no_style:}
% 行号格式。
%    \begin{macrocode}
\cs_new_protected_nopar:Npn \@@_code_line_no_style:
  { \color { code@gray } \normalfont \sffamily \tiny }
%    \end{macrocode}
% \end{macro}
%
% \begin{macro}[int]{\HD@SetMacroIndent}
% 设置代码缩进(行号一栏的宽度)。该命令会写进 |.aux| 辅助文件,
% 以便在二次编译时确定最大行号宽度。
%    \begin{macrocode}
\cs_set_protected:Npn \HD@SetMacroIndent #1
  {
    \group_begin:
      \settowidth \MacroIndent
        {
          \@@_code_line_no_style:
          \prg_replicate:nn { \tl_count:n { #1 } } { 0 }
          \enspace
        }
      \dim_gset_eq:NN \MacroIndent \MacroIndent
    \group_end:
  }
%    \end{macrocode}
% \end{macro}
%
% \begin{macro}{\meta}
%    \begin{macrocode}
\RenewDocumentCommand \meta { m }
  {
    \group_begin:
%    \end{macrocode}
% 禁用中文、西文之间的空格。
%    \begin{macrocode}
      \sys_if_engine_xetex:T { \xeCJKsetup{CJKecglue={}} }
      \__codedoc_meta:n { #1 }
    \group_end:
  }
%    \end{macrocode}
% \end{macro}
%
% \begin{macro}[aux]{\__codedoc_typeset_functions:}
% 调整 \env{function} 环境的字体。
%    \begin{macrocode}
\ctex_patch_cmd_once:NnnnTF \__codedoc_typeset_functions: { }
  { \small \ttfamily }
  { \footnotesize \CodeFont \CJKCodeFont }
  { } { \ctex_patch_failure:N \__codedoc_meta_original:n }
%    \end{macrocode}
% \end{macro}
%
% \subsubsection{\LaTeXe{} 风格的命令与补丁}
%    \begin{macrocode}
\ExplSyntaxOff
%    \end{macrocode}
%
% 调整浮动体、代码等与文字的间距。
% 见 \url{http://tex.stackexchange.com/a/40896}。
% \begin{macro}[int]{\@addtocurcol}
%    \begin{macrocode}
\patchcmd{\@addtocurcol}%
  {\vskip \intextsep}%
  {\edef\save@first@penalty{\the\lastpenalty}\unpenalty
    \ifnum \lastpenalty = \@M  % hopefully the OR penalty
       \unpenalty
    \else
       \penalty \save@first@penalty \relax % put it back
    \fi
    \ifnum\outputpenalty <-\@Mii
                       \addvspace\intextsep
                       \vskip\parskip
    \else
                       \addvspace\intextsep
    \fi}%
  {\typeout{Info: Command `@addtocurcol' patched successfully.}}
  {\typeout{Warning: Command `@addtocurcol' patched failed.}}
\patchcmd{\@addtocurcol}%
  {\vskip\intextsep
    \ifnum\outputpenalty <-\@Mii \vskip -\parskip\fi}%
  {\ifnum\outputpenalty <-\@Mii
      \aftergroup\vskip\aftergroup\intextsep
      \aftergroup\nointerlineskip
    \else
      \vskip\intextsep
    \fi}%
  {\typeout{Info: Command `@addtocurcol' patched successfully.}}
  {\typeout{Warning: Command `@addtocurcol' patched failed.}}
%    \end{macrocode}
% \end{macro}
%
% \begin{macro}[int]{\@getpen}
%    \begin{macrocode}
\patchcmd{\@getpen}{\@M}{\@Mi}
  {\typeout{Info: Command `@getpen' patched successfully.}}
  {\typeout{Warning: Command `@getpen' patched failed.}}
%    \end{macrocode}
% \end{macro}
%
% \begin{macro}[int]{\ctexdocverbaddon,\ctexdisableecglue,
%   \ctexplainps}
% 调整文字间距,以便于让 CJK 字符占的宽度等于西文等宽字体中两个
% 空格的宽度。需要按编译情况分别定义。
%    \begin{macrocode}
\ifxetex
  \let\ctexdocverbaddon\xeCJKVerbAddon
  \def\ctexdisableecglue{\xeCJKsetup{CJKecglue}}
  \def\ctexplainps{\xeCJKsetup{PunctStyle=plain}}
  \appto\meta@font@select{\ifinner\ctexdisableecglue\fi}
\else
  \let\ctexdocverbaddon\relax
  \def\ctexdisableecglue{\ltjsetparameter{autoxspacing=false}}
  \let\ctexplainps\relax
  \appto\meta@font@select{\ctexdisableecglue}
\fi
%    \end{macrocode}
% \end{macro}
%
% 设置标准列表环境样式。
%    \begin{macrocode}
\setlist{noitemsep, topsep=\smallskipamount}
\setlist[1]{labelindent=\parindent}
\setlist[enumerate]{leftmargin=*}
\setlist[itemize]{leftmargin=*}
%    \end{macrocode}
%
% \begin{macro}{optdesc}
% 用于描述各选项。设置条目间距为 \tn{marginparsep},与
% \cls{l3doc} 一致。
%    \begin{macrocode}
\newlist{optdesc}{description}{3}
\setlist[optdesc]{%
  font=\mdseries\small\ttfamily, align=right,
  listparindent=\parindent,
  labelsep=\marginparsep, labelindent=-\marginparsep,
  leftmargin=*}
%    \end{macrocode}
% \end{macro}
%
% \begin{macro}{tablenotes}
% 重新定义 \pkg{threeparttable} 包的 |tablenotes| 环境,
% 用于表格的注释。
%    \begin{macrocode}
\renewlist{tablenotes}{description}{1}
\setlist[tablenotes]{%
  format=\normalfont\tnote@item, align=right,
  listparindent=\parindent, labelindent=\tabcolsep,
  leftmargin=*, rightmargin=\tabcolsep,
  after=\@noparlisttrue}
\AtBeginEnvironment{tablenotes}{%
  \setlength\parindent{2\ccwd}%
  \normalfont\footnotesize}
\AtBeginEnvironment{threeparttable}{%
  \stepcounter{tpt@id}%
  \edef\curr@tpt@id{tpt@\arabic{tpt@id}}}
\newcounter{tpt@id}
%    \end{macrocode}
% \end{macro}
%
% \begin{macro}[aux]{\tnote@item,\TPTtagStyle}
%    \begin{macrocode}
\def\tnote@item#1{%
  \Hy@raisedlink{\hyper@anchor{\curr@tpt@id-#1}}#1}
\def\TPTtagStyle#1{\hyperlink{\curr@tpt@id-#1}{#1}}
%    \end{macrocode}
% \end{macro}
%
% 调整 |function| 环境前后间距。
%    \begin{macrocode}
\BeforeBeginEnvironment{function}{\par\nointerlineskip}
\AtEndEnvironment{function}{%
  \par\xdef\ctexfixprevdepth{\prevdepth=\the\prevdepth\space}}
\AfterEndEnvironment{function}{\ctexfixprevdepth}
%    \end{macrocode}
%
% 调整 |syntax| 环境的样式。
%    \begin{macrocode}
\AtBeginEnvironment{syntax}{%
  \linespread{1.2}\ctexplainps\ctexdisableecglue}
%    \end{macrocode}
%
% \begin{macro}{\exptarget,\rexptarget,\expstar,\rexpstar}
% 部分命令之后的星号($\star$ 或 \ding{73}),表明其不同用法。
% 这里的“exp”和“rexp”分别源自 \LaTeX3 中的“expandable”
% 和“restricted-expandable”。
%    \begin{macrocode}
\newcommand*\exptarget{\Hy@raisedlink{\hypertarget{expstar}{}}}
\newcommand*\rexptarget{\Hy@raisedlink{\hypertarget{rexpstar}{}}}
\newcommand*\expstar{\hyperlink{expstar}{$\star$}}
\newcommand*\rexpstar{\hyperlink{rexpstar}{\ding{73}}}
%    \end{macrocode}
% \end{macro}
%
%    \begin{macrocode}
\AtBeginDocument{%
  \addtocontents{toc}{\StopSpecialIndexModule}}
\pdfstringdefDisableCommands{%
  \let\path\meta
  \let\opt\@firstofone}
\preto\@thehead{\cslet{MakeUppercase\space}{\@iden}}
%    \end{macrocode}
%
% \begin{macro}{\orbar,\defaultval,\TF,\TTF,\TFF,
%   \opt,\bashcmd}
% \begin{macro}[aux]{\defaultvalaux}
% |syntax| 中需要的命令。
%    \begin{macrocode}
\def\orbar{\textup{\textbar}}
\def\defaultval#1{\textbf{\textup{#1}}}
\def\defaultvalaux#1){\defaultval{#1}}
\def\TF{true\orbar false}
\def\TTF{\defaultval{true}\orbar false}
\def\TFF{true\orbar\defaultval{false}}
%%%%%%%%%%%%%%%%%%%%%%%%%%%%%%%%%%%%%%%%%%%%%
\protected\def\opt{\texttt}
\def\bashcmd{\texttt}
%    \end{macrocode}
% \end{macro}
% \end{macro}
%
% \begin{macro}{\TeX,\LaTeX,\LaTeXe,\LaTeXiii,
%   \pdfTeX,\pdfLaTeX,\XeTeX,\XeLaTeX,\LuaTeX,\LuaLaTeX,
%   \AmSLaTeX,\TeXLive,\MiKTeX}
% \TeX{} 相关标志。
%    \begin{macrocode}
\def\TeX{\hologo{TeX}}
\def\LaTeX{\hologo{LaTeX}}
\def\LaTeXe{\hologo{LaTeX2e}}
\def\LaTeXiii{\hologo{LaTeX3}}
\def\pdfTeX{\hologo{pdfTeX}}
\def\pdfLaTeX{\hologo{pdfLaTeX}}
\def\XeTeX{\hologo{XeTeX}}
\def\XeLaTeX{\hologo{XeLaTeX}}
\def\LuaTeX{\hologo{LuaTeX}}
\def\LuaLaTeX{\hologo{LuaLaTeX}}
\def\AmSLaTeX{\hologo{AmSLaTeX}}
\def\TeXLive{\TeX\ Live}
\def\MiKTeX{\hologo{MiKTeX}}
%    \end{macrocode}
% \end{macro}
%
% \begin{macro}{\IndexLayout,\usage}
% 修改索引外观。\tn{usage} 用来控制“描述对应索引项的页码”样式,
% 在 \pkg{doc} 宏包中原来定义为 \tn{textit}。
%    \begin{macrocode}
\IndexPrologue{%
  \section{\indexname}%
  \textit{无衬线字体的数字表示描述对应索引项的页码;
  带下划线的数字表示定义对应索引项的代码行号;
  罗马字体的数字表示使用对应索引项的代码行号(或页码).}}
\def\IndexLayout{%
  \newgeometry{%
    left   = 1.00 in,
    right  = 1.00 in,
    top    = 1.25 in,
    bottom = 1.00 in}%
  \setlength\IndexMin{.5\textheight}%
  \ctexset{section/numbering=false}%
  \StopSpecialIndexModule}
\def\usage#1{\textbf{\textsf{#1}}}
%    \end{macrocode}
% \end{macro}
%
% \begin{macro}{\glossaryname}
% 修改版本历史外观。
%    \begin{macrocode}
\appto\GlossaryParms{%
  \raggedcolumns
  \let\Hy@writebookmark\HDorg@writebookmark
  \def\@idxitem{\par\hangindent 2em }%
  \def\subitem{\@idxitem\hspace*{1em}}%
  \def\subsubitem{\@idxitem\hspace*{2em}}}
\GlossaryPrologue{\section{\glossaryname}}
\def\glossaryname{版本历史}
%    \end{macrocode}
% \end{macro}
%
% \begin{macro}{\thebibliography}
% \begin{macro}[aux]{\HDorg@thebibliography}
% 参考文献一节需要编号。
%    \begin{macrocode}
\patchcmd{\HDorg@thebibliography}
  {\section*}{\section}
  {\typeout{Info: Command `HDorg@thebibliography'
    patched successfully.}}
  {\typeout{Warning: Command `HDorg@thebibliography'
    patched failed.}}
\let\thebibliography\HDorg@thebibliography
%    \end{macrocode}
% \end{macro}
% \end{macro}
%
% 设置字体。
%    \begin{macrocode}
% \setmainfont{TeX Gyre Pagella}
% \setsansfont{TeX Gyre Heros}
% \setmonofont{CMU Typewriter Text}[%
%   UprightFont = * Light,
%   BoldFont    = * Bold,
%   SlantedFont = * Light Oblique,
%   HyphenChar  = None]
% \setmathfont{TeX Gyre Pagella Math}
% \setCJKmainfont{FZShuSong-Z01}%
%   [BoldFont = FZHei-B01, ItalicFont = FZKai-Z03]
% \setCJKsansfont{FZHei-B01}%
%   [BoldFont = FZHei-B01, ItalicFont = FZKai-Z03]
% \setCJKmonofont{FZFangSong-Z02}%
%   [BoldFont = FZHei-B01, ItalicFont = FZKai-Z03]
% % \setCJKfamilyfont{宋}{FZShuSong-Z01}
% % \setCJKfamilyfont{黑}{FZHei-B01}
% % \setCJKfamilyfont{仿}{FZFangSong-Z02}
% \setCJKfamilyfont{楷}{FZKai-Z03}
% \newfontfamily\CodeFont{Source Code Pro}%
%   [BoldFont = Source Code Pro Semibold]
% \newCJKfontfamily\CJKCodeFont{Source Han Sans SC}%
%   [ItalicFont = FZKai-Z03]
\setmainfont{Linux Libertine O}
\setsansfont{Linux Biolinum O}
\setmonofont{CMU Typewriter Text}[%
  UprightFont = * Light,
  BoldFont    = * Bold,
  SlantedFont = * Light Oblique,
  HyphenChar  = None]
\setmathfont{Libertinus Math}
\setCJKmainfont{Source Han Serif SC}%
  [BoldFont = Source Han Sans SC Medium, ItalicFont = FandolKai]
\setCJKsansfont{Source Han Sans SC Medium}%
  [BoldFont = Source Han Sans SC Bold, ItalicFont = FandolKai]
\setCJKmonofont{FandolFang}%
  [BoldFont = Source Han Sans SC Medium, ItalicFont = FandolKai]
\setCJKfamilyfont{楷}{FandolKai}
\newfontfamily\CodeFont{Source Code Pro}%
  [BoldFont = Source Code Pro Semibold]
\newCJKfontfamily\CJKCodeFont{Source Han Sans SC}%
  [BoldFont = Source Han Sans SC Bold, ItalicFont = FandolKai]
%    \end{macrocode}
%
% 中文排版格式。
%    \begin{macrocode}
\ctexset
  {
    section = {name = {第,节}, format += \raggedright },
    subsubsection / tocline = {\CJKfamily{楷}\CTEXnumberline{#1}#2},
    paragraph / runin = false,
    indexname = {代码索引}
  }
%    \end{macrocode}
%
% 单位设置。
%    \begin{macrocode}
\sisetup
  {
    number-math-rm       = \ensuremath,
    inter-unit-product   = \ensuremath{{}\cdot{}},
    group-digits         = true,
    group-minimum-digits = 4,
    group-separator      = \text{~},
    range-phrase         = \symbol{"FF5E},
    separate-uncertainty = true
  }
%    \end{macrocode}
%
% 超链接设置。
%    \begin{macrocode}
\hypersetup
  {
    bookmarksopen      = true,
    bookmarksopenlevel = 4,
    bookmarksnumbered  = true,
    colorlinks         = true,
    linkcolor          = MaterialPink,
    citecolor          = MaterialGreen,
    urlcolor           = MaterialIndigo
  }
%    \end{macrocode}
%
% 浮动体标题设置。
%    \begin{macrocode}
% \captionsetup[figure]%
%   {labelsep = quad, justification = centering}
\captionsetup[table]%
  {labelsep = quad, font = sf, justification = centering}
%    \end{macrocode}
%
% \pkg{listings} 代码样式。
%    \begin{macrocode}
\lst@CCPutMacro\lst@ProcessOther {"2D}{\lst@ttfamily{-{}}{-{}}}
\@empty\z@\@empty

\lstdefinestyle{lstStyleBase}{%
  extendedchars   = true,
  gobble          = 3,
  lineskip        = 2 pt,
  frame           = l,
  framerule       = 1 pt,
  framesep        = 0 pt,
  escapeinside    = {(*}{*)},
  %basicstyle      = \small\ttfamily\color{MaterialGrey900},
  basicstyle      = \small\CodeFont\CJKCodeFont%
    \color{MaterialGrey900},
  keywordstyle    = \bfseries\color{MaterialIndigo},
  commentstyle    = \itshape\color{MaterialGrey600},
  stringstyle     = \color{MaterialDeepOrange},
  backgroundcolor = \color{MaterialGrey50}
}

\lstdefinestyle{lstStyleShell}{%
  style           = lstStyleBase,
  rulecolor       = \color{MaterialPink},
  language        = bash,
  alsoletter      = {-},
  morekeywords    = {mkdir,cp,git,fc,%
    tex,pdftex,xetex,luatex,%
    latex,pdflatex,xelatex,lualatex,%
    bibtex,makeindex,latexmk},
  emph            = {-o,-s,-t,-xelatex,-lualatex},
  emphstyle       = \color{MaterialGreen800}
}

\lstdefinestyle{lstStyleLaTeX}{%
  style           = lstStyleBase,
  rulecolor       = \color{MaterialIndigo},
  language        = [LaTeX]TeX,
  texcsstyle      = *\bfseries\color{MaterialDeepOrange},
  deletetexcs     = {\documentclass},
  moretexcs       = {\chapter},
  morekeywords    = {\documentclass,\fdusetup},
  emph            = {style,info,%
    font,fontsize,%
    author,department,title},
  emphstyle       = \color{MaterialGreen800}
}

\lstdefinestyle{lstStyleSyntax}{
  extendedchars   = true,
  gobble          = 6,
  % lineskip        = 2 pt,
  escapeinside    = {(*}{*)},
  language        = [LaTeX]TeX,
  alsoletter      = {*},
  basicstyle      = \footnotesize\CodeFont\CJKCodeFont%
    \color{MaterialGrey900},
  keywordstyle    = \bfseries\color{MaterialIndigo},
  commentstyle    = \itshape\color{MaterialGrey600},
  stringstyle     = \color{MaterialRed},
  texcsstyle      = *\bfseries\color{MaterialDeepOrange},
  deletetexcs     = {\documentclass},
  moretexcs       = {\chapter},
  morekeywords    = {\documentclass,\fdusetup},
  emph            = [1]{oneside,twoside,nofonts,draft,%模板选项
      info,style,%meta 选项
      automakecover,cjkfont,font,fontsize,footnotestyle,%
      fullwidthstop,%style
      author,author*,clc,date,department,department*,instructors,%
      keywords,keywords*,major,schoolid,secretlevel,studentid,%
      supervisor,title,title*%info
    },
  emphstyle       = [1]\color{MaterialGreen800},
  emph            = [2]{abstract,abstract*,notation,%
      axiom,corollary,definition,example,lemma,proof,theorem%
    },
  emphstyle       = [2]\color{MaterialBlue900},
}

\lstnewenvironment{shellexample}{\lstset{style=lstStyleShell}}{}
\lstnewenvironment{latexexample}{\lstset{style=lstStyleLaTeX}}{}
\lstnewenvironment{fdusyntax}{%
  \lstset{style=lstStyleSyntax}\vspace{-1.8ex}}{}
%    \end{macrocode}
%
% 引用环境。
%    \begin{macrocode}
\ExplSyntaxOn
\NewDocumentEnvironment { fduquote } { o o }
  { \quote \ttfamily \qquad }
  {
    \endquote
    \IfNoValueF {#1} { \hfill —— \IfNoValueF {#2} {〔#2〕} #1 }
  }
\ExplSyntaxOff
%    \end{macrocode}
%
% \begin{macro}[int]{\@makefntext}
% 脚注文本。
%    \begin{macrocode}
\renewcommand\@makefntext[1]{%
  \parindent 0em \noindent\@makefnmark~#1}
%    \end{macrocode}
% \end{macro}
%
% 网址断行。代码来自于 \url{https://liam0205.me/2017/05/17/help-the-url-command-from-hyperref-to-break-at-line-wrapping-point/}。
%    \begin{macrocode}
\def\UrlAlphabet{%
      \do\a\do\b\do\c\do\d\do\e\do\f\do\g\do\h\do\i\do\j%
      \do\k\do\l\do\m\do\n\do\o\do\p\do\q\do\r\do\s\do\t%
      \do\u\do\v\do\w\do\x\do\y\do\z\do\A\do\B\do\C\do\D%
      \do\E\do\F\do\G\do\H\do\I\do\J\do\K\do\L\do\M\do\N%
      \do\O\do\P\do\Q\do\R\do\S\do\T\do\U\do\V\do\W\do\X%
      \do\Y\do\Z}
\def\UrlDigits{\do\1\do\2\do\3\do\4\do\5\do\6\do\7\do\8\do\9\do\0}
\g@addto@macro{\UrlBreaks}{\UrlOrds}
\g@addto@macro{\UrlBreaks}{\UrlAlphabet}
\g@addto@macro{\UrlBreaks}{\UrlDigits}
%    \end{macrocode}
%
% 命令。
%    \begin{macrocode}
% \let\DocCs=\cs
% \let\DocTn=\tn
% \renewcommand\cs[2][]{\textcolor{MaterialIndigo}{\DocCs[#1]{#2}}}
% \renewcommand\tn[2][]{\textcolor{MaterialPink}{\DocTn[#1]{#2}}}

% \renewcommand{\pkg}[1]{\textsf{#1}}
% \renewcommand{\cls}[1]{\textsf{#1}}

\let\OldMeta=\meta
\renewcommand\meta[1]{{\rmfamily\OldMeta{#1}}}
\renewcommand\marg[1]{\{\meta{#1}\}}
\renewcommand\oarg[1]{[\meta{#1}]}
\renewcommand\parg[1]{(\meta{#1})}

% 与之前 \defaultvalaux 定义相同
\def\fduoptions#1{\textit{#1}}
\def\fduoptionsaux#1>{\fduoptions{#1}}

\newcommand*\email{\nolinkurl}

\renewcommand\file[1]{{%
  \ttfamily%
  \textcolor{MaterialGrey900}{#1}%
}}

\newcommand\scite[1]{\textsuperscript{\cite{#1}}}

\hyphenation{fdu-the-sis}
\hyphenation{clear-dou-ble-page}
\hyphenation{set-main-font}
\hyphenation{set-main-CJK-font}
\hyphenation{set-math-font}

\DoNotIndex{\begin,\end,
  \a,\b,\c,\d,\e,\f,\g,\h,\i,\j,\k,\l,\m,
  \n,\o,\p,\q,\r,\s,\t,\u,\v,\w,\x,\y,\z,
  \A,\B,\C,\D,\E,\F,\G,\H,\I,\J,\K,\L,\M,
  \N,\O,\P,\Q,\R,\S,\T,\U,\V,\W,\X,\Y,\Z,
  \0,\1,\2,\3,\4,\5,\6,\7,\8,\9}
\EnableCrossrefs
\CodelineIndex
\RecordChanges
%</doc>
%    \end{macrocode}
%
% \subsubsection{\pkg{latexmk} 配置文件}
%    \begin{macrocode}
%<*latexmk>
# Latexmk configuration file.

# Use XeLaTeX to compile.
$pdf_mode = 5;

# Process index.
$makeindex = 'zhmakeindex -s gind.ist %O -o %D %S';

# Process glossary (change log).
add_cus_dep('glo', 'gls', 0, 'makeglo2gls');
sub makeglo2gls {
    system("zhmakeindex -s gglo.ist -o \"$_[0].gls\"
        -t \"$_[0].glg\" \"$_[0].glo\"");
}
%</latexmk>
%    \end{macrocode}
%
% \end{implementation}
%
% \DisableImplementation
%
% \begin{implementation}
%
% \subsubsection{参考文献列表}
%    \begin{macrocode}
%<*bbl>
\begin{thebibliography}{99}
\providecommand{\urlprefix}{\newline\hspace*{\fill}}
\let\OldUrl=\url
\renewcommand\url[1]{{\small\OldUrl{#1}}}
\newcommand\sourceurl[1]{{\small{\CJKfamily{楷}源代码:} \OldUrl{#1}}}

\bibitem{source2e}
\textsc{Braams J}, \textsc{Carlisle D}, \textsc{Jeffrey A}, et al.
\newblock \textit{The \LaTeXe{} Sources} [CP/OL].
\newblock (2017-04-15) \urlprefix
  \url{https://www.ctan.org/tex-archive/macros/latex/base/source2e.pdf}

\bibitem{CTeX}
\textsc{CTEX.ORG}.
\newblock \textit{\CTeX{} 宏集手册} [EB/OL].
\newblock version 2.4.9,
\newblock (2017-04-01) \urlprefix
  \url{https://www.ctan.org/tex-archive/language/chinese/ctex/ctex.pdf}
  \urlprefix \sourceurl{https://www.ctan.org/tex-archive/language/chinese/ctex/ctex.dtx}

\bibitem{xeCJK}
\textsc{CTEX.ORG}.
\newblock \textit{\pkg{xeCJK} 宏包} [EB/OL].
\newblock version 3.4.8,
\newblock (2017-05-15) \urlprefix
  \url{https://www.ctan.org/tex-archive/macros/xetex/latex/xecjk/xeCJK.pdf}
  \urlprefix \sourceurl{https://www.ctan.org/tex-archive/macros/xetex/latex/xecjk/xeCJK.dtx}

\bibitem{hmshan2017fduphdthesistemplate}
\textsc{hmshan}.
\newblock \textit{复旦博士论文模板} [EB/OL].
\newblock (2017-04-04) \urlprefix
  \url{https://github.com/hmshan/FDU_PhD_Thesis_Template}

\bibitem{knuth1986texbook}
\textsc{Knuth D E}.
\newblock \textit{The \TeX book: Computers \& Typesetting, volumn A} [M].
\newblock Boston: Addison--Wesley Publishing Company, 1986
  \urlprefix {\small {\CJKfamily{楷}源代码
    \footnote{此代码只可作为学习之用。未经 Knuth 本人同意,您不应当编译该文档。}:}
    \OldUrl{https://www.ctan.org/tex-archive/systems/knuth/dist/tex/texbook.tex}}

\bibitem{interfaces3}
\textsc{The \LaTeX3 Project}.
\newblock \textit{The \LaTeX3 Interfaces} [EB/OL].
\newblock (2017-07-19) \urlprefix
  \url{https://www.ctan.org/tex-archive/macros/latex/contrib/l3kernel/interface3.pdf}

\bibitem{source3}
\textsc{The \LaTeX3 Project}.
\newblock \textit{The \LaTeX3 Sources} [CP/OL].
\newblock (2017-07-19) \urlprefix
  \url{https://www.ctan.org/tex-archive/macros/latex/contrib/l3kernel/source3.pdf}

\bibitem{mittelbach2004latexcompanion}
\textsc{Mittelbach F} and \textsc{Goossens M}.
\newblock \textit{The \LaTeX{} Companion} [M].
\newblock 2nd ed.
\newblock Boston: Addison--Wesley Publishing Company, 2004

\bibitem{lshort}
\textsc{Oetiker T}, \textsc{Partl H}, \textsc{Hyna I}, et al.
\newblock \textit{The Not So Short Introduction to \LaTeXe{}: Or \LaTeXe{} in 157 minutes} [EB/OL].
\newblock version 5.05,
\newblock (2015-07-18) \urlprefix
  \url{https://www.ctan.org/tex-archive/info/lshort/english/lshort.pdf}

\bibitem{lshort-zh-cn}
\textsc{Oetiker T}, \textsc{Partl H}, \textsc{Hyna I}, et al.
\newblock \textit{一份不太简短的 \LaTeXe{} 介绍: 或 102 分钟了解 \LaTeXe{}} [EB/OL].
\newblock China\TeX{} 论坛, 译.
\newblock 原版版本 version 5.05, 中文版本 version 5.10,
\newblock (2017-03-26) \urlprefix
  \url{https://www.ctan.org/tex-archive/info/lshort/chinese/lshort-zh-cn.pdf}

\bibitem{pandoxie2014fduthesislatex}
\textsc{Pandoxie}.
\newblock \textit{Fudan University-Latex Template} [EB/OL].
\newblock (2014-06-07) \urlprefix
  \url{https://github.com/Pandoxie/FDU-Thesis-Latex}

\bibitem{richard2016fudanthesis}
\textsc{richard}.
\newblock \textit{复旦大学硕士学位论文模板} [EB/OL].
\newblock (2016-01-31) \urlprefix
  \url{https://github.com/richarddzh/fudan-thesis}

\bibitem{weijianwen2017sjtuthesis}
\textsc{weijianwen}.
\newblock \textit{A \XeLaTeX{} template for Shanghai Jiao Tong University (SJTU) thesis.} [EB/OL].
\newblock (2017-01-27) \urlprefix
  \url{https://github.com/weijianwen/SJTUThesis}

\bibitem{wright2009dtxfile}
\textsc{Wright J}.
\newblock \textit{A model dtx file} [EB/OL].
\newblock (2009-10-06) \urlprefix
  \url{http://www.texdev.net/2009/10/06/a-model-dtx-file}

\bibitem{zepinglee2017ustcthesis}
\textsc{zepinglee} and \textsc{seisman}.
\newblock \textit{\LaTeX{} template for USTC thesis} [EB/OL].
\newblock version 3.0,
\newblock (2017-07-02) \urlprefix
  \url{https://github.com/ustctug/ustcthesis}

\bibitem{数院毕业论文格式}
复旦大学数学科学学院.
\newblock \textit{毕业论文格式 tex 版和 word 版} [EB/OL]. \urlprefix
  \url{http://math.fudan.edu.cn/show.aspx?info_lb=664&flag=101&info_id=1816}

\bibitem{数院毕业论文格式更新}
复旦大学数学科学学院.
\newblock \textit{毕业论文格式: Word、\TeX{} 模板更新} [EB/OL]. \urlprefix
  \url{http://math.fudan.edu.cn/Show.aspx?info_lb=664&info_id=1855&flag=101}

\bibitem{复旦大学论文规范}
复旦大学图书馆, 复旦大学研究生院.
\newblock \textit{复旦大学博士、硕士学位论文规范} [EB/OL].
\newblock 2017 年 3 月修订版.
\newblock (2017-03-27) \urlprefix
  \url{http://www.gs.fudan.edu.cn/_upload/article/4c/a8/a82545ef443b9c057c14ba13782c/c883c6f3-6d7f-410c-8f30-d8bde6fcb990.doc}

\bibitem{gb-t-15834-2011}
国务院学位委员会办公室, 全国信息与文献标准化技术委员会.
\newblock \textit{学位论文编写规则: GB/T 7713.1--2006} [S].
\newblock 北京: 中国标准出版社, 2007

\bibitem{胡伟2017latex2e}
胡伟.
\newblock \textit{\LaTeXe{} 文类和宏包学习手册} [M].
\newblock 北京: 清华大学出版社, 2017

\bibitem{gb-t-15834-2011}
教育部语言文字信息管理司.
\newblock \textit{标点符号用法: GB/T 15834--2011} [S].
\newblock 北京: 中国标准出版社, 2012

\bibitem{cquthesis}
李振楠.
\newblock \textit{\textsc{CquThesis}:重庆大学毕业论文 \LaTeX{} 模板} [EB/OL].
\newblock version 1.20,
\newblock (2017-05-30) \urlprefix
  \url{https://www.ctan.org/tex-archive/macros/latex/contrib/cquthesis/cquthesis.pdf}

\bibitem{刘海洋2013latex入门}
刘海洋.
\newblock \textit{\LaTeX{} 入门} [M].
\newblock 北京: 电子工业出版社, 2013

\bibitem{gb-t-7714-2015}
全国信息与文献标准化技术委员会.
\newblock \textit{信息与文献 参考文献著录规则: GB/T 7714--2015} [S].
\newblock 北京: 中国标准出版社, 2015

\bibitem{thuthesis}
薛瑞尼.
\newblock \textit{\textsc{ThuThesis}:清华大学学位论文模板} [EB/OL].
\newblock version 5.3.2,
\newblock (2017-03-26) \urlprefix
  \url{https://www.ctan.org/tex-archive/macros/latex/contrib/thuthesis/thuthesis.pdf}
\end{thebibliography}
%</bbl>
%    \end{macrocode}
%
% \end{implementation}
