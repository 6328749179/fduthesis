% \iffalse meta-comment
%
% Copyright (C) 2017, 2018 by Xiangdong Zeng <xdzeng96@gmail.com>
%
% This work may be distributed and/or modified under the
% conditions of the LaTeX Project Public License, either
% version 1.3c of this license or (at your option) any later
% version. The latest version of this license is in:
%
%   http://www.latex-project.org/lppl.txt
%
% and version 1.3 or later is part of all distributions of
% LaTeX version 2005/12/01 or later.
%
% This work has the LPPL maintenance status `maintained'.
%
% The Current Maintainer of this work is Xiangdong Zeng.
%
% \fi
%
% \EnableImplementation
%
% \begin{implementation}
%
%^^A 代码部分的页边距
% \newgeometry{
%   left      = 2.25 in,
%   right     = 1.00 in,
%   top       = 1.25 in,
%   bottom    = 1.00 in,
%   marginpar = 2.25 in
% }
%
% \subsection{模板文档样式 \cls{fdudoc}}
%
% \changes{v0.4}{2017/07/29}{在 \cls{ctxdoc} 的基础上完成
%   \cls{fdudoc} 文档类,用于模板手册的编写。}
%
% 编写 \LaTeX{} 宏包文档,传统上会采用 \pkg{doc} 宏包或
% \cls{ltxdoc} 文档类。而对于使用 \LaTeX3 开发的宏包,\cls{l3doc}
% 文档类将是一个更好的选择。\CTeX{} 宏集所附带的
% \href{https://github.com/CTeX-org/ctex-kit/blob/master/tool/ctxdoc.cls}^^A
% {\cls{ctxdoc}} 文档类,则在 \cls{l3doc} 的基础上进行了一些修正,
% 特别是重新实现了 \env{macrocode} 环境,使之能更好地应用于中文
% 文档。\cls{ctxdoc} 的主要功能如下:
%
% \begin{itemize}
%   \item 注释使用灰色、倾斜字体,以便与一般代码区分;
%   \item 模块、名字空间等使用彩色标注,并添加超链接;
%   \item 自动更新行号宽度;
%   \item 边注中的长命令使用盒子进行缩放,防止溢出;
%   \item 修订记录中将显示修改日期;
%   \item 添加中文支持。
% \end{itemize}
%
% 然而,\cls{ctxdoc} 主要供内部使用,代码较为混乱和随意。
% 本模板的文档样式(\cls{fdudoc})为适应具体需求,对其代码
% 进行了整理,添加了相关注释,并做了一些改动:
%
% \begin{itemize}
%   \item 允许模块标记 |<*|\meta{module}|>| 和
%     |</|\meta{module}|>| 出现在行号左侧;
%   \item 不再以直立、倾斜字体区分不同嵌套层次的模块;
%   \item 调整索引中版本号的排序方式;
%   \item 新增一系列实用命令;
%   \item 修改文档字体、配色等。
% \end{itemize}
%
% 本文档样式的核心代码主要来自 \cls{ctxdoc} 文档类 v2.4.10。
% 在此,本人要向原开发者
% \href{https://github.com/CTeX-org/}{CTEX.ORG}
% 团队表示由衷的感谢。
%
% 以下为 \cls{fdudoc} 的具体实现。
%
% \subsubsection{载入宏包、文档类}
%
%    \begin{macrocode}
%<*doc>
\ExplSyntaxOff
%    \end{macrocode}
%
% 无需载入 \pkg{thumbpdf}。
%    \begin{macrocode}
\@namedef{ver@thumbpdf.sty}{9999/99/99}
%    \end{macrocode}
%
% 关闭 \pkg{xparse} 中的命令声明信息。
%    \begin{macrocode}
\PassOptionsToPackage{log-declarations = false}{xparse}
%    \end{macrocode}
%
% 载入宏包和文档类。
%    \begin{macrocode}
\LoadClass[a4paper, full]{l3doc}
\RequirePackage[UTF8, heading, sub3section, fontset = none]{ctex}
%    \end{macrocode}
% 与 \pkg{hypdoc}(由 \cls{l3doc} 调用)冲突,导致脚注超链接
% 无法正常跳转。临时禁用。
%    \begin{macrocode}
% \RequirePackage[stable, bottom]{footmisc}
\RequirePackage{caption}
% \RequirePackage[showframe]{geometry}
\RequirePackage{geometry}
\RequirePackage{listings}
\RequirePackage{makecell}
\RequirePackage[toc]{multitoc}
\RequirePackage{siunitx}
\RequirePackage{tabularx}
\RequirePackage{threeparttable}
\RequirePackage{unicode-math}
\RequirePackage{xcolor}
\RequirePackage{xcolor-material}
\RequirePackage{zref-base}
%    \end{macrocode}
%
% \subsubsection{\env{macrocode} 环境}
%
% \paragraph{继承的代码}
%
% \begin{macro}[int]{\macro@code}
% 在 \pkg{doc} 宏包中,\env{macrocode} 环境的核心功能由命令
% \tn{macro@code} 负责实现,而 \tn{xmacro@code} 只用来结束
% \env{macrocode} 环境。但在 \cls{l3doc} 以及 \cls{ctxdoc} 中,
% \tn{xmacro\-@\-code} 则基本接管了 \tn{macro@code} 的功能。
% 后者此时只起辅助作用。
%    \begin{macrocode}
\def\macro@code{%
%    \end{macrocode}
% 调整前后间距,禁止 \env{macrocode} 环境前的分页。
%    \begin{macrocode}
   \topsep \MacrocodeTopsep
   \@beginparpenalty \predisplaypenalty
%    \end{macrocode}
% 将列表前后的附加垂直空白设为 0。根据 \cls{ctxdoc} 修改。
%    \begin{macrocode}
   \partopsep \z@skip
%    \if@inlabel\leavevmode\fi
%    \end{macrocode}
% 构建 \env{trivlist} 环境,设置段间距为 0。
% 之后修改字体,并调节左右间距。\tn{MacroIndent} 会根据代码行数
% 更新,具体细节见后文。
% \tn{macro@font} 用来在不同模块见切换字体。本文档类不使用
% \tn{AltMacroFont},因此这里改用 \tn{MacroFont} 代替。
%    \begin{macrocode}
   \trivlist \parskip \z@ \item[]%
%    \macro@font
   \MacroFont
   \leftskip\@totalleftmargin \advance\leftskip\MacroIndent
   \rightskip\z@ \parindent\z@ \parfillskip\@flushglue
%    \end{macrocode}
% 按照 \LaTeXe{} 中 \tn{verbatim} 环境中定义 \tn{par},使得空行
% 可以原样输出,否则空行会被吃掉。
%    \begin{macrocode}
   \blank@linefalse \def\par{\ifblank@line
                             \leavevmode\fi
                             \blank@linetrue\@@par
                             \penalty\interlinepenalty}
%    \end{macrocode}
% \tn{obeylines} 将把回车符 |^^M| 变成 \tn{par}。
% 接下来将所有特殊符号的类别码设为 12,即“其他”类。
%    \begin{macrocode}
   \obeylines
   \let\do\do@noligs \verbatim@nolig@list
   \let\do\@makeother \dospecials
%    \end{macrocode}
% 相当于退出 |\begin{list}| 和 |\begin{minipage}|。
%    \begin{macrocode}
   \global\@newlistfalse
   \global\@minipagefalse
%    \end{macrocode}
% 初始化交叉引用功能。
%    \begin{macrocode}
   \init@crossref}
%    \end{macrocode}
% \end{macro}
%
%    \begin{macrocode}
%<@@=fdudoc>
\ExplSyntaxOn
%    \end{macrocode}
%
% \begin{variable}{\l_@@_tmpa_tl,\l_@@_tmpb_tl}
% 临时变量。
%    \begin{macrocode}
\tl_new:N \l_@@_tmpa_tl
\tl_new:N \l_@@_tmpb_tl
%    \end{macrocode}
% \end{variable}
%
% \begin{macro}{\@@_patch_cmd:Nnn,\@@_preto_cmd:Nn,\@@_appto_cmd:Nn}
% 补丁工具。
%    \begin{macrocode}
\cs_new_protected:Npn \@@_patch_cmd:Nnn #1#2#3
  {
    \ctex_patch_cmd_once:NnnnTF #1 { } {#2} {#3}
      { } { \ctex_patch_failure:N #1 }
  }
\cs_new_protected:Npn \@@_preto_cmd:Nn #1#2
  {
    \ctex_preto_cmd:NnnTF #1 { } {#2}
      { } { \ctex_patch_failure:N #1 }
  }
\cs_new_protected:Npn \@@_appto_cmd:Nn #1#2
  {
    \ctex_appto_cmd:NnnTF #1 { } {#2}
      { } { \ctex_patch_failure:N #1 }
  }
%    \end{macrocode}
% \end{macro}
%
% \paragraph{代码行处理}
%
% \begin{macro}[int]{\xmacro@code,\sxmacro@code}
% 重新实现 \env{macrocode} 与 \env{macrocode*} 环境的核心功能,
% 将对代码逐行处理。后者会将空格显示为“\textvisiblespace”。
%    \begin{macrocode}
\cs_set_protected_nopar:Npn \xmacro@code
  { \@@_marco_code:w }
\cs_set_protected_nopar:Npn \sxmacro@code
  {
    \fontspec_print_visible_spaces:
    \xmacro@code
  }
%    \end{macrocode}
% \end{macro}
%
% \begin{macro}{\@@_marco_code:w}
%    \begin{macrocode}
\cs_new_protected_nopar:Npn \@@_marco_code:w
  {
%    \end{macrocode}
% 根据 \tn{codeline@index} 是否为 |true| 选择是否显示行号。
%    \begin{macrocode}
    \ifcodeline@index
      \@@_marco_code_every_par:n { \@@_code_line_no: }
    \else:
      \@@_marco_code_every_par:n { }
    \fi:
%    \end{macrocode}
% 设置代码段结束标记为“\verb*|%    \end{macrocode}^^M|”。
%    \begin{macrocode}
    \@@_make_finish_tag:x { \@currenvir }
%    \end{macrocode}
% 开始 \env{macrocode}。
%    \begin{macrocode}
    \@@_macro_code_start:w
  }
%    \end{macrocode}
% \end{macro}
%
% \begin{macro}{\@@_marco_code_every_par:n}
% 在每段之前插入内容。这里每段即相当于每行。
%    \begin{macrocode}
\cs_new_protected:Npn \@@_marco_code_every_par:n #1
  {
    \everypar
      {
        \everypar {#1}
        \if@inlabel
          \global \@inlabelfalse \@noparlistfalse
          \llap { \box \@labels \hskip \leftskip }
        \fi
        #1
      }
  }
%    \end{macrocode}
% \end{macro}
%
% 设置 \tn{endlinechar} 为 $-1$,表示行末不插入任何字符
% (实际上相当于在行尾插入注释符 |%|)。
%    \begin{macrocode}
\group_begin:
  \int_set:Nn \tex_endlinechar:D { -1 }
%    \end{macrocode}
%
% \begin{variable}{\c_@@_active_space_tl}
% 活动字符类的空格(ASCII 码为 32)。
%    \begin{macrocode}
  \use:n
    {
      \char_set_catcode_active:n { 32 }
      \tl_const:Nn \c_@@_active_space_tl
    }
    { }
\group_end:
%    \end{macrocode}
% \end{variable}
%
% ASCII 码 13 是回车符 |^^M|。将其设置为活动字符。
%    \begin{macrocode}
\group_begin:
  \char_set_catcode_active:n { 13 }
%    \end{macrocode}
%
% \begin{macro}{\@@_make_finish_tag:n,\@@_make_finish_tag:x}
% \env{macrocode} 结尾标记。展开后变成
% “\verb*|%    \end{#1}^^M|”。
%    \begin{macrocode}
  \cs_new_protected:Npx \@@_make_finish_tag:n #1
    {
      \tl_set:Nn \exp_not:N \l_@@_macro_code_finish_tl
        {
          \c_percent_str
          \prg_replicate:nn { 4 }
            { \exp_not:o { \c_@@_active_space_tl } }
          \exp_not:o { \active@escape@char } end
          \c_left_brace_str #1 \c_right_brace_str
          \exp_not:N ^^M
        }
    }
  \cs_generate_variant:Nn \@@_make_finish_tag:n { x }
%    \end{macrocode}
% \end{macro}
%
% \begin{macro}{\@@_macro_code_start:w}
% 开始代码环境。此命令主要是为了防止 |\begin{macrocode}|
% 后出现多余的空行。
%    \begin{macrocode}
  \cs_new_protected:Npn \@@_macro_code_start:w #1
    {
      \str_if_eq:nnTF {#1} { ^^M }
        { \@@_macro_code_read_line:w }
        { \@@_macro_code_read_line:w #1 }
    }
%    \end{macrocode}
% \end{macro}
%
% \begin{macro}{\@@_macro_code_read_line:w}
% 逐行读取代码,并连同行尾回车符一并存入
% \cs{l_@@_macro_code_line_tl}。如果该行与结束标记
% “\verb*|%    \end{macrocode}^^M|”相同,则结束此
% \env{macrocode};否则继续处理该行代码。
%    \begin{macrocode}
  \cs_new_protected:Npn \@@_macro_code_read_line:w #1 ^^M
    {
      \tl_set:Nn \l_@@_macro_code_line_tl { #1 ^^M }
      \tl_if_eq:NNTF
        \l_@@_macro_code_line_tl \l_@@_macro_code_finish_tl
        { \exp_args:Nx \end { \@currenvir } }
        {
          \@@_macro_code_process_line:
          \@@_macro_code_read_line:w
        }
    }
%    \end{macrocode}
% \end{macro}
%
% \changes{v0.4}{2017/08/09}{[\pkg{fdudoc}] 修复 \cls{ctxdoc}
%   文档类 v2.4.10 之前版本中行距偏小的问题,见 ctex-kit
%   \href{https://github.com/CTeX-org/ctex-kit/issues/303}{\#~303}。}
%
% \begin{macro}{\@@_swap_cr:,\@@_swap_cr_aux:w}
% 把 \cs{l_@@_macro_code_line_tl} 中的回车符 |^^M| 挪到外面。
%    \begin{macrocode}
  \cs_new_protected:Npn \@@_swap_cr:
    {
      \exp_after:wN
        \@@_swap_cr_aux:w \l_@@_macro_code_line_tl
    }
  \cs_new_protected:Npn \@@_swap_cr_aux:w #1 ^^M
    {
      \group_insert_after:N ^^M
      \tl_set:Nn \l_@@_macro_code_line_tl {#1}
    }
%    \end{macrocode}
% \end{macro}
%
% \begin{variable}{\c_@@_active_cr_tl}
% 活动字符类的回车符。
%    \begin{macrocode}
  \tl_const:Nn \c_@@_active_cr_tl { ^^M }
\group_end:
%    \end{macrocode}
% \end{variable}
%
% \begin{variable}{\l_@@_macro_code_line_tl,
%   \l_@@_macro_code_finish_tl,
%   \g_@@_macro_code_verbatim_stop_tl}
% 分别用来存储代码行、\env{macrocode} 结束标记以及抄录停止标记。
%    \begin{macrocode}
\tl_new:N \l_@@_macro_code_line_tl
\tl_new:N \l_@@_macro_code_finish_tl
\tl_new:N \g_@@_macro_code_verbatim_stop_tl
%    \end{macrocode}
% \end{variable}
%
% \begin{macro}{\@@_process_normal_line:}
% 普通代码行根据开头字符分别处理。
%    \begin{macrocode}
\cs_new_protected_nopar:Npn \@@_process_normal_line:
  {
    \str_case_x:nnF
      { \str_head:N \l_@@_macro_code_line_tl }
      {
%    \end{macrocode}
% 以 |%| 开头的行先由 \cs{tl_tail:N} 去掉 |%|,之后再检查 |<|。
%    \begin{macrocode}
        { \c_percent_str }
        {
          \@@_check_angle:x
            { \tl_tail:N \l_@@_macro_code_line_tl }
        }
%    \end{macrocode}
% 以 |#| 开头的行按注释的格式输出。
%    \begin{macrocode}
        { \c_hash_str }
        { \@@_output_comment_line: }
      }
%    \end{macrocode}
% 其余正常输出。
%    \begin{macrocode}
      { \@@_output_line: }
  }
%    \end{macrocode}
% \end{macro}
%
% \begin{macro}{\@@_process_verbatim_line:}
% 处理抄录代码行(不检查 |%| 与 |<|)。
%    \begin{macrocode}
\cs_new_protected_nopar:Npn \@@_process_verbatim_line:
  {
%    \end{macrocode}
% 将该行与抄录停止标记进行比较。
%    \begin{macrocode}
    \tl_if_eq:NNTF \l_@@_macro_code_line_tl
        \g_@@_macro_code_verbatim_stop_tl
%    \end{macrocode}
% 若相同,则结束抄录环境,清空抄录停止标记,并输出该标记;
%    \begin{macrocode}
      {
        \tl_gclear:N \g_@@_macro_code_verbatim_stop_tl
        \cs_gset_eq:NN \@@_macro_code_process_line:
          \@@_process_normal_line:
        \@@_output_module:nn
          { \color { verb@guard } }
          {
            \@@_swap_cr:
            \@@_module_pop:n { \l_@@_macro_code_line_tl }
          }
      }
%    \end{macrocode}
% 否则直接输出抄录代码。
%    \begin{macrocode}
      { \tl_use:N \l_@@_macro_code_line_tl }
  }
%    \end{macrocode}
% \end{macro}
%
% \begin{macro}{\@@_macro_code_process_line:}
% 处理代码行。该命令的作用如下:
% \begin{itemize}
%   \item 正常情况下,等同于 \cs{@@_process_normal_line:};
%   \item 在 \cs{DontCheckModules} 之后,等价于
%     \cs{@@_output_line:},即不检查模块标记,直接输出;
%   \item 在抄录环境中,等价于 \cs{@@_process_verbatim_line:},
%     此时将不再特殊处理以 |%| 开头的代码行。
% \end{itemize}
%    \begin{macrocode}
\cs_new_eq:NN \@@_macro_code_process_line:
  \@@_process_normal_line:
%    \end{macrocode}
% \end{macro}
%
% \paragraph{模块标记处理}
%
% \begin{macro}{\CheckModules,\DontCheckModules}
% 选择是否检查模块标记。这两个命令在 \pkg{doc} 宏包中已有定义,
% 此处重新声明。
%    \begin{macrocode}
\DeclareDocumentCommand \CheckModules { }
  {
    \cs_set_eq:NN \@@_macro_code_process_line:
      \@@_process_normal_line:
  }
\DeclareDocumentCommand \DontCheckModules { }
  {
    \cs_set_eq:NN \@@_macro_code_process_line:
      \@@_output_line:
  }
%    \end{macrocode}
% \end{macro}
%
% \begin{macro}{\@@_check_angle:n,\@@_check_angle:x}
% 检查第一个字符是否是 |<|。若是,则检查模块;否则立刻输出改行。
% 该函数的参数不带 |%|。
%    \begin{macrocode}
\cs_new_protected:Npn \@@_check_angle:n #1
  {
    \str_if_eq_x:nnTF { \str_head:n {#1} } { < }
      { \@@_check_module:x { \tl_tail:n {#1} } }
      { \@@_output_comment_line: }
  }
\cs_generate_variant:Nn \@@_check_angle:n { x }
%    \end{macrocode}
% \end{macro}
%
% \begin{macro}{\@@_check_module:n,\@@_check_module:x}
% 检查紧跟 |<| 之后的字符。共有四种情况:
% \begin{itemize}
%   \item |*|:模块开始(|%<*|\meta{module}|>|);
%   \item |/|:模块结束(|%</|\meta{module}|>|);
%   \item |@|:名字空间(|%<@@=|\meta{namespace}|>|);
%   \item |<|:抄录环境开始(|%<<|\meta{end-tag})。
% \end{itemize}
% 若不是这几种情况,则为单独一行的独立模块
% (|%<|\meta{module}|>|)。
%    \begin{macrocode}
\cs_new_protected:Npn \@@_check_module:n #1
  {
    \str_case_x:nnF { \str_head:n {#1} }
      {
        { * } { \@@_module_star:w }
        { / } { \@@_module_slash:w }
        { @ } { \@@_module_at:w }
        { < } { \@@_module_verb:w }
      }
      { \@@_module_pm:w }
%    \end{macrocode}
% 参数 |#1| 将被上面几个 |:w| 型函数吃掉。
%    \begin{macrocode}
    #1 \q_stop
  }
\cs_generate_variant:Nn \@@_check_module:n { x }
%    \end{macrocode}
% \end{macro}
%
% 设置 |>| 为活动字符。
%    \begin{macrocode}
\group_begin:
  \char_set_catcode_active:N \>
%    \end{macrocode}
%
% \begin{macro}{\@@_module_star:w}
% 模块开始标记。
% \begin{arguments}
%   \item |*|\meta{module}
%   \item 之后的代码
% \end{arguments}
%    \begin{macrocode}
  \cs_new_protected:Npn \@@_module_star:w #1 > #2 \q_stop
    {
%    \end{macrocode}
% 临时变量 \cs{l_@@_tmp_tl} 保存 |<*|\meta{module}|>|
% 之后的部分,即真实代码。
%    \begin{macrocode}
      \tl_set:Nn \l_@@_tmpa_tl {#2}
%    \end{macrocode}
% 判断该行是否为空(只含一个回车符 |^^M|)。
%    \begin{macrocode}
      \tl_if_eq:NNTF \l_@@_tmpa_tl \c_@@_active_cr_tl
%    \end{macrocode}
% 若是,则将 |<|\meta{module}|>| 放在行号的右侧;
%    \begin{macrocode}
        {
          \@@_output_module:nn
            { \@@_star_color: }
            {
              \@@_module_push:n
                { \@@_module_angle:n {#1} }
            }
        }
%    \end{macrocode}
% 否则放在左侧,并输出相应代码。
%    \begin{macrocode}
        {
          \@@_output_module_left:nn
            { \@@_star_color: }
            {
              \@@_module_push:n
                { \@@_module_angle:n {#1} }
            }
        }
      \@@_output_line:n {#2}
    }
%    \end{macrocode}
% \end{macro}
%
% \begin{macro}{\@@_module_slash:w}
% 模块结束标记。结构与 \cs{@@_module_star:w} 相同。
% \begin{arguments}
%   \item |/|\meta{module}
%   \item 之后的代码
% \end{arguments}
%    \begin{macrocode}
  \cs_new_protected:Npn \@@_module_slash:w #1 > #2 \q_stop
    {
      \tl_set:Nn \l_@@_tmpa_tl {#2}
      \tl_if_eq:NNTF \l_@@_tmpa_tl \c_@@_active_cr_tl
        {
          \@@_output_module:nn
            { \@@_slash_color: }
            {
              \@@_module_pop:n
                { \@@_module_angle:n {#1} }
            }
        }
        {
          \@@_output_module_left:nn
            { \@@_slash_color: }
            {
              \@@_module_pop:n
                { \@@_module_angle:n {#1} }
            }
        }
      \@@_output_line:n {#2}
    }
%    \end{macrocode}
% \end{macro}
%
% \begin{macro}{\@@_module_at:w}
% 名字空间。
% \begin{arguments}
%   \item 名字空间的名称(不含 |@@=|)
%   \item 之后的代码
% \end{arguments}
%    \begin{macrocode}
  \cs_new_protected:Npn \@@_module_at:w @ @ = #1 > #2 \q_stop
    {
      \@@_output_module:nn
        { \color { at@guard } }
        { \@@_module_angle:n { @ @ = #1 } }
%    \end{macrocode}
% 设置名字空间为 |#1|。\cls{l3doc} 中将名字空间称为
% “模块”(module),注意不要混淆。
%    \begin{macrocode}
      \tl_gset:Nn \g__codedoc_module_name_tl {#1}
      \@@_output_line:n {#2}
    }
%    \end{macrocode}
% \end{macro}
%
% \begin{macro}{\@@_module_verb:w}
% 抄录开始。|#1|: |<|\meta{end-tag},只有一个 |<|。
% \meta{end-tag} 的最后会带有一个回车符 |^^M|。
%    \begin{macrocode}
  \cs_new_protected:Npn \@@_module_verb:w #1 \q_stop
    {
%    \end{macrocode}
% 重定义 \cs{@@_macro_code_process_line:},用以输出抄录行。
%    \begin{macrocode}
      \cs_gset_eq:NN \@@_macro_code_process_line:
        \@@_process_verbatim_line:
%    \end{macrocode}
% 设置抄录停止标记。用 \cs{tl_tail:n} 去掉开头的 |<|。
%    \begin{macrocode}
      \tl_gset:Nx \g_@@_macro_code_verbatim_stop_tl
        { \c_percent_str \tl_tail:n {#1} }
%    \end{macrocode}
% 输出 |%<<|\meta{end-tag}。
%    \begin{macrocode}
      \@@_output_module:nn
        { \color { verb@guard } }
        {
          \@@_swap_cr:
          \@@_module_push:n { \l_@@_macro_code_line_tl }
        }
    }
%    \end{macrocode}
% \end{macro}
%
% \begin{macro}{\@@_module_pm:w}
% 处理单独一行的模块。|<|\meta{module}|>| 放在行号的左侧。
% \begin{arguments}
%   \item \meta{module}
%   \item 之后的代码
% \end{arguments}
%    \begin{macrocode}
  \cs_new_protected:Npn \@@_module_pm:w #1 > #2 \q_stop
    {
      \@@_output_module_left:nn
        { \@@_pm_color: }
        { \@@_module_angle:n {#1} }
      \@@_output_line:n {#2}
    }
\group_end:
%    \end{macrocode}
% \end{macro}
%
% \begin{macro}{\@@_output_line:n,\@@_output_line:}
% 输出代码行。参数将被存入 \cs{l_@@_macro_code_line_tl},
% 再由不带参数的版本调用。
%    \begin{macrocode}
\cs_new_protected:Npn \@@_output_line:n #1
  {
    \tl_set:Nn \l_@@_macro_code_line_tl {#1}
%    \end{macrocode}
% 若为空行(只含一个 |^^M|),则直接输出(换行)。
%    \begin{macrocode}
    \tl_if_eq:NNTF
      \l_@@_macro_code_line_tl \c_@@_active_cr_tl
      { \tl_use:N \l_@@_macro_code_line_tl }
      {
%    \end{macrocode}
% 检查开头是否为 |%|,据此分别处理。
%    \begin{macrocode}
        \str_if_eq_x:nnTF
          { \str_head:N \l_@@_macro_code_line_tl } { \c_percent_str }
          { \@@_output_comment_line: } { \@@_output_line: }
      }
  }
\cs_new_protected_nopar:Npn \@@_output_line:
  {
    \tex_noindent:D
%    \end{macrocode}
% 此处将把 |@@| 替换为相应的名字空间。
%    \begin{macrocode}
    \@@_replace_at_at:N \l_@@_macro_code_line_tl
    \tl_use:N \l_@@_macro_code_line_tl
  }
%    \end{macrocode}
% \end{macro}
%
% \begin{macro}{\@@_output_comment_line:}
% 输出注释代码行。用灰色、斜体显示。
%    \begin{macrocode}
\cs_new_protected:Npn \@@_output_comment_line:
  {
    \tex_noindent:D
    \group_begin:
      \__fdudoc_swap_cr:
      \color { code@gray } \slshape \@@_output_line:
    \group_end:
  }
%    \end{macrocode}
% \end{macro}
%
% \begin{macro}{\@@_replace_at_at:N,
%   \@@_replace_at_at_aux:Nn,\@@_replace_at_at_aux:No}
% 把 |@@| 替换为相应的名字空间。其名称存放在全局变量
% \cs{g__codedoc_module_name_tl} 中。
% 它为空时(|%<@@=>|),不做替换。
%    \begin{macrocode}
\cs_new_protected:Npn \@@_replace_at_at:N #1
  {
    \tl_if_empty:NF \g__codedoc_module_name_tl
      { \@@_replace_at_at_aux:No #1 \g__codedoc_module_name_tl }
  }
\cs_new_protected:Npn \@@_replace_at_at_aux:Nn #1#2
  {
%    \end{macrocode}
% 以下代码分别对应两种名字空间的替换:
% \begin{itemize}
%   \item 内部变量:|\|\meta{type}|_@@_|\meta{name} $\to$
%     |\|\meta{type}|__|\meta{namespace}|_|\meta{name};
%   \item 内部函数:|\@@_|\meta{name}” $\to$
%     |\__|\meta{namespace}|_|\meta{name}”)。
% \end{itemize}
%    \begin{macrocode}
    \tl_replace_all:Nnn #1 { _ @ @ } { _ _ #2 }
    \tl_replace_all:Nnn #1 {   @ @ } { _ _ #2 }
  }
\cs_generate_variant:Nn \@@_replace_at_at_aux:Nn { No }
%    \end{macrocode}
% \end{macro}
%
% \begin{macro}{\@@_module_push:n,
%   \@@_module_push_aux:nn,\@@_module_push_aux:on}
% 将模块名压入栈中。此处的栈主要用来处理模块名(包括抄录标记)
% 之间的超链接。
%    \begin{macrocode}
\cs_new_protected_nopar:Npn \@@_module_push:n
  { \@@_module_push_aux:on { \int_use:N \c@HD@hypercount } }
\cs_new_protected:Npn \@@_module_push_aux:nn #1
  {
    \seq_gpush:Nn \g_@@_module_dest_seq {#1}
    \hypersetup { hidelinks }
    \exp_args:Nx \hdclindex
      { \zref@extractdefault { HD.#1 } { guard@end } { 1 } } { }
  }
\cs_generate_variant:Nn \@@_module_push_aux:nn { on }
%    \end{macrocode}
% \end{macro}
%
% \begin{macro}{\@@_module_pop:n,
%   \@@_module_pop_aux:nn,\@@_module_pop_aux:on}
% 将模块名释放出栈。
%    \begin{macrocode}
\cs_new_protected_nopar:Npn \@@_module_pop:n
  {
    \seq_gpop:NNTF \g_@@_module_dest_seq \l_@@_tmpa_tl
      { \@@_module_pop_aux:on { \l_@@_tmpa_tl } }
      { \BOOM \use:n }
  }
\cs_new_protected:Npn \@@_module_pop_aux:nn #1
  {
    \zref@labelbylist { HD.#1 } { fdudoc }
    \hypersetup { hidelinks }
    \hdclindex {#1} { }
  }
\cs_generate_variant:Nn \@@_module_pop_aux:nn { on }
%    \end{macrocode}
% \end{macro}
%
% \begin{variable}{\g_@@_module_dest_seq}
% 存放模块名的序列(栈)。
%    \begin{macrocode}
\seq_new:N \g_@@_module_dest_seq
%    \end{macrocode}
% \end{variable}
%
% 处理行号超链接。使用 \pkg{zref} 宏包。
%    \begin{macrocode}
\zref@newlist { fdudoc }
\zref@newprop { guard@end } [ 1 ]
  { \int_eval:n { \c@HD@hypercount - 1 } }
\zref@addprop { fdudoc } { guard@end }
%    \end{macrocode}
%
% \paragraph{格式处理}
%
% \begin{macro}{\MacroFont}
% 代码部分的字体。
%    \begin{macrocode}
\cs_set_protected:Npn \MacroFont
  {
    \linespread { 1.05 }
    \small \ttfamily \mdseries \upshape
    \@@_verb_addon:
  }
%    \end{macrocode}
% \end{macro}
%
% \begin{macro}{\@@_output_module:nn,\@@_output_module_left:nn}
% 输出模块名(分行内和行号左侧两种)。
% \begin{arguments}
%   \item 颜色等样式
%   \item 模块名
% \end{arguments}
%    \begin{macrocode}
\cs_new_protected:Npn \@@_output_module:nn #1#2
  {
    \tex_noindent:D
    \group_begin:
      #1
      \footnotesize \normalfont \sffamily #2
    \group_end:
  }
\cs_new_protected:Npn \@@_output_module_left:nn #1#2
  {
    \tex_noindent:D
    \hbox_overlap_left:n
      {
        \@@_output_module:nn {#1} {#2}
        \skip_horizontal:n { \leftskip + \smallskipamount }
      }
  }
%    \end{macrocode}
% \end{macro}
%
% \begin{macro}{\@@_star_color:,\@@_slash_color:,\@@_pm_color:}
% 选择模块标记的颜色。模块标记的颜色会根据嵌套层次而改变。
%    \begin{macrocode}
\cs_new_protected_nopar:Npn \@@_star_color:
  {
    \seq_gpop:NNTF \g_@@_star_color_seq \current@color
      { \set@color }
      { \@@_select_color: }
    \seq_gpush:No \g_@@_slash_color_seq { \current@color }
  }
\cs_new_protected_nopar:Npn \@@_slash_color:
  {
    \seq_gpop:NNTF \g_@@_slash_color_seq \current@color
      {
        \set@color
        \seq_gpush:No \g_@@_star_color_seq { \current@color }
      }
% TODO: 需要报错:star 与 slash 没有匹配
      { \BOOM }
  }
\cs_new_protected_nopar:Npn \@@_pm_color:
  {
    \seq_get:NNTF \g_@@_star_color_seq \current@color
      { \set@color }
      {
        \@@_select_color:
        \seq_gpush:No \g_@@_star_color_seq { \current@color }
      }
  }
%    \end{macrocode}
% \end{macro}
%
% \begin{variable}{\g_@@_star_color_seq,\g_@@_slash_color_seq}
% 存放模块标记颜色的序列。
%    \begin{macrocode}
\seq_new:N \g_@@_star_color_seq
\seq_new:N \g_@@_slash_color_seq
%    \end{macrocode}
% \end{variable}
%
% \begin{macro}{\@@_select_color:}
% \begin{macro}[int]{guard@series}
% 设置模块标记的色系。
%    \begin{macrocode}
\cs_new_protected_nopar:Npn \@@_select_color:
  { \color { guard@series!!+ } }
\definecolorseries { guard@series }
  { cmyk } { last } { blue } { purple }
%    \end{macrocode}
% \end{macro}
% \end{macro}
%
% 设置色系的增量大小。可选参数 |3| 意味着基色(blue)与
% 末色(purple)之间将等分为三份。该数字比嵌套最大深度小 1。
%    \begin{macrocode}
\resetcolorseries [ 3 ] { guard@series }
%    \end{macrocode}
%
% \begin{macro}[int]{verb@guard,at@guard,code@gray}
% 设置颜色。
%    \begin{macrocode}
\definecolor { verb@guard } { named } { MaterialLime600 }
\definecolor { at@guard   } { named } { MaterialPink    }
\definecolor { code@gray  } { named } { MaterialGrey    }
%    \end{macrocode}
% \end{macro}
%
% \begin{macro}{\@@_module_angle:n}
% 输出“$\langle\cdots\rangle$”。^^A\\
%^^A 注:原来所使用的 \tn{textlangle} 和 \tn{textrangle} 在
%^^A Source Sans Pro 字体下不可用。
%    \begin{macrocode}
\cs_new_protected:Npn \@@_module_angle:n #1
  { \textlangle #1 \textrangle }
%   { < #1 > }
%   { \ensuremath \langle #1 \ensuremath \rangle }
%    \end{macrocode}
% \end{macro}
%
% \begin{macro}{\@@_code_line_no:}
% 行号。设置为阿拉伯数字。
%    \begin{macrocode}
\cs_new_protected_nopar:Npn \@@_code_line_no:
  {
    \int_gincr:N \c@CodelineNo
    \hbox_overlap_left:n
      {
        \hbox_to_wd:nn
          { \MacroIndent }
          {
            \HD@target
            \tex_hss:D \@@_code_line_no_style:
            \theCodelineNo \enspace
          }
        \tex_kern:D \@totalleftmargin
      }
  }
\tl_set:Nn \theCodelineNo { \arabic { CodelineNo } }
%    \end{macrocode}
% \end{macro}
%
% \begin{macro}{\@@_code_line_no_style:}
% 行号格式。
%    \begin{macrocode}
\cs_new_protected_nopar:Npn \@@_code_line_no_style:
  { \color { code@gray } \normalfont \sffamily \tiny }
%    \end{macrocode}
% \end{macro}
%
% \begin{macro}[int]{\HD@SetMacroIndent}
% 设置代码缩进(行号一栏的宽度)。该命令会写进 |.aux| 辅助文件,
% 以便在二次编译时确定最大行号宽度。
%    \begin{macrocode}
\cs_set_protected:Npn \HD@SetMacroIndent #1
  {
    \group_begin:
      \settowidth \MacroIndent
        {
          \@@_code_line_no_style:
          \prg_replicate:nn { \tl_count:n {#1} } { 0 }
          \enspace
        }
      \dim_gset_eq:NN \MacroIndent \MacroIndent
    \group_end:
  }
%    \end{macrocode}
% \end{macro}
%
% \subsubsection{\env{function} 环境}
%
% \begin{macro}{\@@_verb_addon:,
%   \@@_disable_ecglue:,
%   \@@_plain_punct_style:}
% \begin{macro}[int]{\meta@font@select}
% 调整文字间距,以便于让 CJK 字符占的宽度等于西文等宽字体中两个
% 空格的宽度。需要按编译情况分别定义。
%    \begin{macrocode}
\sys_if_engine_xetex:TF
  {
    \cs_set_eq:NN \@@_verb_addon: \xeCJKVerbAddon
    \cs_set:Nn \@@_plain_punct_style:
      { \xeCJKsetup { PunctStyle = plain } }
    \cs_set:Nn \@@_disable_ecglue:
      { \xeCJKsetup { CJKecglue } }
    \@@_appto_cmd:Nn \meta@font@select
      { \mode_if_inner:T { \@@_disable_ecglue: } }
  }
  {
    \cs_set_eq:NN \@@_verb_addon:        \prg_do_nothing:
    \cs_set_eq:NN \@@_plain_punct_style: \prg_do_nothing:
    \cs_set:Nn \@@_disable_ecglue:
      { \ltjsetparameter { autoxspacing = false } }
    \@@_appto_cmd:Nn \meta@font@select
      { \@@_disable_ecglue: }
  }
%    \end{macrocode}
% \end{macro}
% \end{macro}
%
% \begin{environment}{function}
% \begin{macro}{\@@_fix_previous_depth:}
% 调整 \env{function} 环境前后间距。
%    \begin{macrocode}
\BeforeBeginEnvironment { function }
  { \par \nointerlineskip }
\AtEndEnvironment { function }
  {
    \par
    \cs_gset:Nx \@@_fix_previous_depth:
      { \prevdepth = \the \prevdepth \space }
  }
\AfterEndEnvironment { function }
  { \@@_fix_previous_depth: }
%    \end{macrocode}
% \end{macro}
% \end{environment}
%
% \begin{environment}{syntax}
% \begin{environment}{fdusyntax}
% 在 \env{syntax} 和 \env{fdusyntax} 环境前设置若干活动字符。
% \texttt{\textbar} 用于分隔多个选项,无需倾斜;|<xxx>| 表示选项,
% |(xxx)| 表示默认选项。原来的括号用宏保存,并且使用直立字体。
% \env{syntax} 环境另需要额外调整行距、标点样式及字符间距。
%    \begin{macrocode}
\AtBeginEnvironment { syntax }
  {
    \linespread { 1.2 }
    \@@_plain_punct_style:
    \@@_disable_ecglue:
%     \char_set_catcode_active:N |
%     \char_set_catcode_active:N (
%     \char_set_active_eq:NN | \orbar
%     \char_set_active_eq:NN ( \defaultval@aux
  }
\AtBeginEnvironment { fdusyntax }
  {
    \cs_set:Npn \lparen { \textup { ( } }
    \cs_set:Npn \rparen { \textup { ) } }
    \char_set_catcode_active:N |
    \char_set_catcode_active:N <
    \char_set_catcode_active:N (
    \char_set_active_eq:NN | \orbar
    \char_set_active_eq:NN < \syntaxopt@aux
    \char_set_active_eq:NN ( \defaultval@aux
  }
%    \end{macrocode}
% \end{environment}
% \end{environment}
%
% \subsubsection{修订记录索引项}
%
% \begin{macro}{\@@_ltx_changes:nnn}
% 保存 \pkg{doc} 中 \tn{changes@} 的定义。
%    \begin{macrocode}
\cs_new_eq:NN \@@_ltx_changes:nnn \changes@
%    \end{macrocode}
% \end{macro}
%
% \begin{macro}[int]{\changes@}
% \changes{v0.4}{2017/07/30}{调整索引排序方式。}
% 重定义 \tn{changes@},在版本号一行显示修改日期。
%    \begin{macrocode}
\cs_set_protected:Npn \changes@ #1#2
  {
    \@@_save_version_date:nn {#1} {#2}
    \@@_ltx_changes:nnn {#1} {#2}
  }
%    \end{macrocode}
% \end{macro}
%
% \begin{variable}{\g_@@_version_date_prop}
% 存放版本号与对应的修改日期。
% key = 版本号,value = \{ 开始日期,结束日期 \}。
% 开始日期与结束日期可以相同。
%    \begin{macrocode}
\prop_new:N \g_@@_version_date_prop
%    \end{macrocode}
% \end{variable}
%
% \begin{macro}{\@@_save_version_date:nn}
% |nn| 版本最终将被 \tn{changes@} 调用。
% \begin{arguments}
%   \item 版本号
%   \item 日期
% \end{arguments}
% 它们分别对应 \tn{change} 的前两个参数(第三个是说明文字)。
%    \begin{macrocode}
\cs_new_protected:Npn \@@_save_version_date:nn #1#2
  {
    \prop_get:NnNTF \g_@@_version_date_prop
      {#1} \l_@@_tmpa_tl
      {
%    \end{macrocode}
% \cs{l_@@_tmp_tl} 相当于两个参数(开始日期、结束日期),
% 因此需要提前展开。
%    \begin{macrocode}
        \exp_after:wN
          \@@_save_version_date_aux:nnnn \l_@@_tmpa_tl
        {#2} {#1}
      }
      { \@@_save_version_date_aux:nnn {#1} {#2} {#2} }
  }
%    \end{macrocode}
% \end{macro}
%
% \begin{macro}{\@@_save_version_date_aux:nnnn}
% \begin{arguments}
%   \item 原开始日期
%   \item 原结束日期(显然应有 |#1| < |#2|)
%   \item 新读入的日期
%   \item 版本号
% \end{arguments}
% 如果 |#3| < |#1|,则读入日期 |#3|、|#2|;
% 如果 |#3| > |#2|,则读入日期 |#1|、|#3|。
%    \begin{macrocode}
\cs_new_protected:Npn \@@_save_version_date_aux:nnnn #1#2#3#4
  {
    \@@_if_date_later:nnTF {#1} {#3}
      { \@@_save_version_date_aux:nnn {#4} {#3} {#2} }
      {
        \@@_if_date_later:nnT {#3} {#2}
          { \@@_save_version_date_aux:nnn {#4} {#1} {#3} }
      }
  }
%    \end{macrocode}
% \end{macro}
%
% \begin{macro}{\@@_save_version_date_aux:nnn}
% 将版本号和日期存入 \cs{g_@@_version_date_prop}。
% \begin{arguments}
%   \item 版本号
%   \item 开始日期
%   \item 结束日期
% \end{arguments}
%    \begin{macrocode}
\cs_new_protected:Npn \@@_save_version_date_aux:nnn #1#2#3
  { \prop_gput:Nnn \g_@@_version_date_prop {#1} { {#2} {#3} } }
%    \end{macrocode}
% \end{macro}
%
% \begin{macro}[TF]{\@@_if_date_later:nn}
% \begin{macro}{\@@_parse_date:w}
% 比较两个日期。如果 |#1| 在 |#2| 之后,则为 |true|;反之为 |false|。
% 日期的格式为 YYYY/MM/DD。比较方法是直接将日期化成 8 位数字,
% 所以月、日前的 0 不可以省略。
%    \begin{macrocode}
\prg_new_conditional:Npnn \@@_if_date_later:nn #1#2 { TF, T }
  {
    \if_int_compare:w
        \@@_parse_date:w #1 / / / 0 \q_stop >
        \@@_parse_date:w #2 / / / 0 \q_stop \exp_stop_f:
      \prg_return_true:
    \else:
      \prg_return_false:
    \fi:
  }
\cs_new:Npn \@@_parse_date:w #1/#2/#3/ #4 \q_stop
  { #1#2#3 }
%    \end{macrocode}
% \end{macro}
% \end{macro}
%
% \begin{macro}[int]{\CTEX@versionitem}
% 版本条目标签。如果版本号不在 \cs{g_@@_version_date_prop} 的
% key 里面,则利用未定义的 \cs{BOOM} 报错。
%    \begin{macrocode}
\cs_new_protected:Npn \CTEX@versionitem #1 \efill
  {
    \@idxitem
    \prop_get:NnNTF \g_@@_version_date_prop
      {#1} \l_@@_tmpa_tl
      {
        \exp_after:wN
          \@@_print_version_date:nnn \l_@@_tmpa_tl
        {#1}
      }
      { \BOOM }
  }
%    \end{macrocode}
% \end{macro}
%
% \begin{macro}{\@@_print_version_date:nnn}
% 输出版本号和日期。如果开始日期和结束日期相同,则只输出一项。
% \begin{arguments}
%   \item 开始日期
%   \item 结束日期
%   \item 版本号
% \end{arguments}
%    \begin{macrocode}
\cs_new_protected:Npn \@@_print_version_date:nnn #1#2#3
  {
    \noindent
    \Hy@raisedlink { \belowpdfbookmark {#3} { HD.#3 } }
    \textbf {#3} \hfill
    \hbox:n
      {
        \footnotesize
        \str_if_eq:nnTF {#1} {#2}
          { ( #1 ) } { ( #1 ~ -- ~ #2 ) }
      }
    \par \nopagebreak
  }
%    \end{macrocode}
% \end{macro}
%
% \begin{macro}[int]{\HDorg@theglossary}
% 该命令由 \pkg{hypdoc} 宏包定义,用于存放标准文档类 \cls{book}
% 中定义的 \tn{theindex} 命令。
% 此处的补丁将在版本号一行最后加上修改日期。
%    \begin{macrocode}
\ctex_patch_cmd:Nnn \HDorg@theglossary
  { \let \item \@idxitem }
  { \let \item \CTEX@versionitem }
%    \end{macrocode}
% \end{macro}
%
% \begin{macro}[int]{\@wrglossary}
% 该命令由 \LaTeXe{} 内核定义,又由 \pkg{hypdoc} 宏包作了修改。
% 此处的补丁使得修订记录条目的页码能够指向对应行。
%    \begin{macrocode}
\ctex_patch_cmd:Nnn \@wrglossary
  { hdpindex }
  {
    \ifnum \c@HD@hypercount = \z@
      hdpindex
    \else
      hdclindex { \the \c@HD@hypercount }
    \fi
  }
%    \end{macrocode}
% \end{macro}
%
% \subsubsection{命令补丁}
%
% \paragraph{\LaTeXe{} 补丁}
%
% \begin{macro}[int]{\@addtocurcol}
% 调整浮动体、代码等与文字的间距。
% 见 \url{http://tex.stackexchange.com/a/40896}。
%    \begin{macrocode}
\ctex_patch_cmd:Nnn \@addtocurcol
  { \vskip \intextsep }
  {
    \edef \save@first@penalty { \the \lastpenalty } \unpenalty
    \ifnum \lastpenalty = \@M
      \unpenalty
    \else
      \penalty \save@first@penalty \relax
    \fi
    \ifnum \outputpenalty < -\@Mii
      \addvspace\intextsep
      \vskip\parskip
    \else
      \addvspace\intextsep
    \fi
  }
\ctex_patch_cmd:Nnn \@addtocurcol
  {
    \vskip \intextsep
    \ifnum \outputpenalty < -\@Mii
      \vskip -\parskip
    \fi
  }
  {
    \ifnum \outputpenalty < -\@Mii
      \aftergroup \vskip \aftergroup \intextsep
      \aftergroup \nointerlineskip
    \else
      \vskip \intextsep
    \fi
  }
%    \end{macrocode}
% \end{macro}
%
% \begin{macro}[int]{\@getpen}
% 将换行或换页的最大罚值由 \num{10000} 改为 \num{10001}。
%    \begin{macrocode}
\ctex_patch_cmd:Nnn \@getpen { \@M } { \@Mi }
%    \end{macrocode}
% \end{macro}
%
% \begin{macro}[int]{\l@section,\l@subsection}
% 修改目录条目的缩进。
%    \begin{macrocode}
\ctex_patch_cmd:Nnn \l@section    { 2.5em } { 1.5em }
\ctex_patch_cmd:Nnn \l@subsection { 2.5em } { 1.5em }
%    \end{macrocode}
% \end{macro}
%
% \begin{macro}[int]{\@thehead}
% 修改页眉,禁用 \tn{MakeUppercase}。
%    \begin{macrocode}
\@@_preto_cmd:Nn \@thehead
  { \cs_set_eq:cN { MakeUppercase \space } \@iden }
%    \end{macrocode}
% \end{macro}
%
% \begin{macro}{\thebibliography}
% \begin{macro}[int]{\HDorg@thebibliography}
% 参考文献一节需要编号。
%    \begin{macrocode}
\ctex_patch_cmd:Nnn \HDorg@thebibliography
  { \section* } { \section }
\cs_set_eq:NN \thebibliography \HDorg@thebibliography
%    \end{macrocode}
% \end{macro}
% \end{macro}
%
% \begin{macro}{\GlossaryParms}
% 修改修订记录中的一些缩进。
%    \begin{macrocode}
\@@_appto_cmd:Nn \GlossaryParms
  {
    \raggedcolumns
    \cs_set_eq:NN \Hy@writebookmark \HDorg@writebookmark
    \cs_set:Npn \@idxitem   { \par \hangindent 2em }
    \cs_set:Npn \subitem    { \@idxitem \hspace* { 1em } }
    \cs_set:Npn \subsubitem { \@idxitem \hspace* { 2em } }
  }
%    \end{macrocode}
% \end{macro}
%
% \begin{macro}[int]{\HoLogo@LaTeXe}
% 由于使用了 \pkg{unicode-math},需要额外修改 \pkg{hologo} 中的
% \tn{LaTeXe} 命令,以使粗体正常显示。
%    \begin{macrocode}
\ctex_patch_cmd:Nnn \HoLogo@LaTeXe
  { \hbox { \HOLOGO@MathSetup 2 $ _{ \textstyle \varepsilon } $ } }
  {
    \hbox
      {
        \mathsurround 0pt \relax
        2
        \if b \expandafter \@car \f@series \@nil
          $ _{ \textstyle \symbf { \varepsilon } } $
        \else
          $ _{ \textstyle \varepsilon } $
        \fi
      }
  }
%    \end{macrocode}
% \end{macro}
%
% \begin{macro}{\SpecialMainEnvIndex}
% 索引汉化。
%    \begin{macrocode}
\ctex_patch_cmd:Nnn \SpecialMainEnvIndex
  { (environment) } { ~ 环境 }
\ctex_patch_cmd:Nnn \SpecialMainEnvIndex
  { environments: } { 环境: }
%    \end{macrocode}
% \end{macro}
%
% \begin{macro}[int]{\HDorg@SpecialEnvIndex}
% \pkg{hypdoc} 宏包重新定义了 \tn{SpecialEnvIndex} 命令,因此需要
% 修改内部定义。
%    \begin{macrocode}
\ctex_patch_cmd:Nnn \HDorg@SpecialEnvIndex
  { (environment) } { ~ 环境 }
\ctex_patch_cmd:Nnn \HDorg@SpecialEnvIndex
  { environments: } { 环境: }
%    \end{macrocode}
% \end{macro}
%
% \paragraph{\cls{l3doc} 补丁}
%
%    \begin{macrocode}
%<@@=codedoc>
%    \end{macrocode}
%
% \begin{macro}{\list}
% \cls{l3doc} 会设置列表环境中 \tn{listparindent} |=| \tn{z@},
% 这里将其恢复。
%    \begin{macrocode}
\cs_set_eq:NN \list \@@_oldlist:nn
%    \end{macrocode}
% \end{macro}
%
% \begin{macro}{\@@_function_descr_start:w}
% 抑制首段的 \tn{parskip}。
%    \begin{macrocode}
\__fdudoc_patch_cmd:Nnn \@@_function_descr_start:w
  { \noindent }
  { \skip_vertical:n { -\parskip } \noindent }
%    \end{macrocode}
% \end{macro}
%
% \begin{macro}{\@@_function_assemble:}
% 在 \cls{l3doc} 中,\env{function} 环境里的 \env{syntax} 和
% \env{descr} 盒子中间存在一段 \tn{med\-skip\-amount} 的距离。
% 但是如果 \env{syntax} 盒子为空(未使用 \env{syntax} 环境),
% 就会显得不好看。此时通过把 \tn{medskipamount} 设置为零来修正。
% 若盒子非空,则把 \tn{parskip} 还回去。
%    \begin{macrocode}
\__fdudoc_preto_cmd:Nn \@@_function_assemble:
  {
    \box_if_empty:NTF \g_@@_syntax_box
      { \skip_zero:N \medskipamount }
      { \skip_add:Nn \medskipamount { \parskip } }
  }
%    \end{macrocode}
% \end{macro}
%
% \begin{macro}{\@@_typeset_functions:}
% 调整 \env{function} 环境的字体。
%    \begin{macrocode}
\__fdudoc_patch_cmd:Nnn \@@_typeset_functions:
  { \small \ttfamily } { \footnotesize \CodeFont }
%    \end{macrocode}
% \end{macro}
%
% \begin{macro}{\@@_typeset_functions:,\@@_macro_init:,
%   \@@_macro_dump:}
% 左侧边注的函数列表采用单倍行距。
%    \begin{macrocode}
\__fdudoc_preto_cmd:Nn \@@_typeset_functions:
  { \MacroFont }
\__fdudoc_patch_cmd:Nnn \@@_macro_init:
  { \hbox:n } { \MacroFont \hbox:n }
\__fdudoc_patch_cmd:Nnn \@@_macro_dump:
  { \hbox_unpack_clear:N } { \MacroFont \hbox_unpack_clear:N }
%    \end{macrocode}
% \end{macro}
%
% \begin{macro}{\@@_macro_end_style:n}
% 禁止显示 \env{macro} 环境最后的 “(\emph{End definition for ...})”。
%    \begin{macrocode}
\cs_set_eq:NN \@@_macro_end_style:n \use_none:n
%    \end{macrocode}
% \end{macro}
%
% \begin{macro}{\@@_macro_typeset_one:nN}
% 在 \env{macro} 环境的侧边栏中,\cls{l3doc} 根据命令的长短,分别用
% 普通字体和紧缩字体输出。然而很长的命令还是会超出页边。这里用缩放
% 盒子的手段使得长命令也可正常显示。
%    \begin{macrocode}
\cs_set_protected:Npn \@@_macro_typeset_one:nN #1#2
  {
    \vbox_set:Nn \l_@@_macro_box
      {
        \MacroFont
        \vbox_unpack_clear:N \l_@@_macro_box
        \hbox_set:Nn \l_tmpa_box
          { \@@_print_macroname:nN {#1} #2 }
%    \end{macrocode}
% \tn{marginparwidth} 和 \tn{marginparsep} 分别是边注的宽度及其到
% 版心的距离,\tn{la\-bel\-sep} 则是编号盒子右端与条目首行文本之间
% 的距离。
%    \begin{macrocode}
        \dim_set:Nn \l_tmpa_dim
          { \marginparwidth - \labelsep - \marginparsep }
        \dim_compare:nNnT { \box_wd:N \l_tmpa_box } > \l_tmpa_dim
          {
            \box_resize_to_wd_and_ht:Nnn \l_tmpa_box
              { \l_tmpa_dim } { \box_ht:N \l_tmpa_box }
          }
        \hbox_overlap_left:n
          {
            \box_use:N \l_tmpa_box
            \skip_horizontal:n { \marginparsep - \labelsep }
          }
      }
    \int_incr:N \l_@@_macro_int
  }
%    \end{macrocode}
% \end{macro}
%
% \begin{macro}{\@@_print_macroname:nN}
% 该函数不再需要根据命令的长短切换字体。
%    \begin{macrocode}
\cs_set_protected:Npn \@@_print_macroname:nN #1#2
  {
    \strut
    \@@_get_hyper_target:xN
      {
        \exp_not:n {#1}
        \bool_if:NT #2 { \tl_to_str:n {TF} }
      }
      \l_@@_tmpa_tl
    \cs_if_exist:cTF { r@ \l_@@_tmpa_tl }
      { \exp_args:NNo \label@hyperref [ \l_@@_tmpa_tl ] }
      { \use:n }
      {
        \tl_set:Nn \l_@@_tmpa_tl {#1}
%    \end{macrocode}
% 命令中的空格改用“\textvisiblespace”显示。
%    \begin{macrocode}
        \tl_replace_all:Non \l_@@_tmpa_tl
          { \c_catcode_other_space_tl }
          { \fontspec_visible_space: }
        \@@_macroname_prefix:o \l_@@_tmpa_tl
        \@@_macroname_suffix:N #2
      }
  }
%    \end{macrocode}
% \end{macro}
%
% \begin{macro}{\@@_special_index_module:nnnnN}
% 索引汉化。
%    \begin{macrocode}
\cs_set_protected:Npn \@@_special_index_module:nnnnN #1#2#3#4#5
  {
    \use:x
      {
        \exp_not:n { \@@_special_index_aux:nnnnnnn {#1} {#2} }
          \tl_if_empty:nTF {#3}
            { { } { } { } { } }
            {
              \str_if_eq_x:nnTF {#3} { TeX }
                {
                  { TeX~ and~ LaTeX2e }
                  { \string \TeX{}~ 和~ \string \LaTeXe{} }
                }
                { {#3} { \string \pkg {#3} } }
              \bool_if:NTF #5
                { { commands~ internal } { ~ 内部命令: } }
                { { commands           } { ~ 命令:     } }
            }
      }
    {#4}
  }
%    \end{macrocode}
% \end{macro}
%
% \begin{macro}{\@@_special_index_aux:nnnnnnn}
% 该函数在 \cls{l3doc} 中本来只有 6 个参数。这里增加了一个,用来辅助
% 排序。
% \begin{arguments}
%   \item 键(即宏名称字符串,用来排序)
%   \item 宏名称
%   \item 索引头排序字符串(排序)
%   \item 索引头文字
%   \item 索引头后缀字符串(排序,新增)
%   \item 索引头后缀文字
%   \item 索引类型(\opt{main}/\opt{usage} 等)
% \end{arguments}
%    \begin{macrocode}
\cs_new_protected:Npn \@@_special_index_aux:nnnnnnn #1#2#3#4#5#6#7
  {
    \tl_set:Nn \l_@@_index_escaped_key_tl {#1}
    \@@_quote_special_char:N \l_@@_index_escaped_key_tl
    \@@_special_index_set:Nn
      \l_@@_index_escaped_macro_tl {#2}
    \str_if_eq:onTF { \@currenvir } { macrocode }
      { \codeline@wrindex }
      {
        \str_case:nnF {#7}
          {
            { main  } { \codeline@wrindex }
            { usage } { \index }
          }
          { \HD@target \index }
      }
      {
        \tl_if_empty:nF { #3 #4 #5 #6 }
          { #3 #5 \actualchar #4 #6 \levelchar }
        \l_@@_index_escaped_key_tl
        \actualchar
        {
          \token_to_str:N \verbatim@font \c_space_tl
          \l_@@_index_escaped_macro_tl
        }
        \encapchar
        hdclindex { \the \c@HD@hypercount } {#7}
      }
  }
%    \end{macrocode}
% \end{macro}
%
% \subsubsection{杂项}
%
% \begin{macro}{\StopSpecialIndexModule,
%   \@@_special_index_module:nnnnN}
% 不对 \cs{cs} 和 \cs{tn} 等编索引。用于目录、索引等。
%    \begin{macrocode}
\DeclareDocumentCommand \StopSpecialIndexModule { }
  {
    \cs_set_eq:NN
      \@@_special_index_module:nnnnN \use_none:nnnnn
  }
\tl_map_inline:nn { \actualchar \encapchar \levelchar }
  { \exp_args:Nx \DoNotIndex { \bslash \tl_to_str:N #1 } }
%    \end{macrocode}
% \end{macro}
%
% \begin{macro}{\meta}
% 重定义 \cs{meta} 命令,需要禁用中文、西文之间的空格,
% 并确保为罗马字体族。
%    \begin{macrocode}
\RenewDocumentCommand \meta { m }
  {
    \group_begin:
      \sys_if_engine_xetex:T { \xeCJKsetup { CJKecglue = { } } }
      \rmfamily \@@_meta:n {#1}
    \group_end:
  }
%    \end{macrocode}
% \end{macro}
%
% 重定向 \cls{l3doc} 中的一些提示信息。
%    \begin{macrocode}
\msg_redirect_name:nnn { l3doc } { foreign-internal    } { log }
\msg_redirect_name:nnn { l3doc } { print-changes-howto } { log }
\msg_redirect_name:nnn { l3doc } { print-index-howto   } { log }
%    \end{macrocode}
%
%    \begin{macrocode}
%<@@=>
\ExplSyntaxOff
%    \end{macrocode}
%
% \subsubsection{排版样式设置}
%
% 目录中禁止对 \cs{cs} 和 \cs{tn} 等的索引。
%    \begin{macrocode}
\AtBeginDocument{%
  \addtocontents{toc}{\StopSpecialIndexModule}}
%    \end{macrocode}
%
% \begin{macro}{\path,\opt,\kaishu}
% 设置 PDF 字符串中的命令替换。
%    \begin{macrocode}
\pdfstringdefDisableCommands{%
  \let\path\meta
  \let\opt\@firstofone
  \let\kaishu\relax}
%    \end{macrocode}
% \end{macro}
%
% \begin{macro}[int]{\@makefntext}
% 调整脚注文本缩进。
%    \begin{macrocode}
\renewcommand\@makefntext[1]{\parindent 0em\noindent\@makefnmark~#1}
%    \end{macrocode}
% \end{macro}
%
% \begin{macro}{\IndexLayout,\indexname}
% 调整索引外观。
%    \begin{macrocode}
\IndexPrologue{%
  \section{\indexname}%
  \textit{无衬线字体的数字表示对应索引项出现的页码;
    带下划线的数字表示定义对应索引项的代码行号;
    其他普通数字则表示使用对应索引项的代码行号.}}
\def\IndexLayout{%
  \newgeometry{%
    left   = 0.85 in,
    right  = 0.85 in,
    top    = 1.25 in,
    bottom = 1.00 in}%
  \setlength\IndexMin{0.5\textheight}%
  \ctexset{section/numbering=false}%
  \StopSpecialIndexModule}
\def\indexname{代码索引}
%    \end{macrocode}
% \end{macro}
%
% \begin{macro}{\usage}
% 控制“描述对应索引项的页码”样式。在 \pkg{doc} 宏包中的定义为
% \tn{textit}。
%    \begin{macrocode}
\def\usage#1{\textsf{#1}}
%    \end{macrocode}
% \end{macro}
%
% \begin{macro}{\glossaryname}
% 调整修订记录外观。
%    \begin{macrocode}
\GlossaryPrologue{\section{\glossaryname}}
\def\glossaryname{修订记录}
%    \end{macrocode}
% \end{macro}
%
% 西文字体。
%    \begin{macrocode}
\setmainfont{TeX Gyre Pagella}
\setsansfont{TeX Gyre Heros}
\setmathfont{TeX Gyre Pagella Math}
%    \end{macrocode}
%
% \begin{macro}{\kaishu,\fangsong}
% 中文字体。
%    \begin{macrocode}
\setCJKmainfont{FZShuSong-Z01}%
  [
    BoldFont       = FZHei-B01,
    ItalicFont     = FZKai-Z03,
    BoldItalicFont = FZKai-Z03
  ]
\setCJKsansfont{FZHei-B01}%
  [
    BoldFont       = FZHei-B01,
    ItalicFont     = FZKai-Z03,
    BoldItalicFont = FZKai-Z03
  ]
\setCJKmonofont{FZFangSong-Z02}%
  [
    BoldFont       = FZHei-B01,
    ItalicFont     = FZKai-Z03,
    BoldItalicFont = FZKai-Z03
  ]
\newCJKfontfamily\kaishu{FZKai-Z03}%
  [BoldFont = *, ItalicFont = *, BoldItalicFont = *]
\newCJKfontfamily\fangsong{FZFangSong-Z02}%
  [BoldFont = *, ItalicFont = *, BoldItalicFont = *]
%    \end{macrocode}
% \end{macro}
%
% \begin{macro}{\CodeFont}
% \begin{macro}[int]{\fdudoc@code@font,\fdudoc@cjk@code@font}
% 代码部分的字体,这里用了两种系列。
%    \begin{macrocode}
\setmonofont{CMU Typewriter Text}%
  [
    UprightFont = * Light,
    BoldFont    = * Bold,
    SlantedFont = * Light Oblique,
    HyphenChar  = None
  ]
\newfontfamily\fdudoc@code@font{Source Code Pro}%
  [BoldFont = * Semibold]
\newCJKfontfamily\fdudoc@cjk@code@font{Source Han Sans SC}%
  [BoldFont = * Bold, AutoFakeSlant = 0.22]
\newcommand*\CodeFont{\fdudoc@code@font\fdudoc@cjk@code@font}
%    \end{macrocode}
% \end{macro}
% \end{macro}
%
% 中文排版格式(\pkg{ctex} 宏包)。
%    \begin{macrocode}
\ctexset
  {
    section = {name = {第,节}, format+ = \raggedright},
    subsubsection/tocline = {\kaishu\CTEXnumberline{#1}#2},
    paragraph = {runin = false, numbering = false}
  }
%    \end{macrocode}
%
% 设定章节标题、目录深度。
%    \begin{macrocode}
\setcounter{secnumdepth}{4}
\setcounter{tocdepth}{3}
%    \end{macrocode}
%
% 单位设置(\pkg{siunitx} 宏包)。
%    \begin{macrocode}
\sisetup
  {
    number-math-rm       = \ensuremath,
    inter-unit-product   = \ensuremath{{}\cdot{}},
    group-digits         = true,
    group-minimum-digits = 4,
    group-separator      = \text{~},
    range-phrase         = \symbol{"FF5E},
    separate-uncertainty = true
  }
%    \end{macrocode}
%
% 超链接设置(\pkg{hyperref} 宏包)。
%    \begin{macrocode}
\hypersetup
  {
    bookmarksdepth    = 4,
    bookmarksnumbered = true,
    colorlinks        = true,
    citecolor         = MaterialGreen,
    linkcolor         = MaterialPink,
    urlcolor          = MaterialIndigo
  }
%    \end{macrocode}
%
% 浮动体标题设置(\pkg{caption} 宏包)。
%    \begin{macrocode}
\captionsetup{labelsep = quad, labelfont+ = bf}
%    \end{macrocode}
%
% 设置标准列表环境样式。
%    \begin{macrocode}
\setlist{noitemsep, topsep=\smallskipamount}
\setlist[1]{labelindent=\parindent}
\setlist[enumerate]{leftmargin=*}
\setlist[itemize]{leftmargin=*}
%    \end{macrocode}
%
% \begin{environment}{optdesc}
% 用于描述各选项。设置条目间距为 \tn{marginparsep},与
% \cls{l3doc} 一致。
%    \begin{macrocode}
\newlist{optdesc}{description}{3}
\setlist[optdesc]{%
  font=\mdseries\small\ttfamily, align=right,
  listparindent=\parindent,
  labelsep=\marginparsep, labelindent=-\marginparsep,
  leftmargin=*}
%    \end{macrocode}
% \end{environment}
%
% \begin{environment}{tablenotes}
% \begin{variable}{tpt@id}
% 重新定义 \pkg{threeparttable} 包的 \env{tablenotes} 环境,
% 用于表格的注释。
%    \begin{macrocode}
\renewlist{tablenotes}{description}{1}
\setlist[tablenotes]{%
  format=\normalfont\tnote@item, align=right,
  listparindent=\parindent, labelindent=\tabcolsep,
  leftmargin=*, rightmargin=\tabcolsep,
  after=\@noparlisttrue}
\AtBeginEnvironment{tablenotes}{%
  \setlength\parindent{2\ccwd}%
  \normalfont\footnotesize}
\AtBeginEnvironment{threeparttable}{%
  \stepcounter{tpt@id}%
  \edef\curr@tpt@id{tpt@\arabic{tpt@id}}}
\newcounter{tpt@id}
%    \end{macrocode}
% \end{variable}
% \end{environment}
%
% \begin{macro}[int]{\tnote@item,\TPTtagStyle}
% 为 \tn{tnote} 增加超链接。
%    \begin{macrocode}
\def\tnote@item#1{%
  \Hy@raisedlink{\hyper@anchor{\curr@tpt@id-#1}}#1}
\def\TPTtagStyle#1{\hyperlink{\curr@tpt@id-#1}{#1}}
%    \end{macrocode}
% \end{macro}
%
% \begin{macro}{\UrlAlphabet,\UrlDigits}
% 网址断行。\tn{UrlOrds}、\tn{UrlAlphabet} 和 \tn{UrlDigits}
% 分别记录了特殊符号、字母和数字,把它们依次附加在 \pkg{url} 宏包
% 提供的命令 \tn{UrlBreaks} 之后,即可允许在这些位置处断行。与
% \cs{fdu_allow_url_break:} 的原理是相同的。
%    \begin{macrocode}
\def\UrlAlphabet{%
  \do\a\do\b\do\c\do\d\do\e\do\f\do\g\do\h\do\i\do\j%
  \do\k\do\l\do\m\do\n\do\o\do\p\do\q\do\r\do\s\do\t%
  \do\u\do\v\do\w\do\x\do\y\do\z\do\A\do\B\do\C\do\D%
  \do\E\do\F\do\G\do\H\do\I\do\J\do\K\do\L\do\M\do\N%
  \do\O\do\P\do\Q\do\R\do\S\do\T\do\U\do\V\do\W\do\X%
  \do\Y\do\Z}
\def\UrlDigits{%
  \do\1\do\2\do\3\do\4\do\5\do\6\do\7\do\8\do\9\do\0}
\g@addto@macro\UrlBreaks{\UrlOrds}
\g@addto@macro\UrlBreaks{\UrlAlphabet}
\g@addto@macro\UrlBreaks{\UrlDigits}
%    \end{macrocode}
% \end{macro}
%
% 不对下列各项添加索引。
%    \begin{macrocode}
\DoNotIndex{\begin,\end,
  \a,\b,\c,\d,\e,\f,\g,\h,\i,\j,\k,\l,\m,
  \n,\o,\p,\q,\r,\s,\t,\u,\v,\w,\x,\y,\z,
  \A,\B,\C,\D,\E,\F,\G,\H,\I,\J,\K,\L,\M,
  \N,\O,\P,\Q,\R,\S,\T,\U,\V,\W,\X,\Y,\Z,
  \0,\1,\2,\3,\4,\5,\6,\7,\8,\9}
%    \end{macrocode}
%
% 启用索引、交叉引用、历史记录。
%    \begin{macrocode}
\EnableCrossrefs
\CodelineIndex
\RecordChanges
%    \end{macrocode}
%
% \subsubsection{文档层命令}
%
% \begin{macro}{\exptarget,\rexptarget,\expstar,\rexpstar,
%   \__codedoc_typeset_exp:,\__codedoc_typeset_rexp:}
% 部分命令之后的星号($\star$ 或 \ding{73}),表明其不同用法。
% 这里的“exp”和“rexp”分别源自 \LaTeX3 中的“expandable”
% 和“restricted-expandable”。
%    \begin{macrocode}
\newcommand*\exptarget{\Hy@raisedlink{\hypertarget{expstar}{}}}
\newcommand*\rexptarget{\Hy@raisedlink{\hypertarget{rexpstar}{}}}
\newcommand*\expstar{\hyperlink{expstar}{$\star$}}
\newcommand*\rexpstar{\hyperlink{rexpstar}{\ding{73}}}
% TODO: (2017/08/12) 允许修改 exptstar 和 rexpstar
\renewcommand*\expstar{\hyperlink{expstar}{$\mfrake$}}
\renewcommand*\rexpstar{\hyperlink{rexpstar}{$\mfrakc$}}
\ExplSyntaxOn
\cs_set_eq:NN \__codedoc_typeset_exp:  \expstar
\cs_set_eq:NN \__codedoc_typeset_rexp: \rexpstar
\ExplSyntaxOff
%    \end{macrocode}
% \end{macro}
%
% \begin{macro}{\marg,\oarg,\parg}
% 几种命令参数:
% \begin{itemize}
%   \item 必选参数:|{|\meta{arg}|}|;
%   \item 可选参数:|[|\meta{arg}|]|;
%   \item 图形参数:|(|\meta{arg}|)|。
% \end{itemize}
%    \begin{macrocode}
\renewcommand*\marg[1]{\{\meta{#1}\}}
\renewcommand*\oarg[1]{[\meta{#1}]}
\renewcommand*\parg[1]{(\meta{#1})}
%    \end{macrocode}
% \end{macro}
%
% \begin{macro}{\opt}
% 选项名。
%    \begin{macrocode}
\DeclareDocumentCommand\opt{m}{\texttt{#1}}
%    \end{macrocode}
% \end{macro}
%
% \begin{macro}{\kvopt}
% \meta{key} |=| \meta{value} 型选项。
%    \begin{macrocode}
\DeclareDocumentCommand\kvopt{mm}
  {\texttt{#1\breakablethinspace=\breakablethinspace#2}}
%    \end{macrocode}
% \end{macro}
%
% \begin{macro}{\breakablethinspace}
% 允许换行的细间距。
%    \begin{macrocode}
\def\breakablethinspace{\hskip 0.16667em\relax}
%    \end{macrocode}
% \end{macro}
%
% \begin{macro}{\syntaxopt,\defaultval}
% \begin{macro}[int]{\syntaxopt@aux,\defaultval@aux}
% \env{syntax} 中的选项及命令选项。
% |aux| 结尾的两个命令用于定义利用 |<>| 和 |()| 的简写形式。
%    \begin{macrocode}
\def\syntaxopt#1{\textit{#1}}
\def\defaultval#1{\textbf{\textup{#1}}}
\def\syntaxopt@aux#1>{\syntaxopt{#1}}
\def\defaultval@aux#1){\defaultval{#1}}
%    \end{macrocode}
% \end{macro}
% \end{macro}
%
% \begin{macro}{\orbar,\TF,\TTF,\TFF}
% \env{syntax} 中的选项分隔符,以及 \opt{true} 或 \opt{false}
% 选项的几种快捷方式。
%    \begin{macrocode}
\def\orbar{\textup{\textbar}}
\def\TF{true\orbar false}
\def\TTF{\defaultval{true}\orbar false}
\def\TFF{true\orbar\defaultval{false}}
%    \end{macrocode}
% \end{macro}
%
% \begin{environment}{arguments}
% 放在 \env{macro} 环境中,用于描述对应命令的参数。
% \cls{l3doc} 中的定义 \opt{labelsep} 设置不太合理,会使标签被覆盖,
% 这里重新定义。
%    \begin{macrocode}
\DeclareDocumentEnvironment{arguments}{}
  {\enumerate[%
    label={\texttt{\#\arabic*:~}}, labelsep=0pt, nolistsep]}%
  {\endenumerate}
%    \end{macrocode}
% \end{environment}
%
% \begin{macro}{\TeX,\LaTeX,\LaTeXe,
%   \pdfTeX,\pdfLaTeX,\XeTeX,\XeLaTeX,\LuaTeX,\LuaLaTeX,
%   \AmSLaTeX,\TeXLive,\MiKTeX,\BibTeX,\biber,\TikZ}
% \TeX{} 相关标志。
%    \begin{macrocode}
\def\TeX{\hologo{TeX}}
\def\LaTeX{\hologo{LaTeX}}
\def\LaTeXe{\hologo{LaTeXe}}
\def\pdfTeX{\hologo{pdfTeX}}
\def\pdfLaTeX{\hologo{pdfLaTeX}}
\def\XeTeX{\hologo{XeTeX}}
\def\XeLaTeX{\hologo{XeLaTeX}}
\def\LuaTeX{\hologo{LuaTeX}}
\def\LuaLaTeX{\hologo{LuaLaTeX}}
\def\AmSLaTeX{\hologo{AmSLaTeX}}
\def\TeXLive{\TeX\ Live}
\def\MiKTeX{\hologo{MiKTeX}}
\def\BibTeX{\hologo{BibTeX}}
\def\BibTeX{\hologo{BibTeX}}
\def\biber{\hologo{biber}}
%    \end{macrocode}
% 该定义来自 \file{pgfmanual-en-macros.tex}。
%    \begin{macrocode}
\def\TikZ{Ti\emph{k}Z}
%    \end{macrocode}
% \end{macro}
%
% \begin{macro}{\cs,\tn}
% \begin{macro}[int]{\codedoc@cs,\codedoc@tn}
% 控制序列。
%^^A Colors are used for debug.
%    \begin{macrocode}
% \let\codedoc@cs=\cs
% \let\codedoc@tn=\tn
% \renewcommand*\cs[2][]{%
%   \textcolor{MaterialIndigo}{\codedoc@cs[#1]{#2}}}
% \renewcommand*\tn[2][]{%
%   \textcolor{MaterialPink}{\codedoc@tn[#1]{#2}}}
%    \end{macrocode}
% \end{macro}
% \end{macro}
%
% \begin{macro}{\file,\env,\pkg,\cls}
% 文件、环境、宏包、文档类。
%    \begin{macrocode}
% \renewcommand*\file[1]{%
%   \textcolor{MaterialGrey900}{\texttt{#1}}}
\renewcommand*\env[1]{\textbf{\texttt{#1}}}
% \renewcommand*\pkg[1]{\textsf{#1}}
% \renewcommand*\cls[1]{\textit{\textsf{#1}}}
%    \end{macrocode}
% \end{macro}
%
% \begin{macro}{\bashcmd}
% Bash 中的命令。
%    \begin{macrocode}
\newcommand*\bashcmd[1]{\texttt{#1}}
%    \end{macrocode}
% \end{macro}
%
% \begin{macro}{\scite}
% 位于上标的文献引用。
%    \begin{macrocode}
\newcommand*\scite[1]{\textsuperscript{\cite{#1}}}
%    \end{macrocode}
% \end{macro}
%
% \begin{environment}{quote*}
% 引述环境。
% \begin{arguments}
%   \item 作者
%   \item 朝代
% \end{arguments}
%    \begin{macrocode}
\DeclareDocumentEnvironment{quote*}{oo}
  {\quote\fangsong\qquad}%
  {\endquote\IfNoValueF{#1}{%
    \hfill —— \IfNoValueF{#2}{〔#2〕}#1}}
%    \end{macrocode}
% \end{environment}
%
% \paragraph{示例代码环境}
%
% \pkg{listings} 宏包中连字符 |-| 原本以数学模式输出,
% 此处改为普通文本。
% 见 \url{https://tex.stackexchange.com/a/33188/136923}。
%    \begin{macrocode}
\lst@CCPutMacro\lst@ProcessOther {"2D}{\lst@ttfamily{-{}}{-{}}}
\@empty\z@\@empty
%    \end{macrocode}
%
% \changes{v0.5}{2017/09/05}{[\pkg{fdudoc}] 移除 \pkg{listings}
%   关键字定义文件。}
%
% 定义几种代码样式。
% \begin{macro}[int]{style@base}
%    \begin{macrocode}
\lstdefinestyle{style@base}
  {
    extendedchars   = true,
    gobble          = 3,
    lineskip        = 2 pt,
    frame           = l,
    framerule       = 1 pt,
    framesep        = 0 pt,
    escapeinside    = {(*}{*)},
    basicstyle      = \small\CodeFont\color{MaterialGrey900},
    keywordstyle    = \bfseries\color{MaterialIndigo},
    commentstyle    = \itshape\color{MaterialGrey600},
    stringstyle     = \color{MaterialRed},
    backgroundcolor = \color{MaterialGrey50}
  }
%    \end{macrocode}
% \end{macro}
%
% \begin{macro}[int]{style@shell}
%    \begin{macrocode}
\lstdefinestyle{style@shell}
  {
    style      = style@base,
    rulecolor  = \color{MaterialPink},
    language   = bash,
    alsoletter = {-},
    emphstyle  = \color{MaterialGreen800}
  }
%    \end{macrocode}
% \end{macro}
%
% \begin{macro}[int]{style@latex}
%    \begin{macrocode}
\lstdefinestyle{style@latex}
  {
    style      = style@base,
    rulecolor  = \color{MaterialIndigo},
    language   = [LaTeX]TeX,
    alsoletter = {*, -},
    texcsstyle = *\color{MaterialDeepOrange},
    emphstyle  = [1]\color{MaterialGreen800},
    emphstyle  = [2]\color{MaterialTeal}
  }
%    \end{macrocode}
% \end{macro}
%
% \begin{macro}[int]{style@syntax}
%    \begin{macrocode}
\lstdefinestyle{style@syntax}
  {
    extendedchars = true,
    gobble        = 6,
    escapeinside  = {(*}{*)},
    language      = [LaTeX]TeX,
    alsoletter    = {*, -},
    basicstyle    = \footnotesize\CodeFont\color{MaterialGrey900},
    keywordstyle  = \bfseries\color{MaterialIndigo},
    commentstyle  = \itshape\color{MaterialGrey600},
    texcsstyle    = *\color{MaterialDeepOrange},
    emphstyle     = [1]\color{MaterialGreen800},
    emphstyle     = [2]\color{MaterialTeal}
  }
%    \end{macrocode}
% \end{macro}
%
% \begin{environment}{shellexample}
% \begin{environment}{latexexample}
% Shell 和 \LaTeX{} 示例代码。
%    \begin{macrocode}
\lstnewenvironment{shellexample}[1][]{%
  \lstset{style=style@shell, #1}}{}
\lstnewenvironment{latexexample}[1][]{%
  \lstset{style=style@latex, #1}}{}
%    \end{macrocode}
% \end{environment}
% \end{environment}
%
% \begin{environment}{fdusyntax}
% 语法说明。用于代替 \cls{l3doc} 中的 \env{syntax} 环境。
%    \begin{macrocode}
\lstnewenvironment{fdusyntax}[1][]{%
  \lstset{style=style@syntax, #1}\vspace{-1.8ex}}{}
%</doc>
%    \end{macrocode}
% \end{environment}
%
% \subsubsection{\pkg{latexmk} 配置文件}
%
%    \begin{macrocode}
%<*latexmk|latexmk-en>
# Latexmk configuration file.

# Use XeLaTeX to compile.
$pdf_mode = 5;

# Process index.
$makeindex = 'zhmakeindex -s gind.ist %O -o %D %S';

# Show CPU time used.
$show_time = 1;
%</latexmk|latexmk-en>
%<*latexmk>

# Process glossary (change history).
add_cus_dep('glo', 'gls', 0, 'makeglo2gls');
sub makeglo2gls {
    system("zhmakeindex -s gglo.ist -o \"$_[0].gls\"
        -t \"$_[0].glg\" \"$_[0].glo\"");
}
%</latexmk>
%    \end{macrocode}
%
% \clearpage
%
% \end{implementation}
%
