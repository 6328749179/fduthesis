\chapter{快速入门}

这一章的目的是用最短的时间介绍 \LaTeX{} 中最重要的概念、思想与方法。因此,我们将忽略所有无关紧要
的细节,包括但不限于:

\begin{itemize}
  \item 安装 \TeX{} 发行版;
  \item 配置编辑器;
  \item 排版复杂的数学公式;
  \item 插入图片与表格;
  \item 添加参考文献等。
\end{itemize}

这些高级内容将在后续介绍。

\section{基本概念}

\LaTeX{} 是一种\kwd{标记语言},它是在\kwd{宏语言} \TeX{} 的基础上封装而成的。之所以称为“标记语言”,
是因为 \LaTeX{} 会通过一系列的文法标记来指定文章的样式。这里,我们需要指出 \LaTeX{} 中最重要的思想
——\emph{内容与格式分离}。也就是说,文章的作者只需要关注文章的内容,而无需操心其样式;“设计师”可以
指定文章的样式,但无需关注具体内容。

举例来说,如果需要强调文字,可以使用 \cs{emph} 这样的命令(这就是所谓“标记”),使得文字呈现出倾斜
的效果。如果直接用 \cs{textit},虽然似乎可以达到同样的显示效果,但却违背了内容与格式分离的原则。
因为“倾斜”并不是强调文字的固有属性,只是设计师让它呈现出了这种样貌。

\LaTeX{} 作为一种语言,需要有具体的实现,即\kwd{排版引擎}(或称\kwd{编译器})。排版引擎可以读入
代码,并编译生成目标 PDF 文档。常见的引擎有 \pdfTeX{}、\XeTeX{}、\LuaTeX{} 等。

排版引擎实际上只定义了一些\kwd{原语}。如果只使用这些原语,编写复杂的文章将会变得极其困难。\LaTeX{}
在原语的基础上定义了更多的命令,也就是上文中所说的文法标记。它们组成了一套\kwd{格式}。

通常情况下,安装在计算机中的 \TeX{} 系统(称为\kwd{发行版})已经提供了将排版引擎与格式组合在一起的
命令。例如调用 \bashcmd{xelatex} 命令(在 Windows 下是调用 \file{xelatex.exe} 程序),就相当于使用
\XeTeX{} 引擎,并指定采用 \LaTeX{} 格式。

为了高效地编写 \LaTeX{} 源文件,我们需要使用一些\kwd{文本编辑器},如 TeXworks、TeXstudio 等。
由于种种原因,\emph{非常不建议}使用 Windows 下的记事本程序作为编辑器。在下文中,我们将以 TeXstudio
为例进行介绍。

\section{Hello, world!}

按照惯例,我们的第一篇文档将从“Hello, world!”开始。打开 TeXstudio,新建空白文档,输入以下内容,
并保存为 \file{hello.tex}。

\begin{verbatim}
\documentclass{article}
% This is a comment.
\begin{document}
Hello, world!
\end{document}
\end{verbatim}

点击工具栏中的“构建并查看”按钮(或按 F5 键),即可开始编译。如图~\ref{fig:hello} 所示,编译结果
默认会显示在屏幕右侧。也可以直接打开 PDF 文件查看结果。生成的 PDF 文件中,除了“Hello, world!”
这句话,还有页面下方的页码(注意在图~\ref{fig:hello} 并未显示出来)。

\begin{figure}[htb]
  \includegraphics[width=\textwidth]{images/hello.png}
  \caption{在 TeXstudio 中编辑文档 \file{hello.tex}}
  \label{fig:hello}
\end{figure}

上面这段代码中,除了要排版的文字,还有一些以反斜杠 |\| 开头的命令(称为\kwd{控制序列})。控制序列
可以带有参数,参数通常需要放在一组花括号中。

第一行代码 |\documentclass{article}| 中包含了命令 \cs{documentclass}。这一命令用来指定\kwd{文档类},
它是对文档样式的总体设定。\cls{article} 文档类是 \LaTeX{} 的标准文档类之一,表示篇幅较短的文章。

第三行与第五行分别包含了命令 \cs{begin} 与 \cs{end}。这两个命令以及它们中间的内容称为\kwd{环境}。
这里就声明了一个名为 \env{document} 的环境。只有在 \env{document} 环境中的内容,才会被排版到文档中。
\cs{documentclass} 与 |\begin{document}| 之间的部分称为\kwd{导言区}。对文档样式的所有设置,都应当
放在导言区中。

我们可以发现,内容与格式分离的原则在这里得到了很好的贯彻:文档内容必须放在 \env{document} 环境中,
而排版样式则需要在导言区中定义。如果不遵守这种约定,很有可能导致编译错误。

第二行以百分号 |%| 开头,这表示一段\kwd{注释}。在该行中,|%| 之后的所有字符都将被忽略,它们不会对
排版结果有任何影响。

\section{你好,world!}

本节将介绍中英文混排。如果不使用中文,可以忽略本节内容。

排版中文文档与排版英文文档并没有显著的区别。我们同样给出一个简短的示例:

\begin{verbatim}
\documentclass[UTF8]{ctexart}
\begin{document}
你好,world!
\end{document}
\end{verbatim}

\begin{figure}[htb]
  \includegraphics[width=\textwidth]{images/hello-zh.png}
  \caption{含有中文的文档 \file{hello-zh.tex}}
  \label{fig:hello-zh}
\end{figure}

与上一小节中的代码相比,除了更改了文字、删去了无关紧要的注释,主要的变化只在第一行。我们改用了
\cls{ctexart} 文档类,前面还添加了一个 |[UTF8]|。方括号中的部分是\kwd{可选参数}。这里的 |UTF8| 是
\cls{ctexart} 文档类的一个选项,指明了编写文档时使用的编码格式。

需要强调的是,为了正确输出中文,请务必确保 \file{.tex} 文档已被保存成了 UTF-8 编码(这也是
TeXstudio 的默认编码)。

另外需要注意,这里的逗号 |,| 和感叹号 |!| 都是中文全角标点,请不要与英文半角标点 |,| 和 |!| 混淆。

\section{数学公式}

\LaTeX{} 最吸引人的功能便是优美的数学排版。数学公式分为\kwd{行内公式}与\kwd{行间公式}两种。前者会将
公式插入在文字之间,后者则会独立成行,有时还可以添加编号。

行内公式用一组美元符号 |$|\ldots|$| 包围,如 |$a+b=c$|,其输出结果为 $a+b=c$。

行间公式又分为带编号和不带编号两种。带编号的公式放在 \env{equation} 环境中(|^| 符号表示上标):

\begin{verbatim}
\begin{equation}
  a^2 + b^2 = c^2
\end{equation}
\end{verbatim}

其排版结果为:
\begin{equation}
  a^2 + b^2 = c^2
\end{equation}

不带编号的公式放在命令 \cs{[} 与 \cs{]}之间:

\begin{verbatim}
\[ E = mc^2 \]
\end{verbatim}

排版结果为:
\[ E = mc^2 \]

公式中额外的空格字符将被忽略
\footnote{当然并非所有的空格都可以忽略,比如 \verb|\sin| 与 \verb|\s in| 显然不是一回事。}。
但我们通常会人为添加一些空格以区分各个子表达式(正如上面的示例),否则很容易导致公式难以阅读。
另需注意,公式中不可以出现空行。
