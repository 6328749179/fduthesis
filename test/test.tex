\input fduthesis-test-toolkit

\documentclass[twoside]{fduthesis}
\usepackage{kantlipsum}
\usepackage{zhlipsum}

\fdusetup{
  style = {
    % font = none,
    cjk-font = windows,
    % font-size = 5,
    fullwidth-stop = mapping,
    % footnote-style = pifont*,
    % auto-make-cover = false
    logo = {fudan-name.pdf, fudan-emblem-new.pdf},
    % logo-size = {2cm,2cm,2cm}
    hyperlink = color,
    hyperlink-color = default,
%
%
    bib-backend = bibtex,
    bib-style = numerical,
    bib-resource = {test.bib}
  },
  info = {
    title = {自河南经乱关内阻饥兄弟离散各在一处},
    title* = {Measurements of the interaction between energetic
      photons and hadrons show that the interaction},
%    date = {2017年2月10日},
    author = {某某某},
    author* = {Xiangdong Zeng},
    supervisor = {S. Nawata},
    supervisor-title = {青年副研究员},
    major = {物理学},
    department = {凝聚态物理系},
    affiliation = {复旦大学},
    student-id = {14307110000},
    keywords = {\LaTeX, hello, world, 物理, 中心法则, China, 为侨服务},
    keywords* = {\LaTeX, hello, world, physics, central rule, China},
    clc = {O414.1/65}
  }
}


% \setmainfont{Adobe Garamond Pro}[ItalicFont=Consolas,BoldFont=Candara]
% \setsansfont{Calibri}
% \setmonofont{Consolas}
% \setmathfont{TeX Gyre Pagella Math}

\def\WORLDb{你好,世界。}

\newcommand{\fonttesttext}{你好,世界\symbol{12290}Hello, world!}
\newcommand\fonttest{%
  正常:\qquad\fonttesttext \par
  粗体:\qquad\textbf{\fonttesttext} \par
  倾斜:\qquad\textit{\fonttesttext} \par
  小型大写:\qquad\textsc{\fonttesttext}
}

\def\BibTeX{B\textsc{ib}\TeX}

\begin{document}

%\raggedbottom
%
%\frontmatter
%
%\tableofcontents
%
%\begin{abstract}
%\zhlipsum[1-10]
%\end{abstract}
%
%\begin{abstract*}
%\kant[1-10]
%\end{abstract*}
%
%\begin{notation}[lp{20em}]
%$\sin$      &  正弦 \\
%HPC         &  高性能计算 (High Performance Computing) \\
%cluster     &  集群 \\
%Itanium     &  安腾 \\
%SMP         &  对称多处理 \\
%API         &  应用程序编程接口 \\
%PI          &  聚酰亚胺 \\
%MPI         &  聚酰亚胺模型化合物,N-苯基邻苯酰亚胺 \\
%PBI         &  聚苯并咪唑 \\
%MPBI        &  聚苯并咪唑模型化合物,N-苯基苯并咪唑 \\
%PY          &  聚吡咙 \\
%PMDA-BDA    &  均苯四酸二酐与联苯四胺合成的聚吡咙薄膜 \\
%$\Delta G$  &  活化自由能 (Activation Free Energy) \\
%$\chi$      &  传输系数 (Transmission Coefficient) \\
%$E$         &  能量 \\
%$m$         &  质量 \\
%$c$         &  光速 \\
%$P$         &  概率 \\
%$T$         &  时间 \\
%$v$         &  速度
%\end{notation}

\mainmatter

%\def\PLACEHOLDERi{————汉字————汉字————汉字————汉字————汉字————六七八九十}
%\def\PLACEHOLDER{\zhlipsum[1]}%{\PLACEHOLDERi\PLACEHOLDERi\PLACEHOLDERi\PLACEHOLDERi\PLACEHOLDERi}
%
%%1
%\chapter{文本,字体,脚注 \quad Text, font and footnote}
%\zhlipsum[1]
%\section{文字与段落 Text and paragraph}
%\zhlipsum[1]
%\clearpage
%\zhlipsum[1-2]
%
%% 2
%\chapter{文本,字体,脚注 \quad Text, font and footnote}
%\section{文字与段落 Text and paragraph}
%\zhlipsum[1-4]
%
%% 3
%\chapter{文本,字体,脚注 \quad Text, font and footnote}
%
%\section{文字与段落 Text and paragraph}
%
%\subsection{文字与段落 Text and paragraph}
%\zhlipsum[1-2]
%
%% 4
%\chapter{文本,字体,脚注 \quad Text, font and footnote}
%
%{}
%\section{文字与段落 Text and paragraph}
%\zhlipsum[1-2]

%\subsection{中文文本 Chinese}
%\PLACEHOLDER
%
%\subsubsection{中文文本}
%\PLACEHOLDER
%
%\clearpage
%
%\zhlipsum[-10]
%
%\subsubsection{Chinese}
%\zhlipsum
%
%\subsection{英文文本 English}
%\kant
%
%\clearpage
%
%\section{字体 Font}
%
%\subsection{普通字体 Roman}
%\fonttest
%
%\subsection{无衬线字体 Sans-serif}
%\textsf{\fonttest}
%
%\subsection{打字机字体 Typewriter}
%\texttt{\fonttest}
%
%\subsection{句号}
%如果\symbol{"002E}会突然:\textsc{Full Stop} \par
%如果\symbol{"3002}会突然:\textsc{Ideographic Full SStop} \par
%如果\symbol{"FF0E}会突然:\textsc{Fullwidth Full Stop} \par
%如果\symbol{"FF61}会突然:\textsc{Halfwidth Ideographic Full Stop} \par
%
%\chapter{脚注 Footnote}

脚注\footnote{脚注1如果会突然}。

脚注 \footnote{脚注2。千千万的。}。未取得的

脚注。\footnote{脚注3是一个长脚注。\zhlipsum*[2]}未取得的

\textit{脚注倾斜。 \footnote{脚注4}未取得的}

\textbf{脚注加粗。\footnote{脚注5} 未取得的}

vs\footnote{脚注6}未取得的

vs \footnote{脚注7要分段。\par 不舒服不得不运河滩上野跑,头顶着毒热的阳光,身上再裹起兜肚,一不风凉,
二又窝汗,穿不了一天,就得起大半身痱子。再有,全村跟他一般大的小姑娘,
谁的兜肚也没有这么花儿草儿的鲜艳,他穿在身上,男不男,女不女,
小姑娘们要用手指刮破脸蛋儿。}未取得的

vs\footnote{脚注8}ge

vs\footnote{脚注9} jsty

vs\footnote{脚注10分三段。青大娘大高个儿,一双大脚,青铜肤色,
嗓门也亮堂,骂起人来,方圆二三十里,敢说找不出能够招架几个回合的敌手。
一丈青大娘骂人,就像雨打芭蕉,长短句。\par
青大娘大高个儿,一双大脚,青铜肤色,
嗓门也亮堂,骂起人来,方圆二三十里,敢说找不出能够招架几个回合的敌手。
一丈青大娘骂人,就像雨打芭蕉,长短句。 \par
青大娘大高个儿,一双大脚,青铜肤色,
嗓门也亮堂,骂起人来,方圆二三十里,敢说找不出能够招架几个回合的敌手。
一丈青大娘骂人,就像雨打芭蕉,长短句。}

Text%
\footnote{This is a footnote.}%
\footnote{This is a footnote.}%
\footnote{This is a footnote.}%
\footnote{This is a footnote.}%
\footnote{This is a footnote.}%
\footnote{This is a footnote.}%
\footnote{This is a footnote.}%
\footnote{This is a footnote.}%
\footnote{This is a footnote.}%
\footnote{This is a footnote.}%
\footnote{This is a footnote.}%
\footnote{This is a footnote.}%
\footnote{This is a footnote.}%
\footnote{This is a footnote.}%
\footnote{This is a footnote.}%
\footnote{This is a footnote.}%
\footnote{This is a footnote.}%
\footnote{This is a footnote.}%
\footnote{This is a footnote.}%
\footnote{This is a footnote.}%
\footnote{This is a footnote.}%
\footnote{This is a footnote.}%
\footnote{This is a footnote.}%
\footnote{This is a footnote.}%
\footnote{This is a footnote.}%
\footnote{This is a footnote.}%
\footnote{This is a footnote.}%
\footnote{This is a footnote.}%
\footnote{This is a footnote.}%
\footnote{This is a footnote.}%
\footnote{This is a footnote.}%
\footnote{This is a footnote.}%
\footnote{This is a footnote.}%
\footnote{This is a footnote.}%
\footnote{This is a footnote.}%
\footnote{This is a footnote.}%
\footnote{This is a footnote.}%
\footnote{This is a footnote.}%
\footnote{This is a footnote.}%
\footnote{This is a footnote.}%
\footnote{This is a footnote.}%
\footnote{This is a footnote.}%
\footnote{This is a footnote.}%
\footnote{This is a footnote.}%
\footnote{This is a footnote.}%
\footnote{This is a footnote.}%
\footnote{This is a footnote.}%
\footnote{This is a footnote.}%
\footnote{This is a footnote.}%
\footnote{This is a footnote.}%

%
%\chapter{定理}

\begin{proof}
道千乘之国,敬事而信,节用而爱人,使民以时。
\end{proof}

\begin{definition}
证明完毕/证讫,又写作Q.E.D.。这是拉丁词组“quod erat demonstrandum”
(这就是所要证明的)的缩写,译自希腊语“ὅπερ ἔδει δεῖξαι”
(hoper edei deixai),很多早期数学家用过,包括欧几里得和阿基米德。
“Q.E.D.”可以在证明的尾段写出,以显示证明所需的结论已经完整了。
\end{definition}

\begin{lemma}
  这是一条华丽丽的引理。
\end{lemma}

\begin{proof}[出师表]
先帝创业未半而中道崩殂,今天下三分,益州疲弊,此诚危急存亡之秋也。
\begin{equation}
  \sum_{k=0}^{\infty} \frac{1}{x^k} = \int \sin x dx
\end{equation}
\end{proof}

\begin{proof}
先帝创业未半而中道崩殂,今天下三分,益州疲弊,此诚危急存亡之秋也。
\begin{equation*}
  \sum_{k=0}^{\infty} \frac{1}{x^k} = \int \sin x dx
\end{equation*}
\end{proof}

\begin{lemma}
  这又是一条华丽丽的引理。
\end{lemma}

\chapter{定理(续)}
\newcounter{thm}

\newtheorem[style=plain,qed=\ensuremath{\sin}]{p}{平凡}
\newtheorem[style=margin]{mm}{打断}
\newtheorem[style=change]{fduc}{变革}
\newtheorem[style=break]{fdub}{平凡}
\newtheorem[style=marginbreak]{mb}{打断}
\newtheorem[style=break]{cb}{变革}
\newtheorem*[style=plain]{np}{平凡}
\newtheorem*[style=margin]{nmm}{打断}
\newtheorem*[style=change]{nfduc}{变革}
\newtheorem*[style=break,qed=]{nfdub}{平凡}
\newtheorem*[style=marginbreak,qed={}]{nmb}{打断}
\newtheorem[style=break,counter=thm]{ncb}{变革}

\newtheorem{prop}[thm]{命题}

\begin{prop}
  直角三角形。
\end{prop}
\begin{prop}[圆形]
  直角三角形。
\end{prop}
\begin{prop}
  直角三角形。
\end{prop}

\begin{p}
这是一条定理。
\[ \sum_{k=0}^{\infty} \frac{1}{x^k} = \int \sin x dx \]
\end{p}

\begin{mm}[明早]
这是一条定理。
\[ \sum_{k=0}^{\infty} \frac{1}{x^k} = \int \sin x dx \]
\end{mm}

\begin{fduc}
这是一条定理。
\[ \sum_{k=0}^{\infty} \frac{1}{x^k} = \int \sin x dx \]
\end{fduc}

\begin{fdub}
这是一条定理。
\[ \sum_{k=0}^{\infty} \frac{1}{x^k} = \int \sin x dx \]
\end{fdub}

\begin{mb}
这是一条定理。
\[ \sum_{k=0}^{\infty} \frac{1}{x^k} = \int \sin x dx \]
\end{mb}

\begin{cb}
这是一条定理。
\[ \sum_{k=0}^{\infty} \frac{1}{x^k} = \int \sin x dx \]
\end{cb}

%%%%%%%%%%%%%%%%%%%%%%%%%%%%%%%%%%%%%%%%%%%%%

\begin{np}
这是一条定理。
\[ \sum_{k=0}^{\infty} \frac{1}{x^k} = \int \sin x dx \]
\end{np}

\begin{nmm}[明天一早]
这是一条定理。
\[ \sum_{k=0}^{\infty} \frac{1}{x^k} = \int \sin x dx \]
\end{nmm}

\begin{nfduc}
这是一条定理。
\[ \sum_{k=0}^{\infty} \frac{1}{x^k} = \int \sin x dx \]
\end{nfduc}

\begin{nfdub}
这是一条定理。
\[ \sum_{k=0}^{\infty} \frac{1}{x^k} = \int \sin x dx \]
\end{nfdub}

\begin{nmb}
这是一条定理。
\[ \sum_{k=0}^{\infty} \frac{1}{x^k} = \int \sin x dx \]
\end{nmb}

\begin{ncb}
这是一条定理。
\[ \sum_{k=0}^{\infty} \frac{1}{x^k} = \int \sin x dx \]
\end{ncb}

\begin{ncb}
这是一条定理。
\[ \sum_{k=0}^{\infty} \frac{1}{x^k} = \int \sin x dx \]
\end{ncb}

\begin{ncb}
这是一条定理。
\[ \sum_{k=0}^{\infty} \frac{1}{x^k} = \int \sin x dx \]
\end{ncb}

\begin{ncb}
这是一条定理。
\[ \sum_{k=0}^{\infty} \frac{1}{x^k} = \int \sin x dx \]
\end{ncb}

%
%\chapter{图表 vs 浮动体}

\section{title}
Myriad,英语单词,意为「无数的」。同时,「Myriad」也是一款字体的名字。
由罗伯特·斯林巴赫(Robert Slimbach,1956年-)和卡罗·图温布利
(Carol Twombly,1959年-)在1990年到1992年期间以 Frutiger 字体为蓝本
为 Adobe 公司设计。 Myriad 是早期数码字体时代的先驱,伴随着技术的成长
一路走来。

\begin{figure}[h]
  \centering
  \includegraphics[width=3cm]{../logo/pdf/fudan-emblem-a-blue.pdf}
  \includegraphics[width=4cm]{../logo/pdf/fudan-emblem-new-b-red.pdf}
  \caption{Multiple Master 是 Type 1字体格式的扩展部分。Type 1 是利用
    PostScript 语言描述字形信息的字体系统。Type 1字体是第一款矢量字体
    (outline font),通过二维坐标系中的关键点和三次贝塞尔曲线描述字体
    的边缘,在屏幕显示和输出时,在光栅图像处理器内,根据字号大小计算
    出字体边缘(栅格化)。}
\end{figure}

如今,它更多地和我们相见在显示屏幕上。当然,还有那著名的标榜设计的
电子品牌。1992 年,耗时两年开发的 Myriad 终于发布了历史上第一个版本:
Myriad MM。

\section{title}
这款温和且具有良好可读性的人文主义无衬线字体,集诸多当时最新的数字
字体技术于一身。 后缀 MM,意为 Multiple Master,没有找到对应的中文
译名,我们权且称之为「多母板技术」。Myriad 是最早采用 Multiple Master
技术的无衬线字体之一。这项技术的原理是在坐标轴(Axis)的区间两端设计
极限母板,中间的变量则采取线性或非线性变化,对于字体来说,字型的宽度、
粗细甚至有无衬线,都可以在坐标轴上设置。此外,MM 技术还提供了在小字号
下屏幕显示的视觉修正(Optical Adjustment),也就是说,同一款字体,在
小字号时,其字间距和笔画粗细,会被适当地放大。而衬线字体,随着字号的
变小,衬线会相对变粗。视觉修正可以提高小字号字体的识别性,对于远低于
印刷分辨率的电脑屏幕来说,也具有重要意义。

\begin{table}[h]
  \centering
  \caption{一个 normal 表格}
  \begin{tabular}{ccc}
    \hline
    \bfseries 功能 & \bfseries 环境 & \bfseries code \\
    \hline
    表格 & tabular & \ttfamily \backslash begin\{tabular\} ... \backslash end\{tabular\} \\
    插图 & figure  & \ttfamily \backslash begin\{figure\}  ... \backslash end\{figure\}  \\
    居中 & center  & \ttfamily \backslash begin\{center\}  ... \backslash end\{center\}  \\
    \hline
  \end{tabular}
\end{table}

在 Multiple Master 的时代,字号是从6pt到72pt之间非线性设置的。这一传统
保留到了今天 Truetype 和 Opentype 的 Single Master 时代。Adobe 软件的
字体下拉菜单,仍然只显示6到72pt 的字号。

%
%\chapter{文本}
%
%\section{文字与段落}
%
%\textbf{本段使用 \texttt{\string\cite}}
%Myriad,英语单词,意为「无数的」\cite{sunstein,gjhjbhjkjbzs,hblzsthjkjyxgs}。
%同时,「Myriad」也是一款字体的名字。
%由罗伯特·斯林巴赫(Robert Slimbach,1956年--)和卡罗·图温布利
%(Carol Twombly,1959年-)\cite{wyf,sunstein,zgtsgxh,lzp1}
%在1990年到1992年期间以 Frutiger 字体为蓝本为 Adobe 公司设计\cite{cdy}。
%Myriad 是早期数码字体时代的先驱,\cite{wfz,wfz1}
%伴随着技术的成长一路走来 \cite{hlswedl,zgdylsdag}。
%
%\textbf{本段使用 \texttt{\string\citep}}
%如今,它更多地和我们相见在显示屏幕上 \citep{wfz2}。当然,还有那著名的标榜设计的
%电子品牌 \citep{cgw,mks}。1992 年,耗时两年开发的 Myriad 终于发布了历史上第一个版本:
%Myriad MM \citep{wyf,hblzsthjkjyxgs,sunstein,zgtsgxh,aaas}。
%
%\section{title}
%
%\textbf{本段使用 \texttt{\string\citet}}
%这款温和且具有良好可读性的人文主义无衬线字体\citet{yjb},集诸多当时最新的数字
%字体技术于一身。 后缀 MM,意为 Multiple Master,没有找到对应的中文
%译名\citet{lbm,calkin},我们权且称之为「多母板技术」。Myriad 是最早采用 Multiple Master
%技术的无衬线字体之一。这项技术的原理是在坐标轴(Axis)的区间两端设计
%极限母板,中间的变量则采取线性或非线性变化,对于字体来说,字型的宽度、
%粗细甚至有无衬线\citet{xadzkjdx,yufin,cgw},都可以在坐标轴上设置。此外,MM 技术还提供了在小字号
%下屏幕显示的视觉修正(Optical Adjustment),也就是说,同一款字体,在
%小字号时,其字间距和笔画粗细,会被适当地放大。而衬线字体,随着字号的
%变小,衬线会相对变粗。视觉修正可以提高小字号字体的识别性,对于远低于
%印刷分辨率的电脑屏幕来说,也具有重要意义。
%
%在 Multiple Master 的时代,字号是从6pt到72pt之间非线性设置的。这一传统
%保留到了今天 Truetype 和 Opentype 的 Single Master 时代。Adobe 软件的
%字体下拉菜单,仍然只显示6到72pt 的字号。
%
%\chapter{假文}
%
%\begin{definition}
%证明完毕/证讫,又写作Q.E.D.。这是拉丁词组“quod erat demonstrandum”
%(这就是所要证明的)的缩写,译自希腊语“ὅπερ ἔδει δεῖξαι”(hoper edei
%deixai),很多早期数学家用过,包括欧几里得和阿基米德。“Q.E.D.”可以
%在证明的尾段写出,以显示证明所需的结论已经完整了。
%\end{definition}
%
%\zhlipsum[1-15]
%
%\nocite{*}
%
%\backmatter
%
%\bibliography{test}
%\printbibliography

\end{document}
