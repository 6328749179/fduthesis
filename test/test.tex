\PassOptionsToPackage{log-declarations=false}{xparse}
\documentclass[twoside]{../fduthesis}
\usepackage{kantlipsum}
\usepackage{zhlipsum}

\fdusetup{
  style = {
    font = times,
    % CJKfont = founder,
    fontsize = -4,
    %fullwidth-stop
    footnotestyle = times
  },
  info = {
    title = {Helmholtz 方程解的唯一延拓及逆源问题},
    title* = {Unique Continuation and Inverse Source Problem \\
    for Helmholtz Equation},
%    date = {2017年2月10日},
    author = {某某某},
    author* = {Xiangdong Zeng},
    supervisor = {陈丙丁 \quad 工程师},
    instructors = {
      张五六 \quad 工程师,
      赵\quad 甲 \quad 工程师,
      王三四 \quad 讲\quad 师
    },
    major = {物理学},
    department = {凝聚态物理系},
    studentID = {14307110000},
    schoolID = {10246},
    keywords = {\LaTeX, hih, vwvs, 物理, 中心法则, vsbd, 为侨服务},
    keywords* = {\LaTeX, hih, vwvs, Physics, Central Rule, vsbd},
    CLC = {O414.1/65}
  }
}

\newcommand{\fonttesttext}{你好,世界\symbol{12290}Hello, world!}
\newcommand\fonttest{%
  正常:\qquad\fonttesttext \par
  粗体:\qquad\textbf{\fonttesttext} \par
  倾斜:\qquad\textit{\fonttesttext} \par
  小型大写:\qquad\textsc{\fonttesttext}
}

\begin{document}
\frontmatter
\makecover
\tableofcontents

\mainmatter
\begin{abstract}
中心法则形成了分子生物学的生命观:生命世界在信息和规律上是统一的;
生命的物质基础和主宰物质是不同的,生命的物质基础是以核酸蛋白质整合
体系为主宰的原生质各种必要的物质组分及其实在的相互作用;生命运动
的本质在于组成生命的各物质之间以及生命的物质、能量与信息之间和生命
自身与环境之间的不断的相互作用。不少人不少人,如果是到无穷大。

上标\TeX;此时的的

上标\TeX ;此时的的

上标\LaTeX;此时的的

上标\LaTeX ;此时的的

\zhlipsum[1-15]
\end{abstract}

\begin{abstract*}
\LaTeX3 does not use @ as a ``letter'' for defining internal
macros. Instead, the symbols are used in internal macro names to
provide structure. The name of each function is divided into
logical units using separates the name of the function from the
argument specifier (``arg-spec''). This describes the arguments
expected by the function.

\kant[2-20]
\end{abstract*}

\begin{notation}
$\sin$      &  正弦 \\
HPC         &  高性能计算 (High Performance Computing) \\
cluster     &  集群 \\
Itanium     &  安腾 \\
SMP         &  对称多处理 \\
API         &  应用程序编程接口 \\
PI          &  聚酰亚胺 \\
MPI         &  聚酰亚胺模型化合物,N-苯基邻苯酰亚胺 \\
PBI         &  聚苯并咪唑 \\
MPBI        &  聚苯并咪唑模型化合物,N-苯基苯并咪唑 \\
PY          &  聚吡咙 \\
PMDA-BDA    &  均苯四酸二酐与联苯四胺合成的聚吡咙薄膜 \\
$\Delta G$  &  活化自由能 (Activation Free Energy) \\
$\chi$      &  传输系数 (Transmission Coefficient) \\
$E$         &  能量 \\
$m$         &  质量 \\
$c$         &  光速 \\
$P$         &  概率 \\
$T$         &  时间 \\
$v$         &  速度 \\
劝学        &  君子曰:学不可以已。青,取之于蓝,而青于蓝;冰,水为之,而寒于水。木
  直中绳。\symbol{"2B413}\symbol{"8F2E} d 輮以为轮,其曲中规。虽有槁暴,不复挺者,輮使之然也。故木受绳则直,金就
  砺则利,君子博学而日参省乎己,则知明而行无过矣。—— 荀况 \\
$\sin$      &  正弦 \\
HPC         &  高性能计算 (High Performance Computing) \\
cluster     &  集群 \\
Itanium     &  安腾 \\
SMP         &  对称多处理 \\
API         &  应用程序编程接口 \\
PI          &  聚酰亚胺 \\
MPI         &  聚酰亚胺模型化合物,N-苯基邻苯酰亚胺 \\
PBI         &  聚苯并咪唑 \\
MPBI        &  聚苯并咪唑模型化合物,N-苯基苯并咪唑 \\
PY          &  聚吡咙
\end{notation}

\chapter{测试 Test}
\section{文字与段落 Text and paragraph}
\subsection{中文文本 Chinese}
\zhlipsum

\subsection{英文文本 English}
\kant

\clearpage

\section{字体 Font}
\subsection{普通字体 Roman}
\fonttest

\subsection{无衬线字体 Sans-serif}
\textsf{\fonttest}

\subsection{打字机字体 Typewriter}
\texttt{\fonttest}

\subsection{句号}
如果\symbol{"002E}会突然:\textsc{Full Stop} \par
如果\symbol{"3002}会突然:\textsc{Ideographic Full SStop} \par
如果\symbol{"FF0E}会突然:\textsc{Fullwidth Full Stop} \par
如果\symbol{"FF61}会突然:\textsc{Halfwidth Ideographic Full Stop} \par

\clearpage

\section{脚注 Footnote}
脚注\footnote{脚注1如果会突然}。

脚注 \footnote{脚注2。千千万的。}。未取得的

脚注。\footnote{脚注3是一个长脚注。\zhlipsum*[2]}未取得的

\textit{脚注倾斜。 \footnote{脚注4}未取得的}

\textbf{脚注加粗。\footnote{脚注5} 未取得的}

vs\footnote{脚注6}未取得的

vs \footnote{脚注7要分段。\par 不舒服不得不运河滩上野跑,头顶着毒热的阳光,身上再裹起兜肚,一不风凉,
二又窝汗,穿不了一天,就得起大半身痱子。再有,全村跟他一般大的小姑娘,
谁的兜肚也没有这么花儿草儿的鲜艳,他穿在身上,男不男,女不女,
小姑娘们要用手指刮破脸蛋儿。}未取得的

vs\footnote{脚注8}ge

vs\footnote{脚注9} jsty

vs\footnote{脚注10分三段。青大娘大高个儿,一双大脚,青铜肤色,
嗓门也亮堂,骂起人来,方圆二三十里,敢说找不出能够招架几个回合的敌手。
一丈青大娘骂人,就像雨打芭蕉,长短句。\par
青大娘大高个儿,一双大脚,青铜肤色,
嗓门也亮堂,骂起人来,方圆二三十里,敢说找不出能够招架几个回合的敌手。
一丈青大娘骂人,就像雨打芭蕉,长短句。 \par
青大娘大高个儿,一双大脚,青铜肤色,
嗓门也亮堂,骂起人来,方圆二三十里,敢说找不出能够招架几个回合的敌手。
一丈青大娘骂人,就像雨打芭蕉,长短句。}

Text%
\footnote{This is a footnote.}%
\footnote{This is a footnote.}%
\footnote{This is a footnote.}%
\footnote{This is a footnote.}%
\footnote{This is a footnote.}%
\footnote{This is a footnote.}%
\footnote{This is a footnote.}%
\footnote{This is a footnote.}%
\footnote{This is a footnote.}%
\footnote{This is a footnote.}%
\footnote{This is a footnote.}%
\footnote{This is a footnote.}%
\footnote{This is a footnote.}%
\footnote{This is a footnote.}%
\footnote{This is a footnote.}%
\footnote{This is a footnote.}%
\footnote{This is a footnote.}%
\footnote{This is a footnote.}%
\footnote{This is a footnote.}%
\footnote{This is a footnote.}%
\footnote{This is a footnote.}%
\footnote{This is a footnote.}%
\footnote{This is a footnote.}%
\footnote{This is a footnote.}%
\footnote{This is a footnote.}%
\footnote{This is a footnote.}%
\footnote{This is a footnote.}%
\footnote{This is a footnote.}%
\footnote{This is a footnote.}%
\footnote{This is a footnote.}%
\footnote{This is a footnote.}%
\footnote{This is a footnote.}%
\footnote{This is a footnote.}%
\footnote{This is a footnote.}%
\footnote{This is a footnote.}%
\footnote{This is a footnote.}%
\footnote{This is a footnote.}%
\footnote{This is a footnote.}%
\footnote{This is a footnote.}%
\footnote{This is a footnote.}%
\footnote{This is a footnote.}%
\footnote{This is a footnote.}%
\footnote{This is a footnote.}%
\footnote{This is a footnote.}%
\footnote{This is a footnote.}%
\footnote{This is a footnote.}%
\footnote{This is a footnote.}%
\footnote{This is a footnote.}%
\footnote{This is a footnote.}%
\footnote{This is a footnote.}%

\chapter{测试2 Test}
\section{数学 Math}
\[\pi=\sb{33}\]

\[
  \int\sin x\,\mathrm{d}x=\cos x + C
\]

\begin{proof}
道千乘之国,敬事而信,节用而爱人,使民以时。
\end{proof}

\begin{definition}
证明完毕/证讫,又写作Q.E.D.。这是拉丁词组“quod erat demonstrandum”(这就是所要证明的)的缩写,译自希腊语“ὅπερ ἔδει δεῖξαι”(hoper edei deixai),很多早期数学家用过,包括欧几里得和阿基米德。“Q.E.D.”可以在证明的尾段写出,以显示证明所需的结论已经完整了。
\end{definition}

\begin{lemma}
  这是一条华丽丽的引理。
\end{lemma}

\begin{proof}
先帝创业未半而中道崩殂,今天下三分,益州疲弊,此诚危急存亡之秋也。
\begin{equation}
  \sum_{k=0}^{\infty} \frac{1}{x^k} = \int \sin x dx
\end{equation}
\end{proof}

\begin{proof}
先帝创业未半而中道崩殂,今天下三分,益州疲弊,此诚危急存亡之秋也。
\begin{equation*}
  \sum_{k=0}^{\infty} \frac{1}{x^k} = \int \sin x dx
\end{equation*}
\end{proof}

\begin{lemma}
  这又是一条华丽丽的引理。
\end{lemma}

\chapter{测试3 Test}
\newcounter{thm}

\fdunewtheorem[style=plain,qed=\ensuremath{\sin}]{p}{平凡}
\fdunewtheorem[style=margin]{mm}{打断}
\fdunewtheorem[style=change]{fduc}{变革}
\fdunewtheorem[style=break]{fdub}{平凡}
\fdunewtheorem[style=marginbreak]{mb}{打断}
\fdunewtheorem[style=break]{cb}{变革}
\fdunewtheorem*[style=plain]{np}{平凡}
\fdunewtheorem*[style=margin]{nmm}{打断}
\fdunewtheorem*[style=change]{nfduc}{变革}
\fdunewtheorem*[style=break,qed=]{nfdub}{平凡}
\fdunewtheorem*[style=marginbreak,qed={}]{nmb}{打断}
\fdunewtheorem[style=break,counter=thm]{ncb}{变革}

\newtheorem{prop}[thm]{命题}[chapter]

\begin{prop}
  直角三角形。
\end{prop}
\begin{prop}[圆形]
  直角三角形。
\end{prop}
\begin{prop}
  直角三角形。
\end{prop}

\begin{p}
这是一条定理。
\[ \sum_{k=0}^{\infty} \frac{1}{x^k} = \int \sin x dx \]
\end{p}

\begin{mm}[明早]
这是一条定理。
\[ \sum_{k=0}^{\infty} \frac{1}{x^k} = \int \sin x dx \]
\end{mm}

\begin{fduc}
这是一条定理。
\[ \sum_{k=0}^{\infty} \frac{1}{x^k} = \int \sin x dx \]
\end{fduc}

\begin{fdub}
这是一条定理。
\[ \sum_{k=0}^{\infty} \frac{1}{x^k} = \int \sin x dx \]
\end{fdub}

\begin{mb}
这是一条定理。
\[ \sum_{k=0}^{\infty} \frac{1}{x^k} = \int \sin x dx \]
\end{mb}

\begin{cb}
这是一条定理。
\[ \sum_{k=0}^{\infty} \frac{1}{x^k} = \int \sin x dx \]
\end{cb}

%%%%%%%%%%%%%%%%%%%%%%%%%%%%%%%%%%%%%%%%%%%%%

\begin{np}
这是一条定理。
\[ \sum_{k=0}^{\infty} \frac{1}{x^k} = \int \sin x dx \]
\end{np}

\begin{nmm}[明天一早]
这是一条定理。
\[ \sum_{k=0}^{\infty} \frac{1}{x^k} = \int \sin x dx \]
\end{nmm}

\begin{nfduc}
这是一条定理。
\[ \sum_{k=0}^{\infty} \frac{1}{x^k} = \int \sin x dx \]
\end{nfduc}

\begin{nfdub}
这是一条定理。
\[ \sum_{k=0}^{\infty} \frac{1}{x^k} = \int \sin x dx \]
\end{nfdub}

\begin{nmb}
这是一条定理。
\[ \sum_{k=0}^{\infty} \frac{1}{x^k} = \int \sin x dx \]
\end{nmb}

\begin{ncb}
这是一条定理。
\[ \sum_{k=0}^{\infty} \frac{1}{x^k} = \int \sin x dx \]
\end{ncb}

\begin{ncb}
这是一条定理。
\[ \sum_{k=0}^{\infty} \frac{1}{x^k} = \int \sin x dx \]
\end{ncb}

\begin{ncb}
这是一条定理。
\[ \sum_{k=0}^{\infty} \frac{1}{x^k} = \int \sin x dx \]
\end{ncb}

\begin{ncb}
这是一条定理。
\[ \sum_{k=0}^{\infty} \frac{1}{x^k} = \int \sin x dx \]
\end{ncb}

\chapter{章节}
\section{title}
\section{title}
\section{title}
\section{title}
\section{title}
\section{title}
\section{title}
\chapter{章节}
\section{title}
\section{title}
\section{title}
\section{title}
\section{title}
\section{title}
\section{title}
\chapter{章节}
\section{title}
\section{title}
\section{title}
\section{title}
\section{title}
\section{title}
\section{title}
\chapter{章b df节}
\section{title}
\section{title}
\section{title}
\section{tibff btle}
\section{title}
\section{title}
\section{title}
\section{title}
\section{title}
\section{title}
\section{title}
\section{title}
\section{title}
\section{title}
\chapter{章bdab节}
\section{title}
\section{title}
\section{title}
\section{title}
\section{title}
\section{title}
\section{titlbdae}
\chapter{章节badfb}
\section{title}
\section{title}
\section{title}
\section{title}
\section{title}
\section{title}
\section{title}
\chapter{章节vafdb}
\section{title}
\section{title}
\section{title}
\section{title}
\section{title}
\section{title}
\section{title}
\chapter{章节cadca}
\section{title}
\section{title}
\section{title}
\section{title}
\section{title}
\section{title}
\section{title}
\section{title}
\section{title}
\section{title}
\section{title}
\section{title}
\section{title}
\section{title}
\section{title}
\section{title}
\section{title}
\section{title}
\section{title}
\section{title}
\section{title}


\end{document}
