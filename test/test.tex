\PassOptionsToPackage{log-declarations=false}{xparse}
\documentclass[draft]{../source/fduthesis}


\fdusetup{
  style = {
    font = times,
    CJKfont = founder,
    fontsize = -4,
    fullwidth-stop = true
  },
  info = {
    title = {半导体 PN 结的物理特性},
    title* = {The Physical Properties of Semiconductory p-n Junction},
%    date = {2017年2月10日},
    author = {曾祥东},
    author* = {Xiangdong Zeng},
    supervisor = {陈丙丁 \quad 工程师},
    instructors = {
      张五六 \quad 工程师,
      赵\quad 甲 \quad 工程师,
      王三四 \quad 讲\quad 师
    },
    major = {物理学},
    department = {物理物理系},
    studentID = {14307110142},
    schoolID = {10246},
    keywords = {\LaTeX, hih, vwvs, 物理, 中心法则, vsbd, 为侨服务},
    keywords* = {\LaTeX, hih, vwvs, Physics, Central Rule, vsbd},
    CLC = {O414.1/65}
  }
}

% \ExplSyntaxOn
% \tl_const:Nn \c__fdu_full_stop_ideographic_tl {\symbol{ "3002 }}
% \tl_const:Nn \c__fdu_full_stop_fullwidth_tl   {\symbol{ "FF0E }}

% \char_set_active_eq:nN {"3002} \c__fdu_full_stop_fullwidth_tl
% \char_set_catcode_active:n {"3002}

% \ExplSyntaxOff

\newcommand{\fonttesttext}{你好,世界。Hello, world!}
\newcommand\fonttest{%
  正常:\qquad\fonttesttext \par
  粗体:\qquad\textbf{\fonttesttext} \par
  倾斜:\qquad\textit{\fonttesttext} \par
  小型大写:\qquad\textsc{\fonttesttext}
}




\begin{document}
\frontmatter
\makecover

\mainmatter
\begin{abstract}
中心法则形成了分子生物学的生命观:生命世界在信息和规律上是统一的;
生命的物质基础和主宰物质是不同的,生命的物质基础是以核酸蛋白质整合
体系为主宰的原生质各种必要的物质组分及其实在的相互作用;生命运动
的本质在于组成生命的各物质之间以及生命的物质、能量与信息之间和生命
自身与环境之间的不断的相互作用。不少人不少人,如果是到无穷大。

上标vasdv;此时的的

上标vasdv ;此时的的

上标\TeX;此时的的

上标\TeX ;此时的的

上标\LaTeX;此时的的

上标\LaTeX ;此时的的
本科毕业论文工作是高校实现人才培养目标的重要教学环节。为
贯彻落实教育部和上海市关于加强普通高等学校毕业论文工作的有
关文件和学校的要求,为进一步推进物理系本科毕业论文工作的规范
管理,作好充分准备迎接教育部明年对我校的本科教学工作水平评
估,结合物理系的学科特点,明确物理系本科毕业论文应达到的质量
标准,建立有效的管理和监控机制,特制订《复旦大学物理系本科毕
业论文工作管理办法》。
每年的十二月上旬开始启动毕业论文工作。由物理系分管系主
任、教学秘书、凝聚态物理、理论物理和光学科研组的教授各一人组
成物理系本科毕业论文领导小组,负责整个毕业论文工作。
在启动后的第一周内,由各学科的有关教授为学生介绍本学科的
研究方向和目前的研究课题,接受学生的有关咨询,指导教师提出可
供学生选择的毕业论文题目。鼓励学生和不同的指导教师接触交流,
确定自己感兴趣的论文题目和指导教师,一般情况下不鼓励学生做 2
个以上的题目。一位指导教师可以同时指导 2-3 名学生,一般不超过
5 名。在学期结束以前确定指导教师和论文题目。对外推直研的学生
允许到对应的研究所或学校去做毕业论文,也要事先确定指导教师和
论文题目。

确定指导教师和论文题目后,要求学生在寒假期间阅读导师布
置的有关文献资料,熟悉自己的研究方向,并在开学后的 2 周内在导
师指导下填写开题报告,明确研究内容和预期目标。如果原定的论文
题目确实不适合自己,可以提出申请更换题目或导师,在有关导师的
同意下,经系毕业论文工作领导小组批准更换论文题目或导师,并重
新填写开题报告。

学生要在导师的指导下拟定工作进度,导师应当督促学生积极
投入毕业论文的工作,并给予具体的指导,把握学生的研究方向和进
展。

四月上旬进行毕业论文的中期检查,学生在导师指导下填写中
期汇报表。对于工作努力,论文进展顺利的同学予以表扬和鼓励。对
于没能按时间进度完成任务的同学要予以予以批评和警告。导师应当
督促检查学生的论文进展情况,及时帮助解决学生在研究中遇到的问
题和困难。

五月中旬进入毕业论文写作阶段,论文必须经导师审阅修改。
论文应当按统一格式撰写,封面由学校提供,内页用 A4 纸打印。论
文可以使用 Word 或 \LaTeX{} 等软件打印,应当包括论文题目、作者和
指导教师、中英文摘要、关键词、正文、图表、参考文献等部分,凡
属于引用别人的工作必须给出相关的参考文献,坚决杜绝抄袭行为。
导师应当认真审阅学生的毕业论文,提出具体的修改意见,把握论文
的科学性,并在最后给出论文的审阅意见。6 月 10 号前应当完成论
文的写作。
\end{abstract}

\begin{abstract*}
\LaTeX3 does not use @ as a ``letter'' for defining internal
macros. Instead, the symbols are used in internal macro names to
provide structure. The name of each function is divided into
logical units using separates the name of the function from the
argument specifier (``arg-spec''). This describes the arguments
expected by the function.

In most cases, each argument is represented by a single letter. The
complete list of arg-spec letters for a function is referred to as
the signature of the function.

Each function name starts with the module to which it belongs. Thus
apart from a small number of very basic functions, all expl3
function names contain at least one under-score to divide the
module name from the descriptive name of the function. For example,
all functions concerned with comma lists are in module clist and
begin. Every function must include an argument specifier. For
functions which take no arguments, this will be blank and the
function name will end :. Most functions take one or more
arguments, and use the following argument specifiers.
\end{abstract*}

\begin{notation}
$\sin$      &  正弦 \\
HPC         &  高性能计算 (High Performance Computing) \\
cluster     &  集群 \\
Itanium     &  安腾 \\
SMP         &  对称多处理 \\
API         &  应用程序编程接口 \\
PI          &  聚酰亚胺 \\
MPI         &  聚酰亚胺模型化合物,N-苯基邻苯酰亚胺 \\
PBI         &  聚苯并咪唑 \\
MPBI        &  聚苯并咪唑模型化合物,N-苯基苯并咪唑 \\
PY          &  聚吡咙 \\
PMDA-BDA    &  均苯四酸二酐与联苯四胺合成的聚吡咙薄膜 \\
$\Delta G$  &  活化自由能 (Activation Free Energy) \\
$\chi$      &  传输系数 (Transmission Coefficient) \\
$E$         &  能量 \\
$m$         &  质量 \\
$c$         &  光速 \\
$P$         &  概率 \\
$T$         &  时间 \\
$v$         &  速度 \\
劝学        &  君子曰:学不可以已。青,取之于蓝,而青于蓝;冰,水为之,而寒于水。木
  直中绳。\symbol{"2B413}\symbol{"8F2E} d 輮以为轮,其曲中规。虽有槁暴,不复挺者,輮使之然也。故木受绳则直,金就
  砺则利,君子博学而日参省乎己,则知明而行无过矣。—— 荀况 \\
$\sin$      &  正弦 \\
HPC         &  高性能计算 (High Performance Computing) \\
cluster     &  集群 \\
Itanium     &  安腾 \\
SMP         &  对称多处理 \\
API         &  应用程序编程接口 \\
PI          &  聚酰亚胺 \\
MPI         &  聚酰亚胺模型化合物,N-苯基邻苯酰亚胺 \\
PBI         &  聚苯并咪唑 \\
MPBI        &  聚苯并咪唑模型化合物,N-苯基苯并咪唑 \\
PY          &  聚吡咙
% PMDA-BDA    &  均苯四酸二酐与联苯四胺合成的聚吡咙薄膜 \\
% $\Delta G$  &  活化自由能 (Activation Free Energy) \\
% $\chi$      &  传输系数 (Transmission Coefficient) \\
% $E$         &  能量 \\
% $m$         &  质量 \\
% $c$         &  光速 \\
% $P$         &  概率 \\
% $T$         &  时间 \\
% $v$         &  速度 \\
% 劝学        &  君子曰:学不可以已。青,取之于蓝,而青于蓝;冰,水为之,而寒于水。木
%   直中绳。輮以为轮,其曲中规。虽有槁暴,不复挺者,輮使之然也。故木受绳则直,金就
%   砺则利,君子博学而日参省乎己,则知明而行无过矣。—— 荀况 \\
% $\sin$      &  正弦 \\
% HPC         &  高性能计算 (High Performance Computing) \\
% cluster     &  集群 \\
% Itanium     &  安腾 \\
% SMP         &  对称多处理 \\
% API         &  应用程序编程接口 \\
% PI          &  聚酰亚胺 \\
% MPI         &  聚酰亚胺模型化合物,N-苯基邻苯酰亚胺 \\
% PBI         &  聚苯并咪唑 \\
% MPBI        &  聚苯并咪唑模型化合物,N-苯基苯并咪唑 \\
% PY          &  聚吡咙 \\
% PMDA-BDA    &  均苯四酸二酐与联苯四胺合成的聚吡咙薄膜 \\
% $\Delta G$  &  活化自由能 (Activation Free Energy) \\
% $\chi$      &  传输系数 (Transmission Coefficient) \\
% $E$         &  能量 \\
% $m$         &  质量 \\
% $c$         &  光速 \\
% $P$         &  概率 \\
% $T$         &  时间 \\
% $v$         &  速度 \\
% 劝学        &  君子曰:学不可以已。青,取之于蓝,而青于蓝;冰,水为之,而寒于水。木
%   直中绳。輮以为轮,其曲中规。虽有槁暴,不复挺者,輮使之然也。故木受绳则直,金就
%   砺则利,君子博学而日参省乎己,则知明而行无过矣。—— 荀况 \\
% $\sin$      &  正弦 \\
% HPC         &  高性能计算 (High Performance Computing) \\
% cluster     &  集群 \\
% Itanium     &  安腾 \\
% SMP         &  对称多处理 \\
% API         &  应用程序编程接口 \\
% PI          &  聚酰亚胺 \\
% MPI         &  聚酰亚胺模型化合物,N-苯基邻苯酰亚胺 \\
% PBI         &  聚苯并咪唑 \\
% MPBI        &  聚苯并咪唑模型化合物,N-苯基苯并咪唑 \\
% PY          &  聚吡咙 \\
% PMDA-BDA    &  均苯四酸二酐与联苯四胺合成的聚吡咙薄膜 \\
% $\Delta G$  &  活化自由能 (Activation Free Energy) \\
% $\chi$      &  传输系数 (Transmission Coefficient) \\
% $E$         &  能量 \\
% $m$         &  质量 \\
% $c$         &  光速 \\
% $P$         &  概率 \\
% $T$         &  时间 \\
% $v$         &  速度 \\
% 劝学        &  君子曰:学不可以已。青,取之于蓝,而青于蓝;冰,水为之,而寒于水。木
%   直中绳。輮以为轮,其曲中规。虽有槁暴,不复挺者,輮使之然也。故木受绳则直,金就
%   砺则利,君子博学而日参省乎己,则知明而行无过矣。—— 荀况
\end{notation}




\chapter{测试 Test}
\section{文字与段落 Text and paragraph}
\subsection{中文文本 Chinese}
如果\symbol{"002E}会突然:FULL STOP

如果\symbol{"3002}会突然:IDEOGRAPHIC FULL STOP

如果\symbol{"FF0E}会突然:FULLWIDTH FULL STOP

如果\symbol{"FF61}会突然:HALFWIDTH IDEOGRAPHIC FULL STOP

脚注\footnote{脚注1如果会突然}。

脚注 \footnote{脚注2。千千万的。}。未取得的

脚注。\footnote{脚注3是一个长脚注。未取得的那一年是一九三六年。何满子六岁,剃个光葫芦头,天灵盖上留着个木梳背儿;
一交立夏就光屁股,晒得两道眉毛只剩下淡淡的痕影,鼻梁子裂了皮,
全身上下就像刚从烟囱里爬出来,连眼珠都比立夏之前乌黑。她家坐落在北运河岸上,门口外就是大河。有一回,一只外江大帆船打门口路
过,也正是歇晌时分。一丈青大娘站在篱笆外的伞柳阴下放鸭子,
一见几个纤夫赤身露体,只系着一条围腰,裤子卷起来盘在头上,便断喝一声:

“站住!”这几个纤夫头顶着火盆子,拉了百八十里路,顶水又逆风,
还没有歇脚打尖,个顶个窝着一肚子饿火。一丈青大娘的这一声断喝,
他们只当耳旁风。一丈青大娘见他们头也不抬,理也不理,气更大了,又吆喝
了一声:“都给我穿上裤子!”有个年轻不知好歹的纤夫,白瞪了一丈青大娘
一眼,没好气地说:“一大把岁数儿,什么没见过;不爱看合上眼,
掉过脸去!”一丈青大娘火了起来,挽了挽袖口,手腕子上露出两只叮叮当当
响的黄铜镯子,一阵风冲下河坡,阻挡在这几个纤夫的面前,手戳着他们的
鼻子说:“不能叫你们腌臢了我们大姑娘小媳妇的眼睛!”那个不知好歹的年
轻纤夫,是个生楞儿,用手一推一丈青大娘,

说:“好狗不挡道!”
这一下可捅了蜂窝。一丈青大娘勃然大怒,老大一个耳刮子抢圆了扇过去;
那个年轻的纤夫就像风吹乍篷,转了三转,拧了三圈儿,满脸开花,
口鼻出血,一头栽倒在滚烫的沙滩上,紧一口慢一口倒气,高一声低一声呻吟。
几个纤夫见他们的伙伴挨了打,唿哨而上;只听咯吧一声,一丈青大娘折断了
一棵茶碗口粗细的河柳,带着呼呼风声挥舞起来,把这几个纤夫扫下河去,
就像正月十五煮元宵,纷纷落水。}未取得的

\textit{脚注倾斜。 \footnote{脚注4}未取得的}

\textbf{脚注加粗。\footnote{脚注5} 未取得的}

vs\footnote{脚注6}未取得的

vs \footnote{脚注7要分段。\par 不舒服不得不运河滩上野跑,头顶着毒热的阳光,身上再裹起兜肚,一不风凉,
二又窝汗,穿不了一天,就得起大半身痱子。再有,全村跟他一般大的小姑娘,
谁的兜肚也没有这么花儿草儿的鲜艳,他穿在身上,男不男,女不女,
小姑娘们要用手指刮破脸蛋儿。}未取得的

vs\footnote{脚注8}ge

vs\footnote{脚注9} jsty

vs\footnote{脚注10分三段。青大娘;大高个儿,一双大脚,青铜肤色,
嗓门也亮堂,骂起人来,方圆二三十里,敢说找不出能够招架几个回合的敌手。
一丈青大娘骂人,就像雨打芭蕉,长短句。\par 青大娘;大高个儿,一双大脚,青铜肤色,
嗓门也亮堂,骂起人来,方圆二三十里,敢说找不出能够招架几个回合的敌手。
一丈青大娘骂人,就像雨打芭蕉,长短句。 \par 青大娘;大高个儿,一双大脚,青铜肤色,
嗓门也亮堂,骂起人来,方圆二三十里,敢说找不出能够招架几个回合的敌手。
一丈青大娘骂人,就像雨打芭蕉,长短句。}

七月天,中伏大晌午,热得像天上下火。何满子被爷爷拴在葡萄架的, oooooooo oooo oooooooo ooooooo 立柱上,
系的是挂贼扣儿。
那一年是一九三六年。何满子六岁,剃个光葫芦头,天灵盖上留着个木梳背儿;
一交立夏就光屁股,晒得两道眉毛只剩下淡淡的痕影,鼻梁子裂了皮,
全身上下就像刚从烟囱里爬出来,连眼珠都比立夏之前乌黑。
奶奶叫东隔壁的望日莲姑姑给何满子做了一条大红兜肚,兜肚上还用五彩细线
绣了一大堆花草。人配衣裳马配鞍,何满子穿上这条花红兜肚,
一定会在小伙伴们中间出人头地。可是,何满子一天也不穿。
何满子整天在运河滩上野跑,头顶着毒热的阳光,身上再裹起兜肚,一不风凉,
二又窝汗,穿不了一天,就得起大半身痱子。再有,全村跟他一般大的小姑娘,
谁的兜肚也没有这么花儿草儿的鲜艳,他穿在身上,男不男,女不女,
小姑娘们要用手指刮破脸蛋儿,臊得他找个田鼠窝钻进去;
小小子儿们也要敲起锣鼓似的叫他小丫头儿,管叫他一辈子抬不起头。

何满子不穿花红兜肚,奶奶气得咬牙切齿地骂他,手握着擀面杖要梆他,还威
吓要三天不给他饭吃。原来,这条兜肚大有讲究。何满子是个娇哥儿,
奶奶老是怕阎王爷打发白无常把他勾走;听说阎王爷非常重男轻女,
何满子穿上花红兜肚,男扮女妆,阎王爷老眼昏花地看不真切,
也就起不了勾魂索命的恶念。

何满子的奶奶,人人都管她叫一丈青大娘;大高个儿,一双大脚,青铜肤色,
嗓门也亮堂,骂起人来,方圆二三十里,敢说找不出能够招架几个回合的敌手。
一丈青大娘骂人,就像雨打芭蕉,长短句,四六体,鼓点似的骂一天,
一气呵成,也不倒嗓子。她也能打架,动起手来,别看五六十岁了,
三五个大小伙子不够她打一锅的。

她家坐落在北运河岸上,门口外就是大河。有一回,一只外江大帆船打门口路
过,也正是歇晌时分。一丈青大娘站在篱笆外的伞柳阴下放鸭子,
一见几个纤夫赤身露体,只系着一条围腰,裤子卷起来盘在头上,便断喝一声:
“站住!”这几个纤夫头顶着火盆子,拉了百八十里路,顶水又逆风,
还没有歇脚打尖,个顶个窝着一肚子饿火。一丈青大娘的这一声断喝,
他们只当耳旁风。一丈青大娘见他们头也不抬,理也不理,气更大了,又吆喝
了一声:“都给我穿上裤子!”有个年轻不知好歹的纤夫,白瞪了一丈青大娘
一眼,没好气地说:“一大把岁数儿,什么没见过;不爱看合上眼,
掉过脸去!”一丈青大娘火了起来,挽了挽袖口,手腕子上露出两只叮叮当当
响的黄铜镯子,一阵风冲下河坡,阻挡在这几个纤夫的面前,手戳着他们的
鼻子说:“不能叫你们腌臢了我们大姑娘小媳妇的眼睛!”那个不知好歹的年
轻纤夫,是个生楞儿,用手一推一丈青大娘,说:“好狗不挡道!”
这一下可捅了蜂窝。一丈青大娘勃然大怒,老大一个耳刮子抢圆了扇过去;
那个年轻的纤夫就像风吹乍篷,转了三转,拧了三圈儿,满脸开花,
口鼻出血,一头栽倒在滚烫的沙滩上,紧一口慢一口倒气,高一声低一声呻吟。
几个纤夫见他们的伙伴挨了打,唿哨而上;只听咯吧一声,一丈青大娘折断了
一棵茶碗口粗细的河柳,带着呼呼风声挥舞起来,把这几个纤夫扫下河去,
就像正月十五煮元宵,纷纷落水。一丈青大娘不依不饶,站在河边大骂不住声,
还不许那几个纤夫爬上岸来;大帆船失去了纤力,掌舵的绽裂了虎口,
也驾驭不住,在河上转开了磨。最后,还是船老板请出了摆渡船的柳罐斗,
钉掌铺的吉老秤,老木匠郑端午,开小店的花鞋杜四,说和了两三个时辰,
一丈青大娘才算开恩放行。

一丈青大娘有一双长满老茧的大手,种地、撑船、打鱼都是行家。她还会扎针、
拔罐子、接生。接骨、看红伤。这个小村大人小孩有个头痛脑热,都来找她
妙手回春;全村三十岁以下的人,都是她那一双粗大的手给接来了人间。

\subsection{英文文本 English}
The House of Representatives shall be composed of Members chosen
every second Year by the People of the several States, and the
Electors in each State shall have the Qualifications requisite for
Electors of the most numerous Branch of the State Legislature.

No Person shall be a Representative who shall not have attained to
the Age of twenty five Years, and been seven Years a Citizen of the
United States, and who shall not, when elected, be an Inhabitant of
that State in which he shall be chosen.

Representatives and direct Taxes shall be apportioned among the
several States which may be included within this Union, according to
their respective Numbers, which shall be determined by adding to the
whole Number of free Persons, including those bound to Service for a
Term of Years, and excluding Indians not taxed, three fifths of all
other Persons.

The actual Enumerationoraten shall be made within three Years after the
first Meeting onoratef the Conoratengress onoratef the United States, and within every
subsequent Term onoratef ten Years, in such Manner as they shall by Law
direct. The Number onoratef Representatives shall nonoratet exceed onoratene fonorater every
thirty Thonorateusand, but each State shall have at Least onoratene
Representative; and until such enumerationoraten shall be made, the State
onoratef New Hampshire shall be entitled tonorate chuse three, Massachusetts
eight, Rhonoratede Island and Pronoratevidence Plantationoratens onoratene, Conoratennecticut five,
New Yonoraterk six, New Jersey fonorateur, Pennsylvania eight, Delaware onoratene,
Maryland six, Virginia ten, Nonoraterth Caronoratelina five, Sonorateuth Caronoratelina
five and Geonoratergia three.

When vacancies happen in the Representation from any State, the
Executive Authority thereof shall issue Writs of Election to fill
such Vacancies.

The House of Representatives shall chuse their Speaker and other
Officers; and shall have the sole Power of Impeachment.

\section{字体 Font}
\subsection{普通字体 Roman}
\fonttest

\subsection{无衬线字体 Sans-serif}
\textsf{\fonttest}

\subsection{打字机字体 Typewriter}
\texttt{\fonttest}

\hbox{one}\footnote{This is a footnote.}\footnote{This is a footnote.}\footnote{This is a footnote.}\footnote{This is a footnote.}\footnote{This is a footnote.}\footnote{This is a footnote.}\footnote{This is a footnote.}\footnote{This is a footnote.}\footnote{This is a footnote.}\footnote{This is a footnote.}\footnote{This is a footnote.}\footnote{This is a footnote.}\footnote{This is a footnote.}\footnote{This is a footnote.}\footnote{This is a footnote.}\footnote{This is a footnote.}\footnote{This is a footnote.}\footnote{This is a footnote.}\footnote{This is a footnote.}\footnote{This is a footnote.}\footnote{This is a footnote.}\footnote{This is a footnote.}\footnote{This is a footnote.}\footnote{This is a footnote.}\footnote{This is a footnote.}\footnote{This is a footnote.}\footnote{This is a footnote.}\footnote{This is a footnote.}\footnote{This is a footnote.}\footnote{This is a footnote.}\footnote{This is a footnote.}\footnote{This is a footnote.}\footnote{This is a footnote.}\footnote{This is a footnote.}\footnote{This is a footnote.}\footnote{This is a footnote.}\footnote{This is a footnote.}\footnote{This is a footnote.}\footnote{This is a footnote.}\footnote{This is a footnote.}\footnote{This is a footnote.}\footnote{This is a footnote.}\footnote{This is a footnote.}\footnote{This is a footnote.}\footnote{This is a footnote.}\footnote{This is a footnote.}\footnote{This is a footnote.}\footnote{This is a footnote.}\footnote{This is a footnote.}\footnote{This is a footnote.}
%\hbox{two}
%three
%
%\mbox{one}
%\mbox{two}
%three

\chapter{测试2 Test}
\section{数学 Math}
\[\pi=\sb{33}\]

\[
  \int\sin x\,\mathrm{d}x=\cos x + C
\]

\chapter{测试3 Test}
%\makeatletter
%\iffdu@font@times
%	times true
%\else
%	times false
%\fi
%
%\iffdu@font@lm
%	latin modern true
%\else
%	latin modern false
%\fi
%\makeatother
\end{document}