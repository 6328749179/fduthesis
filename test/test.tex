\PassOptionsToPackage{log-declarations=false}{xparse}
\documentclass[oneside]{fduthesis}
\usepackage{kantlipsum}
\usepackage{zhlipsum}

%\usepackage{gbt-7714-2015}

\fdusetup{
  style = {
    % font = none,
    % cjkfont = windows,
    % fullwidthstop = all,
    footnotestyle = libertinus*,
    % automakecover = false
    logo = {../logo/pdf/fudan-name-black.pdf},
    % logosize = {2cm,2cm,2cm}
    hyperlink = color,
    hyperlinkcolor = default,
    % bibstyle = authoryear
  },
  info = {
    title = {自河南经乱关内阻饥兄弟离散各在一处},
    title* = {Measurements of the interaction between energetic
      photons and hadrons show that the interaction},
%    date = {2017年2月10日},
    author = {某某某},
    author* = {Xiangdong Zeng},
    supervisor = {陈丙丁 \quad 教授},
    instructors = {
      张五六 \quad 工程师,
      赵\quad 甲 \quad 工程师,
      王三四 \quad 讲\quad 师
    },
    major = {物理学},
    department = {凝聚态物理系},
    secretlevel = ii,
    secretyear = {三年},
    studentid = {14307110000},
    keywords = {\LaTeX, hih, vwvs, 物理, 中心法则, vsbd, 为侨服务},
    keywords* = {\LaTeX, hih, vwvs, Physics, Central Rule, vsbd},
    clc = {O414.1/65}
  }
}

\makeatletter\ExplSyntaxOn
\ctex_patch_cmd:Nnn \bibliography
  { \@input@ { \jobname.bbl } }
  {
    \begingroup
    \renewcommand \href [ 2 ] {##2}
    \@input@ { \jobname.bbl }
    \endgroup
  }
\ExplSyntaxOff\makeatother

\def\BibTeX{B\textsc{ib}\TeX}

\begin{document}

\frontmatter
\tableofcontents

\begin{abstract}
\zhlipsum[1-2]
\end{abstract}

\begin{abstract*}
\kant[1-2]
\end{abstract*}

\begin{notation}[lp{15em}]
$\sin$      &  正弦 \\
HPC         &  高性能计算 (High Performance Computing) \\
cluster     &  集群 \\
Itanium     &  安腾 \\
SMP         &  对称多处理 \\
API         &  应用程序编程接口 \\
PI          &  聚酰亚胺 \\
MPI         &  聚酰亚胺模型化合物,N-苯基邻苯酰亚胺 \\
PBI         &  聚苯并咪唑 \\
MPBI        &  聚苯并咪唑模型化合物,N-苯基苯并咪唑 \\
PY          &  聚吡咙 \\
PMDA-BDA    &  均苯四酸二酐与联苯四胺合成的聚吡咙薄膜 \\
$\Delta G$  &  活化自由能 (Activation Free Energy) \\
$\chi$      &  传输系数 (Transmission Coefficient) \\
$E$         &  能量 \\
$m$         &  质量 \\
$c$         &  光速 \\
$P$         &  概率 \\
$T$         &  时间 \\
$v$         &  速度
\end{notation}

\mainmatter

\chapter{文本}

\section{文字与段落}

\textbf{本段使用 \texttt{\string\cite}}
Myriad,英语单词,意为「无数的」\cite{sunstein,gjhjbhjkjbzs,hblzsthjkjyxgs}。
同时,「Myriad」也是一款字体的名字。
由罗伯特·斯林巴赫(Robert Slimbach,1956年--)和卡罗·图温布利
(Carol Twombly,1959年-)\cite{wyf,sunstein,zgtsgxh,lzp1}
在1990年到1992年期间以 Frutiger 字体为蓝本为 Adobe 公司设计\cite{cdy}。
Myriad 是早期数码字体时代的先驱,\cite{wfz,wfz1}
伴随着技术的成长一路走来 \cite{hlswedl,zgdylsdag}。

\textbf{本段使用 \texttt{\string\citep}}
如今,它更多地和我们相见在显示屏幕上 \citep{wfz2}。当然,还有那著名的标榜设计的
电子品牌 \citep{cgw,mks}。1992 年,耗时两年开发的 Myriad 终于发布了历史上第一个版本:
Myriad MM \citep{wyf,hblzsthjkjyxgs,sunstein,zgtsgxh,aaas}。

\section{title}

\textbf{本段使用 \texttt{\string\citet}}
这款温和且具有良好可读性的人文主义无衬线字体\citet{yjb},集诸多当时最新的数字
字体技术于一身。 后缀 MM,意为 Multiple Master,没有找到对应的中文
译名\citet{lbm,calkin},我们权且称之为「多母板技术」。Myriad 是最早采用 Multiple Master
技术的无衬线字体之一。这项技术的原理是在坐标轴(Axis)的区间两端设计
极限母板,中间的变量则采取线性或非线性变化,对于字体来说,字型的宽度、
粗细甚至有无衬线\citet{xadzkjdx,yufin,cgw},都可以在坐标轴上设置。此外,MM 技术还提供了在小字号
下屏幕显示的视觉修正(Optical Adjustment),也就是说,同一款字体,在
小字号时,其字间距和笔画粗细,会被适当地放大。而衬线字体,随着字号的
变小,衬线会相对变粗。视觉修正可以提高小字号字体的识别性,对于远低于
印刷分辨率的电脑屏幕来说,也具有重要意义。

在 Multiple Master 的时代,字号是从6pt到72pt之间非线性设置的。这一传统
保留到了今天 Truetype 和 Opentype 的 Single Master 时代。Adobe 软件的
字体下拉菜单,仍然只显示6到72pt 的字号。

\bibliography{test}

\end{document}
